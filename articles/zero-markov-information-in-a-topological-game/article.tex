\documentclass{amsart}
\usepackage{amsmath}
\usepackage{amsthm}
\usepackage{amssymb}

\usepackage{graphicx}

\usepackage{../../clontzDefinitions}

\newtheorem{theorem}{Theorem}[section]
\newtheorem{proposition}[theorem]{Proposition}
\newtheorem{lemma}[theorem]{Lemma}
\newtheorem{corollary}[theorem]{Corollary}


\theoremstyle{definition}
\newtheorem{definition}[theorem]{Definition}
\newtheorem{game}[theorem]{Game}
\newtheorem{example}[theorem]{Example}
\newtheorem{question}[theorem]{Question}



\begin{document}

% \title{Proximal compact spaces are Corson compact\tnoteref{t1}}
% \tnotetext[t1]{2010 Mathematics Subject Classification. 54E15, 54D30, 54A20.}
\title{Zero-Markov information in topological games}

% \author[aub]{S.~Clontz\fnref{fn1}}
% \ead{steven.clontz@auburn.edu}
% \author[aub]{G.~Gruenhage\fnref{fn2}}
% \ead{gruengf@auburn.edu}

% \address[aub]{Department of Mathematics, Auburn University,
%  Auburn, AL 36830}


\author{Steven Clontz}
\address{Department of Mathematics and Statistics, Univeristy of North Carolina
at Charlotte, Charlotte, NC 28223}
\email{steven.clontz@gmail.edu}
\urladdr{clontz.org}

\keywords{topological game, limited information strategy, \(k\)-spaces,
\(\sigma\)-compact spaces, hemicompact spaces}

\subjclass[2010]{54D20, 54D45}


\begin{abstract}
A \(0\)-Markov strategy in a topological game considers only the round number
and ignores all moves by the opponent. The existence of a winning \(0\)-Markov
strategy in either of two games due to Gruenhage characterizes hemicompactness
in either locally compact or compactly generated spaces. However, there
exists a non-compactly generated space for which there exists a winning
\(0\)-Markov strategy in one game but not the other.
\end{abstract}


\maketitle

\section{Introduction}

The following two topological games were introduced by Gary Gruenhage
in \cite{MR858337}.

\begin{game}
  Let $\gruKPGame{X}$ denote the \term{Gruenhage compact/point game}
  with players $\pl K$, $\pl P$ played on a topological space \(X\).
  During round $n$, $\pl K$ chooses
  a compact subset $K_n$ of $X$, followed by $\pl P$ choosing a point
  $p_n\in X$ such that $p_n\not\in \bigcup_{m\leq n}K_m$.

  $\pl K$ wins the game if the collection $\{\{p_n\}:n<\omega\}$ is locally
  finite in the space, and $\pl P$ wins otherwise.
\end{game}

\begin{game}
  Let $\gruKLGame{X}$ denote the \term{Gruenhage compact/compact game}
  with players $\pl K$, $\pl L$ played on a topological space \(X\).
  This game proceeds analogously to
  $\gruKPGame{X}$, except the second player $\pl L$ chooses compact sets $L_n$
  missing $\bigcup_{m\leq n}K_n$,
  and $\pl K$ wins if the collection $\{L_n:n<\omega\}$ is locally finite.
\end{game}

A \term{strategy} for a game defines the move a player makes each round as
a function of the history of the game (previous moves, the round number, etc.).
A \term{winning strategy} defeats every possible counterattack by the
opponent.
Note that a winning strategy in \(\gruKLGame{X}\) is also a winning strategy
in \(\gruKPGame{X}\) since singletons are compact.
In his paper, Gruenhage used these games to characterize several covering
properties using the existence of various kinds of winning strategies for
\(\pl K\) in the games. These results hold in the context of
\term{locally compact} spaces for which every point has a compact neighborhood.

\begin{definition}
  A space is \term{paracompact} if for every open cover $\mc U$ there exists a
  locally-finite open refinement $\mc V$ of $\mc U$ also covering the space.
\end{definition}

\begin{theorem}
  The following are equivalent for a locally compact space $X$:
    \begin{itemize}
      \item $X$ is paracompact
      \item $\pl K \win \gruKLGame{X}$. (\(\pl K\) has a winning strategy
      for the game.)
    \end{itemize}
\end{theorem}

\begin{definition}
  A space is \term{metacompact} if for every open cover $\mc U$ there exists a
  point-finite open refinement $\mc V$ of $\mc U$ also covering the space.
\end{definition}

\begin{theorem}
  The following are equivalent for a locally compact space $X$:
    \begin{itemize}
      \item $X$ is metacompact
      \item $\pl K \tactwin \gruKPGame{X}$ (\(\pl K\) has a \term{tactical}
      winning strategy which only considers the most recent move of
      the opponent each round)
    \end{itemize}
\end{theorem}

\begin{definition}
  A space is \term{$\sigma$-metacompact} if for every open cover $\mc U$ there
  exist point-finite open refinements $\mc V_n$ of $\mc U$ such that
  $\bigcup_{n<\omega}\mc V_n$ also covers the space.
\end{definition}

\begin{theorem}
  The following are equivalent for a locally compact space $X$:
    \begin{itemize}
      \item $X$ is $\sigma$-metacompact
      \item $\pl K \markwin \gruKPGame{X}$ (\(\pl K\) has a \term{Markov}
      winning strategy which only considers the most recent move of
      the opponent and the round number each round)
    \end{itemize}
\end{theorem}

Tactical and Markov strategies are examples of \term{limited information}
strategies. These may be generalized to \term{\(k\)-tactical} and
\term{\(k\)-Markov} strategies by allowing the player to use the \(k\)
most recent moves of the opponent; so \(1\)-tactical strategies are
simply tactical strategies, and similar for Markov. Of course, if
\(k<l\) then a winning \(k\)-tactical (resp. Markov) strategy
is itself a winning \(l\)-tactical (resp. Markov) strategy.
In \cite{QAClontzPreprint}
the author investigated \((k+1)\)-tactical/Markov strategies in \(\gruKPGame{X}\)
and showed that even for a complexly constructed space they can often be
improved to simply a tactical strategy; it remains open if this is always
the case.

In this paper we investigate the applications of \(0\)-Markov strategies
in both \(\gruKPGame{X}\) and \(\gruKLGame{X}\),
which we will call \term{predetermined} strategies as each move is determined
completely by the round number of the game and ignores all moves of the
opponent. It will be shown that for compactly generated spaces that a
predetermined winning strategy in \(\gruKPGame{X}\) can be used to get a
predetermined winning strategy in \(\gruKLGame{X}\).
However, there exists a non-compactly generated space
for which this does not work out.

\section{Locally compact spaces and predetermined strategies}

It's a known fact \cite{MR2048350} that amongst locally compact spaces the
following properties are equivalent.

\begin{definition}
  A space \(X\) is \term{Lindel\"of} if for every open cover of \(X\) there
  exists a countable subcover.
\end{definition}

\begin{definition}
  A space \(X\) is \term{\(\sigma\)-compact} if \(X=\bigcup_{n<\omega}K_n\)
  for \(K_n\) compact.
\end{definition}

\begin{definition}
  A space \(X\) is \term{hemicompact} if \(X=\bigcup_{n<\omega}K_n\)
  for \(K_n\) compact and every compact subset of \(X\) is contained in some
  \(K_n\).
\end{definition}

In general, hemicompact spaces are \(\sigma\)-compact, and \(\sigma\)-compact
spaces are Lindel\"of. By considering the games \(\gruKPGame{X}\) and
\(\gruKLGame{X}\) we will obtain an alternate proof that locally compact
Lindel\"of spaces are hemicompact.

\begin{theorem}
  If \(X\) is a locally compact Lindel\"of space, then
  \(\pl K \prewin \gruKLGame{X}\) (\(\pl K\) has a winning \(0\)-Markov
  a.k.a. predetermined strategy for the game.)
\end{theorem}

\begin{proof}
  For each \(x\in X\), let \(U_x\) be an open neighborhood of \(x\) with \(\cl{U_x}\)
  compact. Then as \(X\) is Lindel\"of, choose \(x_n\in X\) for \(n<\omega\) such that
  \(\{U_{x_n}:n<\omega\}\) covers \(X\). Define the predetermined strategy \(\sigma\)
  for \(\pl K\) by \(\sigma(n)=\cl{U_{x_n}}\).

  Let \(L:\omega\to \mc K(X)\) legally attack \(\sigma\), so
  \(L(n)\cap\bigcup_{m\leq n}\sigma(m)=\emptyset\). For each \(x\in X\),
  choose \(n<\omega\) with \(x\in U_{x_n}\). Then \(U_{x_n}\) is a neighborhood of
  \(x\) which intersects finitely many \(L(n)\), so \(\{L(n):n<\omega\}\) is
  locally finite.
\end{proof}

\begin{theorem}
  If \(\pl K\prewin \gruKPGame{X}\), then \(X\) is hemicompact.
\end{theorem}

\begin{proof}
  Let \(\sigma\) be a winning predetermined strategy for \(\pl K\) in
  \(\gruKPGame{X}\). If \(C\in \mc K(X)\) is compact, then for each \(x\in C\) let
  \(U_x\) be an open neighborhood of \(x\) which intersects finitely many
  \(\sigma(n)\). Choose \(x_i\in C\) for \(i<n<\omega\) such that
  \(\{U_{x_i}:i<n\}\) covers \(C\). Then \(\bigcup_{i<n}U_{x_i}\) contains \(C\)
  and intersects finitely many \(\sigma(n)\), and thus
  \(\{\bigcup_{m\leq n}\sigma(m):n<\omega\}\) witnesses hemicompactness.
\end{proof}

\begin{corollary}
  The following are equivalent for any locally compact space \(X\):
    \begin{itemize}
      \item \(X\) is Lindel\"of.
      \item \(X\) is \(\sigma\)-compact.
      \item \(X\) is hemicompact.
      \item \(\pl K \prewin \gruKPGame{X}\).
      \item \(\pl K \prewin \gruKLGame{X}\).
    \end{itemize}
\end{corollary}

\section{Compactly generated spaces and predetermined strategies}

\begin{definition}
  A space \(X\) is \term{compactly generated} if a set is closed if and
  only if its intersection with every compact set is closed. Such
  spaces are also known as \term{\(k\)-spaces}.
\end{definition}

All locally compact spaces are \(k\)-spaces. As will be shown,
the games \(\gruKPGame{X}\), \(\gruKLGame{X}\) are equivalent for \(\pl K\)'s
predetermined strategies in Hausdorff \(k\)-spaces.

\begin{definition}
  A space \(X\) is a \(k_\omega\)-space if there exist compact sets \(K_n\) for
  \(n<\omega\) such that a set is closed if and
  only if its intersection with every \(K_n\) is closed.
\end{definition}

\begin{theorem}
  If \(X\) is a \(k_\omega\)-space, then
  \(\pl K \prewin \gruKLGame{X}\).
\end{theorem}

\begin{proof}
  Let \(K_n\) witness that \(X\) is a \(k_\omega\)-space. Define the predetermined
  strategy \(\sigma\) for \(\pl K\) by \(\sigma(n)=K_n\).

  Let \(L:\omega\to\mc{K}(X)\) be a legal attack against \(\sigma\), and let
  \(L_{\omega\setminus n} = \bigcup_{n\leq m<\omega}L(m)\). Then as
    \[
      L_{\omega\setminus n}\cap K_p
        =
      \bigcup_{n\leq m< p}L(m) \cap \sigma(p)
    \]
  is compact for each \(p<\omega\), \(L_{\omega\setminus n}\) is closed.

  For each \(x\in X\), \(x\in \sigma(p)\) for some \(p\), so
  \(x\in X\setminus L_{\omega\setminus p}\) which misses all but finitely
  many \(L(n)\), showing that \(\{L(n):n<\omega\}\) is locally finite and
  \(\sigma\) is a winning predetermined strategy.
\end{proof}

The following result was observed in \cite{MR540599}; a proof is provided
for convenience.

\begin{proposition}
  Hemicompact \(k\)-spaces are \(k_\omega\)-spaces.
\end{proposition}

\begin{proof}
  Let \(K_n\) for \(n<\omega\) witness hemicompactness.
  If \(C\cap K_n\) is closed for each \(n<\omega\), then let \(K\) be any compact
  set. Since \(K\subseteq K_n\) for some \(n<\omega\), \(C\cap K\) is closed, and
  therefore \(C\) is closed.
\end{proof}

As we've already seen that \(\pl K \prewin \gruKPGame{X}\) implies
hemicompactness:

\begin{corollary}
  The following are equivalent for any \(k\)-space \(X\):
    \begin{itemize}
      \item \(X\) is \(k_{\omega}\).
      \item \(X\) is hemicompact.
      \item \(\pl K \prewin \gruKPGame{X}\).
      \item \(\pl K \prewin \gruKLGame{X}\).
    \end{itemize}
\end{corollary}

\section{
Non-equivalence of \(\pl K\prewin\gruKPGame{X}\), \(\pl K\prewin\gruKLGame{X}\)
}

For \(k\)-spaces, it has been shown that \(\gruKPGame{X}\) and \(\gruKLGame{X}\)
are equivalent with respect to \(\pl K\)'s winning predetermined strategies.
Looking at a subspace of the Stone-Cech compactification \(\beta\omega\) of
\(\omega\) reveals an example for which the predetermined strategies are not
equivalent.

\begin{definition}
  An \term{ultrafilter} on a cardinal \(\kappa\) is a maximal filter of non-empty
  subsets of \(\kappa\). For each \(\alpha\in \kappa\), the ultrafilter
  \(\mc F_\alpha\) containing all supersets of \(\{\alpha\}\) is called a
  \term{principal ultrafilter}. All
  ultrafilters not of this form are called \term{free ultrafilters}.
\end{definition}

\begin{definition}
  The \term{Stone-Cech compactification} of a cardinal
  \(\kappa\) is the space \(\beta\kappa\) consisting
  of all ultrafilters on \(\kappa\), with open sets of the form
  \(U_S=\{\mc F\in\beta\kappa : S\in\mc F\}\) for \(S\subseteq \kappa\).
\end{definition}

From these definitions it is easily verified that principal ultrafilters
are isolated, so \(\kappa\) with the discrete topology may be viewed as
a dense open subspace of \(\beta\kappa\).
We wish to consider the subspace
of \(\beta\omega\) consisting of all principal ultrafilters and a single free
ultrafilter \(\mc F\), denoted by \(\omega\cup\{\mc F\}\).

\begin{lemma}
  All compact subsets of \(\omega\cup\{\mc F\}\subset\beta\omega\) are finite.
  In particular, the difference of compact sets in \(\omega\cup\{\mc F\}\)
  is compact.
\end{lemma}

\begin{proof}
  Let \(I=\{n_i:i<\omega\}\cup\{\mc F\}\) be infinite.
  Then \(\{U_{\omega\setminus\{n_i:i\geq j\}}:j<\omega\}\) is an open cover of
  \(I\cup\{\mc F\}\) with no finite subcover.
\end{proof}

\begin{theorem}
  \(\pl K\notprewin \gruKLGame{\omega\cup\{\mc F\}}\) for any free
  ultrafilter \(\mc F\).
\end{theorem}

\begin{proof}
  Let \(\sigma\) be a predetermined strategy for \(\pl K\), and define the legal
  counter-attack \(H:\omega\to\mc K(X)\) by
  \(H(n)=(n\cup\sigma(n+1))\setminus\sigma(n)\). Then for any neighborhood
  \(U_S\) of \(\mc F\), \(S\) is infinite, and since
  \(\bigcup_{n<\omega} H(n)\supseteq\omega\setminus\sigma(0)\), \(U_S\) meets
  infinitely many of the finite \(H(n)\). Thus \(\sigma\) is not a winning
  predetermined strategy.
\end{proof}

\begin{theorem}
  There exists a free ultrafilter \(\mc F\) such that
  \(\pl K\prewin \gruKPGame{\omega\cup\{\mc F\}}\).
\end{theorem}

\begin{proof}
  Let \(\mc F\) be any free ultrafilter, and
  define the predetermined strategy \(\sigma\) by
  \(\sigma(n)=n^2\cup\{\mc F\}\).

  Consider the set of all legal attacks \(A\subseteq\omega^\omega\) by
  \(\pl P\) against \(\sigma\). For \(\{f_i:i\leq m\}\in [A]^{<\omega}\) and
  \(m<n<\omega\), each \(f_i\) maps only \(n\) points into \(n^2\), so
  \(\bigcup_{i\leq m}\ran{f_i}\) is coinfinite.
  Then \(\mc G'=\{\omega\setminus\ran f : f\in A\}\) is contained in a free
  ultrafilter \(\mc G\), and if \(\mc F=\mc G\), then \(\sigma\) is a
  winning predetermined strategy.
\end{proof}

It is not possible to prove in \(ZFC\) that
\(\pl K\prewin \gruKPGame{\omega\cup\{\mc F\}}\)
for arbitrary free ultrafilters.

\begin{definition}
  A \term{selective ultrafilter} \(\mc S\) is a free ultrafilter with the
  property that for every
  partition \(\{B_n : n < \omega\}\) of nonempty subsets of \(\omega\) such that
  \(B_n \not\in \mc{S}\) for all \(n\), there exists \(A \in \mc{S}\) such
  that \(|A \cap B_n|=1\) for all \(n\).
\end{definition}

\begin{theorem}
  CH implies the existence of a selective ultrafilter.
  \cite{MR0080902}
\end{theorem}

\begin{theorem}
  If \(\mc S\) is a selective ultrafilter, then
  \(\pl K\notprewin \gruKPGame{\omega\cup\{\mc S\}}\).
\end{theorem}

\begin{proof}
  Let \(\sigma\) be a predetermined strategy for \(\pl K\) such that
  \(\sigma(n)\supset\bigcup_{m<n}\sigma(m)\).
  Then define \(B_n=\omega\cap(\sigma(n+1)\setminus\sigma(n))\). Since \(B_n\)
  is always nonempty finite, \(B_n\not\in\mc F\) and there exists \(A\in \mc S\)
  such that \(|A\cap B_n|=1\).

  Define the legal counter-attack \(p:\omega\to\omega\cup\{\mc S\}\) by
  \(p(n)\in A\cap B_n=A\cap(\sigma(n+1)\setminus\sigma(n))\). Since
  \(A=(A\cap\sigma(0))\cup\{p(n):n<\omega\}\), \(\{p(n):n<\omega\}\in\mc S\).
  Therefore, every neighborhood of \(\mc F\) intersects infinitely many of
  the \(p(n)\), and \(p\) defeats the predetermined strategy \(\sigma\).
\end{proof}

Of particular note is that the author knows of no examples of a
non-\(k\)-space such that \(K\prewin\gruKPGame{X}\).

\begin{question}
  Does \(K\prewin\gruKPGame{X}\) imply \(X\) is a \(k\)-space?
\end{question}




\section{Acknowledgements}

The author would like to thank his PhD advisor Gary Gruenhage for his
support and mentorship while the author developed these results as a part of
his dissertation.


\bibliographystyle{plain}
\bibliography{../../bibliography}

\end{document}