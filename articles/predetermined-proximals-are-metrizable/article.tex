\documentclass{amsart}
\usepackage{amsmath}
\usepackage{amsthm}
\usepackage{amssymb}

\usepackage{../../clontzDefinitions}

\newtheorem{theorem}{Theorem}[section]
\newtheorem{proposition}[theorem]{Proposition}
\newtheorem{lemma}[theorem]{Lemma}
\newtheorem{corollary}[theorem]{Corollary}


\theoremstyle{definition}
\newtheorem{definition}[theorem]{Definition}
\newtheorem{game}[theorem]{Game}
\newtheorem{example}[theorem]{Example}
\newtheorem{question}[theorem]{Question}

\parskip=0.5em



\begin{document}

% \title{Proximal compact spaces are Corson compact\tnoteref{t1}}
% \tnotetext[t1]{2010 Mathematics Subject Classification. 54E15, 54D30, 54A20.}
\title{Predetermined proximal spaces are metrizable}

% \author[aub]{S.~Clontz\fnref{fn1}}
% \ead{steven.clontz@auburn.edu}
% \author[aub]{G.~Gruenhage\fnref{fn2}}
% \ead{gruengf@auburn.edu}

% \address[aub]{Department of Mathematics, Auburn University,
%  Auburn, AL 36830}


\author{Steven Clontz}
\address{Department of Mathematics and Statistics, University of South Alabama,
Mobile, AL 36688}
\email{sclontz@southalabama.edu}
\urladdr{clontz.org}

\keywords{Proximal; predetermined proximal; topological game; limited information strategies}
% \begin{keyword}
%   Proximal\sep absolutely proximal\sep Corson compact\sep uniform space\sep $W$-space
% \end{keyword}

\subjclass[2010]{54E15, 54D30, 54A20}


\begin{abstract}
TODO
%  The author and G. Gruenhage previously showed that J. Bell's proximal game
%  may be used to characterize Corson compactness in compact Hausdorff spaces.
%  Using tactical strategies considering only the most recent
%  move of the opponent, the proximal game may also be
%  used to characterize the strong Eberlein compactness property.
%  In doing so, a purely topological characterization of the proximal game
%  is introduced, and several existing results on
%  the proximal game are given
%  analogues considering limited information strategies.
\end{abstract}


\maketitle

\section{Predetermined Proximal}

We take the following from Willard's text.

\begin{definition}
A \term{normal covering sequence} for a space \(X\) is a sequence
\(\setB{\mc U_n}{n<\omega}\) of open covers such that
\(\mc U_{n+1}\) star-refines \(\mc U_n\). Such a sequence is
\term{compatible} with \(X\) if \(\setB{St(x,\mc U_n)}{n<\omega}\)
is a local base at each \(x\in X\).
\end{definition}
\begin{theorem}
A space \(X\) is psuedometrizable if and only if it has a compatible
normal covering sequence.
\end{theorem}

For convenience, we will recast these results in terms of entourages.

\begin{definition}
An \term{entourage sequence} for a space \(X\) 
\(\setB{D_n}{n<\omega}\) 
is \term{compatible} with \(X\) if \(\setB{D_n[x]}{n<\omega}\)
is a local base at each \(x\in X\).
\end{definition}
\begin{theorem}
A space \(X\) is psuedometrizable if and only if it has a compatible
entourage sequence.
\end{theorem}
\begin{proof}
Let \(d\) be a psuedometric generating \(X\); then \(\setB{D_n}{n<\omega}\)
given by \(D_n=\setB{\tuple{x,y}}{d(x,y)<2^{-n}}\) is a entourage sequence,
which is compatible since \(D_n[x]=B_{2^{-n}}(x)\). 

On the other hand, given a compatible entourage sequence \(\setB{D_n}{n<\omega}\),
let \(E_0=D_0\), \(E_{n+1}=\frac{1}{4}D_n\cap \frac{1}{4}E_n\), and
\(\mc U_n=\setB{E_{n+1}[x]}{x\in X}\). Fix \(x\in X\) and consider
\(St(x,\mc U_n)\subseteq St(E_{n+1}[x],\mc U_n)=
\bigcup\setB{E_{n+1}[y]\in\mc U_n}{E_{n+1}[x]\cap E_{n+1}[y]\not=\emptyset}\).

Let \(z\in St(E_{n+1},\mc U_n)\), so \(z\in E_{n+1}[y]\) for some \(y\in X\)
where \(w\in E_{n+1}[x]\cap E_{n+1}[y]\) for some \(w\in X\).
It follows that \(\tuple{z,y},\tuple{y,w},\tuple{w,x}\in E_{n+1}
=\frac{1}{4}D_{n}\cap\frac{1}{4}E_n\); therefore 
\(\tuple{z,x}\in D_{n}\cap E_n\) and \(z\in D_{n}[x]\cap E_n[x]\).
Therefore
\(St(x,\mc U_n)\subseteq St(E_{n+1}[x],\mc U_n)\subseteq D_{n}[x]\cap E_n[x]\).

We can then observe that \(\mc U_{n+1}\) star refines \(\mc U_{n}\)
since for each \(U\in\mc U_{n+1}\), \(U=E_{n+2}[x]\) for some \(x\in X\)
and \(St(E_{n+2}[x],\mc U_{n+1})\subseteq E_{n+1}[x]\in\mc U_n\),
making \(\setB{\mc U_n}{n<\omega}\) a normal covering sequence.
Finally, the sequence is compatible since for \(x\in X\),
\(St(x,\mc U_n)\subseteq D_{n}[x]\) and \(\setB{D_{n}[x]}{n<\omega}\)
is a local base at \(x\). 
\end{proof}

\begin{theorem}
A space \(X\) is psuedometrizable
if and only if 
\(\plI\prewin\bellConGame{X}\). 
\end{theorem}
\begin{proof}
Suppose \(X\) is psuedometrizable by \(d\);
then let \(\sigma\) be the predetermined strategy for \(\bellConGame{X}\) defined by
\(\sigma(n)=\setB{\tuple{x,y}}{d(x,y)<2^{-n}}\). 
For any legal attack \(\alpha\) against \(\sigma\), \(\alpha(n+1)\in\sigma(n)[\alpha(n)]\).
It follows that if \(x\in\bigcap_{n<\omega}\sigma(n)[\alpha(n)]\) and \(\epsilon>0\),
we may choose \(N<\omega\) such that \(2^{-N}<\epsilon\). Therefore 
\(d(x,\alpha(n))<2^{-n}\leq 2^{-N}<\epsilon\) for all \(n\geq N\), showing \(\alpha\) converges
to \(x\). Thus \(\sigma\) is a winning strategy.

Now let \(\sigma\) be any predetermined winning strategy satisfying \(\sigma(n)\subseteq\sigma(m)\)
for all \(n\geq m\), and suppose \(\setB{\frac{1}{2^{n+1}}\sigma(n)[x]}{n<\omega}\) is not a
local base at some \(x\in X\). Then we may pick an entourage \(D\) such that
\(\frac{1}{2^{n+1}}\sigma(n)[x]\not\subseteq D[x]\) for all \(n<\omega\). So choose
\(\alpha(n)\in\frac{1}{2^{n+1}}\sigma(n)[x]\setminus D[x]\).

Observe that \(\tuple{\alpha(n),x}\in\frac{1}{2^{n+1}}\sigma(n)\) and
\(\tuple{\alpha(n+1),x}\in\frac{1}{2^{n+2}}\sigma(n+1)\subseteq\frac{1}{2^{n+1}}\sigma(n)\).
It follows that \(\tuple{\alpha(n),\alpha(n+1)}\in\frac{1}{2^n}\sigma(n)\subseteq\sigma(n)\),
witnessing that \(\alpha(n+1)\in\sigma(n)[\alpha(n)]\), that is, \(\alpha\) is a legal counterattack
to \(\sigma\). Since \(x\in\frac{1}{2^{n+1}}\sigma(n)[\alpha(n)]\subseteq\sigma(n)[\alpha(n)]\)
for all \(n<\omega\), \(\sigma\) can only win for \(\plI\) if \(\alpha\) converges.
But \(\alpha(n)\not\in D[x]\) for all \(n<\omega\), so \(\alpha\) fails to converge as well.
Thus \(\sigma\) is not a winning strategy.

As a result, if \(\sigma\) is a winning predetermined strategy,
we have that \(\setB{\frac{1}{2^{n+1}}\sigma(n)[x]}{n<\omega}\) is a
local base at each \(x\in X\). 
This shows that \(\setB{\frac{1}{2^{n+1}}\sigma(n)}{n<\omega}\)
is a compatible entourage sequence; therefore by the previous lemma,
\(X\) is psuedometrizable.
\end{proof}

\section{A Selectively Proximal Game and a Dual Proximal Game}

Because the choices of \(P2\) do not depend solely on the choices of
\(P1\) each round, the proximal game is not a selection game.
However, it can be somewhat emulated as a selection game as follows.

\begin{definition}
Let \(\mc{A}_X\) be the collection of basic neighborhood assignments
\(N:X\to\mc T_X\) such that \(x\in N(x)\) (equivalently,
\(x\in U\) for each \(\tuple{x,U}\in N\)), and let \(\mc {PR}_X\) 
be the collection of countable sets of tuples \(\setB{\tuple{x_n,U_n}}{n<\omega}\)
satisfying all of the following:
\begin{itemize}
\item For each \(n<\omega\), \(x_n\in U_m\) for cofinitely-many \(m<\omega\)
\item \(\setB{x_n}{n<\omega}\) fails to converge to any point of \(X\)
\item \(\bigcap\setB{U_n}{n<\omega}\not=\emptyset\)
\end{itemize}
Then we call \(\ssg{\mc A_X}{\mc{PR}_X}\) the \term{selectively proximal game}.
\end{definition}

\begin{theorem}
The selectively proximal game is equivalent to the proximal game when
considering only paracompact spaces.
\end{theorem}
\begin{proof}
TODO

Let \(\sigma\) be a winning predetermined strategy for \(P1\) in the proximal game
such that \(\sigma(n)\subseteq\sigma(m)\) for all \(n\geq m\),
and define the neighborhood assignment
\(\tau(n)\) for \(P1\) in the selectively proximal game 
by \(\tau(n)(x)=\sigma(n)[x]\). Then consider when \(P2\) responds to \(\tau\) 
by \(\tuple{x_n,\sigma(n)[x_n]}\) during round \(n<\omega\), and
suppose that for each \(n<\omega\), \(x_n\in \sigma(m)[x_m]\) for cofinitely-many \(m<\omega\).
Pick some \(S(n)>n\) such that \(x_n\in U_{S(n)}\), and let 
\(\tuple{y_n,V_n}=\tuple{x_{S^n(0)},U_{S^n(0)}}\).
\end{proof}

\begin{definition}
Let \(\mc{N}_X=\setB{N_x}{x\in X}\) where \(N_x=\setB{\tuple{x,U}}{U\in\mc T_{X,x}}\).
Then we call \(\ssg{\mc N_X}{\neg\mc{PR}_X}\) the \term{dual selectively proximal game}.
\end{definition}

We should defend this nominclature.

\begin{proposition}
\(\ssg{\mc A_X}{\mc{PR}_X}\) is dual to 
\(\ssg{\mc N_X}{\neg\mc{PR}_X}\).
\end{proposition}
\begin{proof}
We proceed by showing that \(\mc N_X\) is a reflection of \(\mc A_X\); that is,
\[\mc A_X'=\setB{\ran{f}}{f\in\mathbf C(\mc N_X)}\]
is a selection basis for \(\mc A_X\). To see this, 
first observe that for each
\(f\in\mathbf C(\mc N_X)\) and \(x\in X\), \(f(N_x)\in N_x\), so
\(f(N_x)=\tuple{x,U}\) for some open neighborhood \(U\) of \(x\). It follows
that \(\ran{f}\in\mc A_X\) and \(\mc A_X'\subseteq\mc A_X\). Furthermore 
for each neighborhood assignment 
\(N\in\mc A_X\), we may define \(f_N\in\mathbf C(\mc N_X)\) by 
\(f_N(N_x)=\tuple{x,N(x)}\). It follows that
\(\ran{f_N}\subseteq N\) yielding our result, but in fact
we have shown that \(N=\ran{f_N}\in\mc A_X'\)
and thus \(\mc A_X'=\mc A_X\).
\end{proof}

\bibliographystyle{plain}
\bibliography{../../bibliography}

\end{document}
