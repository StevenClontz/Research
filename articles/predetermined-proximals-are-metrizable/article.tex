\documentclass{amsart}
\usepackage{amsmath}
\usepackage{amsthm}
\usepackage{amssymb}

\usepackage{../../clontzDefinitions}

\newtheorem{theorem}{Theorem}[section]
\newtheorem{proposition}[theorem]{Proposition}
\newtheorem{lemma}[theorem]{Lemma}
\newtheorem{corollary}[theorem]{Corollary}


\theoremstyle{definition}
\newtheorem{definition}[theorem]{Definition}
\newtheorem{game}[theorem]{Game}
\newtheorem{example}[theorem]{Example}
\newtheorem{question}[theorem]{Question}

\parskip=0.5em



\begin{document}

% \title{Proximal compact spaces are Corson compact\tnoteref{t1}}
% \tnotetext[t1]{2010 Mathematics Subject Classification. 54E15, 54D30, 54A20.}
\title{Predetermined proximal spaces are metrizable}

% \author[aub]{S.~Clontz\fnref{fn1}}
% \ead{steven.clontz@auburn.edu}
% \author[aub]{G.~Gruenhage\fnref{fn2}}
% \ead{gruengf@auburn.edu}

% \address[aub]{Department of Mathematics, Auburn University,
%  Auburn, AL 36830}


\author{Steven Clontz}
\address{Department of Mathematics and Statistics, University of South Alabama,
Mobile, AL 36688}
\email{sclontz@southalabama.edu}
\urladdr{clontz.org}

\keywords{Proximal; predetermined proximal; topological game; limited information strategies}
% \begin{keyword}
%   Proximal\sep absolutely proximal\sep Corson compact\sep uniform space\sep $W$-space
% \end{keyword}

\subjclass[2010]{54E15, 54D30, 54A20}


\begin{abstract}
TODO
%  The author and G. Gruenhage previously showed that J. Bell's proximal game
%  may be used to characterize Corson compactness in compact Hausdorff spaces.
%  Using tactical strategies considering only the most recent
%  move of the opponent, the proximal game may also be
%  used to characterize the strong Eberlein compactness property.
%  In doing so, a purely topological characterization of the proximal game
%  is introduced, and several existing results on
%  the proximal game are given
%  analogues considering limited information strategies.
\end{abstract}


\maketitle

We take the following from Willard's text.

\begin{definition}
A \term{normal covering sequence} for a space \(X\) is a sequence
\(\setB{\mc U_n}{n<\omega}\) of open covers such that
\(\mc U_{n+1}\) star-refines \(\mc U_n\). Such a sequence is
\term{compatible} with \(X\) if \(\setB{St(x,\mc U_n)}{n<\omega}\)
is a local base at each \(x\in X\).
\end{definition}
\begin{theorem}
A space \(X\) is psuedometrizable if and only if it has a compatible
normal covering sequence.
\end{theorem}

For convenience, we will recast these results in terms of entourages.
TODO: ``normal'' can be dropped

\begin{definition}
A \term{normal entourage sequence} for a space \(X\) is a sequence
\(\setB{D_n}{n<\omega}\) of entourages such that
\(2D_{n+1}\subseteq D_n\). Such a sequence is
\term{compatible} with \(X\) if \(\setB{D_n[x]}{n<\omega}\)
is a local base at each \(x\in X\).
\end{definition}
\begin{theorem}
A space \(X\) is psuedometrizable if and only if it has a compatible
normal entourage sequence.
\end{theorem}
\begin{proof}
Let \(d\) be a psuedometric generating \(X\); then \(\setB{D_n}{n<\omega}\)
given by \(D_n=\setB{\tuple{x,y}}{d(x,y)<2^{-n}}\) is a normal entourage sequence,
and is compatible since \(D_n[x]=B_{2^{-n}}(x)\). 

On the other hand, given a normal entourage sequence \(\setB{D_n}{n<\omega}\),
let \(\mc U_n=\setB{\frac{1}{2}D_{n+1}[x]}{x\in X}\).  It follows that
\(St(x,\mc U_n)=\bigcup\setB{\frac{1}{2}D_{n+1}[y]}{x\in\frac{1}{2}D_{n+1}[y]}\).
Furthermore \(z\in St(x,\mc U_n)\Rightarrow z\in\frac{1}{2}D_{n+1}[y]\) for
some \(y\); therefore \(\tuple{x,y},\tuple{y,z}\in \frac{1}{2}D_{n+1}\) shows that 
\(\tuple{x,z}\in D_{n+1}\) and \(z\in D_{n+1}[x]\). 
Thus \(St(x,\mc U_n)\subseteq D_{n+1}[x]\).

We now may observe that \(\mc U_{n+1}\) star-refines \(\mc U_n\), since
\(St(x,\mc U_{n+1})\subseteq D_{n+1}[x]\subseteq D_n[x]\in\mc U_n\)
witnesses that \(\setB{St(x,\mc U_n)}{n<\omega}\) is compatible with \(X\),
guaranteeing pseudometrizability.
\end{proof}

\begin{theorem}
A space \(X\) is psuedometrizable
if and only if 
\(\plI\prewin\bellConGame{X}\). 
\end{theorem}
\begin{proof}
Suppose \(X\) is psuedometrizable by \(d\);
then let \(\sigma\) be the predetermined strategy for \(\bellConGame{X}\) defined by
\(\sigma(n)=\setB{\tuple{x,y}}{d(x,y)<2^{-n}}\). 
For any legal attack \(\alpha\) against \(\sigma\), \(\alpha(n+1)\in\sigma(n)[\alpha(n)]\).
It follows that if \(x\in\bigcap_{n<\omega}\sigma(n)[\alpha(n)]\) and \(\epsilon>0\),
we may choose \(N<\omega\) such that \(2^{-N}<\epsilon\). Therefore 
\(d(x,\alpha(n))<2^{-n}\leq 2^{-N}<\epsilon\) for all \(n\geq N\), showing \(\alpha\) converges
to \(x\). Thus \(\sigma\) is a winning strategy.

Now let \(\sigma\) be any predetermined winning strategy satisfying \(\sigma(n)\subseteq\sigma(m)\)
for all \(n\geq m\), and suppose \(\setB{\frac{1}{2^{n+1}}\sigma(n)[x]}{n<\omega}\) is not a
local base at some \(x\in X\). Then we may pick an entourage \(D\) such that
\(\frac{1}{2^{n+1}}\sigma(n)[x]\not\subseteq D[x]\) for all \(n<\omega\). So choose
\(\alpha(n)\in\frac{1}{2^{n+1}}\sigma(n)[x]\setminus D[x]\).

Observe that \(\tuple{\alpha(n),x}\in\frac{1}{2^{n+1}}\sigma(n)\) and
\(\tuple{\alpha(n+1),x}\in\frac{1}{2^{n+2}}\sigma(n+1)\subseteq\frac{1}{2^{n+1}}\sigma(n)\).
It follows that \(\tuple{\alpha(n),\alpha(n+1)}\in\frac{1}{2^n}\sigma(n)\subseteq\sigma(n)\),
witnessing that \(\alpha(n+1)\in\sigma(n)[\alpha(n)]\), that is, \(\alpha\) is a legal counterattack
to \(\sigma\). Since \(x\in\frac{1}{2^{n+1}}\sigma(n)[\alpha(n)]\subseteq\sigma(n)[\alpha(n)]\)
for all \(n<\omega\), \(\sigma\) can only win for \(\plI\) if \(\alpha\) converges.
But \(\alpha(n)\not\in D[x]\) for all \(n<\omega\), so \(\alpha\) fails to converge as well.
Thus \(\sigma\) is not a winning strategy.

As a result, if \(\sigma\) is a winning predetermined strategy,
we have that \(\setB{\frac{1}{2^{n+1}}\sigma(n)[x]}{n<\omega}\) is a
local base at each \(x\in X\). Therefore by the previous lemma,
\(X\) is psuedometrizable.
\end{proof}

\bibliographystyle{plain}
\bibliography{../../bibliography}

\end{document}
