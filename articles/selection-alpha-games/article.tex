\documentclass{amsart}
\usepackage{amsmath}
\usepackage{amsthm}
\usepackage{amssymb}

\usepackage{tikz}
\usetikzlibrary{arrows}

\usepackage{../../clontzDefinitions}

\renewcommand{\vec}{\mathbf}

      \theoremstyle{plain}
      \newtheorem{theorem}{Theorem}
      \newtheorem{lemma}[theorem]{Lemma}
      \newtheorem{corollary}[theorem]{Corollary}
      \newtheorem{proposition}[theorem]{Proposition}
      \newtheorem{conjecture}[theorem]{Conjecture}
      \newtheorem{question}[theorem]{Question}
      \newtheorem{claim}[theorem]{Claim}

      \theoremstyle{definition}
      \newtheorem{definition}[theorem]{Definition}
      \newtheorem{notation}[theorem]{Notation}
      \newtheorem{example}[theorem]{Example}
      \newtheorem{game}[theorem]{Game}

      \theoremstyle{remark}
      \newtheorem{remark}[theorem]{Remark}

      \theoremstyle{plain}
      \newtheorem*{theorem*}{Theorem}
      \newtheorem*{lemma*}{Lemma}
      \newtheorem*{corollary*}{Corollary}
      \newtheorem*{proposition*}{Proposition}
      \newtheorem*{conjecture*}{Conjecture}
      \newtheorem*{question*}{Question}
      \newtheorem*{claim*}{Claim}

      \theoremstyle{definition}
      \newtheorem*{definition*}{Definition}
      \newtheorem*{example*}{Example}
      \newtheorem*{game*}{Game}

      \theoremstyle{remark}
      \newtheorem*{remark*}{Remark}


\usepackage{lineno}
\linenumbers



\begin{document}

\title{Selection games and Arhangelskii's convergence principles}



\author{Steven Clontz}
\address{Department of Mathematics and Statistics,
The University of South Alabama,
Mobile, AL 36688}
\email{sclontz@southalabama.edu}



\keywords{Selection principle, selection game,
\(\alpha_i\) property, convergence}

%\subjclass[2010]{54D20, 54D45, 91A44}



\begin{abstract}
We prove the things.
\end{abstract}


\maketitle






%
%\section{Introduction}
%
%\begin{definition}
%  The \term{selection game} \(\schStrongSelGame{\mc A}{\mc B}\) 
%  is an \(\omega\)-length game involving Players \(\plI\) and \(\plII\). 
%  During round \(n\), \(\plI\) chooses
%  \(A_n\in\mc A\), followed by \(\plII\) choosing \(B_n\in A_n\).
%  Player \(\plII\) wins in the case that \(\{B_n:n<\omega\}\in\mc B\),
%  and Player \(\plI\) wins otherwise.
%\end{definition}
%
%\begin{definition}
%  The \term{selection game} \(\schSelGame{\mc A}{\mc B}\) 
%  is an \(\omega\)-length game involving Players \(\plI\) and \(\plII\). 
%  During round \(n\), \(\plI\) chooses
%  \(A_n\in\mc A\), followed by \(\plII\) choosing \(B_n\in [A_n]^{<\aleph_0}\).
%  Player \(\plII\) wins in the case that \(\bigcup\{B_n:n<\omega\}\in\mc B\),
%  and Player \(\plI\) wins otherwise.
%\end{definition}
%
\section{Clontz results}

\begin{definition}
Say a collection \(\mc A\) is \term{\(\Gamma\)-like} if it satisfies the following
for each \(A\in\mc A\).
\begin{itemize}
\item \(|A|\geq\aleph_0\).
\item If \(A'\subseteq A\) and \(|A'|\geq\aleph_0\), then \(A'\in\mc A\).
\end{itemize}
\end{definition}

\begin{definition}
Let \(\Gamma_X\) be the collection of open \term{\(\gamma\)-covers} \(\mc U\) of \(X\),
that is, infinite open covers of \(X\) such that for each \(x\in X\),
\(\{U\in\mc U:x\in U\}\) is cofinite in \(\mc U\).
\end{definition}

\begin{definition}
Let \(\Gamma_{X,x}\) be the collection of non-trivial sequences \(S\subseteq X\) converging to \(x\),
that is, infinite subsets of \(X\) such that for each neighborhood \(U\) of \(x\),
\(S\cap U\) is cofinite in \(S\).
\end{definition}

It follows that \(\Gamma_X,\Gamma_{X,x}\) are both \(\Gamma\)-like. We also require
the following.

\begin{definition}
Say a collection \(\mc A\) is \term{almost-\(\Gamma\)-like} if
for each \(A\in\mc A\), there is \(A'\subseteq A\) such that:
\begin{itemize}
\item \(|A'|=\aleph_0\).
\item If \(A''\) is a cofinite subset of \(A'\), then \(A''\in\mc A\).
\end{itemize}
\end{definition}

So all \(\Gamma\)-like sets are almost-\(\Gamma\)-like.

\begin{theorem}
Let \(\mc B\) be \(\Gamma\)-like. Then \(\alpha_1(\mc A,\mc B)\) holds if and only
if \(\plI\notprewin G_{cf}(\mc A,\mc B)\).
\end{theorem}

\begin{proof}
We first assume \(\alpha_1(\mc A,\mc B)\) and let \(A_n\in\mc A\) for \(n<\omega\)
define a predetermined strategy for \(\plI\). By \(\alpha_1(\mc A,\mc B)\), we
immediately obtain \(B\in\mc B\) such that \(|A_n\setminus B|<\aleph_0\). Thus
\(B_n=A_n\cap B\) is a cofinite choice from \(A_n\), and 
\(B'=\bigcup\{B_n:n<\omega\}\) is an infinite subset of \(B\),
so \(B'\in\mc B\). Thus \(\plII\) may defeat \(\plI\) by choosing
\(B_n\subseteq A_n\) each round, witnessing \(\plI\notprewin G_{cf}(\mc A,\mc B)\).

On the other hand, let \(\plI\notprewin G_{cf}(\mc A,\mc B)\). Given \(A_n\in\mc A\)
for \(n<\omega\), we note that \(\plII\) may choose a cofinite subset \(B_n\subseteq A_n\)
such that \(B=\bigcup\{B_n:n<\omega\}\in\mc B\). Then \(B\) witnesses \(\alpha_1(\mc A,\mc B)\)
since \(|A_n\setminus B|\leq|A_n\setminus B_n|\leq\aleph_0\).
\end{proof}

\begin{theorem}
Let \(\mc A\) be almost-\(\Gamma\)-like and \(\mc B\) be \(\Gamma\)-like. 
Then \(\alpha_2(\mc A,\mc B)\) holds if and only
if \(\plI\notprewin G_1(\mc A,\mc B)\).
\end{theorem}

\begin{proof}
We first assume \(\alpha_2(\mc A,\mc B)\) and let \(A_n\in\mc A\) for \(n<\omega\)
define a predetermined strategy for \(\pl I\).
We may apply \(\alpha_2(\mc A,\mc B)\) to choose \(B\in\mc B\) such that
\(|A_n\cap B|\geq\aleph_0\). We may then choose \(a_n\in(A_n\cap B)\setminus\{a_i:i<n\}\)
for each \(n<\omega\). It follows that \(B'=\{a_n:n<\omega\}\in\mc B\) since
\(B'\) is an infinite subset of \(B\in\mc B\); therefore \(A_n\) does not define
a winning predetermined strategy for \(\plI\).

Now suppose \(\plI\notprewin G_1(\mc A,\mc B)\). Given \(A_n\in\mc A\) for \(n<\omega\),
first choose \(A_n'\in\mc A\) such that \(A_n'=\{a_{n,j}:j<\omega\}\subseteq A_n\),
\(j<k\) implies \(a_{n,j}\not=a_{n,k}\),
and \(A_{n,m}=\{a_{n,j}:m\leq j<\omega\}\in\mc A\).
Finally choose
some \(\theta:\omega\to\omega\) such that \(|\theta^{\leftarrow}(n)|=\aleph_0\) for
each \(n<\omega\).

Since playing \(A_{\theta(m),m}\) during round \(m\)
does not define a winning strategy for \(\plI\) in
\(G_1(\mc A,\mc B)\), \(\plII\) may choose \(x_m\in A_{\theta(m),m}\)
such that \(B=\{x_m:m<\omega\}\in\mc B\).
Choose \(i_m<\omega\) for each \(m<\omega\) such that
\(x_m=a_{\theta(m),i_m}\), noting \(i_m\geq m\).
It follows that 
\(A_n\cap B\supseteq\{a_{\theta(m),i_m}:m\in\theta^{\leftarrow}(n)\}\).
Since for each \(m\in\theta^{\leftarrow}(n)\) there exists
\(M\in\theta^{\leftarrow}(n)\) such that \(m\leq i_m<M\leq i_{M}\),
and therefore \(a_{\theta(m),i_m}\not=a_{\theta(m),i_{M}}=a_{\theta(M),i_{M}}\),
we have shown that \(A_n\cap B\) is infinite. Thus \(B\) witnesses
\(\alpha_2(\mc A,\mc B)\).
\end{proof}

\begin{theorem}
Let \(\mc A\) be almost-\(\Gamma\)-like and \(\mc B\) be \(\Gamma\)-like. 
Then \(\alpha_4(\mc A,\mc B)\) holds if and only
if \(\plI\notprewin G_{<2}(\mc A,\mc B)\) if and only if
\(\plI\notprewin G_{fin}(\mc A,\mc B)\).
\end{theorem}

\begin{proof}
We first assume \(\alpha_4(\mc A,\mc B)\) and let \(A_n\in\mc A\) for \(n<\omega\)
define a predetermined strategy for \(\plI\) in 
\(G_{<2}(\mc A,\mc B)\). We then may choose \(A_n'\in\mc A\) where
\(A_n'=\{a_{n,j}:j<\omega\}\subseteq A_n\), \(j<k\) implies
\(a_{n,j}\not=a_{n,k}\), and \(A_n''=A_n'\setminus\{a_{i,j}:i,j<n\}\in\mc A\).

By applying \(\alpha_4(\mc A,\mc B)\) to \(A_n''\), we obtain \(B\in\mc B\)
such that \(A_n''\cap B\not=\emptyset\) for infintely-many \(n<\omega\).
We then let \(F_n=\emptyset\) when \(A_n''\cap B=\emptyset\), and
\(F_n=\{x_n\}\) for some \(x_n\in A_n''\cap B\) otherwise. Then we will have that
\(B'=\bigcup\{F_n:n<\omega\}\subseteq B\) belongs to \(\mc B\) once we show that
\(B'\) is infinite. To see this, for \(m\leq n<\omega\) note that either \(F_m\) is
empty (and we let \(j_m=0\)) or \(F_m=\{a_{m,j_m}\}\)
for some \(j_m\geq m\); choose \(N<\omega\) such that \(j_m<N\) for all
\(m\leq n\) and \(F_N=\{x_N\}\). Thus \(F_m\not=F_N\) for all \(m\leq n\) since
\(x_{N}\not\in\{a_{i,j}:i,j< N\}\). Thus \(\plII\) may defeat the predetermined
strategy \(A_n\) by playing \(F_n\) each round.

Since \(\plI\notprewin G_{<2}(\mc A,\mc B)\) immediately implies
\(\plI\notprewin G_{fin}(\mc A,\mc B)\), we assume the latter. Given \(A_n\in\mc A\)
for \(n<\omega\), we note this defines a (non-winning) predetermined 
strategy for \(\plI\), so \(\plII\) may choose \(F_n\in[A_n]^{<\aleph_0}\) such that
\(B=\bigcup\{F_n:n<\omega\}\in\mc B\). Since \(B\) is infinite, we note
\(F_n\not=\emptyset\) for infinitely-many \(n<\omega\). Thus \(B\) witnesses
\(\alpha_4(\mc A,\mc B)\) since \(A_n\cap B\supseteq F_n\not=\emptyset\) for
infinitely-many \(n<\omega\).
\end{proof}

\begin{theorem}
Let \(\mc B\) be \(\Gamma\)-like. Then \(\plI\prewin G_{<2}(\mc A,\mc B)\)
if and only if \(\plI\prewin G_{fin}(\mc A,\mc B)\).
\end{theorem}

\begin{proof}
Assume \(\bigcup\mc A\) is well-ordered.
Given a winning predetermined strategy \(A_n\) for \(\plI\) in
\(G_{<2}(\mc A,\mc B)\), consider \(F_n\in[A_n]^{<\aleph_0}\). We set
\[
  F_n^*
=
  \begin{cases}
    \emptyset
      & \text{ if }
    F_n\setminus\bigcup\{F_m:m<n\}=\emptyset
      \\
    \{\min(
      F_n\setminus\bigcup\{F_m:m<n\}
    )\}
      & \text{ otherwise}
  \end{cases}
\]
Since \(|F_n^*|<2\), we have that \(\bigcup\{F_n^*:n<\omega\}\not\in\mc B\).
In the case that \(\bigcup\{F_n^*:n<\omega\}\) is finite, we immediately
see that \(\bigcup\{F_n:n<\omega\}\) is also finite and therefore not
in \(\mc B\). Otherwise \(\bigcup\{F_n^*:n<\omega\}\not\in\mc B\)
is an infinite subset of \(\bigcup\{F_n:n<\omega\}\), and thus
\(\bigcup\{F_n:n<\omega\}\not\in\mc B\) too. Therefore
\(A_n\) is a winning predetermined strategy for \(\plI\) in
\(G_{fin}(\mc A,\mc B)\) as well.
\end{proof}

\begin{theorem}
Let \(\mc B\) be \(\Gamma\)-like. Then \(\plI\win G_{<2}(\mc A,\mc B)\)
if and only if \(\plI\win G_{fin}(\mc A,\mc B)\).
\end{theorem}

\begin{proof}
Assume \(\bigcup\mc A\) is well-ordered.
Suppose \(\plI\win G_{<2}(\mc A,\mc B)\) is witnessed by the strategy
\(\sigma\). Let \(\tuple{}^\star=\tuple{}\), and for 
\(s\concat\tuple{F}\in([\bigcup\mc A]^{<\aleph_0})^{<\omega}
\setminus\{\tuple{}\}\) let 
\[
  (s\concat\tuple{F})^\star
=
  \begin{cases}
    s^\star\concat\tuple{\emptyset}
      & \text{ if }
    F\setminus\bigcup\ran{s}=\emptyset
      \\
    s^\star\concat\tuple{
      \{\min(F\setminus\bigcup\ran{s})\}
    }
      & \text{ otherwise}
  \end{cases}
\]

We then define the strategy \(\tau\) for \(\plI\) in \(G_{fin}(\mc A,\mc B)\)
by \(\tau(s)=\sigma(s^\star)\). Then given any counterattack
\(\alpha\in([\bigcup\mc A]^{<\aleph_0})^\omega\) by \(\plII\) played against 
\(\tau\), we note that \(\alpha^*=\bigcup\{(\alpha\rest n)^*:n<\omega\}\)
is a counterattack to \(\sigma\), and thus loses.
This means \(B=\bigcup\ran{\alpha^*}\not\in\mc B\).

We consider two cases. The first is the case that \(\bigcup\ran{\alpha^*}\)
is finite. Noting that \(\alpha^*(m)\cap\alpha^*(n)=\emptyset\) whenever
\(m\not=n\), there exists \(N<\omega\) such that 
\(\alpha^*(n)=\emptyset\) for all \(n>N\). As a result,
\(\bigcup\ran{\alpha}=\bigcup\ran{\alpha\rest n}\), and thus
\(\bigcup\ran{\alpha}\) is finite, and therefore not in \(\mc B\).

In the other case, \(\bigcup\ran{\alpha^*}\not\in\mc B\) is an 
infinite subset of \(\bigcup\ran{\alpha}\), and therefore 
\(\bigcup\ran{\alpha}\not\in\mc B\) as well. Thus we have shown
that \(\tau\) is a winning strategy for \(\plI\) in
\(G_{fin}(\mc A,\mc B)\).
\end{proof}

We further note that the above proof technique could be
used to establish that perfect-information and
Markov winning strategies for \(\plII\) in
\(G_{fin}(\mc A,\mc B)\) may be improved to
be valid in \(G_{<2}(\mc A,\mc B)\), provided
\(\mc B\) is \(\Gamma\)-like. As such,
\(G_{<2}(\mc A,\mc B)\) and \(G_{fin}(\mc A,\mc B)\)
are effectively equivalent games under this hypothesis.
%\begin{theorem}
%Let \(\mc B\) be \(\Gamma\)-like. Then \(\plII\win G_{<2}(\mc A,\mc B)\)
%if and only if \(\plII\win G_{fin}(\mc A,\mc B)\).
%\end{theorem}

%\begin{theorem}
%Let \(\mc B\) be \(\Gamma\)-like. Then \(\plII\markwin G_{<2}(\mc A,\mc B)\)
%if and only if \(\plII\markwin G_{fin}(\mc A,\mc B)\).
%\end{theorem}

\begin{theorem}\label{pawlikowskii}
Let \(\mc A\) be almost-\(\Gamma\)-like and \(\mc B\) be \(\Gamma\)-like.
Then \(\plI\win G_{fin}(\mc A,\mc B)\) if and only if
\(\plI\prewin G_{fin}(\mc A,\mc B)\), and
\(\plI\win G_1(\mc A,\mc B)\) if and only if
\(\plI\prewin G_1(\mc A,\mc B)\).
\end{theorem}

\begin{proof}
We assume \(\plI\win G_{fin}(\mc A,\mc B)\)
and let the symbol \(\dagger\) mean \(<\aleph_0\)
(respectively, \(\plI\win G_1(\mc A,\mc B)\)
and \(\dagger=1\),
and for convenience we assume \(\plII\) plays
singleton subsets of \(\mc A\) rather than elements).
As \(\mc A\) is almost-\(\Gamma\)-like, there is a 
winning strategy \(\sigma\) where
\(|\sigma(s)|=\aleph_0\) and \(\sigma(s)\cap\bigcup\ran{s}=\emptyset\)
(that is, \(\sigma\) never replays the choices of \(\plII\))
for all partial plays \(s\) by \(\plII\).

For each \(s\in\omega^{<\omega}\), suppose 
\(F_{s\rest m}\in[\bigcup A]^{\dagger}\) 
is defined for each \(0<m\leq|s|\).
Then let \(s^\star:|s|\to[\bigcup\mc A]^{\dagger}\)
be defined by
\(s^\star(m)=F_{s\rest m+1}\), and define \(\tau':\omega^{<\omega}\to\mc A\)
by \(\tau'(s)=\sigma(s^\star)\). Finally, set 
\([\sigma(s^\star)]^{\dagger}=\{F_{s\concat\tuple{n}}:n<\omega\}\), and
for some bijection \(b:\omega^{<\omega}\to\omega\) let \(\tau(n)=\tau'(b(n))\)
be a predetermined strategy for \(\plI\) in \(G_{fin}(\mc A,\mc B)\)
(resp. \(G_1(\mc A,\mc B)\)).

Suppose \(\alpha\) is a counterattack by \(\plII\) against \(\tau\), so 
\[
  \alpha(n)
    \in
  [\tau(n)]^{\dagger}
    =
  [\tau'(b(n))]^{\dagger}
    =
  [\sigma(b(n)^\star)]^{\dagger}
\]
It follows that \(\alpha(n)=F_{b(n)\concat\tuple{m}}\) for some \(m<\omega\).
In particular, there is some infinite subset \(W\subseteq\omega\) and \(f\in\omega^\omega\)
such that \(\{\alpha(n):n\in W\}=\{F_{f\rest n+1}:n<\omega\}\).
Note here that \((f\rest n+1)^\star=(f\rest n)^\star\concat\tuple{F_{f\rest n+1}}\).
This shows that \(F_{f\rest n+1}\in[\sigma((f\rest n)^\star)]^{\dagger}\) 
is an attempt by \(\plII\) to defeat \(\sigma\), which fails. Thus 
\(\bigcup\{F_{f\rest n+1}:n<\omega\}=\bigcup\{\alpha(n):n\in W\}\not\in\mc B\),
and since this set is infinite (as \(\sigma\) prevents \(\plII\)
from repeating choices) we have \(\bigcup\{\alpha(n):n<\omega\}\not\in\mc B\) too.
Therefore \(\tau\) is winning.
\end{proof}

Note that the assumption in Theorem \ref{pawlikowskii} that \(\mc A\)
be almost-\(\Gamma\)-like cannot be omitted. In
[todo cite Clontz k-tactics in Gruenhage game]
an example of a space and point where \(\plI\win G_1(\mc A,\mc B)\) but
\(\plI\notprewin G_1(\mc A,\mc B)\) is given, where \(\mc A\) is
the set of open neighborhoods of the given point 
(which are all uncountable), 
and \(\mc B\) is the set of converging sequences to that point.
(Note that \(G_1(\mc A,\mc B)\) is called \(Gru_{O,P}(X,x)\) in that
paper. In fact, more is shown: \(\plI\) has a winning perfect-information
strategy, but any strategy that only uses 
the most recent \(k\) moves of \(\plII\) and the round number
can be defeated, where \(k\) is any natural number.)


\begin{proposition}\label{auto-asl}
Let \(\mc B\) be \(\Gamma\)-like, \(\mc B\subseteq\mc A\),
and \(\plI\notprewin G_{fin}(\mc A,\mc B)\). Then
\(\mc A\) is almost-\(\Gamma\)-like.
\end{proposition}
\begin{proof}
Let \(A\in\mc A\), and for all \(n<\omega\) let \(A_n=A\).
Then \(A_n\) is not a winning predetermined strategy for
\(\plI\), so \(\plII\) may choose finite sets
\(B_n\subseteq A_n=A\) such that 
\(A'=\bigcup\{B_n:n<\omega\}\in\mc B\subseteq\mc A\).

It follows that \(A'\subseteq A\) and \(|A'|=\aleph_0\), 
and for any infinite subset
\(A''\subseteq A'\) (in particular, any cofinite subset),
\(A''\in\mc B\subseteq\mc A\). Thus \(\mc A\) is almost-\(\Gamma\)-like.
\end{proof}


\begin{corollary}
Let \(\mc B\) be \(\Gamma\)-like and \(\mc B\subseteq\mc A\).
Then \(\plI\win G_{fin}(\mc A,\mc B)\) if and only if
\(\plI\prewin G_{fin}(\mc A,\mc B)\), and
\(\plI\win G_{1}(\mc A,\mc B)\) if and only if
\(\plI\prewin G_{1}(\mc A,\mc B)\).
\end{corollary}
\begin{proof}
Assuming \(\plI\notprewin G_{fin}(\mc A,\mc B)\), we have
\(\plI\notwin G_{fin}(\mc A,\mc B)\) by Proposition \ref{auto-asl}
and Theorem \ref{pawlikowskii}.

Similarly, assuming \(\plI\notprewin G_{1}(\mc A,\mc B)
\Rightarrow\plI\notprewin G_{fin}(\mc A,\mc B)\), we have
\(\plI\notwin G_{1}(\mc A,\mc B)\) by Proposition \ref{auto-asl}
and Theorem \ref{pawlikowskii}.
\end{proof}

This corollary generalizes e.g. Theorems 26 and 30 of [cite Scheepers 1996 Ramsey],
Theorem 5 of [cite MR2119791], and Corollary 36 of [cite Clontz dual games].

\bibliographystyle{plain}
\bibliography{../../bibliography}

\end{document}
