\documentclass{amsart}
\usepackage{amsmath}
\usepackage{amsthm}
\usepackage{amssymb}

\usepackage{tikz}
% \usetikzlibrary{matrix}
\usetikzlibrary{arrows}

\usepackage{../../clontzDefinitions}


      \theoremstyle{plain}
      \newtheorem{theorem}{Theorem}
      \newtheorem{lemma}[theorem]{Lemma}
      \newtheorem{corollary}[theorem]{Corollary}
      \newtheorem{proposition}[theorem]{Proposition}
      \newtheorem{conjecture}[theorem]{Conjecture}
      \newtheorem{question}[theorem]{Question}
      \newtheorem{claim}[theorem]{Claim}

      \theoremstyle{definition}
      \newtheorem{definition}[theorem]{Definition}
      \newtheorem{notation}[theorem]{Notation}
      \newtheorem{example}[theorem]{Example}
      \newtheorem{game}[theorem]{Game}

      \theoremstyle{remark}
      \newtheorem{remark}[theorem]{Remark}

      \theoremstyle{plain}
      \newtheorem*{theorem*}{Theorem}
      \newtheorem*{lemma*}{Lemma}
      \newtheorem*{corollary*}{Corollary}
      \newtheorem*{proposition*}{Proposition}
      \newtheorem*{conjecture*}{Conjecture}
      \newtheorem*{question*}{Question}
      \newtheorem*{claim*}{Claim}

      \theoremstyle{definition}
      \newtheorem*{definition*}{Definition}
      \newtheorem*{example*}{Example}
      \newtheorem*{game*}{Game}

      \theoremstyle{remark}
      \newtheorem*{remark*}{Remark}




\begin{document}

% \title{Proximal compact spaces are Corson compact\tnoteref{t1}}
% \tnotetext[t1]{2010 Mathematics Subject Classification. 54E15, 54D30, 54A20.}
\title{Relating games of Menger, countable fan tightness, and selective separability}

% \author[aub]{S.~Clontz\fnref{fn1}}
% \ead{steven.clontz@auburn.edu}
% \author[aub]{G.~Gruenhage\fnref{fn2}}
% \ead{gruengf@auburn.edu}

% \address[aub]{Department of Mathematics, Auburn University,
%  Auburn, AL 36830}


\author{Steven Clontz}
\address{Department of Mathematics and Statistics,
The University of South Alabama,
Mobile, AL 36606}
\email{steven.clontz@gmail.com}

% \keywords{TODO}
% \subjclass[2010]{}






\begin{abstract}
  By adapting techniques of Arhangel'skii, Barman, and Dow, we may
  equate the existence of perfect-information, Markov, and tactical
  strategies between two interesting selection games.
  These results shed some light on Gruenhage's question asking whether all
  strategically selectively separable spaces are Markov selectively
  separable.
\end{abstract}


\maketitle







\section{Introduction}

\begin{definition}
  The \term{selection principle} \(\schSelProp{\mc A}{\mc B}\) states that
  given \(A_n\in\mc A\) for \(n<\omega\), there exist \(B_n\in[A_n]^{<\omega}\)
  such that \(\bigcup_{n<\omega}B_n\in\mc B\).
\end{definition}

\begin{definition}
  The \term{selection game} \(\schSelGame{\mc A}{\mc B}\) is the
  analogous game to \(\schSelProp{\mc A}{\mc B}\), where during each
  round \(n<\omega\), Player \(\plI\) first
  chooses \(A_n\in\mc A\), and then Player \(\plII\) chooses
  \(B_n\in[A_n]^{<\omega}\).
  Player \(\plII\) wins in the case that \(\bigcup_{n<\omega}B_n\in\mc B\),
  and Player \(\plI\) wins otherwise.
\end{definition}

This game and property were first formally investigated
by Scheepers in ``Combinatorics of open covers'' \cite{MR1378387}, which
inspired a series of ten sequels with several co-authors. We may quickly
observe that if \(\plII\) has a winning strategy for the game
\(\schSelGame{\mc A}{\mc B}\), then \(\schSelProp{\mc A}{\mc B}\) will hold,
but the converse need not follow.

The power of this selection principle and game comes from their ability
to characterize several properties and games from the literature.
Of interest to us are the following.

\begin{definition}
  Let \(\mc O_X\) be the collection of open covers for a topological space
  \(X\). Then \(\schSelProp{\mc O_X}{\mc O_X}\) is the well-known
  \term{Menger property} for \(X\) (\(M\) for short), and
  \(\schSelGame{\mc O_X}{\mc O_X}\) is the
  well-known \term{Menger game}.
\end{definition}

\begin{definition}
  An \term{\(\omega\)-cover} \(\mc U\)
  for a topological space \(X\) is an open cover
  such that for every \(F\in[X]^{<\omega}\), there exists some \(U\in\mc U\)
  such that \(F\subseteq U\).
\end{definition}

\begin{definition}
  Let \(\Omega_X\) be the collection of \(\omega\)-covers for a topological
  space \(X\). Then \(\schSelProp{\Omega_X}{\Omega_X}\) is the
  \term{\(\Omega\)-Menger property} for \(X\) (\(\Omega M\) for short), and
  \(\schSelGame{\Omega_X}{\Omega_X}\) is the \term{\(\Omega\)-Menger game}.
\end{definition}

In \cite[Theorem 3.9]{MR1419798} it was shown that \(X\) is \(\Omega\)-Menger
if and only if \(X^n\) is Menger for all \(n<\omega\).

\begin{definition}
  Let \(\mc B_{X,x}\) be the collection of subsets \(A\subset X\) where
  \(x\in\closure{A}\). (Call \(A\) a \term{blade} of \(x\).)
  Then \(\schSelProp{\mc B_{X,x}}{\mc B_{X,x}}\) is the
  \term{countable fan tightness property} for \(X\) at \(x\)
  (\(CFT_x\) for short), and
  \(\schSelGame{\mc B_{X,x}}{\mc B_{X,x}}\) is the
  \term{countable fan tightness game} for \(X\) at \(x\).
\end{definition}

\begin{definition}
  A space \(X\) has \term{countable fan tightness} (\(CFT\) for short)
  if it has
  countable fan tightness at each point \(x\in X\).
\end{definition}

\begin{definition}
  Let \(\mc D_{X}\) be the collection of dense subsets of a topological
  space \(X\). (So, \(\mc D_X\subseteq \mc B_{X,x}\) for all \(x\in X\).)
  Then \(\schSelProp{\mc D_X}{\mc B_{X,x}}\) is the
  \term{countable dense fan tightness property} for \(X\) at \(x\)
  (\(CDFT_x\) for short), and
  \(\schSelGame{\mc D_X}{\mc B_{X,x}}\) is the
  \term{countable dense fan tightness game} for \(X\) at \(x\).
\end{definition}

\begin{definition}
  A space \(X\) has \term{countable dense fan tightness}
  (\(CDFT\) for short) if it has
  countable dense fan tightness at each point \(x\in X\).
\end{definition}

Note that \(CFT\Rightarrow CDFT\) for any space \(X\) as
\(\mc D_X\subseteq \mc B_{X,x}\).

The notion of countable fan tightness was first studied by
by Arhangel'skii in \cite{MR837289}. A result of that paper showed
that for \(T_{3\frac{1}{2}}\) spaces \(X\), the countable fan tightness
of the space of real-vaued continuous functions with
pointwise convergence \(C_p(X)\) is characterized by
the \(\Omega\)-Menger property of \(X\).


\begin{definition}
  \(\schSelProp{\mc D_X}{\mc D_X}\) is the
  \term{selective separability property} for \(X\)
  (\(SS\) for short), and
  \(\schSelGame{\mc D_X}{\mc D_X}\) is the
  \term{selective separability game} for \(X\).
\end{definition}

Of course, one may easily observe that a selective separable space is
separable. In \cite{MR2678950} Barman and Dow demonstrated that all
separable Frechet spaces are selectively separable. They were also able
to produce a space which is selectively separable, but does not allow
\(\plII\) a winning strategy in the selective separability game.

The object of this paper is to investigate the game-theoretic properties
characterized by the presence of winning \term{limited information}
strategies in these selection games.

\begin{definition}
  A \term{strategy} for \(\plII\) in the game \(\schSelGame{\mc A}{\mc B}\)
  is a function \(\sigma\) satisfying
  \(\sigma(\<A_0,\dots,A_n\>)\in[A_n]^{<\omega}\) for
  \(\<A_0\,\dots,A_n\>\in\mc A^{n+1}\). We say this strategy is
  \term{winning} if whenever \(\plI\) plays \(A_n\in\mc A\) during each
  round \(n<\omega\), \(\plII\) wins the game by playing
  \(\sigma(\<A_0,\dots,A_n\>)\) during each round \(n<\omega\).
  If a winning strategy exists, then we write
  \(\plII\win\schSelGame{\mc A}{\mc B}\).
\end{definition}

\begin{definition}
  A \term{Markov strategy} for \(\plII\) in the game
  \(\schSelGame{\mc A}{\mc B}\)
  is a function \(\sigma\) satisfying
  \(\sigma(A,n)\in[A_n]^{<\omega}\) for
  \(A\in\mc A\) and \(n<\omega\). We say this Markov strategy is
  \term{winning} if whenever \(\plI\) plays \(A_n\in\mc A\) during each
  round \(n<\omega\), \(\plII\) wins the game by playing
  \(\sigma(A_n,n)\) during each round \(n<\omega\).
  If a winning Markov strategy exists, then we write
  \(\plII\markwin\schSelGame{\mc A}{\mc B}\).
\end{definition}

\begin{notation}
  If \(\schSelProp{\mc A}{\mc B}\) characterizes the property \(P\),
  then we say \(\plII\win\schSelGame{\mc A}{\mc B}\) characterizes
  \(P^+\) (``strategically \(P\)''), and
  \(\plII\markwin\schSelGame{\mc A}{\mc B}\) characterizes
  \(P^{+mark}\) (``Markov \(P\)'').
  Of course, \(P^{+mark}\Rightarrow P^+ \Rightarrow P\).
\end{notation}

In this notation,
Barman and Dow showed that \(SS\) does not imply
\(SS^+\). We aim to make progress on the following question attributed to
Gary Gruenhage:

\begin{question}\label{mainQuestion}
  Does \(SS^+\) imply \(SS^{+mark}\)?
\end{question}

The solution is known to be ``yes'' in the context of countable spaces
\cite{MR2678950}. However in general, winning strategies in selection games
cannot be improved to be winning Markov strategies. In
\cite{clontzMengerGamePreprint} the author showed that while \(M^+\) implies
\(M^{+mark}\) for second-countable spaces, there exists a simple
example of a regular non-second-countable space which is
\(M^+\) but not \(M^{+mark}\).



\section{\(CFT\), \(CDFT\) and \(SS\)}

We begin by generalizing the following result:

\begin{theorem}[Lem 2.7 of \cite{MR2678950}]
  The following are equivalent for any topological space \(X\).
  \begin{itemize}
    \item \(X\) is \(SS\).
    \item \(X\) is separable and \(CDFT\).
    \item \(X\) has a countable dense subset \(D\) where
          \(CDFT_x\) holds for all \(x\in D\).
  \end{itemize}
\end{theorem}

\begin{theorem}
  The following are equivalent for any topological space \(X\).
  \begin{itemize}
    \item \(X\) is \(SS\) (resp. \(SS^+\), \(SS^{+mark}\)).
    \item \(X\) is separable and \(CDFT\)
          (resp. \(CDFT^+\), \(CDFT^{+mark}\)).
    \item \(X\) has a countable dense subset \(D\) where
          \(CDFT_x\) (resp. \(CDFT_x^+\), \(CDFT_x^{+mark}\))
          holds for all \(x\in D\).
  \end{itemize}
\end{theorem}

\begin{proof}
  We need only show that the final condition implies the first.
  Let \(D=\{d_i:i<\omega\}\).

  Let \(\sigma_i\) be a winning strategy witnessing \(CDFT_{d_i}^+\)
  for each \(i<\omega\). We define the strategy \(\tau\) for the
  \(SS\) game by
  \[
    \tau(\<D_0,\dots,D_n\>)
      =
    \bigcup_{i\leq n}
    \sigma_i(\<D_i,\dots,D_n\>)
  .\]

  By \(CDFT_{d_i}^+\), we have
  \[
    d_i
      \in
    \cl{\bigcup_{i\leq n<\omega}\sigma_i(\<D_i,\dots,D_n\>)}
      \subseteq
    \cl{\bigcup_{i\leq n<\omega}\tau(\<D_0,\dots,D_n\>)}
      \subseteq
    \cl{\bigcup_{n<\omega}\tau(\<D_0,\dots,D_n\>)}
  \]
  and as \(D\subseteq\cl{\bigcup_{n<\omega}\tau(\<D_0,\dots,D_n\>)}\)
  it follows that
  \[
    X
      \subseteq
    \cl D
      \subseteq
    \cl{\cl{\bigcup_{n<\omega}\tau(\<D_0,\dots,D_n\>)}}
      =
    \cl{\bigcup_{n<\omega}\tau(\<D_0,\dots,D_n\>)}
  .\]
  Therefore \(\tau\) witnesses \(SS^+\).

  \sloppy
  The above proof may be easily modified for the Markov case by
  replacing \(\sigma_i(\<D_i,\dots,D_n\>)\) with \(\sigma_i(D_n,n)\)
  and \(\tau(\<D_0,\dots,D_n\>)\) with \(\tau(D_n,n)\).
\end{proof}

\fussy

So amongst separable spaces, we see that \(SS\)
(resp. \(SS^+\), \(SS^{+mark}\)) and
\(CDFT\) (resp. \(CDFT^+\), \(CDFT^{+mark}\))
are equivalent. We now further bridge the gap between \(CDFT\)
and \(CFT\) in the context of function spaces.
Consider the following result of Arhangel'skii.

\begin{theorem}[\cite{MR837289}]
  The following are equivalent for any \(T_{3\frac{1}{2}}\)
  topological space \(X\).
    \begin{itemize}
      \item \(X\) is \(\Omega M\).
      \item \(C_p(X)\) is \(CFT\).
    \end{itemize}
\end{theorem}

This result may similarly be generalized in a game theoretic sense.
In addition, this proof will demonstrate the equivalence of
\(CFT\) and \(CDFT\) in \(C_p(X)\).
It is unknown to the author whether Arhangel'skii
used a strategy similar to the following proof in \cite{MR837289},
but Sakai employed a similar technique
in \cite{MR964873} to relate the \(\Omega\)-Rothberger and
countable strong fan tightness properties
(and essentially, the countable strong dense fan tightness property).
Due to the difficulty in obtaining an English translation of
\cite{MR837289}, we reprove Arhangel'skii's theorem above in our
more general context below.

\begin{definition}
  Let \(X\) be a \(T_{3\frac{1}{2}}\) topological space.
  For \(\vec x\in C_p(X)\), \(F\in[X]^{<\omega}\), and
  \(\epsilon>0\), let
  \[
    [\vec x,F,\epsilon]
      =
    \{
      \vec y\in C_p(x)
    :
      |\vec y(t)-\vec x(t)|<\epsilon
    \text{ for all }
      t\in F
    \}
  \]
  give a basic open neighborhood of \(\vec x\).
\end{definition}

\begin{lemma}
  Let \(X\) be a \(T_{3\frac{1}{2}}\) topological space.
  If \(X\) is \(\Omega M\)
  (resp. \(\Omega M^+\), \(\Omega M^{+mark}\)).
  then \(C_p(X)\) is \(CFT_{\vec 0}\)
  (resp. \(CFT_{\vec 0}^+\), \(CFT_{\vec 0}^{+mark}\)).
\end{lemma}

\begin{proof}

\end{proof}

\begin{lemma}
  Let \(X\) be a \(T_{3\frac{1}{2}}\) topological space.
  If \(C_p(X)\) is \(CDFT_{\vec 0}\)
  (resp. \(CDFT_{\vec 0}^+\), \(CDFT_{\vec 0}^{+mark}\)),
  then \(X\) is \(\Omega M\)
  (resp. \(\Omega M^+\), \(\Omega M^{+mark}\)).
\end{lemma}

\begin{proof}
  For each \(\mc U\in\Omega_{X}\) define
  \[
    D(\mc U)
      =
    \{
      \vec y\in C_p(X)
    :
      \vec y[X\setminus U_{\vec y,\mc U}]=\{1\}
      \text{ for some }
      U_{\vec y,\mc U}\in\mc U
    \}
  .\]
  Consider the point \(\vec x\in C_p(X)\) and its basic open neighborhood
  \([\vec x,G,\epsilon]\). If \(\mc U\) is an \(\omega\)-cover
  of \(X\), \(G\subseteq U\) for some \(U_{\vec y,\mc U}\in\mc U\).
  Since \(X\) is
  \(T_{3\frac{1}{2}}\), \(X\setminus U_{\vec y,\mc U}\) is closed, and \(G\)
  is finite and disjoint from \(X\setminus U_{\vec y,\mc U}\),
  we may choose some function \(\vec y\in C_p(X)\) where
  \(\vec y[X\setminus U_{\vec y,\mc U}]=\{1\}\) and \(\vec x(t)=\vec y(t)\)
  for each \(t\in G\).
  It follows \(\vec y\in [\vec x,G,\epsilon]\cap D\), so \(D(\mc U)\)
  is dense in \(C_p(X)\).

  Consider the sequence of \(\omega\)-covers
  \(\<\mc U_0,\mc U_1,\dots\>\in\Omega_X^\omega\), and the
  corresponding sequence of dense subsets
  \(\<D(\mc U_0),D(\mc U_1),\dots\>\in\mc D_{C_p(X)}^\omega\).

  Assuming \(C_p(X)\) is \(CDFT_{\vec 0}\), choose a witness
  \(\<F_0,F_1,\dots\>\) such that
  \[
    \vec 0 \in \cl{\bigcup_{n<\omega} F_n}
  .\]
  Now let
  \[
    \mc F_n
      =
    \{
      U_{\vec y,\mc U_n}
    :
      \vec y \in F_n
    \}
      \in
    [\mc U_n]^{<\omega}
  .\]
  We claim that \(\bigcup_{n<\omega}\mc F_n\) is an \(\omega\)-cover.
  Let \(G\in[X]^{<\omega}\). The neighborhood \([\vec 0,G,\frac{1}{2}]\)
  contains some point \(\vec y\in F_n\) for some \(n<\omega\). It follows
  that \(U_{\vec y,\mc U_n}\in\mc U_n\) and
  \(\vec y[X\setminus U_{\vec y,\mc U_n}]=\{1\}\). It follows that
  \(G\cap(X\setminus U_{\vec y,\mc U_n})=\emptyset\), and therefore
  \(G\subseteq U_{\vec y,\mc U_n}\in\mc F_n\).

  Assuming \(C_p(X)\) is \(CDFT_{\vec 0}^+\), choose a witness
  \(\sigma\) such that
  \[
    \vec 0
      \in
    \cl{\bigcup_{n<\omega} \sigma(\<D(\mc U_0),\dots,D(\mc U_n)\>)}
  .\]
  Now let
  \[
    \tau(\<\mc U_0,\dots,\mc U_n\>)
      =
    \{
      U_{\vec y,\mc U_n}
    :
      \vec y \in \sigma(\<D(\mc U_0),\dots,D(\mc U_n)\>)
    \}
      \in
    [\mc U_n]^{<\omega}
  .\]
  We claim that \(\bigcup_{n<\omega}\tau(\<\mc U_0,\dots,\mc U_n\>)\)
  is an \(\omega\)-cover.
  Let \(G\in[X]^{<\omega}\). The neighborhood \([\vec 0,G,\frac{1}{2}]\)
  contains some point \(\vec y\in \sigma(\<D(\mc U_0),\dots,D(\mc U_n)\>)\)
  for some \(n<\omega\). It follows
  that \(U_{\vec y,\mc U_n}\in\mc U_n\) and
  \(\vec y[X\setminus U_{\vec y,\mc U_n}]=\{1\}\). As a result
  \(G\cap(X\setminus U_{\vec y,\mc U_n})=\emptyset\), and therefore
  \(G\subseteq U_{\vec y,\mc U_n}\in\mc \tau(\<\mc U_0,\dots,\mc U_n\>)\).

  Assuming \(C_p(X)\) is \(CDFT_{\vec 0}^{+mark}\), choose a witness
  \(\sigma\) such that
  \[
    \vec 0
      \in
    \cl{\bigcup_{n<\omega} \sigma(D(\mc U_n),n)}
  .\]
  Now let
  \[
    \tau(\mc U_n,n)
      =
    \{
      U_{\vec y,\mc U_n}
    :
      \vec y \in \sigma(D(\mc U_n),n)
    \}
      \in
    [\mc U_n]^{<\omega}
  .\]
  We claim that \(\bigcup_{n<\omega}\tau(\mc U_n,n)\)
  is an \(\omega\)-cover.
  Let \(G\in[X]^{<\omega}\). The neighborhood \([\vec 0,G,\frac{1}{2}]\)
  contains some point \(\vec y\in \sigma(D(\mc U_n),n)\)
  for some \(n<\omega\). It follows
  that \(U_{\vec y,\mc U_n}\in\mc U_n\) and
  \(\vec y[X\setminus U_{\vec y,\mc U_n}]=\{1\}\). As a result
  \(G\cap(X\setminus U_{\vec y,\mc U_n})=\emptyset\), and therefore
  \(G\subseteq U_{\vec y,\mc U_n}\in\mc \tau(\mc U_n,n)\).
\end{proof}

\begin{theorem}
  The following are equivalent for any \(T_{3\frac{1}{2}}\)
  topological space \(X\).
    \begin{itemize}
      \item \(X\) is \(\Omega M\)
            (resp. \(\Omega M^+\), \(\Omega M^{+mark}\)).
      \item \(C_p(X)\) is \(CFT\)
            (resp. \(CFT^+\), \(CFT^{+mark}\)).
      \item \(C_p(X)\) is \(CDFT\)
            (resp. \(CDFT^+\), \(CDFT^{+mark}\)).
    \end{itemize}
\end{theorem}

\begin{proof}
  Since \(\mc D_X\subseteq \mc B_{X,x}\), the second condition trivially
  implies the first. As \(C_p(X)\) is homogeneous, the \(C(D)FT\) properties
  are characterized by \(C(D)FT_{\vec 0}\). So the result follows from the
  previous lemmas.
\end{proof}


\bibliographystyle{plain}
\bibliography{../../bibliography}

\end{document}
