\documentclass{amsart}
\usepackage{amsmath}
\usepackage{amsthm}
\usepackage{amssymb}

\usepackage{tikz}
\usetikzlibrary{arrows}

\usepackage{../../clontzDefinitions}

\renewcommand{\vec}{\mathbf}

      \theoremstyle{plain}
      \newtheorem{theorem}{Theorem}
      \newtheorem{lemma}[theorem]{Lemma}
      \newtheorem{corollary}[theorem]{Corollary}
      \newtheorem{proposition}[theorem]{Proposition}
      \newtheorem{conjecture}[theorem]{Conjecture}
      \newtheorem{question}[theorem]{Question}
      \newtheorem{claim}[theorem]{Claim}

      \theoremstyle{definition}
      \newtheorem{definition}[theorem]{Definition}
      \newtheorem{notation}[theorem]{Notation}
      \newtheorem{example}[theorem]{Example}
      \newtheorem{game}[theorem]{Game}

      \theoremstyle{remark}
      \newtheorem{remark}[theorem]{Remark}

      \theoremstyle{plain}
      \newtheorem*{theorem*}{Theorem}
      \newtheorem*{lemma*}{Lemma}
      \newtheorem*{corollary*}{Corollary}
      \newtheorem*{proposition*}{Proposition}
      \newtheorem*{conjecture*}{Conjecture}
      \newtheorem*{question*}{Question}
      \newtheorem*{claim*}{Claim}

      \theoremstyle{definition}
      \newtheorem*{definition*}{Definition}
      \newtheorem*{example*}{Example}
      \newtheorem*{game*}{Game}

      \theoremstyle{remark}
      \newtheorem*{remark*}{Remark}




\begin{document}

\title{Dual selection games}



\author{Steven Clontz}
\address{Department of Mathematics and Statistics,
The University of South Alabama,
Mobile, AL 36688}
\email{sclontz@southalabama.edu}



\keywords{Selection principle, selection game,
limited information strategies}

\subjclass[2010]{54C30, 54D20, 54D45, 91A44}






\begin{abstract}
  (an investigation of dual selection games)
\end{abstract}


\maketitle







\section{Introduction}

\begin{definition}
  The \term{selection game} \(\schStrongSelGame{\mc A}{\mc B}\) 
  is an \(\omega\)-length game involving Players \(\plI\) and \(\plII\). 
  During round \(n\), \(\plI\) chooses
  \(A_n\in\mc A\), followed by \(\plII\) choosing \(B_n\in A_n\).
  Player \(\plII\) wins in the case that \(\{B_n:n<\omega\}\in\mc B\),
  and Player \(\plI\) wins otherwise.
\end{definition}

  For brevity, let 
  \[
    \schStrongSelGame{\mc A}{\neg \mc B}
      =
    \schStrongSelGame{\mc A}{\mc P\left(\bigcup \mc A\right)\setminus \mc B}
  .\]
  That is, \(\plII\) wins in the case that \(\{B_n:n<\omega\}\not\in\mc B\),
  and \(\plI\) wins otherwise.

\begin{definition}
  For a set \(X\), let \(\mathbf C(X)\) be the collection of all
  choice functions on \(X\), functions \(f:X\to\bigcup X\) 
  such that \(f(x)\in x\) for all \(x\in X\).
\end{definition}

\begin{definition}
  The set \(\mc R\) is said to be a \term{reflection} of the set \(\mc A\)
  if \[\mc A= \{\ran f:f\in\mathbf C(\mc R)\}.\]
\end{definition}

  For example, a reflection of the collection \(\mc O_X\) of basic open covers
  of \(X\) would be \(\mc P_X=\{\mc T_{X,x}:x\in X\}\), where \(\mc T_{X,x}\) 
  is the corresponding point-base at \(x\in X\). 
  Likewise for the collection \(\Omega_{X,x}\)
  of sets with \(x\in X\) as a limit point, \(\mc T_{X,x}\) is itself
  a reflection.

\begin{lemma}
  Let \(\mc R\) be a reflection of \(\mc A\). Then \(\bigcup \mc R=\bigcup \mc A\).
\end{lemma}
\begin{proof}
  If \(x\in \bigcup\mc A\), then \(x\in\ran{f}\) for some \(f\in\mathbf C(\mc R)\).
  Thus \(x=f(R)\in R\) for some \(R\in\mc R\), showing \(x\in\bigcup\mc R\).

  Likewise if \(x\in\bigcup\mc R\), so \(x\in R\) for some \(R\in \mc R\).
  Let \(f\in\mathbf C(\mc R)\) satisfy \(f(R)=x\), so \(x\in\ran{f}\),
  showing \(x\in\bigcup\mc A\).
\end{proof}

\begin{theorem}
  Let \(\mc R\) be a reflection of \(\mc A\). 

  Then
  \(\plI\prewin\schStrongSelGame{\mc A}{\mc B}\) if and only if
  \(\plII\markwin\schStrongSelGame{\mc R}{\neg\mc B}\).
\end{theorem}

\begin{proof}
  Let \(\sigma\) witness 
  \(\plI\prewin\schStrongSelGame{\mc A}{\mc B}\).
  Since \(\sigma(n)\in\mc A=\{\ran{f}:f\in\mathbf C(\mc R)\}\), 
  \(\sigma(n)=\ran{f_n}\)
  for some \(f_n\in\mathbf C(\mc R)\). So let
  \(\tau(R,n)=f_n(R)\) for all \(R\in \mc R\) and \(n<\omega\).
  Suppose \(R_n\in \mc R\) for all \(n<\omega\).
  Note that since \(\sigma\) is winning and 
  \(\tau(R_n,n)=f_n(R_n)\in\ran{f_n}=\sigma(n)\),
  \(\{\tau(R_n,n):n<\omega\}\not\in\mc B\). Thus \(\tau\) witnesses
  \(\plII\markwin\schStrongSelGame{\mc R}{\neg\mc B}\).

  Now let \(\sigma\) witness
  \(\plII\markwin\schStrongSelGame{\mc R}{\neg\mc B}\).
  Let \(f_n\in\mathbf C(\mc R)\) be defined by \(f_n(R)=\sigma(R,n)\).
  Since \(\tau(n)\in\mc A=\{\ran f:f\in\mathbf C(\mc R)\}\), let
  \(\tau(n)=\ran{f_n}\). Suppose that \(B_n\in\tau(n)=\ran{f_n}\) for
  all \(n<\omega\). Choose \(R_n\in\mc R\) such that 
  \(B_n=f_n(R_n)=\sigma(R_n,n)\). Since \(\sigma\) is winning,
  \(\{B_n:n<\omega\}\not\in\mc B\). Thus \(\tau\) witnesses
  \(\plI\prewin\schStrongSelGame{\mc A}{\mc B}\).
\end{proof}

\begin{theorem}
  Let \(\mc R\) be a reflection of \(\mc A\). 

  Then
  \(\plII\markwin\schStrongSelGame{\mc A}{\mc B}\) if and only if
  \(\plI\prewin\schStrongSelGame{\mc R}{\neg\mc B}\).
\end{theorem}

\begin{proof}
  Let \(\sigma\) witness 
  \(\plII\markwin\schStrongSelGame{\mc A}{\mc B}\).
  Let \(n<\omega\). Suppose that for each \(R\in\mc R\),
  there was \(g(R)\in R\) such that for all \(A\in \mc A\),
  \(\sigma(A,n)\not=g(R)\). Then \(g\in\mathbf C(\mc R)\),
  and \(\sigma(\ran g,n)\not=g(R)\) for all \(R\in\mc R\),
  a contradiction.

  So choose \(\tau(n)\in\mc R\) such that for all \(r\in \tau(n)\)
  there exists \(A_{r,n}\in\mc A\) such that \(\sigma(A_{r,n},n)=r\).
  It follows that when \(r_n\in\tau(n)\) for \(n<\omega\),
  \(\{r_n:n<\omega\}=\{\sigma(A_{r_n,n}:n<\omega\}\in B\),
  so \(\tau\) witnesses
  \(\plI\prewin\schStrongSelGame{\mc R}{\neg\mc B}\).

  Now let \(\sigma\) witness 
  \(\plI\prewin\schStrongSelGame{\mc R}{\neg\mc B}\).
  Then \(\sigma(n)\in\mc R\), so for \(A\in\mc A\), let
  \(f_A\in\mathbf C(\mc R)\) satisfy \(A=\ran{f_A}\),
  and let \(\tau(A,n)=f_A(\sigma(n))\).
  Then if \(A_n\in\mc A\) for \(n<\omega\), \(\tau(A_n,n)\in\sigma(n)\),
  so \(\{\tau(A_n,n):n<\omega\}\in\mc B\).
  Thus \(\tau\) witnesses
  \(\plII\markwin\schStrongSelGame{\mc A}{\mc B}\).
\end{proof}

\begin{theorem}
  Let \(\mc R\) be a reflection of \(\mc A\). 

  Then
  \(\plI\win\schStrongSelGame{\mc A}{\mc B}\) if and only if
  \(\plII\win\schStrongSelGame{\mc R}{\neg\mc B}\).
\end{theorem}

\begin{proof}
  Let \(\sigma\) witness 
  \(\plI\win\schStrongSelGame{\mc A}{\mc B}\).
  Let \(c(\emptyset)=\emptyset\). Suppose 
  \(c(s)\in(\bigcup A)^{<\omega}=(\bigcup R)^{<\omega}\)
  is defined for \(s\in\mc R^{<\omega}\). Since \(\sigma(c(s))\in\mc A\),
  let \(f_s\in\mathbf C(\mc R)\) satisfy \(\sigma(c(s))=\ran{f_s}\),
  and let \(c(s\concat\<R\>)=c(s)\concat\<f_s(R)\>\).
  Then let \(c(\alpha)=\bigcup\{c(\alpha\rest n):n<\omega\}\)
  for \(\alpha\in\mc R^\omega\), so
  \[
    c(\alpha)(n)
      =
    f_{\alpha\rest n}(\alpha(n))
      \in
    \ran{f_{\alpha\rest n}}
      =
    \sigma(c(\alpha\rest n))
  \]
  demonstrating that \(c(\alpha)\) is a legal attack against \(\sigma\).

  Let \(\tau(s\concat\<R\>)=f_s(R)\). Consider the attack \(\alpha\in\mc R^\omega\)
  against \(\tau\). Then since \(\sigma\) is winning and
  \(
    \tau(\alpha\rest n+1)=f_{\alpha\rest n}(\alpha(n))\in
    \ran{f_{\alpha\rest n}}=\sigma(c(\alpha\rest n))
  \), it follows that \(\{\tau(\alpha\rest n+1):n<\omega\}\not\in\mc B\).
  Thus \(\tau\) witnesses
  \(\plII\win\schStrongSelGame{\mc R}{\neg\mc B}\).

  Now let \(\sigma\) witness
  \(\plII\win\schStrongSelGame{\mc R}{\neg\mc B}\).
  For \(s\in \mc R^{<\omega}\), define \(f_s\in\mathbf C(\mc R)\)
  by \(f_s(R)=\sigma(s\concat\<R\>)\). Let \(\tau(\emptyset)=\ran{f_\emptyset}\),
  and for \(x\in\tau(\emptyset)\), choose \(R_{\<x\>}\in\mc R\) such that
  \(x=f_{\emptyset}(R_{\<x\>})\) (for other \(x\in\bigcup A\), choose \(R_{\<x\>}\)
  arbitrarily as it won't be used). Now let \(s\in(\bigcup A)^{<\omega}\setminus\emptyset\),
  and suppose \(\tau(s\rest n)\in\mc A\) and \(R_{s\rest n+1}\in\mc R\) have been
  defined for \(n<|s|\). Then let \(\tau(s)=\ran{f_{\<R_{s\rest 0},\dots,R_s\>}}\)
  and for \(x\in\tau(s)\) choose \(R_{s\concat\<x\>}\) such that
  \(x=f_{\<R_{s\rest 0},\dots,R_s\>}(R_{s\concat\<x\>})\) (and again,
  choose \(R_{s\concat\<x\>}\) arbitrarily for other \(x\in\bigcup\mc A\) as it won't be used).

  Then let \(\alpha\) attack \(\tau\), so
  \(\alpha(n)\in\tau(\alpha\rest n)\) and thus 
  \(\alpha(n)=f_{\<R_{\alpha\rest 0},\dots,R_{\alpha\rest n}\>}(R_{\alpha\rest n+1})
  =\sigma(\<R_{\alpha\rest 0},\dots,R_{\alpha\rest n+1}\>)\).  Since \(\sigma\) is winning,
  \(\{\sigma(\<R_{\alpha\rest 0},\dots,R_{\alpha\rest n+1}\>):n<\omega\}
  =\{\alpha(n):n<\omega\}\not\in\mc B\).
  Thus \(\tau\) witnesses
  \(\plI\win\schStrongSelGame{\mc A}{\mc B}\).
\end{proof}

\bibliographystyle{plain}
\bibliography{../../bibliography}

\end{document}
