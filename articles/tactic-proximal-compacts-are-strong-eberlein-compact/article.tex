\documentclass{amsart}
\usepackage{amsmath}
\usepackage{amsthm}
\usepackage{amssymb}

\usepackage{../../clontzDefinitions}

\newtheorem{theorem}{Theorem}[section]
\newtheorem{proposition}[theorem]{Proposition}
\newtheorem{lemma}[theorem]{Lemma}
\newtheorem{corollary}[theorem]{Corollary}


\theoremstyle{definition}
\newtheorem{definition}[theorem]{Definition}
\newtheorem{game}[theorem]{Game}
\newtheorem{example}[theorem]{Example}
\newtheorem{question}[theorem]{Question}

\parskip=0.5em



\begin{document}

% \title{Proximal compact spaces are Corson compact\tnoteref{t1}}
% \tnotetext[t1]{2010 Mathematics Subject Classification. 54E15, 54D30, 54A20.}
\title{Tactic-proximal compact spaces are strong Eberlein compact}

% \author[aub]{S.~Clontz\fnref{fn1}}
% \ead{steven.clontz@auburn.edu}
% \author[aub]{G.~Gruenhage\fnref{fn2}}
% \ead{gruengf@auburn.edu}

% \address[aub]{Department of Mathematics, Auburn University,
%  Auburn, AL 36830}


\author{Steven Clontz}
\address{Department of Mathematics, Auburn University,
Auburn, AL 36830}
\email{steven.clontz@gmail.edu}
\urladdr{www.stevenclontz.com}

\keywords{Proximal, tactic proximal, Corson compact, strong Eberlein compact, topological game}
% \begin{keyword}
%   Proximal\sep absolutely proximal\sep Corson compact\sep uniform space\sep $W$-space
% \end{keyword}

\subjclass[2010]{54E15, 54D30, 54A20}


\begin{abstract}
  The author and G. Gruenhage previously showed that J. Bell's proximal game
  may be used to characterize Corson compactness in compact Hausdorff spaces.
  Using limited information strategies, the proximal game may also be
  used to characterize the strong Eberlein compactness property.
  In doing so, a purely topological characterization of the proximal game
  is introduced, and several existing results on
  the proximal game are given
  analogues considering limited information strategies.
\end{abstract}


\maketitle

  Two papers published in 2014 introduced the
  \term{proximal uniform space game}
  \(\bellUniGame{X}\) due to Jocelyn Bell. If \(X\) is a
  topological space, and there exists a uniform structure inducing its topology
  which gives the first player in this game has a winning stategy,
  then \(X\) is said to be a \term{proximal} space. Bell used this
  game as a tool in \cite{MR3239205} for investigating uniform box products,
  and the author showed with Gary Gruenhage in \cite{MR3227201} that
  this game characterizes Corson compactness amongst compact Hausdorff spaces,
  answering a question of Peter Nyikos in \cite{nyikosProximalPreprint}.

  All spaces in this paper are assumed to be \(T_{3\frac{1}{2}}\), so that they
  have a uniform structure inducing the topology on the space.
  Unlike many game-theoretic topological properties, the game \(\bellUniGame{X}\)
  for which the proximal property was defined by is not itself a topological
  game. However, by considering entourages of the universal uniformity
  inducing the topology of a space, this original uniform space game may be
  easily modified to the purely topological games
  \(\bellConGame{X}\), \(\bellAbsConGame{X}\).

  The aim of this paper is to use this topological interpretation of the
  proximal game to give a new game-theoretic characterization of the
  strong Eberlein compactness property. Strong Eberlein compacts are Corson
  compacts, and therefore proximal compact spaces; in fact, it will be shown
  that strong Eberlein compacts are exactly the compact spaces for which the
  first player has a \term{tactical} winning strategy for the proximal game,
  a strategy which relies on only the most recent move of the opponent.


\section{Topologizing \(\bellUniGame{X}\)}

  We refer to \cite{MR3227201} for definitions, notation, and basic theorems on
  uniform spaces and the proximal game \(\bellUniGame{X}\) (denoted as
  \(Prox_{D,P}(X)\) in that paper). In particular recall that:

  \begin{definition}
    \(\pl P \win G\) denotes that the player \(\pl P\) has a winning strategy
    in the \(\omega\)-length game \(G\).
  \end{definition}

  \begin{game}
    Let \(\bellUniGame{X,\mb D}\) denote the
    \term{proximal uniform space game} with
    players \(\pl D\), \(\pl P\) which proceeds as follows for
    a space \(X\) with uniformity \(\mb D\). In round \(0\),
    \(\pl D\) chooses an entourage \(D_0\in\mb D\), followed by \(\pl P\)
    choosing a point \(p_0\in X\). In round \(n+1\),
    \(\pl D\) chooses an entourage
    \(D_{n+1}\in\mb D\),
    followed by \(\pl P\) choosing a point \(p_{n+1}\in D_n[p_n]\).

    \(\pl D\) wins in the case that either
    \(\<p_0,p_1,\dots\>\) converges in \(X\),
    or \(\bigcap_{n<\omega}D_n[p_n] = \emptyset\). \(\pl P\) wins otherwise.
  \end{game}

  \begin{definition}
    A uniformizable space \(X\) is \term{proximal}
    in the case that there exists a uniformity
    \(\mb D\) for \(X\) such that \(\pl D\win\bellUniGame{X,\mb D}\).
  \end{definition}

  As it turns out, the search for such a uniformity is trivial.

  \begin{definition}
    The \term{universal uniformity} for a uniformizable topology is the
    union of all uniformities which induce the given topology.
  \end{definition}

  \begin{theorem}[\cite{MR2048350}]
    The universal uniformity is itself a uniformity compatibile with
    its given topology.
  \end{theorem}

  \begin{definition}
    For a uniformizable space \(X\), a \term{universal entourage} \(D\) is a
    entourage of the universal uniformity.
  \end{definition}

  \begin{theorem}[\cite{MR2048350}]
    For every uniformizable space, if \(D\) is a neighborhood of the diagonal
    \(\Delta\)  such that there exist neighborhoods \(D_n\) of \(\Delta\) with
    \(D\supseteq D_0\) and
    \(D_n \supseteq D_{n+1}\circ D_{n+1}\), then \(D\) is a
    universal entourage.
  \end{theorem}

  \begin{definition}
    For every entourage \(D\) and \(n<\omega\), let \(\frac{1}{2^n}D\)
    denote an entourage such that \(\frac{1}{1}D=D\) and
    \(\frac{1}{2^{n+1}}D\circ\frac{1}{2^{n+1}}D\subseteq\frac{1}{2^n}D\).
  \end{definition}

  \begin{definition}
    An \term{open symmetric entourage} \(D\) is a entourage
    which is open in the product topology induced by the uniformity and where
    \(D=D^{-1}=\{\<y,x\>:\<x,y\>\in D\}\).
  \end{definition}

  \begin{theorem}
    For every entourage \(D\), there exists an open symmetric
    entourage \(U\subseteq D\).
  \end{theorem}

  Due to this theorem, we will simply use the word \term{entourage} to refer
  to open symmetric universal entourages. Note that if \(D\) is an
  entourage, then \(D[x]=\{y:\<x,y\>\in D\}\) is an open neighborhood of \(x\).
  One may consider
  \(D[x]\) to be an entourage-``ball'' about \(x\), generalizing the notion of
  an \(\epsilon\)-ball given by a metric structure.

  In the case that
  the space is paracompact, entourages are even more easily found.

  \begin{theorem}[\cite{MR2048350}]
    Every open neighborhood of the diagonal is a universal entourage for
    paracompact uniformizable spaces.
  \end{theorem}

\subsection{Using universal entourages to characterize the proximal property}

  The natural adaptation of the original uniform space game \(\bellUniGame{X}\)
  to a topological game requires the use of the universal uniformity on \(X\).

  \begin{game}
    Let \(\bellConHardGame{X}\) denote the \term{hard Bell convergence game} with
    players \(\pl D\), \(\pl P\) which proceeds as follows for
    a uniformizable space \(X\). In round \(0\),
    \(\pl D\) chooses an entourage \(D_0\), followed by \(\pl P\)
    choosing a point \(p_0\in X\). In round \(n+1\),
    \(\pl D\) chooses an entourage
    \(D_{n+1}\), followed by \(\pl P\) choosing a point \(p_{n+1}\in D_n[p_n]\).

    \(\pl D\) wins in the case that either
    \(\<p_0,p_1,\dots\>\) converges in \(X\),
    or \(\bigcap_{n<\omega}D_n[p_n] = \emptyset\). \(\pl P\) wins otherwise.
  \end{game}

  This game is considered ``hard'' due to the requirement that \(\pl D\)
  keep track of the history of the game to ensure that successive
  moves refine previous moves. This record-keeping may be eliminated by
  requiring that \(\pl P\) respect all moves made by \(\pl D\) rather than only
  the most recent move.

  \begin{game}
    Let \(\bellConGame{X}\) denote the \term{Bell convergence game} with players
    \(\pl D\), \(\pl P\) which proceeds analogously to \(\bellConHardGame{X}\),
    except for the following. Let \(E_n=\bigcap_{m\leq n}D_n\), where \(D_n\) is
    the entourage played by \(\pl D\) in round \(n\).
    Then \(\pl P\) must ensure that \(p_{n+1}\in E_n[p_n]\),
    and \(\pl D\) wins when either \(\<p_0,p_1,\dots\>\) converges in \(X\)
    or \(\bigcap_{n<\omega}E_n[p_n] = \emptyset\).
  \end{game}

  These games are all essentially equivalent with respect to perfect
  information for \(\pl D\).

  \begin{theorem}
    \(\pl D\win\bellConHardGame{X}\) if and only if
    \(\pl D\win\bellConGame{X}\) if and only if
    \(X\) is proximal.
  \end{theorem}

  \begin{proof}
    If \(\pl D\win\bellConHardGame{X}\), then we immediately see that
    \(\pl D\win\bellConGame{X}\).
    If \(\sigma\) is a winning strategy for \(\pl D\)
    in \(\bellConGame{X}\), then \(\tau\) defined by
    \(\tau(s)=\bigcap_{t\leq s}\sigma(t)\) is easily seen to be a winning
    strategy for \(\pl D\) in \(\bellConHardGame{X}\).

    If \(\pl D\win\bellConHardGame{X}\), then \(\pl D\win\bellUniGame{X}\)
    for the universal uniformity, showing \(X\) is proximal.
    Finally, if \(X\) is proximal, then there exists a winning strategy
    \(\sigma\) for \(\bellUniGame{X}\) for a uniformity inducing the
    topology on \(X\). Then a winning strategy for \(\pl D\) in
    \(\bellConHardGame{X}\) may be constructed by converting every
    entourage in this uniformity into a smaller open symmetric universal
    entourage.
  \end{proof}

  The secondary winning condition in \(\bellConGame{X}\)
  allows for a space to be incomplete: \(\pl D\win\bellConGame{\mb Q}\) by
  playing \(E_{1/2^n}=\{\<x,y\>:d(x,y)<\frac{1}{2^n}\}\) in each round. This
  forces \(\pl P\)'s sequence to be Cauchy, and thus it either converges to
  a rational, or the sets \(E_{1/2^n}[x_n]\) will have empty intersection
  (where the irrational point of convergence would be). Uniformly locally
  compact spaces (and in particular, compact spaces) lack such holes, so it
  will be convenient to eliminate this technicality when it is irrelevant.

  \begin{definition}
    Let \(\bellAbsConGame{X}\) denote the
    \term{absolute Bell convergence game} which
    proceeds analogously to \(\bellConGame{X}\), except
    that \(\pl D\) must always ensure that \(\<p_0,p_1,\dots\>\) converges
    in \(X\) in order to win.
  \end{definition}

  \begin{definition}
    A uniformizable space \(X\) is \term{absolutely proximal} if
    \(\pl D \win \bellAbsConGame{X}\).
  \end{definition}

  As was shown in \cite{MR3227201}:

  \begin{definition}
    A uniformizable space \(X\) is \term{uniformly locally compact} if there
    exists an entourage \(D\) such that \(\cl{D[x]}\) is compact
    for all \(x\).
  \end{definition}

  \begin{theorem}\label{uniformlyLocallyCompact}
    If \(X\) is a uniformly locally compact space, then
    \(\pl D \win \bellConGame{X}\) if and only if
    \(\pl D \win \bellAbsConGame{X}\).
  \end{theorem}

  So absolutely proximal compacts are simply proximal compacts (and therefore
  Corson compacts).


\section{Limited information analogues}

  First recall the definitions of the following limited information strategies.

  \begin{definition}
    A \term{\(k\)-tactical strategy} or \term{\(k\)-tactic}
    is a strategy which considers only the most recent move of the opponent.

    If \(\pl P\) has a winning \(k\)-tactic for a game \(G\), we write
    \(\pl P\ktactwin{k} G\). If omitted, assume \(k=1\).
  \end{definition}

  \begin{definition}
    A \term{\(k\)-Markov strategy} or \term{\(k\)-mark} is a strategy
    which considers only the round number and most recent move of the
    opponent.

    If \(\pl P\) has a winning \(k\)-mark for a game \(G\), we write
    \(\pl P\kmarkwin{k} G\). If omitted, assume \(k=1\).
  \end{definition}

  Limited information strategies may be used to strengthen game-theoretic
  topological properties.

  \begin{definition}
    A uniformizable space \(X\) is \term{(absolutely) tactic-proximal} if
    \(\pl D \tactwin \bellConGame{X}\) (\(\pl D \tactwin \bellAbsConGame{X}\)).
  \end{definition}

  \begin{definition}
    A uniformizable space \(X\) is \term{(absolutely) Markov-proximal} if
    \(\pl D \markwin \bellConGame{X}\) (\(\pl D \markwin \bellAbsConGame{X}\)).
  \end{definition}

  As in Theorem \ref{uniformlyLocallyCompact}, the ``absolutely''
  is redundant in the above definitions when \(X\) is uniformly locally compact,
  so the absolute Bell convergence game will be used for convenience in the
  context of compact spaces.

  Some results of Bell may be generalized to hold for such limited information
  strategies. The proofs of the following propositions are straight forward.

  \begin{proposition}
    Let \(X\) be a uniformizable space and \(H\) be a closed subset of \(X\).
    If \(k<\omega\), then
    \begin{itemize}
      \item
       \(
          \pl D \ktactwin{k} \bellConGame{X}
            \Rightarrow
          \pl D \ktactwin{k} \bellConGame{H}
       \)
      \item
       \(
          \pl D \kmarkwin{k} \bellConGame{X}
            \Rightarrow
          \pl D \kmarkwin{k} \bellConGame{H}
       \)
      \item
       \(
          \pl D \win \bellConGame{X}
            \Rightarrow
          \pl D \win \bellConGame{H}
       \)
    \end{itemize}
  \end{proposition}

  \begin{proposition}
    Let \(X\) be a uniformizable space and \(H\) be a closed subset of \(X\).
    If \(k<\omega\), then
    \begin{itemize}
      \item
       \(
          \pl D \ktactwin{k} \bellAbsConGame{X}
            \Rightarrow
          \pl D \ktactwin{k} \bellAbsConGame{H}
       \)
      \item
       \(
          \pl D \kmarkwin{k} \bellAbsConGame{X}
            \Rightarrow
          \pl D \kmarkwin{k} \bellAbsConGame{H}
       \)
      \item
       \(
          \pl D \win \bellAbsConGame{X}
            \Rightarrow
          \pl D \win \bellAbsConGame{H}
       \)
    \end{itemize}
  \end{proposition}

  Less obvious is the following.

  \begin{definition}
    Let \(X_\alpha\) be a topological space for \(\alpha<\kappa\)
    and \(z\in\prod_{\alpha<\kappa}X_\alpha\).
    The \term{\(\Sigma\)-product} \(\SigmaProd_{\alpha<\kappa}^{z}X_\alpha\)
    with base point \(z\) is given by
    \[
      \SigmaProd_{\alpha<\kappa}^{z}X_\alpha
        =
      \left\{
        x\in \prod_{\alpha<\kappa}X_\alpha
      :
        |\{\alpha<\kappa:x(\alpha)\not=z(\alpha)\}|\leq\omega
      \right\}
    \]
    When \(X_\alpha=X\) and \(z(\alpha)=0\) for all \(\alpha<\kappa\),
    then we write \(\SigmaPower{X}{\kappa}=\SigmaProd_{\alpha<\kappa}^z X\).
  \end{definition}

  \begin{theorem}[\cite{MR3239205}]
    If \(X_\alpha\) is proximal for all \(\alpha<\kappa\),
    then \(\SigmaProd_{\alpha<\kappa}^{z}X_\alpha\) is proximal for any
    base point \(z\).
  \end{theorem}

\subsection{\(\Sigma^\star\)- and \(\sigma\)-Products}

  We may consider smaller subspaces of
  \(X^\kappa\) than given by \(\SigmaPower{X}{\kappa}\).

  \begin{definition}
    Let \(X\) be a metrizable space with
    compatible metric \(d\), and let \(z\in X^\kappa\).
    The \term{\(\Sigma^\star\)-product}
    \(\SigmaStarProd_{\alpha<\kappa}^{z}X\)
    with base point \(z\) is given by
    \[
      \SigmaStarProd_{\alpha<\kappa}^{z}X
        =
      \left\{
        x\in \prod_{\alpha<\kappa}X
      :
        n<\omega
      \Rightarrow
        \left|\left\{
          \alpha<\kappa:d(x(\alpha),z(\alpha))>\frac{1}{2^n}
        \right\}\right|
        <\omega
      \right\}
    \]
    When \(z(\alpha)=0\) for all \(\alpha<\kappa\),
    then we write
    \(\SigmaStarPower{X}{\kappa}=\SigmaStarProd_{\alpha<\kappa}^z X\).
  \end{definition}

  \begin{definition}
    Let \(X_\alpha\) be a topological space for \(\alpha<\kappa\)
    and \(z\in\prod_{\alpha<\kappa}X_\alpha\).
    The \term{\(\sigma\)-product} \(\sigmaProd_{\alpha<\kappa}^{z}X_\alpha\)
    with base point \(z\) is given by
    \[
      \sigmaProd_{\alpha<\kappa}^{z}X_\alpha
        =
      \left\{
        x\in \prod_{\alpha<\kappa}X_\alpha
      :
        |\{\alpha<\kappa:x(\alpha)\not=z(\alpha)\}|<\omega
      \right\}
    \]
    When \(X_\alpha=X\) and \(z(\alpha)=0\) for all \(\alpha<\kappa\),
    then we write \(\sigmaPower{X}{\kappa}=\sigmaProd_{\alpha<\kappa}^z X\).
  \end{definition}

  Of course,
    \[
      \sigmaPower{X}{\kappa}
        \subseteq
      \SigmaStarPower{X}{\kappa}
        \subseteq
      \SigmaPower{X}{\kappa}
        \subseteq
      X^\kappa
    \]
  for metrizable \(X\), and
    \[
      \sigmaProd_{\alpha<\kappa}^{z}X_\alpha
        \subseteq
      \SigmaProd_{\alpha<\kappa}^{z}X_\alpha
        \subseteq
      \prod_{\alpha<\kappa}X_\alpha
    \]
  for all spaces \(X_\alpha\) and base points \(z\).

  Just as \(\Sigma\) products preserve winning perfect-information strategies,
  we will show that these product subspaces preserve certain
  winning limited-information strategies.

  \begin{definition}
    For a metric space \(\<X,d\>\), let
    \(E_\epsilon=\{\<x,y\>:d(x,y)<\epsilon\}\). Note that \(E_\epsilon\)
    is an (open symmetric) entrouage on \(X\).
  \end{definition}

  \begin{proposition}
    For any metrizable space \(X\),
    \(\pl D\kmarkwin{0}\bellConGame{X}\).
    For any completely metrizable space \(X\),
    \(\pl D\kmarkwin{0}\bellAbsConGame{X}\).
  \end{proposition}

  \begin{proof}
    Essentially shown by Bell in her paper: in either case,
    \(\pl D\) chooses \(E_{1/2^{n}}\)
    during round \(n\), forcing legal attacks by \(\pl P\) to be Cauchy.
  \end{proof}

  As an aside, this next proposition may be proved similarly
  using \(E_{d(x,y)/2}\).

  \begin{proposition}
    For any metrizable space \(X\),
    \(\pl D\ktactwin{2}\bellConGame{X}\).
    For any completely metrizable space \(X\),
    \(\pl D\ktactwin{2}\bellAbsConGame{X}\).
  \end{proposition}

  We will exploit the last move of the
  opponent to obtain winning Markov strategies under
  \(\Sigma^\star\) products.

  \begin{proposition}
    If \(X_\alpha\) is a uniformizable space for \(\alpha<\kappa\),
    and \(D_\alpha\) is an entourage of \(X_\alpha\) for
    \(\alpha\in F\in[\kappa]^{<\omega}\), then
      \[
        P(\{\<\alpha,D_\alpha\>:\alpha\in F\})
          =
        \left\{
          \<x,y\>\in\left(\prod_{\alpha<\kappa}X_\alpha\right)^2
        :
          \alpha\in F
        \Rightarrow
          \<x(\alpha),y(\alpha)\>\in D_\alpha
        \right\}
      \]
    is an entourage of \(\prod_{\alpha<\kappa}X_\alpha\).
  \end{proposition}

  \begin{theorem}
    For any metrizable space \(X\) and \(z\in X^\kappa\),
    \(\pl D\markwin\bellConGame{\SigmaStarProd_{\alpha<\kappa}^z X}\).
  \end{theorem}

  \begin{proof}
    For \(x\in\SigmaStarProd_{\alpha<\kappa}^z X\), let
      \[
        \supp_n(x)
          =
        \left\{
          \alpha<\kappa
        :
          d(x(\alpha),z(\alpha))>\frac{1}{2^n}
        \right\}
          \in
        \kappa^{<\omega}
      \]
    where \(d\) is a metric compatible with the topology on \(X\).

    Define a strategy \(\tau\) for \(\pl D\) by
      \[
        \tau(\emptyset,0)
          =
        \left(\SigmaStarProd_{\alpha<\kappa}^z X\right)^2
      \]
      \[
        \tau(\<p\>,n+1)
          =
        \left(\SigmaStarProd_{\alpha<\kappa}^z X\right)^2
          \cap
        P(\{
          \<\alpha,E_{1/2^{n+1}}\>
            :
          \alpha\in\supp_n(p)
        \})
      \]

    Let \(a:\omega\to \SigmaStarProd_{\alpha<\kappa}^z X\) be a legal
    attack by \(\pl P\) against \(\tau\), so
      \[
        \<a(n+1),a(n+2)\>
          \in
        \bigcap_{m\leq n}
        \tau(\<a(m)\>,m+1)
      \]
      \[
          =
        \left(\SigmaStarProd_{\alpha<\kappa}^z X\right)^2
          \cap
        \bigcap_{m\leq n}
        P(\{
          \<\alpha,E_{1/2^{m+1}}\>
            :
          \alpha\in\supp_m(a(m))
        \})
      \]
    and let \(\alpha<\kappa\).

    If \(d(a(m+1)(\alpha),z(\alpha))\leq\frac{1}{2^m}\) for all but finite
    \(m<\omega\), then  \(\lim_{n<\omega}a(n)(\alpha)=z(\alpha)\).
    Otherwise, we have \(\alpha\in\supp_m(a(m))\) for infinitely many
    \(m<\omega\).

    If
    \(\bigcap_{n<\omega}\tau(\<a(n)\>,n+1)[a(n+1)]=\emptyset\), then
    \(\pl D\) has already won.
    Otherwise there exists a nonstrictly increasing
    unbounded function \(f\in\omega^\omega\) with
    \(\bigcap_{n<\omega}E_{1/2^{f(n)+1}}[a(n+1)(\alpha)]\not=\emptyset\).
    Since the diameter of these sets approaches \(0\), the intersection is
    a singleton \(x\); furthermore, the \(\alpha\) coordinate of \(a\) must
    then converge to \(x\). We conclude that
    \(\pl D\markwin\bellConGame{\SigmaStarProd_{\alpha<\kappa}^z X}\).
  \end{proof}

  A similar result holds for tactics and
  \(\sigma\)-products.

  \begin{theorem}
    Let \(X_\alpha\) be a uniformizable space for \(\alpha<\kappa\)
    and \(z\in \prod_{\alpha<\kappa}X_\alpha\).
    If
    \(\pl D\tactwin\bellConGame{X_\alpha}\) for all \(\alpha<\kappa\),
    then
    \(\pl D\tactwin\bellConGame{\sigmaProd_{\alpha<\kappa}^z X_\alpha}\).
  \end{theorem}

  \begin{proof}
    For \(x\in\sigmaProd_{\alpha<\kappa}^z X_\alpha\), let
      \[
        \supp(x)
          =
        \left\{
          \alpha<\kappa
        :
          x(\alpha)\not=z(\alpha)
        \right\}
          \in
        \kappa^{<\omega}
      \]
    and let \(\tau_\alpha\) be a winning strategy for \(\pl D\) in
    \(\bellConGame{X_\alpha}\) for \(\alpha<\kappa\).

    Define a strategy \(\tau\) for \(\pl D\) by
      \[
        \tau(\emptyset)
          =
        \left(\sigmaProd_{\alpha<\kappa}^z X\right)^2
      \]
      \[
        \tau(\<p\>)
          =
        \left(\sigmaProd_{\alpha<\kappa}^z X\right)^2
          \cap
        P(\{
          \<\alpha,\tau_\alpha(p(\alpha))\>
            :
          \alpha\in\supp(p)
        \})
      \]

    Let \(a:\omega\to \sigmaProd_{\alpha<\kappa}^z X\) be a legal
    attack by \(\pl P\) against \(\tau\), so
      \[
        \<a(n+1),a(n+2)\>
          \in
        \bigcap_{m\leq n}
        \tau(\<a(m)\>)
      \]
      \[
          =
        \left(\sigmaProd_{\alpha<\kappa}^z X\right)^2
          \cap
        \bigcap_{m\leq n}
        P(\{
          \<\alpha,\tau_\alpha(a(m)(\alpha))\>
            :
          \alpha\in\supp(a(m))
        \})
      \]
    and let \(\alpha<\kappa\).

    If \(a(m+1)(\alpha)=z(\alpha)\) for all but finite
    \(m<\omega\), then  \(\lim_{n<\omega}a(n)(\alpha)=z(\alpha)\).
    Otherwise, we have \(\alpha\in\supp(a(m))\) for infinitely many
    \(m<\omega\).

    TODO: wrap this up unless it's wrong
    (it's not needed to get the
    main result)
    % If
    % \(\bigcap_{n<\omega}\tau(\<a(n)\>)[a(n+1)]=\emptyset\), then
    % \(\pl D\) has already won.
    % Otherwise there exists a nonstrictly increasing
    % unbounded function \(f\in\omega^\omega\) with
    % \(
    %   \bigcap_{n<\omega}
    %   \sigma_\alpha(a(f(n))(\alpha))[a(n+1)(\alpha)]
    %     \not=
    %   \emptyset
    % \).
    % Since the diameter of these sets approaches \(0\), the intersection is
    % a singleton \(x\); furthermore, the \(\alpha\) coordinate of \(a\) must
    % then converge to \(x\). We conclude that
    % \(\pl D\tactwin\bellConGame{\sigmaProd_{\alpha<\kappa}^z X}\).
  \end{proof}

\subsection{Eberlein and Strong Eberlein Compacts}

  We recall some convenient definitions for a few strengthenings of
  compactness.

  \begin{definition}
    A \term{Corson compact} space is a compact space which may be embedded in
    \(\SigmaProdR\kappa\).
  \end{definition}

  \begin{definition}
    An \term{Eberlein compact} space is a compact space which may be embedded
    in \(\SigmaStarProdR\kappa\).
  \end{definition}

  \begin{definition}
    A \term{strong Eberlein compact} space is a compact space which may
    be embedded in \(\sigmaProdTwo\kappa\).
  \end{definition}

  Obviously, strong Eberlein compacts are Eberlein compact, and Eberlein
  compacts are Corson compact. Nyikos observed in
  \cite{MR3288115} that as the \(\Sigma\)-product of proximal spaces
  are proximal and the closed subspaces of proximal spaces are proximal:

  \begin{corollary}
    Corson compacts are proximal.
  \end{corollary}

  By the previous section we may now obtain analogues of this observation.

  \begin{corollary}
    Eberlein compacts are Markov-proximal. Strong Eberlein compacts are
    tactic-proximal.
  \end{corollary}

  The author showed with Gary Gruenhage in \cite{MR3227201} that
  Nyikos's observation can actually be reversed. The result required an
  earlier game characterization of Corson compactness due to Gruenhage.

  \begin{game}
    Let \(\gruConGame{X}{S}\) denote
    \term{Gruenhage's convergence game} with
    players \(\pl O\), \(\pl P\) which proceeds as follows for a space
    \(X\) and a set \(S\). In round \(n\),
    \(\pl O\) chooses an open set \(U_n\subseteq X^2\) containing
    \(S\), followed by \(\pl P\)
    choosing a point \(p_n\in \bigcap_{m\leq n}U_m\).

    \(\pl O\) wins in the case that
    \(\<p_0,p_1,\dots\>\) converges to the set \(S\); that is,
    any open set containing \(S\) contains all but finite \(p_n\).
    \(\pl P\) wins otherwise.
  \end{game}

  When \(S=\{x\}\), we abuse notation and write simply
  \(\gruConGame{X}{x}\).

  \begin{corollary}[\cite{MR752278},\cite{MR858337}]
    A compact space \(X\) is Corson compact if and only if
    \(\pl O\win\gruConGame{X^2}{\Delta}\).
    A compact space \(X\) is Eberlein compact if and only if
    \(\pl O\markwin\gruConGame{X^2}{\Delta}\).
  \end{corollary}

  In particular note that Corson compactness is characterized by both
  \(\pl O\win\gruConGame{X^2}{\Delta}\) and
  \(\pl D\win\bellAbsConGame{X}\). We will see that these games are not
  equivalent with respect to limited information strategies, but the following
  results are useful.

  \begin{theorem}
    Let \(k<\omega\). For all \(x\in X\):
    \begin{itemize}
      \item
        \(
          \pl D\ktactwin{2k} \bellConGame{X}
            \Rightarrow
          \pl O \ktactwin{k} \gruConGame{X}{x}
        \)
      \item
        \(
          \pl D\kmarkwin{2k} \bellConGame{X}
            \Rightarrow
          \pl O \kmarkwin{k} \gruConGame{X}{x}
        \)
      \item
        \(
          \pl D\win \bellConGame{X}
            \Rightarrow
          \pl O \win \gruConGame{X}{x}
        \)
    \end{itemize}
  \end{theorem}

  \begin{proof}
    The perfect-information result was proven in Bell's
    original paper \cite{MR3239205}. We proceed by proving the
    Markov-information result, and note that the tactical-information
    result may be proven by simply dropping usage of
    the round number from the given proof.

    Let \(\sigma\) be a winning \(2k\)-mark for \(\pl D\) in
    \(\bellConGame{X}\); without loss of generality, assume that
    \(\sigma(t,n)\subseteq\sigma(s,m)\) whenever \(n\geq m\) and
    \(s\) is a permutation of a subsequence of \(t\).
    We define the \(k\)-mark \(\tau\) for \(\pl O\) in
    \(\gruConGame{X}{x}\) such that
      \[
        \tau(t,n)
          =
        \sigma\Big(\<t(0),\dots,x,t(|t|-1),x\>,2n+1\Big)[x]
      \]

    Let \(p\) be a legal attack against \(\tau\).
    Consider the attack \(q\) against the winning \(2k\)-mark
    \(\sigma\) such that \(q(2n)=x\) and \(q(2n+1)=p(n)\).
    Let \(D_n\) be the entourage played according to \(\sigma\)
    in round \(n\) versus \(q\), and \(E_n=\bigcap_{m\leq n}D_m\).

    TODO: finish

    % Certainly, $x\in E_{2n}[x]= E_{2n}[q(2n)]$ for any $n<\omega$.
    % Note also for any $n<\omega$ that
    %     \[
    %       p(n) \in
    %       \bigcap_{m\leq n}\tau\circ {L}_k(p\rest m)
    %     \]
    %     \[
    %       =
    %       \bigcap_{m\leq n}\left(
    %         \sigma\circ {L}_{2k}(\<x,p(0),\dots,x,p(m-1)\>)[x]\cap
    %         \sigma\circ {L}_{2k}(\<x,p(0),\dots,x,p(m-1),x\>)[x]
    %       \right)
    %     \]
    %     \[
    %       =
    %       \bigcap_{m\leq n}\left(
    %         D_{2m}[x]\cap
    %         D_{2m+1}[x]
    %       \right) =
    %       \bigcap_{m\leq 2n+1} D_m[x]=E_{2n+1}[x]
    %     \]
    % so by the symmetry of $E_{2n+1}$, $x\in E_{2n+1}[p(n)]= E_{2n+1}[q(2n+1)]$.
    % Also, $q(2n+1)=p(n)\in E_{2n}[x]=E_{2n}[q(2n)]$ and
    % $p(n)\in E_{2n+1}[x] \Rightarrow q(2n+2)=x\in E_{2n+1}[p(n)]=E_{2n+1}[q(2n+1)]$,
    % making $q$ a legal attack.

    % Then as $x\in \bigcap_{n<\omega} E_n[q(n)]\not=\emptyset$, and $\sigma$ is
    % a winning strategy, the attack $q$ converges. Since $q(2n)=x$, $q$ must
    % converge to $x$. Thus its subsequence $p$ converges to $x$, and $\tau\circ L_k$
    % is a winning strategy for $\pl O$ in $\gruConGame{X}{x}$.
  \end{proof}

  The presence of a Cantor set determines the success of \(\pl D\)'s
  tactial strategies in \(\bellConGame{X}\).

  \begin{lemma}
    Tactic-proximal spaces cannot contain a copy of the Cantor set.
  \end{lemma}

  \begin{proof}
    The result will follow once we show
    \(\pl D\nottactwin\bellAbsConGame{2^\omega}\).
    Let \(\sigma\) be a tactic for \(\pl D\) in \(\bellAbsConGame{2^\omega}\)
    and let \(D_k=\{\<f,g\>:f\rest k = g\rest k\}\). Since \(\{D_k:k<\omega\}\)
    is a base for the universal uniformity on \(2^\omega\)
    (namely, all open neighborhoods of the diagonal),
    we may fix \(k(f)<\omega\)
    for each \(f\in2^\omega\) such that \(D_{k(f)}\subseteq\sigma(\<f\>)\).

    Then there exists \(k<\omega\) such that \(\{f:k(f)=k\}\) is uncountable,
    and therefore there exist distinct \(f,g\in2^\omega\)
    such that \(k=k(f)=k(g)\) and
    \(f\rest k=g\rest k\). Then \(p:\omega\to2^\omega\) defined by
    \(p(2n)=f\) and \(p(2n+1)=g\) is an attack against \(\sigma\) which
    obviously doesn't converge. This attack is legal since
    \(f\in D_k[g]\subseteq\sigma(\<g\>)[g]\) and
    \(g\in D_k[f]\subseteq\sigma(\<f\>)[f]\), so \(\sigma\) is not a winning
    tactic.
  \end{proof}

  \begin{lemma}
    Every non-scattered Corson compact space contains a homeomorphic
    copy of the Cantor set.
  \end{lemma}

  \begin{proof}
    Every non-scattered space contains a closed subspace without
    isolated points. Let \(X\) be such a subspace, and assume that this
    Corson compact is embedded in \(\SigmaProdR\kappa\). Let
    \(B_{\alpha,\epsilon}(x)=\{y: d(x(\alpha),y(\alpha))<\epsilon\}\).
    For each \(x\in X\) and \(n<\omega\), let \(\beta(x,n)<\kappa\) be defined
    such that
    \(\supp(x)=\{\beta(x,n):n<\omega\}\).

    Choose an arbitrary \(x_\emptyset\in X\) and \(\epsilon_0>0\), and
    and let \(A_0=\emptyset\).

    Suppose then that for some \(n<\omega\),
    \(x_s\in X\) is defined for all \(s\in 2^n\),
    and \(\epsilon_n>0\) and \(A_n\in[\kappa]^{<\omega}\) are defined.
    Since each \(x_s\) is not isolated in \(X\), let \(U_s\) be the open
    set
      \[
        U_s
          =
        X
          \cap
        \bigcap_{\alpha\in A_{|s|}} B_{\alpha,\epsilon_{|s|}}(x_s)
      \]
    and choose \(x_{s\concat\<0\>},x_{s\concat\<1\>}\in U_s\) distinct.
    Then let \(\alpha_s<\kappa\) such that
    \(x_{s\concat\<0\>}(\alpha_s)\not=x_{s\concat\<1\>}(\alpha_s)\).
    Let
      \[
        A_{n+1}
          =
        \{\alpha_s:s\in2^{\leq n}\}
          \cup
        \{\beta(x_s,i):s\in2^{\leq n},i\leq n\}
      \]

    Then choose \(0<\epsilon_{n+1}<\frac{1}{2}\epsilon_n\) such that
    \[
      B_{\alpha_s,\epsilon_{n+1}}(x_s\concat\<0\>)
        \cap
      B_{\alpha_s,\epsilon_{n+1}}(x_s\concat\<1\>)
        =
      \emptyset
    \]
    and
    \[
      \overline{
        \bigcap_{\alpha\in A_{n+1}}
        B_{\alpha,\epsilon_{n+1}}(x_s\concat\<0\>)
      }
        \cup
      \overline{
        \bigcap_{\alpha\in A_{n+1}}
        B_{\alpha,\epsilon_{n+1}}(x_s\concat\<1\>)
      }
        \subseteq
      \bigcap_{\alpha\in A_n} B_{\alpha,\epsilon_n}(x_s)
    \]
    for all \(s\in 2^n\).

    Let \(x_f=\lim_{n<\omega} x_{f\rest n}\in X\)
    for each \(f\in 2^\omega\). We claim \(C=\{x_f:f\in 2^\omega\}\)
    is a copy of the Cantor set. This will follow if we can show that
    \(\{U_s:s\in 2^{<\omega}\}\) is a base for \(C\), since it has
    the structure of the Cantor tree.

    Consider \(x_f\) for some \(f\in 2^\omega\), and a subbasic open ball
    \(B_{\alpha,\epsilon}(x_f)\). Observe that
    \(x_f\in\bigcap_{n<\omega} U_{f\rest n}\) since
    \(x_{f\rest n}\in U_{f\rest m}\) for all \(m<n<\omega\).

    If \(\alpha\in\{\beta(x_s,n):s\in2^{<\omega},n<\omega\}\), choose
    \(k<\omega\) with \(\alpha\in A_k\). Then choose \(l<\omega\) such that
    \(\epsilon_l<\epsilon\). Then
    \(U_{f\rest(l+k)}\subseteq B_{\alpha,\epsilon}(x_f)\).

    Otherwise, \(x_s(\alpha)=0\) for all \(s\in2^{<\omega}\), so
    \(x_g(\alpha)=0\) for all \(g\in2^\omega\) and therefore
    \(C\subseteq B_{\alpha,\epsilon}(x_f)\).
  \end{proof}

  A new game characterization of strong Eberlein compactness follows
  from the above and an earlier characterization by Gruenhage.

  \begin{theorem}
    For compact spaces \(X\),
    \(X\) is strong Eberlein compact if and only if
    \(X\) is scattered and \(\pl O\win\gruConGame{X}{x}\) for all \(x\in X\).
  \end{theorem}

\newpage
\bibliographystyle{plain}
\bibliography{../../bibliography}

% A common generalization of metric spaces is the idea of a \term{uniform space}.  A uniform space is a determined by a collection of supersets of the diagonal in the square (called \term{entourages})  satisfying certain conditions.  A uniformity induces in a natural way a topology on $X$, called the \term{uniform topology}.  A topological space $X$ is \term{uniformizable} if there is  a uniformity on $X$ which generates its topology.   Any completely regular space is uniformizable.  See the next section for more complete definitions of these concepts and some of their basic properties.

% Jocelyn Bell introduced the concept of proximal spaces in her doctoral dissertation while working on some uniform box product problems due to her advisor Scott Williams.  Proximal spaces are defined to be the spaces $X$ for which there is a compatible uniformity on $X$ such that the entourage picker has a winning strategy in a certain $\omega$-length game. Every metric space is easily seen to be proximal. Bell has shown that proximal spaces are collectionwise normal, countably paracompact, and have strong preservation properties, particularly, closed subpaces and  $\Sigma$-products of proximal spaces are  proximal \cite{b}.  The power of these results is illustrated by the fact that the difficult result, due independently to M.E. Rudin \cite{ru} and S.P. Gul'ko \cite{gu}, that $\Sigma$-products of metrizable spaces are normal, follows as an immediate corollary.

% In \cite{nproximal}, Peter Nyikos observed that Bell's results imply that compact subspaces of the $\Sigma$-product of real lines, known as Corson compacts, must be proximal. He asked the natural question as to whether any proximal compact must then be Corson compact. Using a characterization of Corson compact due to the second author in \cite{gcovering}, we  answer that question in the affirmative.


% In this paper, all topological spaces are assumed to be completely regular.

% \section{Definitions and Properites of Uniform Spaces}
%  We relate some definitions and properties of uniform spaces.

% \begin{definition}
%   A \term{uniform space} is a pair $\<X,\mc{D}\>$ where $X$ is a set, and $\mc{D}$ is a uniformity. A \term{uniformity} is a filter on subsets of $X^2$, called \term{entourages}, such that for each entourage $D\in \mc D$:
%   \begin{itemize}
%       \item $D$ is reflexive, i.e., the diagonal $\Delta=\{\<x,x\>:x\in X\}\subseteq D$.
%       \item Its inverse $D^{-1}=\{\<y,x\>:\<x,y\>\in D\}\in \mc D$.
%       \item There exists $\frac{1}{2}D\in\mc D$ such that
%         \[
%           2\left(\frac{1}{2}D\right)=\frac{1}{2}D\circ \frac{1}{2}D=\left\{\<x,z\>:\exists y\left(\<x,y\>,\<y,z\>\in \frac{1}{2}D\right)\right\}\subseteq D.
%         \]
%     \end{itemize}
% \end{definition}

% \begin{definition}
%   The \term{uniform topology} induced by a uniformity declares a set $U$ to be open if for every $x\in U$, there is some $D\in\mc D$ with $x\in D[x]=\{y: \<x,y\>\in D\}\subseteq U$.   A topological space $X$ is said to be \term{uniformizable} if there is a uniformity $D$ on $X$ which induces its topology.
% \end{definition}

% We now list some basic results about uniformities and uniform spaces.  One may see \cite{e}, for example, for proofs.



% \begin{itemize}
%  \item Every uniform topology is completely regular, and every completely regular space is uniformizable.
% \item  For every entourage $D$, there is an open symmetric entourage $E\subseteq D$. That is, $\<x,y\>\in E \Leftrightarrow \<y,x\> \in E$, and $E$ is open in $X^2$ with the usual product topology induced by the uniform topology on $X$.
% \item  If $D$ is an open entourage, then for all $x\in X$, $D[x]$ is an open neighborhood of $x$.
% \end{itemize}

% We will make frequent use of $\frac{1}{2}D$ in this paper.  Of course, $\frac{1}{2}D$ is not unique, but for convenience we will assume that for each $D\in \mathcal D$, a unique $\frac{1}{2}D$ has been chosen, which  by the second item above may be assumed to be open and symmetric.  Then we define $\frac{1}{4}D=\frac{1}{2}(\frac{1}{2}D)$,
% $\frac{1}{8}D=\frac{1}{2}(\frac{1}{4}D)$, and so on.

% We quickly note two properites of $\frac{1}{2}D$.

% \begin{proposition}
%   If $x\in \frac{1}{2}D[y]$ and $y\in \frac{1}{2}D[z]$, then $x\in D[z]$.
% \end{proposition}

% \begin{proof}
%   Directly from the definition of $\frac{1}{2}D$. Note that the same result holds if we assumed instead that $y\in \frac{1}{2}D[x]$ or $z\in \frac{1}{2}D[y]$, since $\frac{1}{2}D$ is assumed to be symmetric.
% \end{proof}

% \begin{proposition}
%   If $X$ is a uniform space, then for all $x\in X$ and symmetric entourages $D$:
%     \[
%       \frac{1}{2}D[x]\subseteq \cl{\frac{1}{2}D[x]}\subseteq D[x]
%     \]
% \end{proposition}

% \begin{proof}
%   If $y$ is a limit point of $\frac{1}{2}D[x]$, then $\frac{1}{2}D[y]$ must intersect $\frac{1}{2}D[x]$ at some $z$. It follows then that $y\in D[x]$.
% \end{proof}

% \subsection{The Proximal Game}

% The theory of proximal uniform spaces relies on the following $\omega$-length
% game introduced in \cite{b}:

% \begin{definition}
%   The \textbf{proximal game} $\proxgame{X}$ of length $\omega$ played on a uniform space $X$ with two players $\pl D$, $\pl P$ proceeds as follows:
%     \begin{itemize}
%       \item In the initial round $0$, $\pl D$ chooses an open symmetric entourage $D_0$, followed by $\pl P$ choosing a point $p_0\in X$.
%       \item In round $n+1$, $\pl D$ chooses an open symmetric entourage $D_{n+1}\subseteq D_n$, followed by $\pl P$ choosing a point $p_{n+1}\in D_n[p_n]$.
%     \end{itemize}
%   At the conclusion of the game, $\pl D$ wins if either $\bigcap_{n<\omega}D_n[p_n]=\emptyset$ or $\<p_0,p_1,\dots\>$ converges, and $\pl P$ wins otherwise.
%   \footnote{It's worth noting that a widely distributed preprint of Bell's
%   paper considered $4D_n$ in place of $D_n$ for the winning
%   condition; however, it's easily verified that player $\pl D$
%   has a winning strategy in one if and only if she has a winning strategy in
%   the other.}
% \end{definition}

% We may assume that if $\sigma$ is a winning strategy for $\pl D$, then $D_{n+1}=\sigma (x_0,x_1,...,x_n)$ is contained in something smaller than $D_n$, e.g., $\frac{1}{4}D_n$.  That is because a sequence of legal moves by $\pl P$ with $\pl D$ using the refined strategy is also a legal sequence of moves with $\pl D$ using the original strategy.

% \begin{definition}
%   If a player $\pl A$ has a winning strategy in a game $G$, we write $\pl A\win G$. Otherwise, we write $\pl A\not\win G$.
% \end{definition}

% \begin{definition}
%   A uniform space $\<X,\mathcal D\>$ is said to be \term{proximal} if and only if $\pl D\win\proxgame{X}$.  A topological space is \term{proximal} iff it admits  a compatible uniformity (i.e., one which induces its topology) which is proximal.
% \end{definition}

% \section{Proximal Games and $W$ Games}

% The second author introduced the following game in \cite{ginfinite}.

% \begin{definition}
%   The \textbf{$W$-convergence game} $\congame{X}{H}$ of length $\omega$ played on a topological space $X$ and set $H\subseteq X$ with two players $\pl O$, $\pl P$ proceeds as follows:
%     \begin{itemize}
%       \item In the initial round $0$, $\pl O$ chooses an open neighborhood $O_0$ of $H$, followed by $\pl P$ choosing a point $p_0\in O_0$.
%       \item In round $n+1$, $\pl O$ chooses an open neigborhood $O_{n+1}\subseteq O_n$ of $H$, followed by $\pl P$ choosing a point $p_{n+1}\in O_{n+1}$.
%     \end{itemize}
%   At the conclusion of the game, $\pl O$ wins if $\<p_0,p_1,\dots\>$ converges to $H$ (every open neighborhood of $H$ contains all but finitely many points of the sequence), and $\pl P$ wins otherwise.

%   In the case of $H=\{x\}$, we abuse notation and write $\congame{X}{x}$.
% \end{definition}

% \begin{definition}
%   A topological space $X$ is said to be a \term{$W$-space} if and only if for every $x\in X$, $\pl O\win \congame{X}{x}$.
% \end{definition}

% We find it useful to consider a (seemingly) weaker version of this game.

% \begin{definition}
%   The \textbf{$W$-clustering game} $\clusgame{X}{H}$ proceeds identically to $\congame{X}{H}$, except that $\pl O$ need only force $p$ to cluster at $H$ (every open neighborhood of $H$ contains infinitely many points of the sequence).
% \end{definition}

% \begin{theorem}
%   $\pl O\win\congame{X}{H}$ if and only if $\pl O \win\clusgame{X}{H}$
% \end{theorem}

% \begin{proof}
%   Shown for $H=\{x\}$ in \cite{ginfinite}, but the analogous proof works for arbitrary $H$. The forward implication is immediate. Let $\sigma$ be a strategy for $\pl O$ witnessing $\pl O\win \clusgame{X}{H}$. Note $\sigma$ is a function whose domain is all possible partial attacks by $\pl P$ (finite sequences of points in $X$), and whose range is the open neighborhoods of $H$, such that for any legal attack $p=\<p_0,p_1,\dots\>$ against $\sigma$, it follows that $p$ must cluster at $H$.

%   For every partial attack $a$, let $S(a)$ contain all subsequences of $a$. We then define $\tau(p\rest n)=\bigcap_{b\in S(p\rest n)}\sigma(b)$. If $p$ attacks the strategy $\tau$ for $\congame{X}{H}$, then it also is a legal attack against $\sigma$ in $\clusgame{X}{H}$, so $p$ clusters at $H$.

%   But not only that: let $q$ be a subsequence of $p$. If $q\rest n$ is a legal partial attack against $\sigma$ and $q\rest n$ is a subsequence of $p\rest m$ with $q(n)=p(m)$, then since $q(n)=p(m)\in\tau(p\rest m)\subseteq\sigma(q\rest n)$, $q\rest n+1$ is a legal partial attack against $\sigma$ as well. Thus $q$ is a legal attack against $\sigma$ in $\clusgame{X}{H}$, so $q$ clusters at $H$.

%   Since no infinite subsequence of $p$ can be completely missed by an open neighborhood of $H$, it follows that $p$ converges to $H$.
% \end{proof}

% \begin{theorem}\cite{b}
%   All proximal spaces are $W$-spaces.
% \end{theorem}


% The idea of Bell's proof of the above result is to consider the result of an attack on the winning strategy for $\pl D$ in $\proxgame{X}$ which alternates between points chosen by $\pl P$ in $\congame{X}{x}$ and the particular point $x$ itself. Since the intersection $\bigcap_{n<\omega} D_n[p_n]$ must contain $x$, the sequence must converge, and since the sequence contains $x$ infinitely often, it must converge to $x$.


% While there is not much trouble in ensuring the nonempty intersection of\\ $\bigcap_{n<\omega} D_n[p_n]$ for this particular proof, we turn to a stronger version of $\proxgame{X}$, also introduced by Bell, which avoids the issue entirely.

% \begin{definition}
%   The \term{absolutely proximal game}  proceeds identially to $\proxgame{X}$, except that $\pl D$ may only win in the case that $\<p_0,p_1,\dots\>$ converges.
% \end{definition}

% Obviously, all absolutely proximal spaces are proximal. We are interested in when we have equivalence.

% \begin{definition}
%   A uniform space is \term{uniformly locally compact} if there exists an open symmetric entourage $L$ such that $\cl{L[x]}$ is a compact neighborhood of $x$ for all $x\in X$.  A topological space is uniformly locally compact if it admits a compatible uniformly locally compact uniformity.
% \end{definition}


% Obviously every compact space is uniformly locally compact, but not every locally compact space is uniformly locally compact.  For example, the space of countable ordinals is locally compact, but does not admit a compatible uniformly locally compact uniformity.

% \begin{theorem}
%   A uniformly locally compact space $X$ is proximal if and only if it is absolutely proximal.
% \end{theorem}

% \begin{proof}
%   Let $L$ be a uniformly locally compact entourage. Let $\sigma$ be a strategy for $\pl D$ witnessing $\pl D\win \proxgame{X}$. Without loss of generality, we may assume such that $\sigma(p\rest n)\subseteq L$ for all partial attacks $p\rest n$ (so $\cl{\sigma(p\rest n)[x]}\subseteq\cl{L[x]}$ is compact), and that $n > m$ implies $\sigma(p\rest n)\subseteq \frac{1}{4}\sigma(p\rest m)$.

%   Let $\tau(p\rest n)=\frac{1}{2}\sigma(p\rest n)$. If $p$ attacks $\tau$, then
%     \[
%       p(n+1)
%         \in
%       \tau(p\rest n)[p(n)]
%         =
%       \frac{1}{2}\sigma(p\rest n)[p(n)]
%     \]

%     and for

%     \[
%       x
%         \in
%       \cl{\sigma(p\rest (n+1))[p(n+1)]}
%         \subseteq
%       \cl{\frac{1}{4}\sigma(p\rest n)[p(n+1)]}
%         \subseteq
%       \frac{1}{2}\sigma(p\rest n)[p(n+1)]
%     \]

%   we can conclude $x\in\sigma(p\rest n)[p(n)]$. Thus

%     \[
%       \sigma(p\rest (n+1))[p(n+1)]
%         \subseteq
%       \cl{\sigma(p\rest (n+1))[p(n+1)]}
%         \subseteq
%       \sigma(p\rest n)[p(n)]
%     \]

%   Finally, note that since $\tau$ yields subsets of $\sigma$, then $p$ attacks the winning strategy $\sigma$ in $\proxgame{X}$, but since the intersection of a descending chain of nonempty compact sets is nonempty, we have

%     \[
%       \bigcap_{n<\omega} \sigma(p\rest n)[p(n)]
%         =
%       \bigcap_{n<\omega} \cl{\sigma(p\rest n)[p(n)]}
%         \not=
%       \emptyset.
%     \]

%   We conclude that $p$ converges.
% \end{proof}


% \section{Corson Compacts and Proximal Compacts}

% We recall  the definition of Corson compact.

% \begin{definition}
%   A space is said to be \term{Corson compact} if and only if it is homeomorphic to a compact set within the $\Sigma$-product $\Sigma\mathbb{R}^\kappa$ of $\kappa$-many real lines, that is:
%     \[
%       \Sigma\mathbb{R}^\kappa
%         =
%       \{x\in \mathbb{R}^\kappa: |\{\alpha:x(\alpha)\not=0\}|\leq\omega\}
%     \]
% \end{definition}


% Since proximal spaces are closed under closed subsets and $\Sigma$-products \cite{b}, it follows (as noted by Nyikos) that every Corson compact is proximal.







% However, the given characterization of Corson compact is less useful when proving the other direction.  Instead we use the following game characterization due to the second author.

% \begin{theorem}\cite{gcovering}
%   A space $X$ is Corson compact if and only if $X$ is compact and $\pl O\win\congame{X^2}{\Delta}$.
% \end{theorem}


% The following contains the meat of our proof of the title result.

% \begin{theorem}
%   For any absolutely proximal space $X$, $\pl O\win \congame{X}{H}$ for all compact $H\subseteq X$.
% \end{theorem}

% \begin{proof}
%   Let $\sigma$ be a winning strategy for $\pl D$ in the absolutely proximal
%   game such that $p\supsetneq q$ implies $\sigma(p)\subseteq \frac{1}{4}\sigma(q)$.
%   For any sequence $t=\<t_0,t_1,\dots\>$, let $o(t)=\<t_1,t_3,\dots\>$
%   be the subsequence of $t$
%   consisting of its odd-indexed terms. We proceed by constructing a winning
%   strategy for $\pl O$ in $\clusgame{X}{H}$. Since $\pl O\win\clusgame{X}{H}$
%   if and only if $\pl O\win\congame{X}{H}$, the result will follow.

%   \bigskip

% First we define a tree $T(\emptyset)$. Our aim is to define an open set
%   \[
%     \bigcup_{i,j<m_\emptyset}
%       \frac{1}{4}\sigma(\<h_{\emptyset,i}\>)[h_{\emptyset,i,j}]
%     =
%     \bigcup_{\<i,h_{\emptyset,i},j\>\in\max(T(\emptyset))}
%       \frac{1}{4}\sigma(o(\emptyset)\concat\<h_{\emptyset,i}\>)[h_{\emptyset,i,j}]
%   \]
% which contains $H$, and $\pl O$ will use this as the inital move in her winning
% strategy for $\clusgame{X}{H}$.

%   \begin{itemize}
%     \item Choose $m_\emptyset<\omega$, $h_{\emptyset,i}\in H$ for $i<m_\emptyset$, and $h_{\emptyset,i,j}\in H\cap\cl{\frac{1}{4}\sigma(\emptyset)[h_{\emptyset,i}]}$ for $i,j<m_\emptyset$ such that
%       \[
%         \left\{\frac{1}{4}\sigma(\emptyset)[h_{\emptyset,i}]:i<m_\emptyset\right\}
%       \]
%     is a cover for $H$ and such that for each $i<m_\emptyset$
%       \[
%         \left\{\frac{1}{4}\sigma(\<h_{\emptyset,i}\>)[h_{\emptyset,i,j}]:j<m_\emptyset\right\}
%       \]
%     is a cover for $H\cap\cl{\frac{1}{4}\sigma(\emptyset)[h_{\emptyset,i}]}$.

%     (The indexings $i\mapsto h_{\emptyset,i}$ and
%     $\<i,j\>\mapsto h_{\emptyset,i, j}$ need not be
%     one-to-one; repetition of points is allowed.)
%     \item Let $\<i,h_{\emptyset,i},j\>$ and its initial segments be in
%     $T(\emptyset)$ for $i,j<m_\emptyset$.
%   \end{itemize}



% \bigskip

% Now suppose that $a$ is a finite
% sequence of moves by $\pl P$ in $\clusgame{X}{H}$ for which $\pl O$ replied with
%   \[
%     \bigcup_{s\concat\<i,h_{s,i},j\>\in\max(T(a))}
%       \frac{1}{4}\sigma(o(s)\concat\<h_{s,i}\>)[h_{s,i,j}]
%   \]
% as a part of her winning strategy for some tree $T(a)\supseteq T(\emptyset)$.
% As we will soon guarantee, any nonempty $s$ will be of the form
% $\<i_0,h_0,j_0,x_0,\dots\>$ so that
% $o(s)\concat\<h_{s,i}\> = \<h_0,x_0,\dots,h_{s,i}\>\in X^{<\omega}$
% corresponds to a partial attack against $\sigma$ by the proximal game's $\pl P$.
% We then define $T(a\concat\<x\>)$ for each of the legal responses
%   \[
%     x\in
%       \bigcup_{s\concat\<i,h_{s,i},j\>\in\max(T(a))}
%       \frac{1}{4}\sigma(o(s)\concat\<h_{s,i}\>)[h_{s,i,j}]
%   \]
% by the clustering game's $\pl P$ as follows:

%   \begin{itemize}
%     \item Fix $s\concat\<i,h_{s,i},j\>\in\max(T(a))$ such that
%       $x\in \frac{1}{4}\sigma(o(s)\concat\<h_{s,i}\>)[h_{s,i,j}]$.
%       Then $s\concat\<i,h_{s,i},j\>$ is the only node of $T(a)$ that will be
%       extended in this step. Let $t=s\concat\<i,h_{s,i},j,x\>$.
%     \item Note that, assuming $o(s)\concat\<h_{s,i}\>$ is a legal partial attack against $\sigma$, then
%       \[
%         x
%           \in
%         \frac{1}{4}\sigma(o(s)\concat\<h_{s,i}\>)[h_{s,i,j}]
%           \subseteq
%         \frac{1}{4}\sigma(o(s))[h_{s,i,j}]
%       \]
%     and
%       \[
%         h_{s,i,j}
%           \in
%         \cl{\frac{1}{4}\sigma(o(s))[h_{s,i}]}
%           \subseteq
%         \frac{1}{2}\sigma(o(s))[h_{s,i}]
%       \]
%     implies
%       \[
%         x
%           \in
%         \sigma(o(s))[h_{s,i}]
%       \]
%     and thus $o(s)\concat\<h_{s,i},x\>=o(t)$ is a legal partial attack against $\sigma$.
%     \item Choose $m_t<\omega$, $h_{t,k}\in H\cap \cl{\frac{1}{4}\sigma(o(s)\concat\<h_{s,i}\>)[h_{s,i,j}]}$ for $k<m_t$, and $h_{t,k,l}\in H\cap\cl{\frac{1}{4}\sigma(o(t))[h_{t,k}]}$ for $k,l<m_t$ such that
%       \[
%         \left\{\frac{1}{4}\sigma(o(t))[h_{t,k}]:k<m_t\right\}
%       \]
%     is a cover for $H\cap \cl{\frac{1}{4}\sigma(o(s)\concat\<h_{s,i}\>)[h_{s,i,j}]}$ and such that for each $k<m_t$
%       \[
%         \left\{\frac{1}{4}\sigma(o(t)\concat\<h_{t,k}\>)[h_{t,k,l}]:l<m_t\right\}
%       \]
%     is a cover for $H\cap\cl{\frac{1}{4}\sigma(o(t))[h_{t,k}]}$.
%     \item Note that, assuming $o(t)$ is a legal partial attack against $\sigma$, then
%       \[
%         h_{t,k}
%           \in
%         \cl{\frac{1}{4}\sigma(o(s)\concat\<h_{s,i}\>)[h_{s,i,j}]}
%           \subseteq
%         \frac{1}{2}\sigma(o(s)\concat\<h_{s,i}\>)[h_{s,i,j}]
%       \]
%     and
%       \[
%         x
%           \in
%         \frac{1}{4}\sigma(o(s)\concat\<h_{s,i}\>)[h_{s,i,j}]
%       \]
%     implies
%       \[
%         h_{t,k}
%           \in
%         \sigma(o(s)\concat\<h_{s,i}\>)[x]
%       \]
%     and thus $o(t)\concat\<h_{t,k}\>$ is a legal partial attack against $\sigma$.

%     \item Let $T(a\concat\<x\>)\supseteq T(a)$, and for each $k,l<m_t$,
%     also let $t\concat\<k,h_{t,k},l\>$ and all of its initial segments be in
%     $T(a\concat\<x\>)$.
%   \end{itemize}

%     \bigskip

%  This completes the construction of  $ T(a\concat\<x\>)$. Note that assuming
%     \[
%       \left\{\frac{1}{4}\sigma(o(s)\concat\<h_{s,i}\>)[h_{s,i,j}] : s\concat\<i,h_{s,i},j\>\in\max(T(a))\right\}
%     \]
%   covers $H$, then since
%     \[
%       \left\{\frac{1}{4}\sigma(o(t)\concat\<h_{t,k}\>)[h_{t,k,l}] : s\concat\<i,h_{s,i},j,x,k,h_{t,k},l\>\in\max(T(a\concat\<x\>))\setminus\max(T(a))\right\}
%     \]
%   covers $H\cap \frac{1}{4}\sigma(o(s)\concat\<h_{s,i}\>)[h_{s,i,j}]$, we have that
%     \[
%       \left\{\frac{1}{4}\sigma(o(t)\concat\<h_{t,k}\>)[h_{t,k,l}] : t\concat\<k,h_{t,k},l\>\in\max(T(a\concat\<x\>))\right\}
%     \]
%   covers $H$.

%   Hence we may define the strategy $\tau$ for $\pl O$ in $\clusgame{X}{H}$ such that:
%   \[
%     \tau(p\rest n) = \bigcup_{s\concat\<i,h_{s,i},j\>\in\max(T(p\rest n))} \frac{1}{4}\sigma(o(s)\concat\<h_{s,i}\>)[h_{s,i,j}]
%   \]

%   If $p$ is an attack by $\pl P$ against $\tau$, then it follows that $T(p\rest n)$ is defined for all $n<\omega$, so let $T(p)=\bigcup_{n<\omega} T(p\rest n)$. We note $T(p)$ is an infinite tree with finite levels:
%     \begin{itemize}
%       \item $\emptyset$ has exactly $m_\emptyset$ successors $\<i\>$.
%       \item $s\concat\<i\>$ has exactly one successor $s\concat\<i,h_{s,i}\>$
%       \item $s\concat\<i,h_{s,i}\>$ has exactly $m_s$ successors $s\concat\<i,h_{s,i},j\>$
%       \item $s\concat\<i,h_{s,i},j\>$ has either no successors or exactly one successor $s\concat\<i,h_{s,i},j,x\>$
%       \item $t=s\concat\<i,h_{s,i},j,x\>$ has exactly $m_t$ successors $t\concat\<k\>$
%     \end{itemize}

%   Hence $T(p)$ has an infinite branch $q'=\<i_0,h_0,j_0,x_0,i_1,h_1,j_1,x_1,\dots\>$.   Let $q=o(q')=\<h_0,x_0,h_1,x_1,\dots\>$. Note that by the construction of $T(p)$, $q$ is an attack on the winning strategy $\sigma$ in the absolutely proximal game, so it must converge. Since every other term of $q$ is in $H$, it must converge to $H$. Then since $o(q)$ is a subsequence of $p$, $p$ must cluster at $H$.
% \end{proof}

% The equivalency result follows as a quick corollary.

% \begin{corollary}
%   For any compact space $X$, $X$ is proximal if and only if $X$ is Corson compact.
% \end{corollary}

% \begin{proof}
%   The reverse implication was noted earlier. If $X$ is compact proximal, then so is $X^2$. By the previous theorem, $\pl O\win\congame{X^2}{\Delta}$, which is a characterization of Corson compact.
% \end{proof}




% \begin{thebibliography}{99}
% \bibitem{b}
%  J. Bell,
%   \emph{An infinite game with topological consequences}, preprint.

% \bibitem{e}
%   R. Engelking,
%   \emph{General topology}, Revised and completed edition, Heldermann Verlag, Berlin, 1989.




%   \bibitem{ginfinite}
%  G. Gruenhage,
%   \emph{Infinite games and generalizations of first-countable spaces}, Gen. Top. Appl. 6(1976), 339-352.

% \bibitem{gcovering}
%   G. Gruenhage,
%   \emph{Covering properties on $X^2\setminus\Delta$, $W$-sets, and compact subsets of $\Sigma$-products},
%   Topology Appl. 17 (1984), no. 3, 287–-304.


% \bibitem{gu} S.P. Gul'ko, \emph{On properties of sets lying in $\Sigma$-products}, Dokl. Akad. Nauk. SSSR 237(1977), 505--508, Soviet Math. Dokl. 18(1977), 1438--1442.

% % \bibitem{k}
% % K. Kunen,
% % \emph{Set theory. An introduction to independence proofs.},
% % Studies in Logic and the Foundations of Mathematics, 102, North-Holland Publishing Co., Amsterdam, 1983.

% \bibitem{nproximal}
%   P.J. Nyikos,
%   \emph{Proximal and semi-proximal spaces}, preprint.

% \bibitem{ru} M.E. Rudin, \emph{The shrinking property}, Canad. Math. Bull. 26(1983), no. 4, 385-388.

% \end{thebibliography}

\end{document}