\documentclass{amsart}
\usepackage{amsmath}
\usepackage{amsthm}
\usepackage{amssymb}

\usepackage{../../clontzDefinitions}

\newtheorem{theorem}{Theorem}[section]
\newtheorem{proposition}[theorem]{Proposition}
\newtheorem{lemma}[theorem]{Lemma}
\newtheorem{corollary}[theorem]{Corollary}


\theoremstyle{definition}
\newtheorem{definition}[theorem]{Definition}
\newtheorem{game}[theorem]{Game}
\newtheorem{example}[theorem]{Example}
\newtheorem{question}[theorem]{Question}

\parskip=0.5em



\begin{document}

% \title{Proximal compact spaces are Corson compact\tnoteref{t1}}
% \tnotetext[t1]{2010 Mathematics Subject Classification. 54E15, 54D30, 54A20.}
\title{Tactic-proximal compact spaces are strong Eberlein compact}

% \author[aub]{S.~Clontz\fnref{fn1}}
% \ead{steven.clontz@auburn.edu}
% \author[aub]{G.~Gruenhage\fnref{fn2}}
% \ead{gruengf@auburn.edu}

% \address[aub]{Department of Mathematics, Auburn University,
%  Auburn, AL 36830}


\author{Steven Clontz}
\address{Department of Mathematics and Statistics, UNC Charlotte,
Charlotte, NC 28262}
\email{steven.clontz@gmail.edu}
\urladdr{stevenclontz.com}

\keywords{Proximal, tactic proximal, Corson compact, strong Eberlein compact, topological game}
% \begin{keyword}
%   Proximal\sep absolutely proximal\sep Corson compact\sep uniform space\sep $W$-space
% \end{keyword}

\subjclass[2010]{54E15, 54D30, 54A20}


\begin{abstract}
  The author and G. Gruenhage previously showed that J. Bell's proximal game
  may be used to characterize Corson compactness in compact Hausdorff spaces.
  Using limited information strategies, the proximal game may also be
  used to characterize the strong Eberlein compactness property.
  In doing so, a purely topological characterization of the proximal game
  is introduced, and several existing results on
  the proximal game are given
  analogues considering limited information strategies.
\end{abstract}


\maketitle

  Two papers published in 2014 introduced the
  \term{proximal uniform space game}
  \(\bellUniGame{X}\) due to Jocelyn Bell. If \(X\) is a
  topological space, and there exists a uniform structure inducing its topology
  which gives the first player in this game has a winning stategy,
  then \(X\) is said to be a \term{proximal} space. Bell used this
  game as a tool in \cite{MR3239205} for investigating uniform box products,
  and the author showed with Gary Gruenhage in \cite{MR3227201} that
  this game characterizes Corson compactness amongst compact Hausdorff spaces,
  answering a question of Peter Nyikos in \cite{nyikosProximalPreprint}.

  All spaces in this paper are assumed to be \(T_{3\frac{1}{2}}\), so that they
  have a uniform structure inducing the topology on the space.
  Unlike many game-theoretic topological properties, the game
  \(\bellUniGame{X}\)
  for which the proximal property was defined by is not itself a topological
  game. However, by considering entourages of the universal uniformity
  inducing the topology of a space, this original uniform space game may be
  trivially modified to the purely topological game \(\bellConGame{X}\).

  The aim of this paper is to use limited information strategies in
  this topological interpretation of the
  proximal game to give a new game-theoretic characterization of the
  strong Eberlein compactness property. Strong Eberlein compacts are Corson
  compacts, and therefore proximal compact spaces; in fact, it will be shown
  that strong Eberlein compacts are exactly the compact spaces for which the
  first player has a \term{tactical} winning strategy for the proximal game,
  a strategy which relies on only the most recent move of the opponent.
  This is the only known purely game-theoretic internal characterization of
  strong Eberlein compactness among compact spaces.


\section{Topologizing \(\bellUniGame{X}\)}

  We refer to \cite{MR3227201} for definitions, notation, and basic theorems on
  uniform spaces and the proximal game \(\bellUniGame{X}\) (denoted as
  \(Prox_{D,P}(X)\) in that paper). In particular recall that:

  \begin{definition}
    \(\pl P \win G\) denotes that the player \(\pl P\) has a winning strategy
    in the \(\omega\)-length game \(G\).
  \end{definition}

  \begin{game}
    Let \(\bellUniGame{X,\mb D}\) denote the
    \term{proximal uniform space game} with
    players \(\pl D\), \(\pl P\) which proceeds as follows for
    a space \(X\) with uniformity \(\mb D\). In round \(0\),
    \(\pl D\) chooses an entourage \(D_0\in\mb D\), followed by \(\pl P\)
    choosing a point \(p_0\in X\). In round \(n+1\),
    \(\pl D\) chooses an entourage
    \(D_{n+1}\in\mb D\),
    followed by \(\pl P\) choosing a point \(p_{n+1}\in D_n[p_n]\).

    \(\pl D\) wins in the case that either
    \(\<p_0,p_1,\dots\>\) converges in \(X\),
    or \(\bigcap_{n<\omega}D_n[p_n] = \emptyset\). \(\pl P\) wins otherwise.
  \end{game}

  \begin{definition}
    A uniformizable space \(X\) is \term{proximal}
    in the case that there exists a compatible uniformity
    \(\mb D\) for \(X\) such that \(\pl D\win\bellUniGame{X,\mb D}\).
  \end{definition}

  The aim of this section is to give an alternate characterization of
  proximal spaces using a slight variation of \(\bellUniGame{X,\mb D}\)
  which is purely topological. While the proof is trivial, the
  following definitions and theorems will be useful in investigating the main
  result.

  \begin{definition}
    The \term{universal uniformity} for a uniformizable topology is the
    union of all uniformities which induce the given topology.
  \end{definition}

  \begin{theorem}[\cite{MR2048350}]
    The universal uniformity is itself a uniformity compatibile with
    its given topology.
  \end{theorem}

  \begin{theorem}[\cite{MR2048350}]\label{halvingEntrouages}
    For every uniformizable space, if \(D\) is a neighborhood of the diagonal
    \(\Delta\)  such that there exist neighborhoods \(D_n\) of \(\Delta\) with
    \(D\supseteq D_0\) and
    \(D_n \supseteq D_{n+1}\circ D_{n+1}\), then \(D\) is an
    entourage of the universal uniformity.
  \end{theorem}

  \begin{definition}
    For every entourage \(D\) and \(n<\omega\), let \(\frac{1}{2^n}D\)
    denote an entourage such that \(\frac{1}{1}D=D\) and
    \(\frac{1}{2^{n+1}}D\circ\frac{1}{2^{n+1}}D\subseteq\frac{1}{2^n}D\).
  \end{definition}

  \begin{definition}
    An \term{open symmetric entourage} \(D\) is a entourage
    which is open in the product topology induced by the uniformity and where
    \(D=D^{-1}=\{\<y,x\>:\<x,y\>\in D\}\).
  \end{definition}

  \begin{proposition}
    For every entourage \(D\), there exists an open symmetric
    entourage \(U\subseteq D\).
  \end{proposition}

  Due to this theorem, we will simply use the word \term{entourage} to refer
  to open symmetric entourages of the universal uniformity for
  a given topological space. Note that if \(D\) is an
  entourage, then \(D[x]=\{y:\<x,y\>\in D\}\) is an open neighborhood of \(x\).
  One may consider
  \(D[x]\) to be an entourage-``ball'' about \(x\), generalizing the notion of
  an \(\epsilon\)-ball given by a metric structure.

  In the case that
  the space is paracompact, entourages are very easily described.

  \begin{theorem}[\cite{MR2048350}]
    Every open symmetric neighborhood of the diagonal is an entourage for
    paracompact uniformizable spaces.
  \end{theorem}

  By replacing the entourages of \(\mb D\) in \(\bellUniGame{X,\mb D}\)
  with the
  (open symmetric) entourages of the universal uniformity, we obtain
  a purely topological game.

  \begin{game}
    Let \(\bellConHardGame{X}\) denote \term{Bell's hard convergence game} with
    players \(\pl D\), \(\pl P\) which proceeds as follows for
    a uniformizable space \(X\). In round \(0\),
    \(\pl D\) chooses an entourage \(D_0\), followed by \(\pl P\)
    choosing a point \(p_0\in X\). In round \(n+1\),
    \(\pl D\) chooses an entourage
    \(D_{n+1}\), followed by \(\pl P\) choosing a point \(p_{n+1}\in D_n[p_n]\).

    \(\pl D\) wins in the case that either
    \(\<p_0,p_1,\dots\>\) converges in \(X\),
    or \(\bigcap_{n<\omega}D_n[p_n] = \emptyset\). \(\pl P\) wins otherwise.
  \end{game}

  To be clear,
  there's practically no difference between \(\bellConHardGame{X}\) and
  \(\bellUniGame{X,\mb D}\) where \(\mb D\) is the universal uniformity;
  any choice by \(\pl D\) in \(\bellUniGame{X,\mb D}\) may be improved to
  an open symmetric choice for \(\pl D\) in either game.

  This game is labeled as ``hard'' due to the requirement that \(\pl D\)
  keep track of the entire history of the game if she wishes to
  ensure that successive moves refine previous moves.
  Put another way, any perfect information strategy \(\sigma\)
  may be improved to a better perfect information strategy \(\tau\) by letting
  \(\tau(s)=\bigcap_{t\leq s}\sigma(t)\) (where \(t\leq s\) means \(t\) is a
  subsequence of \(s\)).

  In the study of limited information strategies, it's preferable to
  consider games for which the opponent must respect all previous moves
  of the player restricted to limited information in the first place.

  \begin{game}
    Let \(\bellConGame{X}\) denote \term{Bell's convergence game} with players
    \(\pl D\), \(\pl P\) which proceeds analogously to \(\bellConHardGame{X}\),
    except for the following. Let \(E_n=\bigcap_{m\leq n}D_n\), where \(D_n\) is
    the entourage played by \(\pl D\) in round \(n\).
    Then \(\pl P\) must ensure that \(p_{n+1}\in E_n[p_n]\),
    and \(\pl D\) wins when either \(\<p_0,p_1,\dots\>\) converges in \(X\)
    or \(\bigcap_{n<\omega}E_n[p_n] = \emptyset\).
  \end{game}

  The proximal property may now be characterized using either game.

  \begin{theorem}\label{topologicalProximalGames}
    \(\pl D\win\bellConHardGame{X}\) if and only if
    \(\pl D\win\bellConGame{X}\) if and only if
    \(X\) is proximal.
  \end{theorem}

  \begin{proof}
    If \(\pl D\win\bellConHardGame{X}\), then we immediately see that
    \(\pl D\win\bellConGame{X}\).
    If \(\sigma\) is a winning strategy for \(\pl D\)
    in \(\bellConGame{X}\), then \(\tau\) defined by
    \(\tau(s)=\bigcap_{t\leq s}\sigma(t)\) is easily seen to be a winning
    strategy for \(\pl D\) in \(\bellConHardGame{X}\).

    If \(\pl D\win\bellConHardGame{X}\), then
    \(\pl D\win\bellUniGame{X,\mb D}\) where \(\mb D\) is
    the universal uniformity, showing \(X\) is proximal.
    Finally, if \(X\) is proximal, then there exists a winning strategy
    \(\sigma\) for \(\bellUniGame{X}\) for a uniformity \(\mb D\) inducing the
    topology on \(X\). Then a winning strategy for \(\pl D\) in
    \(\bellConHardGame{X}\) may be constructed by converting every
    entourage in this uniformity into a smaller open symmetric
    entourage of \(\mb D\), and the result follows as \(\mb D\) is
    contained in the universal uniformity.
  \end{proof}

  The proximal property may now be considered without familiarity of
  the theory of uniform spaces: in either game, \(\pl D\) is simply choosing
  open symmetric neighborhoods of the diagonal with the property in
  Theorem \ref{halvingEntrouages}
  (a property always present when the space is paracompact).

\subsection{Absolutely proximal}

  The secondary winning condition in \(\bellConGame{X}\)
  allows for a space to be incomplete: \(\pl D\win\bellConGame{\mb Q}\) by
  playing \(V_{1/2^n}=\{\<x,y\>:d(x,y)<\frac{1}{2^n}\}\) in each round. This
  forces \(\pl P\)'s sequence to be Cauchy, and thus it either converges to
  a rational, or the sets \(V_{1/2^n}[x_n]\) will have empty intersection
  (where the irrational point of convergence would be). Uniformly locally
  compact spaces (and in particular, compact spaces) lack such holes, so it
  will be convenient to eliminate this technicality when it is irrelevant.

  \begin{game}
    Let \(\bellAbsConGame{X}\) denote the
    \term{Bell absolute convergence game} which
    proceeds analogously to \(\bellConGame{X}\), except
    that \(\pl D\) must always ensure that \(\<p_0,p_1,\dots\>\) converges
    in \(X\) in order to win.
  \end{game}

  \begin{definition}
    A uniformizable space \(X\) is \term{absolutely proximal} if
    \(\pl D \win \bellAbsConGame{X}\).
  \end{definition}

  As was shown in \cite{MR3227201}:

  \begin{definition}
    A uniformizable space \(X\) is \term{uniformly locally compact} if there
    exists an entourage \(D\) such that \(\cl{D[x]}\) is compact
    for all \(x\).
  \end{definition}

  \begin{theorem}\label{uniformlyLocallyCompact}
    If \(X\) is a uniformly locally compact space, then
    \(\pl D \win \bellConGame{X}\) if and only if
    \(\pl D \win \bellAbsConGame{X}\).
  \end{theorem}

  So all proximal compacts are absoluately proximal compacts.


\section{Limited information analogues}

  \begin{definition}
    A \term{\(k\)-tactical strategy} or \term{\(k\)-tactic}
    is a strategy which considers only the most recent move of the opponent.

    If \(\pl P\) has a winning \(k\)-tactic for a game \(G\), we write
    \(\pl P\ktactwin{k} G\). If omitted, assume \(k=1\).
  \end{definition}

  \begin{definition}
    A \term{\(k\)-Markov strategy} or \term{\(k\)-mark} is a strategy
    which considers only the round number and most recent move of the
    opponent.

    If \(\pl P\) has a winning \(k\)-mark for a game \(G\), we write
    \(\pl P\kmarkwin{k} G\). If omitted, assume \(k=1\).
  \end{definition}

  Limited information strategies may be used to strengthen game-theoretic
  topological properties.
  Some results of Bell may be generalized to hold for such limited information
  strategies. The proofs of the following propositions are straightforward.

  \begin{proposition}
    Let \(X\) be a uniformizable space and \(H\) be a closed subset of \(X\).
    If \(k<\omega\), then
    \begin{itemize}
      \item
       \(
          \pl D \ktactwin{k} \bellConGame{X}
            \Rightarrow
          \pl D \ktactwin{k} \bellConGame{H}
       \)
      \item
       \(
          \pl D \kmarkwin{k} \bellConGame{X}
            \Rightarrow
          \pl D \kmarkwin{k} \bellConGame{H}
       \)
      \item
       \(
          \pl D \win \bellConGame{X}
            \Rightarrow
          \pl D \win \bellConGame{H}
       \)
    \end{itemize}
  \end{proposition}

  \begin{proposition}
    Let \(X\) be a uniformizable space and \(H\) be a closed subset of \(X\).
    If \(k<\omega\), then
    \begin{itemize}
      \item
       \(
          \pl D \ktactwin{k} \bellAbsConGame{X}
            \Rightarrow
          \pl D \ktactwin{k} \bellAbsConGame{H}
       \)
      \item
       \(
          \pl D \kmarkwin{k} \bellAbsConGame{X}
            \Rightarrow
          \pl D \kmarkwin{k} \bellAbsConGame{H}
       \)
      \item
       \(
          \pl D \win \bellAbsConGame{X}
            \Rightarrow
          \pl D \win \bellAbsConGame{H}
       \)
    \end{itemize}
  \end{proposition}

  Bell observed a relationship between her game and another convergence
  game due to Gary Gruenhage.

  \begin{game}
    Let \(\gruConGame{X}{S}\) denote
    \term{Gruenhage's convergence game} with
    players \(\pl O\), \(\pl P\) which proceeds as follows for a space
    \(X\) and a set \(S\). In round \(n\),
    \(\pl O\) chooses an open set \(U_n\) containing
    \(S\), followed by \(\pl P\)
    choosing a point \(p_n\in \bigcap_{m\leq n}U_m\).

    \(\pl O\) wins in the case that
    \(\<p_0,p_1,\dots\>\) converges to the set \(S\); that is,
    any open set containing \(S\) contains all but finite \(p_n\).
    \(\pl P\) wins otherwise.
  \end{game}

  When \(S=\{x\}\), we abuse notation and write
  \(\gruConGame{X}{x}\).

  \begin{definition}
    Let \(A\) be any set, \(n<\omega\), and \(f\in A^n\).
    \(f\rest k\in A^{\min(k,n)}\) is the \term{restriction} of \(f\)
    to \(k\), yielding the first \(k\) terms of
    \(f\) when \(k<n\) (and otherwise equaling \(f\) itself).
    \(f\last k\in A^{\min(k,n)}\) is the \term{suffix} of \(f\)
    to \(k\), yielding the last \(k\) terms of
    \(f\) when \(k<n\) (and otherwise equaling \(f\) itself).
  \end{definition}

  \begin{definition}
    Let \(A\) be any set, \(n\leq\omega\), and \(f,g\in A^{n}\).
    \(f\concat g\) is the \term{concatenation} of \(f\) and \(g\), so
    \(f\concat g(i)=f(i)\) and \(f\concat g(n+i)=g(i)\) for \(i<n\).
    \(f\zip g\) is the \term{zip} of \(f\) and \(g\), so
    \(f\zip g(2i)=f(i)\) and \(f\zip g(2i+1)=g(i)\) for \(i<n\).
  \end{definition}

  \begin{theorem}
    Let \(k<\omega\). For all \(x\in X\):
    \begin{itemize}
      \item
        \(
          \pl D\ktactwin{2k} \bellConGame{X}
            \Rightarrow
          \pl O \ktactwin{k} \gruConGame{X}{x}
        \)
      \item
        \(
          \pl D\kmarkwin{2k} \bellConGame{X}
            \Rightarrow
          \pl O \kmarkwin{k} \gruConGame{X}{x}
        \)
      \item
        \(
          \pl D\win \bellConGame{X}
            \Rightarrow
          \pl O \win \gruConGame{X}{x}
        \)
    \end{itemize}
  \end{theorem}

  \begin{proof}
    The perfect-information result was proven in Bell's
    original paper \cite{MR3239205}. We proceed by proving the
    Markov-information result, and note that the tactical-information
    result may be proven by simply dropping usage of
    the round number from the given proof.

    Let \(\sigma\) be a winning \(2k\)-mark for \(\pl D\) in
    \(\bellConGame{X}\).
    Let \(\chi_n\in X^n\) have constant value \(x\) for \(n\leq\omega\).
    We define the \(k\)-mark \(\tau\) for \(\pl O\) in
    \(\gruConGame{X}{x}\) such that
      \[
        \tau(t,n)
          =
        \sigma\Big(
          \chi_{|t|} \zip t
        ,2n\Big)[x]
        \cap
        \sigma\Big(
          \<x\> \concat(t\zip\chi_{|t|}) \rest 2k
        ,2n+1\Big)[x]
      \]

    Let \(p\) be a legal attack against \(\tau\) in \(\gruConGame{X}{x}\).
    Consider the attack \(q=\chi_\omega\zip p\) against the winning
    \(2k\)-mark \(\sigma\) in \(\bellConGame{X}\).
    Let \(D_n=\sigma(q\rest n\last 2k,n)\) be the entourage played according
    to \(\sigma\) in round \(n\), and \(E_n=\bigcap_{m\leq n}D_m\).

    Certainly, $x\in E_{2n}[x]= E_{2n}[q(2n)]$ for any $n<\omega$.
    Note also for any $n<\omega$ that
        \[
          p(n) \in
          \bigcap_{m\leq n}\tau(p\rest m\last k, m)
        \]
        \[
          =
          \bigcap_{m\leq n}\Big(
            \sigma\big(
              \chi_{\min(m,k)} \zip (p\rest m\last k)
            ,2m\big)[x]
            \cap
            \sigma\big(
              \<x\> \concat((p\rest m\last k)\zip
              \chi_{\chi_{\min(m,k)}}) \rest 2k
            ,2m+1\big)[x]
          \Big)
        \]
        \[
          =
          \bigcap_{m\leq n}\Big(
            \sigma\big(
              q\rest 2m\last 2k
            ,2m\big)[x]
            \cap
            \sigma\big(
              q\rest 2m+1\last 2k
            ,2m+1\big)[x]
          \Big)
        \]
        \[
          =
          \bigcap_{m\leq n}\left(
            D_{2m}[x]\cap
            D_{2m+1}[x]
          \right) =
          \bigcap_{m\leq 2n+1} D_m[x]=E_{2n+1}[x]
        \]
    so by the symmetry of \(E_{2n+1}\),
      \[
        q(2n+2)=x\in E_{2n+1}[p(n)]= E_{2n+1}[q(2n+1)]
      \]
    and
      \[
        q(2n+1)=p(n)\in E_{2n+1}[x]\subseteq E_{2n}[x]=E_{2n}[q(2n)]
      \]
    making \(q\) a legal attack against \(\sigma\).

    Then as \(x\in \bigcap_{n<\omega} E_n[q(n)]\not=\emptyset\),
    and \(\sigma\) is a winning strategy, the attack $q$ converges.
    Since \(q(2n)=x\), \(q\) must converge to \(x\). Thus its subsequence
    \(p\) converges to \(x\), and \(\tau\)
    is a winning \(k\)-mark for \(\pl O\) in \(\gruConGame{X}{x}\).
  \end{proof}

\subsection{Products and strengthenings of compactness}

  \begin{definition}
    Let \(X_\alpha\) be a topological space for \(\alpha<\kappa\)
    and \(z\in\prod_{\alpha<\kappa}X_\alpha\).
    The \term{\(\Sigma\)-product} \(\SigmaProd_{\alpha<\kappa}^{z}X_\alpha\)
    with base point \(z\) is given by
    \[
      \SigmaProd_{\alpha<\kappa}^{z}X_\alpha
        =
      \left\{
        x\in \prod_{\alpha<\kappa}X_\alpha
      :
        |\{\alpha<\kappa:x(\alpha)\not=z(\alpha)\}|\leq\omega
      \right\}
    \]
    When \(X_\alpha=X\) and \(z(\alpha)=0\) for all \(\alpha<\kappa\),
    then we write \(\SigmaPower{X}{\kappa}=\SigmaProd_{\alpha<\kappa}^z X\).
  \end{definition}

  \begin{theorem}[\cite{MR3239205}]
    If \(X_\alpha\) is proximal for all \(\alpha<\kappa\),
    then \(\SigmaProd_{\alpha<\kappa}^{z}X_\alpha\) is proximal for any
    base point \(z\).
  \end{theorem}

  It's this result which yielded the following observation by Nyikos in
  \cite{MR3288115}.

  \begin{definition}
    A \term{Corson compact} space is a compact space which may be embedded in
    \(\SigmaProdR\kappa\).
  \end{definition}

  \begin{corollary}\label{nyikosObservation}
    Corson compacts are proximal.
  \end{corollary}

  Nyikos asked if the result could be reversed, which the author answered
  with Gary Gruenhage in \cite{MR3227201} in the affirmative.

  \begin{theorem}
    A compact space \(X\) is Corson compact if and only if it is
    proximal.
  \end{theorem}

  This result required an
  earlier game characterization of Corson compactness due to Gruenhage.

  \begin{theorem}[\cite{MR752278}]
    A compact space \(X\) is Corson compact if and only if
    \(\pl O\win\gruConGame{X^2}{\Delta}\).
  \end{theorem}

  So Corson compactness is characterized by both
  \(\pl O\win\gruConGame{X^2}{\Delta}\) and
  \(\pl D\win\bellAbsConGame{X}\) for compact spaces.
  However, we will see that these games are not
  equivalent with respect to limited information strategies.

  In the study of limited information strategies in \(\bellAbsConGame{X}\),
  it is natural to consider \(\Sigma^\star\)- and \(\sigma\)-products
  in place of \(\Sigma\)-products.


\subsection{\(\Sigma^\star\)-products and Eberlein compacts}

  \begin{definition}
    Let \(X\) be a metrizable space with
    compatible metric \(d\), and let \(z\in X^\kappa\).
    The \term{\(\Sigma^\star\)-product}
    \(\SigmaStarProd_{\alpha<\kappa}^{z}X\)
    with base point \(z\) is given by
    \[
      \SigmaStarProd_{\alpha<\kappa}^{z}X
        =
      \left\{
        x\in \prod_{\alpha<\kappa}X
      :
        n<\omega
      \Rightarrow
        \left|\left\{
          \alpha<\kappa:d(x(\alpha),z(\alpha))>\frac{1}{2^n}
        \right\}\right|
        <\omega
      \right\}
    \]
    When \(z(\alpha)=0\) for all \(\alpha<\kappa\),
    then we write
    \(\SigmaStarPower{X}{\kappa}=\SigmaStarProd_{\alpha<\kappa}^z X\).
  \end{definition}

  Just as \(\Sigma\) products preserve winning perfect-information strategies,
  we will show that \(\Sigma^\star\) products preserve winning
  Markov strategies.

  \begin{definition}
    For a metric space \(\<X,d\>\), let
    \(V_\epsilon=\{\<x,y\>:d(x,y)<\epsilon\}\). Note that \(V_\epsilon\)
    is an (open symmetric) entourage on \(X\).
  \end{definition}

  Note that by definition, a \(0\)-Markov strategy uses zero moves of
  the opponent, and only considers the round number.

  \begin{proposition}
    For any metrizable space \(X\),
    \(\pl D\kmarkwin{0}\bellConGame{X}\).
    For any completely metrizable space \(X\),
    \(\pl D\kmarkwin{0}\bellAbsConGame{X}\).
  \end{proposition}

  \begin{proof}
    Essentially shown by Bell in her paper: in either case,
    \(\pl D\) chooses \(V_{1/2^{n}}\)
    during round \(n\), forcing legal attacks by \(\pl P\) to be Cauchy.
  \end{proof}

  As an aside, this next proposition may be proved similarly
  using \(V_{d(x,y)/2}\).

  \begin{proposition}
    For any metrizable space \(X\),
    \(\pl D\ktactwin{2}\bellConGame{X}\).
    For any completely metrizable space \(X\),
    \(\pl D\ktactwin{2}\bellAbsConGame{X}\).
  \end{proposition}

  We will exploit the last move of the
  opponent to obtain winning Markov strategies under
  \(\Sigma^\star\) products.

  \begin{proposition}
    If \(X_\alpha\) is a uniformizable space for \(\alpha<\kappa\),
    and \(D_\alpha\) is an entourage of \(X_\alpha\) for
    \(\alpha\in F\in[\kappa]^{<\omega}\), then
      \[
        P(\{\<\alpha,D_\alpha\>:\alpha\in F\})
          =
        \left\{
          \<x,y\>\in\left(\prod_{\alpha<\kappa}X_\alpha\right)^2
        :
          \alpha\in F
        \Rightarrow
          \<x(\alpha),y(\alpha)\>\in D_\alpha
        \right\}
      \]
    is an entourage of \(\prod_{\alpha<\kappa}X_\alpha\).
  \end{proposition}

  \begin{theorem}
    For any metrizable space \(X\) and \(z\in X^\kappa\),
    \(\pl D\markwin\bellConGame{\SigmaStarProd_{\alpha<\kappa}^z X}\).
  \end{theorem}

  \begin{proof}
    For \(x\in\SigmaStarProd_{\alpha<\kappa}^z X\), let
      \[
        \supp_n(x)
          =
        \left\{
          \alpha<\kappa
        :
          d(x(\alpha),z(\alpha))>\frac{1}{2^n}
        \right\}
          \in
        \kappa^{<\omega}
      \]
    where \(d\) is a metric compatible with the topology on \(X\).

    Define a strategy \(\tau\) for \(\pl D\) by
      \[
        \tau(\emptyset,0)
          =
        \left(\SigmaStarProd_{\alpha<\kappa}^z X\right)^2
      \]
      \[
        \tau(\<p\>,n+1)
          =
        \left(\SigmaStarProd_{\alpha<\kappa}^z X\right)^2
          \cap
        P(\{
          \<\alpha,V_{1/2^{n+1}}\>
            :
          \alpha\in\supp_n(p)
        \})
      \]

    Let \(a:\omega\to \SigmaStarProd_{\alpha<\kappa}^z X\) be a legal
    attack by \(\pl P\) against \(\tau\), so
      \[
        \<a(n+1),a(n+2)\>
          \in
        \bigcap_{m\leq n}
        \tau(\<a(m)\>,m+1)
      \]
      \[
          =
        \left(\SigmaStarProd_{\alpha<\kappa}^z X\right)^2
          \cap
        \bigcap_{m\leq n}
        P(\{
          \<\alpha,V_{1/2^{m+1}}\>
            :
          \alpha\in\supp_m(a(m))
        \})
      \]
    and let \(\alpha<\kappa\).

    If \(d(a(m+1)(\alpha),z(\alpha))\leq\frac{1}{2^m}\) for all but finite
    \(m<\omega\), then  \(\lim_{n<\omega}a(n)(\alpha)=z(\alpha)\).
    Otherwise, we have \(\alpha\in\supp_m(a(m))\) for infinitely many
    \(m<\omega\).

    If
    \(\bigcap_{n<\omega}\tau(\<a(n)\>,n+1)[a(n+1)]=\emptyset\), then
    \(\pl D\) has already won.
    Otherwise there exists a nonstrictly increasing
    unbounded function \(f\in\omega^\omega\) with
    \(\bigcap_{n<\omega}V_{1/2^{f(n)+1}}[a(n+1)(\alpha)]\not=\emptyset\).
    Since the diameter of these sets approaches \(0\), the intersection is
    a singleton \(\{x\}\); furthermore, the \(\alpha\) coordinate of \(a\)
    must then converge to \(x\). We conclude that
    \(\pl D\markwin\bellConGame{\SigmaStarProd_{\alpha<\kappa}^z X}\).
  \end{proof}

  Note that we may obtain an analogous observation to Corollary
  \ref{nyikosObservation}.

  \begin{definition}
    An \term{Eberlein compact} space is a compact space which may be embedded
    in \(\SigmaStarProdR\kappa\).
  \end{definition}

  \begin{corollary}
    Eberlein compacts are Markov-proximal.
  \end{corollary}

  It's natural to ask (particularly considering the results of the next
  section) whether this too may be reversed.

  \begin{question}\label{mainQuestion}
    Must all Markov-proximal compacts be Eberlein compact?
  \end{question}

  One might suspect that the answer is yes in light of the following fact:

  \begin{theorem}[\cite{MR858337}]
    A compact space \(X\) is Eberlein compact if and only if
    \(\pl O\markwin\gruConGame{X^2}{\Delta}\).
  \end{theorem}

\subsection{\(\sigma\)-products and strong Eberlein compacts}

  \begin{definition}
    Let \(X_\alpha\) be a topological space for \(\alpha<\kappa\)
    and \(z\in\prod_{\alpha<\kappa}X_\alpha\).
    The \term{\(\sigma\)-product} \(\sigmaProd_{\alpha<\kappa}^{z}X_\alpha\)
    with base point \(z\) is given by
    \[
      \sigmaProd_{\alpha<\kappa}^{z}X_\alpha
        =
      \left\{
        x\in \prod_{\alpha<\kappa}X_\alpha
      :
        |\{\alpha<\kappa:x(\alpha)\not=z(\alpha)\}|<\omega
      \right\}
    \]
    When \(X_\alpha=X\) and \(z(\alpha)=0\) for all \(\alpha<\kappa\),
    then we write \(\sigmaPower{X}{\kappa}=\sigmaProd_{\alpha<\kappa}^z X\).
  \end{definition}

  The reader may anticipate our next goal: \(\Sigma\)-products
  preserve winning perfect information strategies, and
  \(\Sigma^\star\)-products preserve winning Markov strategies, so
  we aim to show that \(\sigma\)-products preserve winning
  tactical strategies.

  \begin{theorem}
    Let \(X_\alpha\) be a uniformizable space for \(\alpha<\kappa\)
    and \(z\in \prod_{\alpha<\kappa}X_\alpha\).
    If
    \(\pl D\tactwin\bellConGame{X_\alpha}\) for all \(\alpha<\kappa\),
    then
    \(\pl D\tactwin\bellConGame{\sigmaProd_{\alpha<\kappa}^z X_\alpha}\).
  \end{theorem}

  \begin{proof}
    For \(x\in\sigmaProd_{\alpha<\kappa}^z X_\alpha\), let
      \[
        \supp(x)
          =
        \left\{
          \alpha<\kappa
        :
          x(\alpha)\not=z(\alpha)
        \right\}
          \in
        \kappa^{<\omega}
      \]
    and let \(\tau_\alpha\) be a winning strategy for \(\pl D\) in
    \(\bellConGame{X_\alpha}\) for \(\alpha<\kappa\).

    Define a strategy \(\tau\) for \(\pl D\) by
      \[
        \tau(\emptyset)
          =
        \left(\sigmaProd_{\alpha<\kappa}^z X\right)^2
      \]
      \[
        \tau(\<p\>)
          =
        \left(\sigmaProd_{\alpha<\kappa}^z X\right)^2
          \cap
        P(\{
          \<
            \alpha,
            \tau_\alpha(p(\alpha))\cap\tau_\alpha(z(\alpha))
          \>
            :
          \alpha\in\supp(p)
        \})
      \]

    Let \(a:\omega\to \sigmaProd_{\alpha<\kappa}^z X\) be a legal
    attack by \(\pl P\) against \(\tau\), so
      \[
        \<a(n+1),a(n+2)\>
          \in
        \bigcap_{m\leq n}
        \tau(\<a(m)\>)
      \]
      \[
          =
        \left(\sigmaProd_{\alpha<\kappa}^z X\right)^2
          \cap
        \bigcap_{m\leq n}
        P(\{
          \<
            \alpha,
            \tau_\alpha(a(m)(\alpha))\cap\tau_\alpha(z(\alpha))
          \>
            :
          \alpha\in\supp(a(m))
        \})
      \]
    and let \(\alpha<\kappa\).
    Let \(a_\alpha:\omega\to X_\alpha\) satisfy
    \(a_\alpha(n)=a(n)(\alpha)\).

    Of course if \(a_\alpha(n)=z(\alpha)\) for all \(n<\omega\), then
    \(a_\alpha\to z(\alpha)\).
    Otherwise, let \(N<\omega\) be minimal with \(a_\alpha(N)\not=z(\alpha)\).
    Then for \(n\geq N\),
      \[
        \<a_\alpha(n+1),a_\alpha(n+2)\>
          =
        \<a(n+1)(\alpha),a(n+2)(\alpha)\>
      \]
      \[
          \in
        \bigcap_{m\leq n,\alpha\in\supp(a(m))}
        \tau_\alpha(a(m)(\alpha))\cap\tau_\alpha(z(\alpha))
      \]
      \[
          \subseteq
        \bigcap_{N\leq m\leq n}
        \tau_\alpha(a_\alpha(m))
      \]
    So the subsequence \(a_\alpha'\) obtained by omitting the terms of
    \(a_\alpha\) before \(N\) is a legal attack against the
    winning tactic \(\tau_\alpha\), and therefore either
    \(a_\alpha'\) and \(a_\alpha\) converge, or
      \[
        \bigcap_{N\leq n<\omega}
        \left(
          \bigcap_{N\leq m\leq n}
          \tau_\alpha(a_\alpha(m))
        \right)
        [a_\alpha(n+1)]
          =
        \emptyset
      \]

    We conclude that either \(a\) converges on each coordinate
    \(\alpha<\kappa\), or the intersection
      \[
        \bigcap_{n<\omega}
        \left(
          \bigcap_{m\leq n}
          \tau(\<a(m)\>)
        \right)
        [a(n+1)]
      \]
    is empty, verifying that \(\tau\) is a winning tactic.
  \end{proof}

  The proof of the following is similar.

  \begin{theorem}
    Let \(X_\alpha\) be a uniformizable space for \(\alpha<\kappa\)
    and \(z\in \prod_{\alpha<\kappa}X_\alpha\).
    If
    \(\pl D\tactwin\bellAbsConGame{X_\alpha}\) for all \(\alpha<\kappa\),
    then
    \(\pl D\tactwin\bellAbsConGame{\sigmaProd_{\alpha<\kappa}^z X_\alpha}\).
  \end{theorem}

  We again obtain an analogous observation to Corollary
  \ref{nyikosObservation}.

  \begin{definition}
    A \term{strong Eberlein compact} space is a compact space which
    may be embedded in \(\sigmaProdTwo\kappa\).
  \end{definition}

  \begin{corollary}
    Strong Eberlein compacts are tactic-proximal.
  \end{corollary}

  We can show that among compact spaces, the tactic-proximal property is
  necessary and sufficent for strong Eberlein compactness.
  However, the techniques used in showing this are different than those
  used to characterize Corson compactness with the proximal property in
  \cite{MR3227201}.

  The presence of a Cantor set determines the success of \(\pl D\)'s
  tactical strategies in \(\bellConGame{X}\).

  \begin{lemma}
    Tactic-proximal spaces cannot contain a copy of the Cantor set.
  \end{lemma}

  \begin{proof}
    The result will follow once we show
    \(\pl D\nottactwin\bellAbsConGame{2^\omega}\).
    Let \(\sigma\) be a tactic for \(\pl D\) in \(\bellAbsConGame{2^\omega}\)
    and let \(D_k=\{\<f,g\>:f\rest k = g\rest k\}\). Since \(\{D_k:k<\omega\}\)
    is a base for the universal uniformity on \(2^\omega\)
    (namely, all open neighborhoods of the diagonal),
    we may fix \(k(f)<\omega\)
    for each \(f\in2^\omega\) such that \(D_{k(f)}\subseteq\sigma(\<f\>)\).

    Then there exists \(k<\omega\) such that \(\{f:k(f)=k\}\) is uncountable,
    and therefore there exist distinct \(f,g\in2^\omega\)
    such that \(k=k(f)=k(g)\) and
    \(f\rest k=g\rest k\). Then \(p:\omega\to2^\omega\) defined by
    \(p(2n)=f\) and \(p(2n+1)=g\) is an attack against \(\sigma\) which
    obviously doesn't converge. This attack is legal since
    \(f\in D_k[g]\subseteq\sigma(\<g\>)[g]\) and
    \(g\in D_k[f]\subseteq\sigma(\<f\>)[f]\), so \(\sigma\) is not a winning
    tactic.
  \end{proof}

  \begin{lemma}
    Every non-scattered Corson compact space contains a homeomorphic
    copy of the Cantor set.
  \end{lemma}

  \begin{proof}
    Every non-scattered space contains a closed subspace without
    isolated points. Let \(X\) be such a subspace, and assume that this
    Corson compact is embedded in \(\SigmaProdR\kappa\). Let
    \(B_{\alpha,\epsilon}(x)=\{y: d(x(\alpha),y(\alpha))<\epsilon\}\).
    For each \(x\in X\) and \(n<\omega\), let \(\beta(x,n)<\kappa\) be defined
    such that
    \(\supp(x)=\{\beta(x,n):n<\omega\}\).

    Choose an arbitrary \(x_\emptyset\in X\) and \(\epsilon_0>0\), and
    and let \(A_0=\emptyset\).

    Suppose then that for some \(n<\omega\),
    \(x_s\in X\) is defined for all \(s\in 2^n\),
    and \(\epsilon_n>0\) and \(A_n\in[\kappa]^{<\omega}\) are defined.
    Since each \(x_s\) is not isolated in \(X\), let \(U_s\) be the open
    set
      \[
        U_s
          =
        X
          \cap
        \bigcap_{\alpha\in A_{|s|}} B_{\alpha,\epsilon_{|s|}}(x_s)
      \]
    and choose \(x_{s\concat\<0\>},x_{s\concat\<1\>}\in U_s\) distinct.
    Then let \(\alpha_s<\kappa\) such that
    \(x_{s\concat\<0\>}(\alpha_s)\not=x_{s\concat\<1\>}(\alpha_s)\).
    Let
      \[
        A_{n+1}
          =
        \{\alpha_s:s\in2^{\leq n}\}
          \cup
        \{\beta(x_s,i):s\in2^{\leq n},i\leq n\}
      \]

    Then choose \(0<\epsilon_{n+1}<\frac{1}{2}\epsilon_n\) such that
    \[
      B_{\alpha_s,\epsilon_{n+1}}(x_s\concat\<0\>)
        \cap
      B_{\alpha_s,\epsilon_{n+1}}(x_s\concat\<1\>)
        =
      \emptyset
    \]
    and
    \[
      \overline{
        \bigcap_{\alpha\in A_{n+1}}
        B_{\alpha,\epsilon_{n+1}}(x_s\concat\<0\>)
      }
        \cup
      \overline{
        \bigcap_{\alpha\in A_{n+1}}
        B_{\alpha,\epsilon_{n+1}}(x_s\concat\<1\>)
      }
        \subseteq
      \bigcap_{\alpha\in A_n} B_{\alpha,\epsilon_n}(x_s)
    \]
    for all \(s\in 2^n\).

    Let \(x_f=\lim_{n<\omega} x_{f\rest n}\in X\)
    for each \(f\in 2^\omega\). We claim \(C=\{x_f:f\in 2^\omega\}\)
    is a copy of the Cantor set. This will follow if we can show that
    \(\{U_s:s\in 2^{<\omega}\}\) is a base for \(C\), since it has
    the structure of the Cantor tree.

    Consider \(x_f\) for some \(f\in 2^\omega\), and a subbasic open ball
    \(B_{\alpha,\epsilon}(x_f)\). Observe that
    \(x_f\in\bigcap_{n<\omega} U_{f\rest n}\) since
    \(x_{f\rest n}\in U_{f\rest m}\) for all \(m<n<\omega\).

    If \(\alpha\in\{\beta(x_s,n):s\in2^{<\omega},n<\omega\}\), choose
    \(k<\omega\) with \(\alpha\in A_k\). Then choose \(l<\omega\) such that
    \(\epsilon_l<\epsilon\). Then
    \(U_{f\rest(l+k)}\subseteq B_{\alpha,\epsilon}(x_f)\).

    Otherwise, \(x_s(\alpha)=0\) for all \(s\in2^{<\omega}\), so
    \(x_g(\alpha)=0\) for all \(g\in2^\omega\) and therefore
    \(C\subseteq B_{\alpha,\epsilon}(x_f)\).
  \end{proof}

  A new game characterization of strong Eberlein compactness follows
  from the above and an earlier characterization by Gruenhage.

  \begin{theorem}[\cite{MR752278}]
    For compact spaces \(X\),
    \(X\) is strong Eberlein compact if and only if
    \(X\) is scattered and \(\pl O\win\gruConGame{X}{x}\) for all \(x\in X\).
  \end{theorem}

  \begin{corollary}\label{mainResult}
    For compact spaces \(X\),
    \(X\) is strong Eberlein compact if and only if
    \(X\) is tactic-proximal.
  \end{corollary}

  \begin{proof}
    Assume \(X\) is not strong Eberlein compact. One of two things must hold.
      \begin{enumerate}
        \item
          If \(\pl O\notwin\gruConGame{X}{x}\) for some \(x\in X\), then
          \(\pl D\notwin \bellConGame{X}\) and therefore
          \(\pl D\nottactwin \bellConGame{X}\).
        \item
          If \(X\) is not scattered, it contains a homeomorphic copy
          of the Cantor space \(2^\omega\). Since
          \(\pl D\nottactwin \bellConGame{2^\omega}\),
          we have
          \(\pl D\nottactwin \bellConGame{X}\).
      \end{enumerate}
    Thus \(X\) is not tactic-proximal.
  \end{proof}

\section{Comparing \(\bellConGame{X}\) and \(\gruConGame{X^2}{\Delta}\)}

  As mentioned above, for compact spaces \(X\) it's true that
  \(\pl D\win\bellConGame{X}\) if and only if
  \(\pl O\win\gruConGame{X^2}{\Delta}\). Indeed these games are very similar;
  since \(X\) is (para)compact, both \(\pl D\) and \(\pl O\) are simply
  choosing neighborhoods of the diagonal, aiming for some sort of convergence.

  For tactical information, we may easily see how
  these games differ.

  \begin{example}
    Let \(X\) be compact and metrizable.
      \begin{enumerate}
        \item
          \(\pl O\tactwin\gruConGame{X^2}{\Delta}\)
        \item
          \(\pl O\kmarkwin{0}\gruConGame{X^2}{\Delta}\)
        \item
          \(\pl D\ktactwin{2}\bellConGame{X}\)
        \item
          \(\pl D\kmarkwin{0}\bellConGame{X}\)
      \end{enumerate}
    However, \(\pl D\nottactwin\bellConGame{X}\) if \(X\) is non-scattered.
  \end{example}

  \begin{proof}
    Let \(d\) be a compatible metric for the topology on \(X\).
    Then \(\sigma\) as defined for each case below is a winning strategy.
    \begin{enumerate}
      \item
        \[
          \sigma(\<x\>)
            =
          V_{d(x(0),x(1))/2}
        \]
      \item
        \[
          \sigma(n)
            =
          V_{1/2^n}
        \]
      \item (as noted earlier)
        \[
          \sigma(\<x,y\>)
            =
          V_{d(x,y)/2}
        \]
      \item (as noted earlier)
        \[
          \sigma(n)
            =
          V_{1/2^n}
        \]
    \end{enumerate}
    Finally, recall that \(\pl D\nottactwin\bellConGame{X}\)
    when \(X\) is non-scattered Corson compact.
  \end{proof}

  Effectively, the fact that \(\pl O\) can see two coordinates (\(\pl P\)
  chooses points in \(X^2\)) gives her an extra edge to use in a game
  of limited information.

  With respect to these games,
  question \ref{mainQuestion} could be restated as follows:
  is \(\pl O\markwin\gruConGame{X^2}{\Delta}\) equivalent to
  \(\pl D\markwin\bellConGame{X}\) for compact spaces \(X\)?


\bibliographystyle{plain}
\bibliography{../../bibliography}

\end{document}