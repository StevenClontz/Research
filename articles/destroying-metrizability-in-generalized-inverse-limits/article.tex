\documentclass{amsart}

\usepackage{amsmath}
\usepackage{amsthm}
\usepackage{amssymb}

\usepackage{graphicx}

\newtheorem{theorem}{Theorem}[section]
\newtheorem{proposition}[theorem]{Proposition}
\newtheorem{lemma}[theorem]{Lemma}
\newtheorem{corollary}[theorem]{Corollary}


\theoremstyle{definition}
\newtheorem{definition}[theorem]{Definition}
\newtheorem{game}[theorem]{Game}
\newtheorem{example}[theorem]{Example}
\newtheorem{question}[theorem]{Question}

\newcommand{\<}{\langle}
\renewcommand{\>}{\rangle}

\title{Joint work on long u.s.c. inverse limits}

\author{Steven Clontz, Scott Varagona}

\date{\today}

\begin{document}

\begin{abstract}
If $X$ is a compact Hausdorff space, an upper semi-continuous bonding function $f: X \rightarrow 2^{X}$ is said to be idempotent if $f^2 = f$. In this short paper, we prove that if $f: [0,1] \rightarrow C([0,1])$ is u.s.c., idempotent and surjective, but $f$ is not the identity, then the inverse limit with the single bonding function $f$ and with factor spaces indexed by the ordinal $\kappa$ contains a copy of $\kappa + 1$. It follows that such an inverse limit is only metric in the case where the index set $\kappa$ is countable.
\end{abstract}

\maketitle

\section{Introduction}

As various recent papers have shown (e.g., \cite{char roe}, \cite{varagona}, \cite{vernon}), continuum theorists have begun to broaden their study of generalized inverse limits to the case where the factor spaces are indexed by sets other than the positive integers. This more inclusive approach opens many new avenues for research, and has already produced some interesting results. For example, Patrick Vernon showed that an inverse limit indexed by the set of all integers with a single upper semi-continuous bonding function can be homeomorphic to a 2-cell \cite{vernon}, whereas V. Nall previously showed this is impossible for such an inverse limit indexed by the positive integers alone \cite{nall 2cell}.
\

In another recent paper \cite{varagona}, S. Varagona generalized many previously known theorems (from, e.g., \cite{ingram mahavier} and \cite{nall connected}) to the case of u.s.c. inverse limits indexed by arbitrary totally ordered sets. As he showed, in that context, inverse limits with a single idempotent upper semi-continuous bonding function become especially important. Relatively little is known about the kind of continua that may be produced by such inverse limits, however.
\

This short paper focuses on the case of an inverse limit with a single idempotent, surjective, upper semi-continuous, continuum-valued bonding function on $[0,1]$, when the index set of the inverse limit is some ordinal $\kappa$. In particular, we show that (so long as the bonding function is not the identity) such an inverse limit is metric exactly when $\kappa$ is countable. The strategy, as will be seen, is to show that the inverse limit must contain a copy of $\kappa + 1$.

\

\section{Background Definitions and Lemmas}

We begin with some useful notation. Let $X$ be a non-empty compact Hausdorff space. Then we denote by $2^X$ the set of all non-empty compact subsets of $X$. We say a subset $A$ of $X$ is a \emph{continuum} if $A$ is non-empty, compact and connected. $C(X)$ denotes the set of all elements of $2^X$ that are continua.
\

Suppose $X$ and $Y$ are non-empty compact Hausdorff spaces. Then the set-valued function $f: X \rightarrow 2^{Y}$ is said to be \emph{upper semi-continuous} (u.s.c.) at $x \in X$ if, for every open set $V$ in $Y$ with $f(x) \subseteq V$, there exists an open set $U$ in $X$ with $x \in U$ and $f(u) \subseteq V$ for each $u \in U$. If $f$ is u.s.c. at each $x \in X$, we simply say \emph{$f$ is u.s.c.} If $f: X \rightarrow 2^{Y}$ is a set-valued function, then the \emph{graph of $f$}, denoted $G(f)$, is given by $G(f) = \{\langle x , y \rangle \in X \times Y \ | \ y \in f(x)\}$. (Note that, to help distinguish between ordered pairs and interval subsets of the real line, we will use pointed brackets to denote ordered pairs, e.g., $\langle x , y \rangle$, whereas intervals are given in the traditional way with brackets or parentheses, e.g., $(a,b), [c,d]$, etc.) The set-valued function $f: X \rightarrow 2^{Y}$ is said to be \emph{surjective} if, for every $y \in Y$, there exists $x \in X$ with $y \in f(x)$. If a u.s.c. function $f: X \rightarrow 2^{Y}$ has the property that $f(x)$ is connected for each $x \in X$, then $f$ is \emph{continuum-valued}, and we denote this by writing  $f: X \rightarrow C(Y).$
\

The following characterization of u.s.c. functions (from \cite{i m paper}) is well-known and often more convenient to work with than the original definition:

\begin{lemma} If $X$ and $Y$ are non-empty compact Hausdorff spaces, then $f: X \rightarrow 2^{Y}$ is u.s.c. if and only if $G(f)$ is closed.
\end{lemma}

If $X$, $Y$, and $Z$ are non-empty compact Hausdorff spaces and $f : X \rightarrow 2^{Y}$ and $g: Y \rightarrow 2^{Z}$ are u.s.c., then the composition $g \circ f : X \rightarrow 2^{Z}$ is the u.s.c. function given by $(g \circ f)(x) = \{z \in Z \ | \ $There exists $y \in Y$ such that $y \in f(x)$ and $z \in g(y)\}$. In the special case where $f: X \rightarrow 2^{X}$, we denote $f \circ f$ by $f^2$. If $f^2 = f$, then we say $f$ is \emph{idempotent}.
\

Let $\kappa$ be an ordinal. (We follow the convention of set theory in which an ordinal is equal to the set of its predecessors. For more detailed background on ordinals, see, for example, \cite{Kunen}.) Suppose that for each $\alpha \in \kappa$, $X_{\alpha}$ is a non-empty compact Hausdorff space and whenever $\alpha \le \beta \in \kappa$, the set-valued function $f_{\alpha \beta} : X_{\beta} \rightarrow 2^{X_{\alpha}}$ is u.s.c. (With this notation, $f_{\alpha \alpha}$ always denotes the identity on $X_{\alpha}$.) Suppose further that, whenever $\alpha \le \beta \le \eta \in \kappa$, we have $f_{\alpha \beta} \circ f_{\beta \eta} = f_{\alpha \eta}$. Then the collection $\textbf{f} = \{ f_{\alpha \beta} \ | \ \alpha \le \beta \in \kappa\}$ of u.s.c. functions is said to be \emph{exact}. In this case, we say $\{X_{\alpha}, f_{\alpha \beta}, \kappa\}$ is an \emph{inverse limit system}; the \emph{inverse limit}, $\varprojlim \{X_{\alpha}, f_{\alpha \beta}, \kappa\}$, of this system is $\{ \textbf{x} \in \prod_{\alpha \in \kappa} X_{\alpha} \ | \ x_{\alpha} \in f_{\alpha \beta}(x_{\beta})$ for all $\alpha \le \beta \in \kappa\}$. (In keeping with tradition, a boldface $\textbf{x}$ always denotes an element of $\prod_{\alpha \in \kappa} X_{\alpha}$; the $\alpha$th coordinate of such an $\textbf{x}$ will be denoted $\textbf{x}(\alpha)$.)  We call the u.s.c. functions $f_{\alpha \beta}$ the \emph{bonding functions} of the inverse limit, and each $X_{\alpha}$ is called a \emph{factor space} of the inverse limit.
\

Because it can be difficult to check if a large collection of u.s.c. functions is exact, a much more tractable case would be that of an inverse limit system with a single idempotent bonding function $f$. That is, suppose $X$ is a non-empty compact Hausdorff space and $f: X \rightarrow 2^{X}$ is an idempotent u.s.c. function. If $X_{\alpha} = X$ for each $\alpha \in \kappa$ and $f_{\alpha \beta} = f$ for each $\alpha < \beta \in \kappa$, then $\{X_{\alpha}, f_{\alpha \beta}, \kappa\} = \{X, f, \kappa\}$ is an \emph{inverse limit system with the single bonding function $f$}, and we denote the inverse limit of the system as $\varprojlim \{X, f, \kappa\}$. (Of course, since $f$ is idempotent, the collection $\textbf{f}$ of bonding functions is automatically exact.) We note that, although the only continuous idempotent surjective function from a compact Hausdorff space $X$ to itself is the identity map, there are many non-trivial examples of u.s.c. idempotent surjective functions on $X$.
\

A detailed study of inverse limits with a single idempotent surjective u.s.c. bonding function was begun in \cite{varagona}. The following is a lemma from that paper which will be used repeatedly (often without being referenced explicitly) in the proofs to come.

\begin{lemma} \label{idemlemma}  Let $X$ be a compact Hausdorff space and let $f: X \rightarrow 2^{X}$ be u.s.c. Then $f$ is idempotent if and only if, whenever $f(x) = A$ for some $x \in X$ and $A \subseteq X$, $f(A) = A$.
\end{lemma}

\section{Main Results}

We now focus on the case of an inverse limit with a single idempotent, surjective, u.s.c. bonding function $f: [0,1] \rightarrow C([0,1])$. The main goal of this paper is to prove that if such an $f$ is not the identity map, then the inverse limit $\varprojlim \{[0,1], f, \kappa\}$ is metric exactly when the index set $\kappa$ is a countable ordinal.
\

Let us say a subset $S$ of $[0,1]^2$ satisfies condition $\Gamma$ if, for some $x, y \in [0,1]$ with $x \ne y$, $\<x,x\> \in S$, $\<y,y\> \in S$, and either $\<x,y\>$ or $\<y,x\> \in S$. Let $\Delta = \{\<a,a\> \ | \ a \in [0,1]\}$, and notice that $[0,1]^2 \setminus \Delta$ is the union of two disjoint open sets. Let $\iota$ be the identity map $\iota(x)=\{x\}$.

\begin{lemma}
If $f: [0,1] \rightarrow C([0,1])$ is u.s.c., then for some $x \in [0,1]$, $\<x,x\> \in G(f)$.
\end{lemma}

\begin{proof}
Suppose that $\<x,x\> \not\in S$ for all $x \in [0,1]$. Then $y > 0$ for all $y \in f(0)$, and $z < 1$ for all $z \in f(1)$. This means $([0,1]^2 \setminus \Delta) \cap G(f) = G(f)$ is the union of two non-empty disjoint open sets. However, since $f$ was continuum-valued and u.s.c., $G(f)$ was connected (by Theorem 2.5 from \cite{ingram intro}), yielding a contradiction.
\end{proof}

\begin{lemma}
If $f: [0,1] \rightarrow C([0,1])$ is u.s.c., idempotent, and surjective, then for some $x, y \in [0,1]$ with $x \not= y$, $\<x,x\> \in G(f)$ and $\<y,y\> \in G(f)$.
\end{lemma}

\begin{proof}
By Lemma 1, for some $x \in [0,1]$, $\<x,x\> \in G(f)$. Suppose by way of contradiction that, for all $y \in [0,1]$ with $y \not= x$, $\<y,y\> \not\in G(f)$.

\textbf{Case 1:}
$x = 0$. Since $\<1,1\> \not\in G(f)$, it follows that $f(1) \subseteq [0,1)$. It follows then that $f(z)\subseteq[0,z)$ for all $z\in(0,1]$, since $f(z)$ is an interval not containing $z$ and the graph of $f$ is connected. Therefore, as $f$ is surjective and $1 \not\in f(z)$ for all $z \in (0,1]$, $1 \in f(0)$. In fact, as $f(0)$ is connected and contains both $0$ and $1$, we know $f(0) = [0,1]$. Then for $z \in (0,1]$, $f(f(z))=f(z)\subseteq[0,z)\not=[0,1]$, and thus $0\not\in f(z)$, that is, $f(z)\subseteq(0,z)$.

Now consider $f(1)=[a,b]\subseteq(0,1)$. Then $f(f(1))=f([a,b])=[a,b]$, and in particular, $f(a)\subseteq[a,b]$. However, since $a\in(0,1]$, we also have $f(a)\subseteq(0,a)$, a contradiction.


\textbf{Case 2:}
$x = 1$. The proof is analogous to Case 1.


\textbf{Case 3:}
$x\in(0,1)$. Since $f(0)\subseteq(0,1]$ and the graph of $f$ is connected, $0\not\in f([0,x))$. Likewise, $f(1)\subseteq[0,1)$ and $1\not\in f((x,1])$. Since $f$ is surjective, $1\in f(v)$ for some $v\in[0,x]$ and $0\in f(w)$ for some $w\in[x,1]$. Let us say $f(v) = [k,1]$.

We note that in fact, $v=x$. If $v<x$ could hold, then $f(v)=[k,1]\subseteq(v,1]$. Then $f(f(v))=f([k,1])=[k,1]$, which forces $k\leq x$ as $1\not\in f((x,1])$. This results in a contradiction, as then $f([x,1])\subseteq f([k,1])=[k,1]$ while $0\in f(w)\subseteq f([x,1])$. So $[x,1]\subseteq f(x)$, and as $1\not\in f(1)$, we have that $x\not\in f(1)$ by the idempotence of $f$.

So, $f(1) = [a,b]$ where $[a,b] \subseteq [0,x)$ or $[a,b] \subseteq (x,1)$. If $[a,b] \subseteq (x,1)$, then since $a > x$, $f(a) \subseteq [0,a)$ while $f(a)\subseteq f([a,b])=f(f(1))=f(1)=[a,b]$, contradiction. On the other hand, if $[a,b] \subseteq [0,x)$, then since $b<x$, $f(b) \subseteq (b,1]$ while $f(b) \subseteq f([a,b])=f(f(1)=f(1)=[a,b]$, another contradiction. Since each possible case results in a contradiction, the proof is complete.

\end{proof}

\begin{lemma} \label{big lemma}
If $f: [0,1] \rightarrow C([0,1])$ is u.s.c., idempotent, and surjective, but $f \not=\iota$, then $G(f)$ satisfies condition $\Gamma$.
\end{lemma}

\begin{proof}
If $\Delta$ is a subset of $G(f)$, then since $f \not=\iota$, there must exist some point $\<x,y\> \in G(f)$ with $x \not= y$, and the result follows. So, suppose $\Delta$ is not a subset of $G(f)$. The result follows in either of two possible cases.

\textbf{Case 1.} There exist $x<y$ in $[0,1]$ such that $\<x,x\>$ and $\<y,y\>$ are elements of $G(f)$ but $\<p,p\>$ is not for all $x<p<y$.

Because $f$ is continuum-valued, we may assume $\min(f(p))>p$ for all $p\in(x,y)$. (The case $\max(f(p))<p$ is similar.) If $\max(f(x))<y$, then by the definition of u.s.c., there is some $p\in(x,y)$ such that $f(p)=[a,b]$ where $b<y$. It then follows from idempotence of $f$ that $f(b)\subseteq [a,b]\cap(b,1]=\emptyset$, a contradiction. Thus $\max(f(x))\geq y$ and therefore $\<x,y\>$, $\<x,x\>$, and $\<y,y\>$ all belong to $G(f)$.

\textbf{Case 2.} There exists $[x,y]\subsetneq[0,1]$ with $x<y$ and $\<p,p\>\in G(f)$ exactly when $p\in[x,y]$. Since $\Delta$ is not a subset of $G(f)$, this means either $0 \not\in [x,y]$ or $1 \not\in [x,y]$. Let us suppose $0\not\in[x,y]$. (The case $1\not\in[x,y]$ is similar.) Choose the minimal $z\in[0,1]$ satisfying $0\in f(z)$. Because $0 \not\in f(0)$, we know $f(0) \subseteq (0,1]$ and thus (because $G(f)$ is connected) $f(p) \subseteq (p,1]$ for $p \in [0,x)$. Thus, $z \in [x,1]$.

If $z=x$, then we claim that $\max(f(x))>x$. As justification, suppose not. Then $f(x) = [0,x]$ and $f([0,x]) = [0,x]$. Now if $x \in f(0)$, then $[0,x] = f(x) \subseteq f(f(0)) = f(0)$, which would imply that $0 \in f(0)$, contradicting the earlier statement that $0 \not\in f(0)$. So, $f(0) = [a,b] \subseteq (0,x)$. That means $f([a,b]) = [a,b]$, so $f(b) \subseteq [a,b]$. However, $f(b) \subseteq (b,1]$ also; this is also a contradiction. So indeed, $\max(f(x))>x$, and the result follows.

If $z\in(x,y]$, then the result follows immediately.

Finally, we demonstrate that $z\in(y,1]$ results in a contradiction. Let $f(z)=[0,d]\subseteq[0,z)$. Since $[0,d]=f(z)=f(f(z))=f([0,d])$, there is some $z'\in[0,d]\subseteq[0,z)$ satisfying $0\in f(z')$, even though $z$ was chosen to be minimal.
\end{proof}

\begin{lemma} \label{general lemma}
If $f:[0,1] \to 2^{[0,1]}$ is u.s.c. and idempotent and $G(f)$ satisfies condition $\Gamma$, then $\varprojlim\<[0,1],f,\kappa\>$ contains a copy of $\kappa + 1$.
\end{lemma}

\begin{proof}
As $G(f)$ satisfies $\Gamma$, let us say without loss of generality that $\<x,x\>, \<y,y\>$ and $\<x,y\>$ are elements of $G(f)$. $\varprojlim\<[0,1],f,\kappa\>$ contains the points $\textbf{x}_\alpha$ for $\alpha\leq\kappa$, defined by $\textbf{x}_\alpha(\beta)=y$ for $\beta<\alpha$ and $\textbf{x}_\alpha(\beta)=x$ otherwise. It's easy to see that $\{\textbf{x}_\alpha:\alpha\leq\kappa\}$ is a copy of $\kappa+1$.
\end{proof}

\begin{theorem} \label{main theorem}
If $f:[0,1]\to C([0,1])$ is u.s.c., idempotent, and surjective, but $f\not=\iota$, then $\varprojlim\<[0,1],f,\kappa\>$ contains a copy of $\kappa+1$.
\end{theorem}

\begin{proof}
The result follows immediately from Lemma \ref{big lemma} and Lemma \ref{general lemma}.
\end{proof}

\begin{corollary} \label{main corr}
An inverse limit with a single u.s.c., idempotent, surjective bonding function $f: [0,1] \rightarrow C([0,1])$ and index set $\kappa$, an ordinal, is only metric (in fact, Corson compact) in the case that $\kappa$ is countable or the bonding map is trivial.
\end{corollary}

\begin{proof}
If $f = \iota$, then of course the inverse limit is homeomorphic to $[0,1]$. So, suppose $f \ne \iota$. If $\kappa$ is countable, then because the inverse limit is a subspace of $[0,1]^{\kappa}$, it is metric. However, if $\kappa$ is uncountable, then by Theorem \ref{main theorem}, the inverse limit contains a copy of $\kappa + 1$ and is therefore non-metric.
\end{proof}

\section{Examples}

The following example helped inspire the main theorem of this paper.

\begin{example} \label{ex1} Let $f: [0,1] \rightarrow C([0,1])$ be given by $f(0) = [0,1]$ and $f(x) = \{1\}$ for each $x \ne 0$. Then $\varprojlim \langle [0,1], f, \kappa \rangle$ is metric exactly when $\kappa$ is a countable ordinal.
\end{example}

\begin{proof} $f$ is u.s.c. because $G(f)$ is closed, and obviously $f$ is continuum-valued and surjective. Since $f([0,1]) = [0,1]$ and $f(\{1\}) = \{1\}$, by Lemma \ref{idemlemma}, $f$ is idempotent. Thus, Corollary \ref{main corr} applies. (Note that one could also verify directly that $G(f)$ satisfies condition $\Gamma$ because $G(f)$ contains the ordered pairs $\langle 0,0 \rangle, \langle 1,1 \rangle, \langle 0,1 \rangle$).
\end{proof}

The discussion of this example in \cite{varagona} implies that this inverse limit is homeomorphic to a metric arc whenever $\kappa$ is a limit ordinal satifying $0 < \kappa < \omega_1$, but homeomorphic to the compactified long line when $\kappa = \omega_1$.
\

We give one more simple example to show that the surjectivity of the bonding function $f$ is necessary.

\begin{example} \label{ex2} Let $f: [0,1] \rightarrow C([0,1])$ be given by $f(x) = \{x\}$ for $0 \le x < 1/2$ and $f(x) = 1/2$ for $x \ge 1/2$.
\end{example}

Clearly $f$ is u.s.c. and idempotent and $f \ne \iota$; however, $G(f)$ does not satisfy condition $\Gamma$. Indeed, $\varprojlim \langle [0,1], f, \kappa \rangle$ is homeomorphic to a metric arc for all $\kappa > 0$.

\bibliographystyle{plain}
\begin{thebibliography}{10}

\smallskip

\bibitem{char roe} Wlodzimierz J. Charatonik and Robert P. Roe, {\it On Mahavier Products}, Topology and its Applications,
166, (2014), 92-97.

\smallskip

\bibitem{ingram mahavier} W. T. Ingram, William S. Mahavier, {\it Inverse Limits: from Continua to Chaos}, Springer, Developments in Mathematics (vol. 25), 2012.

\smallskip

\bibitem{i m paper} W. T. Ingram, William S. Mahavier, {\it Inverse limits of upper semi-continuous set valued functions}, Houston Journal of Mathematics, vol. 32 (2006) no. 1, 119-130.

\smallskip

\bibitem{ingram intro} W. T. Ingram, {\it An introduction to inverse limits with set-valued functions}, Springer Briefs in Mathematics, 2012.

\smallskip

\bibitem{Kunen} Kenneth Kunen, {\it Set Theory: An Introduction to Independence Proofs}, Elsevier B.V., 1980.

\smallskip

\bibitem{nall 2cell} Van Nall, {\it Inverse limits with set valued functions}, Houston Journal of Mathematics, 37 (2011), no. 4, 1323-1332.

\smallskip

\bibitem{nall connected} Van Nall, {\it Connected inverse limits with a set-valued function}, Topology Proc. 40 (2012), 167-177.

\smallskip

\bibitem{varagona} Scott Varagona, {\it Generalized Inverse Limits Indexed by Totally Ordered Sets}, preprint, 2015.

\smallskip

\bibitem{vernon} Patrick Vernon, {\it Inverse limits of set-valued functions indexed by the integers}, Topology Applications 171 (2014), 35-40.
\end{thebibliography}


\end{document}