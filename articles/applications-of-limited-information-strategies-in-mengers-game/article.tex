\documentclass{amsart}
\usepackage{amsmath}
\usepackage{amsthm}
\usepackage{amssymb}

\usepackage{../../clontzDefinitions}

\usepackage{tikz}
\usetikzlibrary{matrix}

\newtheorem{theorem}{Theorem}[section]
\newtheorem{proposition}[theorem]{Proposition}
\newtheorem{lemma}[theorem]{Lemma}
\newtheorem{corollary}[theorem]{Corollary}


\theoremstyle{definition}
\newtheorem{definition}[theorem]{Definition}
\newtheorem{game}[theorem]{Game}
\newtheorem{example}[theorem]{Example}
\newtheorem{question}[theorem]{Question}

\parskip=0.5em



\begin{document}

% \title{Proximal compact spaces are Corson compact\tnoteref{t1}}
% \tnotetext[t1]{2010 Mathematics Subject Classification. 54E15, 54D30, 54A20.}
\title{Applications of limited information strategies in Menger's game}

% \author[aub]{S.~Clontz\fnref{fn1}}
% \ead{steven.clontz@auburn.edu}
% \author[aub]{G.~Gruenhage\fnref{fn2}}
% \ead{gruengf@auburn.edu}

% \address[aub]{Department of Mathematics, Auburn University,
%  Auburn, AL 36830}


\author{Steven Clontz}
\address{Department of Mathematics and Statistics, UNC Charlotte,
Charlotte, NC 28262}
\email{steven.clontz@gmail.edu}
\urladdr{stevenclontz.com}

\keywords{Menger's property, Menger's game, limited information strategies}

\subjclass[2010]{03E35, 54D20, 54D45, 91A44}


\begin{abstract}
  I need an abstract
\end{abstract}


\maketitle

I require an introduction.

\section{The Menger property and game}

Recall the following definition.

\begin{definition}
  A space \(X\) is Menger if for every sequence \(\<\mc U_0,\mc U_1,\dots\>\)
  of open covers of \(X\) there exists a sequence
  \(\<\mc F_0,\mc F_1,\dots\>\) such that \(\mc F_n\subseteq \mc U_n\),
  \(|\mc F_n|<\omega\), and \(\bigcup_{n<\omega}\mc F_n\) is a cover of \(X\).
\end{definition}

Note that many authors refer to this property as \(S_{fin}(\mc O,\mc O)\),
where \(\mc O\) is the collection of open covers of \(X\), and
\(S_{fin}(A,B)\) denotes the selection property such that for each
sequence in \(A^\omega\), there are finite subsets of each entry for which
the union of these subsets belongs in \(B\).

\begin{proposition}
  \(X\) is \(\sigma\)-compact
    \(\Rightarrow\)
  \(X\) is Menger
    \(\Rightarrow\)
  \(X\) is Lindel\"of.
\end{proposition}

None of these implications may be reversed; the irrationals are a simple example
of a Lindel\"of space which is not Menger, and we'll see several examples of
Menger spaces which are not \(\sigma\)-compact.

It will be convenient to consider subsets of \(X\) rather than subsets of
the open covers.

\begin{definition}
  For each cover \(\mc U\) of \(X\), \(S\subseteq X\) is
  \term{\(\mc U\)-finite} if
  there exists a finite subcollection of \(\mc U\) which covers \(S\).
\end{definition}

Of course, a compact space is \(\mc U\)-finite for all open covers \(\mc U\).

\begin{proposition}
  A space \(X\) is Menger if and only if
  for every sequence \(\<\mc U_0,\mc U_1,\dots\>\)
  of open covers of \(X\) there exists a sequence
  \(\<F_0,F_1,\dots\>\) such that \(F_n\subseteq X\), \(F_n\) is
  \(\mc U_n\)-finite, and \(X=\bigcup_{n<\omega}F_n\).
\end{proposition}

This is the characterization we will use in this paper.
A game version of the Menger property is also often considered.

\begin{game}
  Let \(\menGame{X}\) denote the \term{Menger game} with players \(\pl C\), \(\pl F\).
  In round \(n\), \(\pl C\) chooses an open cover \(\mc U_n\), followed by \(\pl F\)
  choosing a \(\mc U_n\)-finite subset \(F_n\) of \(X\).

  \(\pl F\) wins the game if \(X = \bigcup_{n<\omega}F_n\),
  and \(\pl C\) wins otherwise.
\end{game}

As with the Menger property, authors usually characterize this game using
finite subsets \(\mc F_n\)
of \(\mc U_n\) instead (and often refer to it as \(G(\mc O,\mc O)\) as
with the selection property above).
This is obviously equivalent in the case of perfect
information, and is also equivalent
in the case of limited information, provided \(\pl F\)
knows \(\mc U_n\) during round \(n\). However, we make this change as we
will investigate \(0\)-Markov strategies which consider no moves of
the opponent, and instead rely only on the current round number.

This game may be used to characterize the Menger property.

\begin{definition}
  If \(\pl A\) has a winning strategy for a game \(G\) (which defeats every
  possible counterattack by her opponent), then we write \(\pl A\win G\).
\end{definition}

\begin{theorem}[Hurewicz \cite{MR1544773}]
  A space \(X\) is Menger if and only if \(\pl C\notwin\menGame{X}\).
\end{theorem}


\section{Limited information strategies}

Recall the following definitions.

\begin{definition}
  A \(k\)-tactical strategy for a game \(G\) with moveset \(M\) is a function
  \(\sigma:M^{\leq k}\to M\); intuitively, it is a strategy which only considers
  the previous \(k\) moves of the opponent. If a winning \(k\)-tactical strategy
  exists for \(\pl P\) in the game \(G\), then we write \(\pl P\ktactwin{k} G\).
\end{definition}

\begin{definition}
  A \(k\)-Markov strategy for a game \(G\) with moveset \(M\) is a function
  \(\sigma:M^{\leq k}\times\omega\to M\); intuitively, it is a strategy which
  only considers the previous \(k\) moves of the opponent and the round number.
  If a winning \(k\)-Markov strategy exists for
  \(\pl P\) in the game \(G\), then we write \(\pl P\kmarkwin{k} G\).
\end{definition}

We will call \(k\)-tactical strategies ``\(k\)-tactics'' and \(k\)-Markov strategies
``\(k\)-marks''. If the \(k\) is omitted then it is assumed that \(k=1\). In
addition, note that some authors refer to tactics as
\term{stationary strategies}.

Tactics will be of interest in a game discussed later; proving
the following is an easy exercise.

\begin{proposition}
  \(X\) is compact if and only if
  \(\pl F \tactwin \menGame{X}\) if and only if
  \(\pl F \ktactwin{k+1} \menGame{X}\) for some \(k<\omega\).
\end{proposition}

Essentially, because \(\pl C\) may repeat the same finite sequence of open covers,
\(\pl F\) needs to be seeded with knowledge of the round number to prevent being
trapped in a loop.

If \(\pl F\)'s memory of \(\pl C\)'s past moves is bounded, then
there is no need to consider more than the two most recent moves. The
intuitive reason is that \(\pl C\) could simply play the same cover repeatedly
until \(\pl F\)'s memory is exhausted, in which case \(\pl F\) would only ever
see the change from one cover to another.

\begin{theorem}
  For each \(k<\omega\), \(F \kmarkwin{(k+2)} \menGame{X}\)
  if and only if \(F \kmarkwin2 \menGame{X}\).
\end{theorem}

\begin{proof}
  Let \(\sigma\) be a winning \((k+2)\)-mark. We define the \(2\)-mark \(\tau\) as
  follows:
    \[
      \tau(\<\mc U\>,0)
        =
      \bigcup_{m<k+2}
        \sigma(\<\underbrace{\mc U,\dots,\mc U}_{m+1}\>,m)
    \]
    \[
      \tau(\<\mc U,\mc V\>,n+1)
        =
      \bigcup_{m<k+2}
        \sigma(\<
          \underbrace{\mc U,\dots,\mc U}_{k+1-m},
          \underbrace{\mc V,\dots,\mc V}_{m+1}
        \>,(n+1)(k+2)+m)
    \]

  Let \(\<\mc U_0,\mc U_1,\dots\>\) be an attack by \(\pl C\) against \(\tau\).
  Then consider the attack
    \[
      \<
        \underbrace{\mc U_0,\dots,\mc U_0}_{k+2},
        \underbrace{\mc U_1,\dots,\mc U_1}_{k+2},
        \dots
      \>
    \]
  by \(\pl C\) against \(\sigma\). Since \(\sigma\) is a winning \((k+2)\)-mark,
    \[
      X
        =
      \bigcup_{m<k+2}
        \sigma(\<\underbrace{\mc U_0,\dots,\mc U_0}_{m+1}\>,m)
      \cup
      \bigcup_{n<\omega}
      \bigcup_{m<k+2}
        \sigma(\<
          \underbrace{\mc U_n,\dots,\mc U_n}_{k+1-m},
          \underbrace{\mc U_{n+1},\dots,\mc U_{n+1}}_{m+1}
        \>,(n+1)(k+2)+m)
    \]
    \[
      =
      \tau(\<\mc U_0\>,0)
      \cup
      \bigcup_{n<\omega}
      \tau(\<\mc U_n,\mc U_{n+1}\>,n+1)
    \]
  Thus \(\tau\) is a winning \(2\)-mark.
\end{proof}

A natural question arises: is there an example of a space \(X\) for which
\(\pl F\kmarkwin{2}\menGame{X}\) but \(\pl F\notmarkwin\menGame{X}\)? We quickly
see that perhaps the simplest example of a Lindel\"of non-\(\sigma\)-compact
space has this property.

\begin{definition}
  For any cardinal \(\kappa\), let \(\oneptlind\kappa=\kappa\cup\{\infty\}\) denote
  the \term{one-point Lindel\"of-ication} of discrete \(\kappa\), where points in
  \(\kappa\) are isolated, and the neighborhoods of \(\infty\) are the co-countable
  sets containing it.
\end{definition}

\begin{proposition}
  \(\pl F \notmarkwin \menGame{\oneptlind{\omega_1}}\); in fact,
  \(\oneptlind{\omega_1}\) is not \(\sigma\)-compact.
\end{proposition}

The greatest advantage of a strategy which has knowledge of two or more previous
moves of the opponent, versus only knowledge of the most recent move, is the
ability to react to changes from one round to the next. It's this ability to
react that will give \(\pl F\) her winning \(2\)-Markov strategy in the Menger
game on \(\oneptlind\omega_1\). This may be proven directly; however, we will
postpone giving a proof so that we may take advantage of an equivalent
game characterization of this property.

\section{Scheepers' filling games}

We now turn to a related game whose \(k\)-tactics were studied by Marion
Scheepers in \cite{MR1129143}.

\begin{game}
  Let \(\schFillStrictGame\kappa\) denote
  \term{Scheepers' strict union filling game}
  with two players \(\pl C\), \(\pl F\). In round \(0\), \(\pl C\) chooses
  \(C_0\in[\kappa]^{\leq\omega}\), followed by \(\pl F\) choosing
  \(F_0\in[\kappa]^{<\omega}\). In round \(n+1\), \(\pl C\) chooses
  \(C_{n+1}\in[\kappa]^{\leq\omega}\) such that \(C_{n+1}\supset C_n\), followed
  by \(\pl F\) choosing \(F_{n+1}\in[\kappa]^{<\omega}\).

  \(\pl F\) wins the game if
  \(\bigcup_{n<\omega} F_n\supseteq\bigcup_{n<\omega} C_n\); otherwise,
  \(\pl C\) wins.
\end{game}

In \(\menGame{\oneptlind\kappa}\), \(\pl C\) essentially chooses a countable set
to not include in her neighborhood of \(\infty\), followed by \(\pl F\) choosing
a finite subset of this complement to cover during each round. Thus,
\(\pl F\) need only be concerned with the \textit{intersection} of the
countable sets chosen by \(\pl C\) in \(\menGame{\oneptlind\kappa}\), rather
than the union as in \(\schFillStrictGame\kappa\).

Another difference between these games:
Scheepers required that \(\pl C\) always choose strictly
growing countable sets. If the goal is to study tactics,
then \(\pl C\) cannot be allowed to trap \(\pl F\) in a loop by
repeating the same
moves. But by eliminating this requirement, the study can then turn to Markov
strategies, bringing the game further in line with the Menger game played upon
\(\oneptlind\kappa\).

We introduce a few games to make the relationship between Scheeper's
\(\schFillStrictGame\kappa\) and \(\menGame{\oneptlind\kappa}\) more precise.

\begin{game}
  Let \(\schFillGame\kappa\) denote the
  \term{union filling game} which proceeds analogously
  to \(\schFillStrictGame\kappa\), except that \(\pl C\)'s restriction in round \(n+1\)
  is reduced to \(C_{n+1}\supseteq C_n\).
\end{game}

\begin{game}
  Let \(\schFillInitialGame\kappa\) denote the
  \term{initial filling game} which proceeds analogously
  to \(\schFillGame\kappa\), except that \(\pl F\) wins whenever
  \(\bigcup_{n<\omega}F_n\supseteq C_0\).
\end{game}

\begin{game}
  Let \(\schFillIntGame\kappa\) denote the
  \term{intersection filling game} which proceeds analogously
  to \(\schFillInitialGame\kappa\), except that \(\pl C\) may choose any
  \(C_n\in[\kappa]^{\leq\omega}\) each round, and \(\pl F\) wins whenever
  \(\bigcup_{n<\omega}F_n\supseteq\bigcap_{n<\omega}C_n\).
\end{game}

\begin{figure}[h]
\begin{center}
\begin{tikzpicture}
  \matrix (m) [matrix of math nodes,row sep=3em,column sep=1em,minimum width=2em]
  {
    \pl F \kmarkwin{k+1}\menGame{\oneptlind\kappa} &
    \pl F \kmarkwin{k+1}\schFillIntGame\kappa \\

    \pl F \kmarkwin{k+1}\schFillInitialGame\kappa &
    \pl F \kmarkwin{k+1}\schFillGame\kappa \\

    &
    \pl F \ktactwin{k+1}\schFillStrictGame\kappa \\
  };
  \path[>=latex,->]
    (m-1-1) edge (m-2-1)
    (m-1-2) edge (m-2-2)
    (m-3-2) edge (m-2-2);
  \path[>=latex,<->]
    (m-1-1) edge (m-1-2)
    (m-2-1) edge (m-2-2);
\end{tikzpicture}
\end{center}
\caption{Diagram of Scheepers/Menger game implications}
\label{fillingGamesDiagram}
\end{figure}

\begin{theorem}
For any cardinal \(\kappa\geq\omega\) and integer \(k<\omega\),
Figure \ref{fillingGamesDiagram} holds.
\end{theorem}

\begin{proof}
  \(\pl F \kmarkwin{k+1}\menGame{\oneptlind\kappa}
    \Rightarrow
  \pl F \kmarkwin{k+1}\schFillIntGame\kappa\):
  Let \(\sigma\) be a winning \(k\)-mark for \(\pl F\) in
  \(\menGame{\oneptlind\kappa}\). Let \(\mc U(C)\) (resp. \(\mc U(s)\)) convert each
  countable subset \(C\) of \(\kappa\) (resp. finite sequence \(s\) of such subsets)
  into the open cover \([C]^1\cup\{\oneptlind\kappa\setminus C\}\)
  (resp. finite sequence of such open covers). Then \(\tau\) defined by
    \[
      \tau(s\concat\<C\>,n)
        =
      C\cap\sigma(\mc U(s\concat\<C\>),n)
    \]
  is a winning \((k+1)\)-mark for \(\pl F\) in \(\schFillIntGame\kappa\).

  \(\pl F \kmarkwin{k+1}\schFillIntGame\kappa
    \Rightarrow
  \pl F \kmarkwin{k+1}\menGame{\oneptlind\kappa}\):
  Let \(\sigma\) be a winning \(k\)-mark for \(\pl F\) in
  \(\schFillIntGame\kappa\). Let \(C(\mc U)\) (resp. \(C(s)\)) convert each open
  cover \(\mc U\) of \(\oneptlind\kappa\) (resp. finite sequence \(s\) of such covers)
  into a countable set \(C\) which is the complement of some neighborhood of
  \(\infty\) in \(\mc U\) (resp. finite sequence of such countable sets).
  Then \(\tau\) defined by
    \[
      \tau(s\concat\<\mc U\>,n)
        =
      (\oneptlind\kappa\setminus C(\mc U))
        \cup
      \sigma(C(s\concat\<\mc U)\>),n)
    \]
  is a winning \((k+1)\)-mark for \(\pl F\) in \(\menGame{\oneptlind\kappa}\).

  \(\pl F \kmarkwin{k+1}\schFillIntGame\kappa
    \Rightarrow
  \pl F \kmarkwin{k+1}\schFillInitialGame\kappa\):
  Let \(\sigma\) be a winning \(k\)-mark for \(\pl F\) in
  \(\schFillIntGame\kappa\). \(\sigma\) is also a winning \(k\)-mark for \(\pl F\)
  in \(\schFillInitialGame\kappa\).

  \(\pl F \kmarkwin{k+1}\schFillInitialGame\kappa
    \Rightarrow
  \pl F \kmarkwin{k+1}\schFillGame\kappa\): Let \(\sigma\) be a winning \(k\)-mark for \(\pl F\) in
  \(\schFillInitialGame\kappa\). For each finite sequence \(s\), let \(t\preceq s\) mean \(t\)
  is a final subsequence of \(s\). Then \(\tau\) defined by
    \[
      \tau(s\concat\<C\>,n)
        =
      \bigcup_{t\preceq s, m\leq n}
      \sigma(t\concat\<C\>,m)
    \]
  is a winning \((k+1)\)-mark for \(\pl F\) in \(\schFillGame\kappa\).

  \(\pl F \kmarkwin{k+1}\schFillGame\kappa
    \Rightarrow
  \pl F \kmarkwin{k+1}\schFillInitialGame\kappa\):
  Let \(\sigma\) be a winning \(k\)-mark for \(\pl F\) in
  \(\schFillGame\kappa\). \(\sigma\) is also a winning
  \((k+1)\)-mark for \(\pl F\)
  in \(\schFillInitialGame\kappa\).

  \(\pl F \ktactwin{k+1}\schFillStrictGame\kappa
    \Rightarrow
  \pl F \kmarkwin{k+1}\schFillGame\kappa\):
  Let \(\sigma\) be a winning \(k\)-tactic for \(\pl F\) in
  \(\schFillStrictGame\kappa\). For each countable subset \(C\) of \(\kappa\), let \(C+n\)
  be the union of \(C\) with the \(n\) least ordinals in \(\kappa\setminus C\).
  Then \(\tau\) defined by
    \[
      \tau(\<C_0,\dots,C_i\>,n)
        =
      \sigma(\<C_0+(n-i),\dots,C_i+n\>)
    \]
  is a winning \((k+1)\)-mark for \(\pl F\) in \(\schFillGame\kappa\).
\end{proof}

While we have not shown a direct implication between the Menger game and
Scheeper's original filling game, Scheepers introduced the statement
\(\alcompS\kappa\) relating to the almost-compatability of functions
from countable subsets of \(\kappa\) into \(\omega\) which may be applied to
both.

\begin{definition}
  For two functions \(f,g\) we say \(f\) is \textbf{\(\mu\)-almost compatible} with
  \(g\) (\(f\alcomp_\mu g\)) if \(|\{x\in\dom(f)\cap\dom(g):f(x)\not=g(x)\}|<\mu\).
  If \(\mu=\omega\) then we say \(f,g\) are \textbf{almost compatible}
  (\(f\alcomp g\)).
\end{definition}

\begin{definition}
  \(\alcompS\kappa\) states that there exist functions
  \(f_A:A\to\omega\) for each \(A\in[\kappa]^{\leq\omega}\) such that
  \(|\{\alpha\in A:f_A(\alpha)\leq n\}|<\omega\) for all \(n<\omega\) and
  \(f_A\alcomp f_B\) for all \(A,B\in[\kappa]^\omega\).
  \footnote{
  This is equivalent to the original characterization given in
  \cite{MR1129143}: there exist injections \(g_A:A\to\omega\)
  such that \(g_A\alcomp g_B\) for all \(A,B\in[\kappa]^\omega\) and \(A\subset B\).
  }
\end{definition}

Scheepers went on to show that \(\alcompS\kappa\) implies
\(\pl F \ktactwin2 \schFillStrictGame\kappa\). This proof, along with the following
facts, give us inspiration for
finding a winning \(2\)-Markov strategy in the Menger game played on
\(\oneptlind\kappa\).

\begin{theorem}
  \(\alcompS{\omega_1}\) and \(\kappa>2^\omega\Rightarrow\neg\alcompS\kappa\)
  are theorems of \(ZFC\).
  \(\alcompS{2^\omega}\) is a theorem of \(ZFC+CH\) and consistent with
  \(ZFC+\neg CH\).
\end{theorem}

\begin{proof}
  For \(\alcompS{\omega_1}\), look at pg. 70 of \cite{MR597342}; this of course
  implies \(\alcompS{2^\omega}\) under \(CH\).
  \(\neg\alcompS{(2^\omega)^+}\) is shown by a cardinality argument
  in \cite{MR1129143}.
  The consistency result under \(ZFC+\neg CH\)
  is a lemma for the main theorem in \cite{MR1129143}.
\end{proof}

\begin{figure}[h]
\begin{center}
\begin{tikzpicture}
  \matrix (m) [matrix of math nodes,row sep=3em,column sep=1em,minimum width=2em]
  {
    &
    \pl F \kmarkwin2\menGame{\oneptlind\kappa} &
    \pl F \kmarkwin2\schFillIntGame\kappa \\

    \alcompS\kappa &
    \pl F \kmarkwin2\schFillInitialGame\kappa &
    \pl F \kmarkwin2\schFillGame\kappa \\

    &
    &
    \pl F \ktactwin2\schFillStrictGame\kappa \\
  };
  \path[>=latex,->]
    (m-2-1) edge (m-1-2)
    (m-2-1) edge (m-2-2)
    (m-2-1) edge (m-3-3)
    (m-1-2) edge (m-2-2)
    (m-1-3) edge (m-2-3)
    (m-3-3) edge (m-2-3);
  \path[>=latex,<->]
    (m-1-2) edge (m-1-3)
    (m-2-2) edge (m-2-3);
\end{tikzpicture}
\end{center}
\caption{Diagram of Filling/Menger game implications with \(\alcompS\kappa\)}
\label{fillingGamesDiagram2}
\end{figure}

\begin{theorem}
  \(\alcompS\kappa\) implies the game-theoretic results in
  Figure \ref{fillingGamesDiagram2}.
\end{theorem}

\begin{proof}
  Since \(\alcompS\kappa\Rightarrow\pl F \ktactwin2\schFillStrictGame\kappa\) was
  a main result of \cite{MR1129143}, it is sufficient to show that
  \(\alcompS\kappa\Rightarrow \pl F \kmarkwin2\schFillIntGame\kappa\).

  Let \(f_A\) for \(A\in[\kappa]^{\leq\omega}\) witness \(\alcompS\kappa\). We define
  the \(2\)-mark \(\sigma\) as follows:
    \[
      \sigma(\<A\>,0) = \{\alpha\in A: f_A(\alpha) \leq 0 \}
    \]
    \[
      \sigma(\<A,B\>,n+1)
        =
      \{
        \alpha \in A\cap B
      :
        f_B(\alpha) \leq n+1 \text{ or }
        f_A(\alpha)\not=f_B(\alpha)
      \}
    \]
  For any attack \(\<A_0,A_1,\dots\>\) by \(\pl C\) and
  \(\alpha\in\bigcap_{n<\omega}A_n\), either \(f_{A_n}(\alpha)\) is constant for
  all \(n\), or \(f_{A_n}(\alpha)\not=f_{A_{n+1}}(\alpha)\) for some \(n\);
  either way, \(\alpha\) is covered.
\end{proof}

\begin{corollary}
  \(\pl F \kmarkwin2\menGame{\oneptlind\omega_1}\).
\end{corollary}



\section{Menger game derived covering properties}

\begin{figure}[h]
\begin{center}
\begin{tikzpicture}
  \matrix (m) [matrix of math nodes,row sep=3em,column sep=1em,minimum width=2em]
  {
     \pl F \markwin \menGame{X} &
     \pl F \kmarkwin2 \menGame{X} &
     \pl F \win \menGame{X} &
     \pl C \notwin \menGame{X} \\

     \text{Markov Menger} &
     2\text{-Markov Menger} &
     \text{Perfect Menger} &
     \text{Menger} \\
  };
  \path[>=latex,<->]
    (m-1-1) edge (m-2-1)
    (m-1-2) edge (m-2-2)
    (m-1-3) edge (m-2-3)
    (m-1-4) edge (m-2-4);
  \path[>=latex,->]
    (m-1-1) edge (m-1-2)
    (m-2-1) edge (m-2-2)
    (m-1-2) edge (m-1-3)
    (m-2-2) edge (m-2-3)
    (m-1-3) edge (m-1-4)
    (m-2-3) edge (m-2-4);
  \path[>=latex,->,bend right=30, dotted]
    (m-1-2) edge node {\( \not \)} (m-1-1)
    (m-2-2) edge node {\( \not \)} (m-2-1)
    (m-1-4) edge node {\( \not \)} (m-1-3)
    (m-2-4) edge node {\( \not \)} (m-2-3);
  \path[>=latex,->,bend right=30, dotted]
    (m-1-3) edge node {\( ? \)} (m-1-2)
    (m-2-3) edge node {\( ? \)} (m-2-2);
\end{tikzpicture}
\end{center}
\caption{Diagram of covering properties related to the Menger game}
\label{menSpec}
\end{figure}

Limited information strategies for the Menger game naturally define a spectrum
of covering properties, see Figure \ref{menSpec}. However,
we do not know if the middle two properties are actually distinct.

\begin{question}\label{perfectTo2Mark}
  Does there exist a space \(X\) such that \(\pl F \win \menGame{X}\) but
  \(\pl F \notkmarkwin2 \menGame{X}\)?
\end{question}

Note that while it's consistent that
\(\pl F\kmarkwin2\menGame{\oneptlind{(2^\omega)}}\), \(\oneptlind{\kappa}\)
for \(\kappa>2^\omega\) is a candidate to answer the above question.

We are also interested in non-game-theoretic characterizations of these
covering properties. It has been known for some time that for metrizable spaces,
winning Menger spaces are exactly the \(\sigma\)-compact spaces, shown first
by Telgarksy in \cite{MR753073} and later directly by Scheepers in
\cite{MR1273523}.

In the interest of generality, we will first characterize the Markov Menger
spaces without any separation axioms.

\begin{definition}
  A subset \(Y\) of \(X\) is \term{relatively compact} to \(X\) if for every open
  cover of \(X\), there exists a finite subcollection which covers \(Y\).
\end{definition}

For example, any bounded subset of Euclidean space is relatively compact
whether it is closed or not. Actually, relative compactness can be thought
of as an analogue of boundedness for regular spaces.

\begin{proposition}
  For regular spaces, \(Y\) is relatively compact to \(X\) if and only if
  \(\cl Y\) is compact in \(X\).
  \footnote{
    It should be noted that some authors define relative compactness in
    this way, but such a definition creates pathological implications for
    non-regular spaces. For example, the singleton containing
    the particular point of an infinite space with the particular point topology
    would not be relatively compact since its closure is not compact, even
    though it is finite.
  }
\end{proposition}

\begin{proof}
  For any space, any subset of a compact set is relatively compact.

  Assume \(Y\) is relatively compact, let \(\mc U\) be an open cover of \(\cl Y\),
  and define \(x\in V_x\subseteq \cl{V_x}\subseteq U_x \in \mc U\) for each
  \(x\in X\). Then if we take a subcollection \(\mc F = \{V_{x_i} : i<n\}\) covering
  \(Y\) by relative compactness, then \(\{U_{x_i}:i<n\}\) is a finite subcollection
  of \(\mc U\) covering \(\cl Y\), showing compactness.
\end{proof}

We now begin the process of factoring out Scheeper's proof to reveal the
limited information implications at work.

\begin{lemma}
  Let \(\sigma(\mc{U}, n)\) be a Markov strategy for \(F\) in
  \(\menGame{X}\), and \(\mathfrak{C}\) collect all open covers of \(X\). Then the
  set
    \[
      R_n = \bigcap_{\mc{U}\in\mathfrak{C}} \sigma(\mc{U},n)
    \]
  is relatively compact to \(X\). If \(\sigma\) is a winning
  Markov startegy, then \(\bigcup_{n<\omega} R_n = X\).
\end{lemma}

\begin{proof}
  First, for every open cover \(\mc U\in\mathfrak C\),
  \(R_n\subseteq\sigma(\mc U,n)\) is covered by a finite subcollection of \(\mc U\).

  Suppose that \(x \not\in R_n\) for any \(n<\omega\). Then for each \(n\),
  pick \(\mc{U}_n\in\mathfrak{C}\) such that \(x\not\in \sigma(\mc{U}_n,n)\). Then
  \(\pl C\) may counter \(\sigma\) with the attack \(\<\mc U_0,\mc U_1,\dots\>\).
\end{proof}

\begin{definition}
  A \term{\(\sigma\)-relatively-compact} space is the countable union of
  relatively compact subsets.
\end{definition}

\begin{corollary}
  The following are equivalent:
  \begin{itemize}
    \item \(X\) is \(\sigma\)-relatively-compact
    \item \(\pl F \prewin \menGame X\)
    \item \(\pl F \markwin \menGame X\)
  \end{itemize}
\end{corollary}

\begin{proof}
  If \(X=\bigcup_{n<\omega} R_n\) for \(R_n\) relatively compact, then
  \(\sigma(n)=R_n\) is a winning predetermined strategy, which yields a
  winning Markov strategy. The previous lemma finishes the proof.
\end{proof}

\begin{corollary}
  Let \(X\) be a regular space. The following are equivalent:
  \begin{itemize}
    \item \(X\) is \(\sigma\)-compact
    \item \(X\) is \(\sigma\)-relatively-compact
    \item \(\pl F \prewin \menGame X\)
    \item \(\pl F \markwin \menGame X\)
  \end{itemize}
\end{corollary}

For Lindel\"of spaces, metrizability is characterized by regularity and
second-countability, the latter of which was essentially used by Scheepers in
this way:

\begin{lemma}
  Let \(X\) be a second-countable space. \(\pl F\win\menGame{X}\) if and only if
  \(\pl F\markwin\menGame{X}\).
\end{lemma}

\begin{proof}
  Let \(\sigma\) be a strategy for \(\pl F\), and note that
  it's sufficient to consider playthroughs with only basic open covers.

  So if \(\mc U_t\) is a basic open cover for \(t<s\in\omega^{<\omega}\), and
  \(\mc V\) is any basic open cover, we may choose a finite subcollection
  \(\mc F(s,\mc V)\) of \(\mc V\) such that
  \[
    \sigma(\<\mc U_{s\rest 1},\dots,\mc U_{s},\mc V\>)
      \subseteq
    \bigcup \mc F(s,\mc V)
  \]

  Note that there are only countably-many finite collections of basic open sets.
  Thus we may choose basic open covers \(\mc U_{s\concat\<n\>}\) for \(n<\omega\)
  such that for any basic open cover \(\mc V\), there exists \(n<\omega\) where
  \(\mc F(s,\mc V)=\mc F(s,\mc U_{s\concat\<n\>})\).

  Let \(t:\omega\to\omega^{<\omega}\) be a bijection. We define the Mark\"ov
  strategy \(\tau\) as follows:
  \[
    \tau(\<\mc V\>, n)
      =
    \bigcup \mc F(t(n),\mc V)
  \]

  Suppose there exists a counter-attack \(\<\mc V_0,\mc V_1,\dots\>\) of
  basic open covers which defeats \(\tau\). Then there exists
  \(f:\omega\to\omega\) such that, letting \(t(m_n)=f\rest n\):
  \[
    \begin{array}{rcl}
    x & \not\in & \tau(\<\mc V_{m_n}\>,m_n) \\
    & = & \bigcup\mc F(f\rest n,\mc V_{m_n}) \\
    & = & \bigcup\mc F(f\rest n,\mc U_{f\rest (n+1)}) \\
    & \supseteq & \sigma(\<\mc U_{f\rest 1},\dots,\mc U_{f\rest(n+1)}\>)
    \end{array}
  \]

  Thus \(\<\mc U_{f\rest 1}, \mc U_{f\rest 2},\dots\>\) is a successful
  counter-attack by \(\pl C\) against the perfect information strategy \(\sigma\).
\end{proof}

\begin{corollary}
  Let \(X\) be a second-countable space. The following are equivalent:
  \begin{itemize}
    \item \(X\) is \(\sigma\)-relatively-compact
    \item \(F \prewin \menGame{X}\)
    \item \(F \markwin \menGame{X}\)
    \item \(F \win \menGame{X}\)
  \end{itemize}
\end{corollary}

\begin{corollary}
  Let \(X\) be a metrizable space. The following are equivalent:
  \begin{itemize}
    \item \(X\) is \(\sigma\)-compact
    \item \(X\) is \(\sigma\)-relatively-compact
    \item \(F \prewin \menGame{X}\)
    \item \(F \markwin \menGame{X}\)
    \item \(F \win \menGame{X}\)
  \end{itemize}
\end{corollary}

\begin{proof}
  Each property implies Lindel\"of, so \(X\) may be assumed to be
  regular and second-countable.
\end{proof}



\section{Robustly Lindel\"of}

To help describe \(\pl F\kmarkwin2\menGame{X}\) topologically, we introduce a
subset variant of the Menger game and a related covering property.

\begin{game}
  Let \(\menGame{X,Y}\) denote the \term{Menger subspace game} which proceeds
  analogously to the Menger game, except that \(\pl F\) wins whenever
  \(\bigcup_{n<\omega} \mc F_n\) is a cover for \(Y\subseteq X\).
\end{game}

Note of course that \(\menGame{X,X}=\menGame{X}\).

\begin{definition}
  A subset \(Y\) of \(X\) is \term{relatively robustly Menger} if there exist
  functions \(r_{\mc V}:Y\to\omega\)
  for each open cover \(\mc V\) of \(X\) such that
  for all open covers \(\mc U,\mc V\) and numbers \(n<\omega\),
  the following sets are \(\mc V\)-finite:
    \[
      c(\mc V,n)=\{ x\in Y : r_{\mc V}(x)\leq n\}
    \]
    \[
      p(\mc U,\mc V,n+1)=\{ x\in Y : n<r_{\mc U}(x)<r_{\mc V}(x)\}
    \]
\end{definition}

\begin{definition}
  A space \(X\) is \term{robustly Menger} if it is relatively robustly
  Menger to itself.
\end{definition}

\begin{proposition}
  All \(\sigma\)-relatively-compact spaces are robustly Menger.
\end{proposition}

\begin{proof}
  If \(X=\bigcup_{n<\omega}R_n\), then for all \(\mc U\), let \(r_{\mc U}(x)\) be the
  least \(n\) such that \(x\in R_n\). Then \(c(\mc V,n)=\bigcup_{m\leq n}R_m\) and
  \(p(\mc U,\mc V) = \emptyset\).
\end{proof}

\begin{theorem}
  If \(Y\subseteq X\) is relatively robustly Menger, then
  \(\pl F\kmarkwin2\menGame{X,Y}\).
\end{theorem}

\begin{proof}
  We define the Markov strategy \(\sigma\) as follows.
  Let \(\sigma(\<\mc U\>,0)=c(\mc U,0)\), and let
  \(\sigma(\<\mc U,\mc V\>,n+1)=c(\mc V,n+1)\cup p(\mc U, \mc V,n+1)\).

  For any attack \(\<\mc{U}_0,\mc{U}_1,\dots\>\) by \(\pl C\) and \(x\in Y\),
  one of the following must occur:

  \begin{itemize}
    \item
      \(r_{\mc U_0}(x)=0\) and thus
      \(x\in c(\mc U_0,0)\subseteq \sigma(\<\mc U_0\>,0)\).

    \item
      \(r_{\mc U_0}(x)=N+1\) for some \(N\geq 0\) and:
      \begin{itemize}
        \item
          For all \(n\leq N\),
          \[
            r_{\mc U_{n+1}}(x)\leq N+1
          \]
          and thus
          \(x\in c(\mc U_{N+1},N+1) \subseteq
            \sigma(\<U_N,U_{N+1}\>,N+1)\).
        \item
          For some \(n \leq N\),
          \[ r_{\mc U_n}(x)\leq n \]
          and thus
          \(x\in c(\mc U_{n+1},n+1) \subseteq
            \sigma(\<U_n,U_{n+1}\>,n+1)\).
        \item
          For some \(n \leq N\),
          \[
            n < r_{\mc U_n}(x)\leq N+1 < r_{\mc U_{n+1}}(x)
          \]
         and thus
         \(x\in p(\mc U_n,\mc U_{n+1},n+1) \subseteq
          \sigma(\<U_n,U_{n+1}\>,n+1)\)
       \end{itemize}
  \end{itemize}
\end{proof}

\begin{theorem}
  \(\alcompS\kappa\) implies \(\oneptlind\kappa\) is
  robustly Menger, and thus
  \(\pl F\kmarkwin2\menGame{\oneptlind\kappa}\).
\end{theorem}

\begin{proof}
  Let \(f_A\) for \(A\in[\kappa]^{\leq\omega}\) witness \(\alcompS\kappa\) and fix
  \(A(\mc U)\in[\kappa]^{\leq\omega}\) for each open cover \(\mc U\) such that
  \(\oneptlind\kappa\setminus A(\mc U)\) is contained in some element of
  \(\mc U\).
  Then let \(r_{\mc U}(x)=0\) for \(x\in\oneptlind\kappa\setminus A(\mc U)\),
  and \(r_{\mc U}(\alpha)=f_{A(\mc U)}(\alpha)\) for \(\alpha\in A(\mc U)\).

  It follows that
    \[
      c(\mc U,n)
        =
      \left(\oneptlind\kappa\setminus A(\mc U)\right)
        \cup
      \{\alpha\in A(\mc U) : f_{A(\mc U)}(\alpha)\leq n \}
    \]
  is \(\mc U\)-finite, \(\bigcup_{n<\omega}c(\mc U,n)=X\), and
    \[
      p(\mc U,\mc V,n+1)
        =
      \{
        \alpha\in A(\mc U)\cap A(\mc V)
          :
        n < f_{A(\mc U)}(\alpha) < f_{A(\mc V)}(\alpha)
      \}
    \]
  is finite.
\end{proof}

We may also consider common (non-regular) counterexamples which are finer
than the usual Euclidean line.

\begin{definition}
  Let \(R_{\mathbb Q}\) be the real line with the topology generated by open
  intervals with or without the rationals removed.
\end{definition}

\begin{theorem}
  \(R_{\mathbb Q}\) is non-regular and non-\(\sigma\)-compact, but is
  second-countable and \(\sigma\)-relatively-compact.
\end{theorem}

\begin{proof}
  Compact sets in \(R_{\mathbb Q}\) can be shown to not contain open intervals,
  and thus are nowhere dense in nonmeager \(\mathbb R\),
  so \(R_{\mathbb Q}\) is not \(\sigma\)-compact. The usual base of
  intervals with rational endpoints (with or without rationals removed)
  witnesses second-countability.

  To see that \(R_{\mathbb Q}\) is \(\sigma\)-relatively compact, consider
  \([a,b]\setminus\mathbb{Q}\). Let \(\mc U\) be a cover of \(R_{\mathbb Q}\), and
  let \(\mc U'\) fill in the missing rationals for any open set in \(\mc U\).
  There is a finite subcover \(\mc V'\subseteq \mc U'\) for \([a,b]\) since
  \(\mc U'\) contains open sets from the Euclidean topology. Let
  \(\mc V= \{V\setminus\mathbb{Q}:V\in V'\}\): this is a finite refinment of
  \(\mc U\) covering \([a,b]\setminus\mathbb{Q}\), so \([a,b]\setminus\mathbb{Q}\)
  is relatively compact. It follows then that \(R_{\mathbb Q}\setminus\mathbb Q\)
  is \(\sigma\)-relatively-compact, and since \(\mathbb Q\) is countable,
  \(R_{\mathbb Q}\) is \(\sigma\)-relatively-compact. Non-regularity follows since
  regular and \(\sigma\)-relatively-compact implies \(\sigma\)-compact.
\end{proof}

\begin{definition}
  Let \(R_\omega\) be the real line with the topology generated by open
  intervals with countably many points removed.
\end{definition}

\begin{theorem}
  \(R_\omega\) is non-regular, non-second-countable, and
  non-\(\sigma\)-relatively-compact, but \(\pl F\win\menGame{R_\omega}\).
\end{theorem}

\begin{proof}
  The closure of any open set is its closure in the usual Euclidean topology,
  so \(R_\omega\) is not regular. If \(S\supseteq\{s_n:n<\omega\}\) for \(s_n\)
  discrete, then \(U_m=R_\omega\setminus\{s_n:m<n<\omega\}\) yields an
  infinite cover \(\{U_m:m<\omega\}\) with no finite subcollection covering \(S\),
  showing that all relatively compact sets are finite, and \(R_\omega\) is not
  \(\sigma\)-relatively-compact.

  Define the winning strategy \(\sigma\) for \(\pl F\) in \(\menGame{R_\omega}\) as
  follows: let \(\sigma(\mc{U}_0,\dots,\mc{U}_{2n})=[-n,n]\setminus C_n\)
  for some countable \(C_n=\{c_{n,m}: m<\omega\}\),
  and let \(\sigma(\mc{U}_0,\dots,\mc{U}_{2n+1})=\{c_{i,j} : i,j < n\}\).
  Non-second-countable follows since
  second-countable and \(\pl F\win\menGame{X}\) implies
  \(\sigma\)-relatively-compact.
\end{proof}

We will soon see that, assuming \(\alcompS{2^\omega}\), \(\pl F\) has a winning
\(2\)-Mark\"ov strategy for \(\menGame{R_\omega}\) as well.

\begin{proposition}
  Let \(\limitwin\) be either \(\kmarkwin{k}\) or \(\win\).
  If \(X=\bigcup_{i<\omega} X_i\) and \(\pl F \limitwin \menGame{X,X_i}\)
  for \(i<\omega\), then \(\pl F \limitwin \menGame{X}\)
\end{proposition}

\begin{proof}
  Let \(L\) be the \(k\)-Markov fog-of-war \(\mu_k\) (resp. the identity),
  and let \(\sigma_i\) be a \(k\)-Markov strategy (resp. perfect information
  strategy) for \(\pl F\) in \(\menGame{X,X_i}\).

  We define the \(k\)-Markov strategy (resp. perfect information strategy)
  \(\sigma\) for \(\menGame{X}\) as follows:
    \[
      \sigma\circ L(\<\mc U_0,\dots,\mc U_n\>)
        =
      \bigcup_{i\leq n}
      \sigma_i\circ L(\<\mc U_i,\dots,\mc U_n\>)
    \]

  Let \(\<\mc U_0,\mc U_1,\dots\>\) be a successful counter-attack by \(\pl C\)
  against \(\sigma\). Then there exists \(x\in X_i\) for
  some \(i<\omega\) such that \(x\) is not covered by
  \(\bigcup_{n<\omega}\sigma\circ L(\<\mc U_0,\dots,\mc U_n\>)\).
  It follows that \(x\) is not covered by
  \(\bigcup_{n<\omega}\sigma_i\circ L(\<\mc U_i,\dots,\mc U_{i+n}\>)\),
  and \(\<\mc U_i,\mc U_{i+1},\dots\>\) is a successful counter-attack
  by \(\pl C\) against \(\sigma_i\).
\end{proof}

\begin{theorem}
  If \(\alcompS{2^\omega}\), then \(\pl F\kmarkwin2\menGame{R_\omega}\).
\end{theorem}

\begin{proof}
  It's sufficient to show that \([0,1]\subseteq R_\omega\) is relatively
  robustly Menger. Let \(f_A\) witness \(\alcompS{2^\omega}\)
  for \(A\in[[a,b]]^{\leq\omega}\).
  For each open cover \(\mc U\), let \(A_{\mc U}\) be
  such that \([0,1]\setminus A_{\mc U}\) is \(\mc U\)-finite.
  Let \(r_{\mc U}(x)=0\) if \(x\in[0,1]\setminus A_{\mc U}\) and
  \(r_{\mc U}(x)=f_{\mc A_{\mc U}}(x)\) otherwise.

  It follows then that
    \[
      c(\mc U,n) = [0,1]\setminus\{x\in A_{\mc U}:f_{\mc A_{\mc U}}(x)>n\}
    \]
  is \(\mc U\)-finite and
    \[
      p(\mc U,\mc V,n+1)
        =
      \{x\in A_{\mc U}\cap A_{\mc V}
        :
      n<f_{\mc A_{\mc U}}(x)<f_{\mc A_{\mc V}}(x)\}
    \]
  is finite.
\end{proof}

\bibliographystyle{plain}
\bibliography{../../bibliography}

\end{document}