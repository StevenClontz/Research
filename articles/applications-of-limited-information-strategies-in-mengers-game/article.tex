\documentclass{amsart}
\usepackage{amsmath}
\usepackage{amsthm}
\usepackage{amssymb}

\usepackage{../../clontzDefinitions}

\usepackage{tikz}
\usetikzlibrary{matrix}
\usetikzlibrary{arrows}

\usepackage{tikz-cd}
\usetikzlibrary{decorations.markings}
\tikzset{not/.style={
  decoration={markings,
    mark= at position 0.5 with {
      \node[transform shape] (tempnode) {$\backslash$};
    }
  },
  postaction={decorate}
}}
\tikzset{maybe/.style={
  color=gray,
  decoration={markings,
    mark=at position 0.5 with {
      \node[color=black] (tempnode) {$?$};
    }
  },
  postaction={decorate}
}}

\newtheorem{theorem}{Theorem}[section]
\newtheorem{proposition}[theorem]{Proposition}
\newtheorem{lemma}[theorem]{Lemma}
\newtheorem{corollary}[theorem]{Corollary}


\theoremstyle{definition}
\newtheorem{definition}[theorem]{Definition}
\newtheorem{game}[theorem]{Game}
\newtheorem{example}[theorem]{Example}
\newtheorem{question}[theorem]{Question}

\parskip=0.5em



\begin{document}

% \title{Proximal compact spaces are Corson compact\tnoteref{t1}}
% \tnotetext[t1]{2010 Mathematics Subject Classification. 54E15, 54D30, 54A20.}
\title{Applications of limited information strategies in Menger's game}

% \author[aub]{S.~Clontz\fnref{fn1}}
% \ead{steven.clontz@auburn.edu}
% \author[aub]{G.~Gruenhage\fnref{fn2}}
% \ead{gruengf@auburn.edu}

% \address[aub]{Department of Mathematics, Auburn University,
%  Auburn, AL 36830}


\author{Steven Clontz}
\address{Department of Mathematics and Statistics, University of South Alabama,
Mobile, AL 36688}
\email{sclontz@southalabama.edu}
\urladdr{clontz.org}

\keywords{Menger property, Menger game, \(\sigma\)-compact spaces,
limited information strategies}

\subjclass[2010]{03E35, 54D20, 54D45, 91A44}


\begin{abstract}
  As shown by Telg\'arsky and Scheepers,
  winning strategies in the Menger game characterize
  \(\sigma\)-compactness amongst metrizable spaces. This is improved
  by showing that winning Markov strategies in the Menger game
  characterize \(\sigma\)-compactness amongst
  regular spaces, and that winning strategies may be improved to winning
  Markov strategies in second-countable spaces.
  An investigation of
  \(2\)-Markov strategies introduces a new topological property between
  \(\sigma\)-compact and Menger spaces.
\end{abstract}


\maketitle

\section{The Menger property and game}

\begin{definition}
  A space \(X\) is Menger if for every sequence \(\<\mc U_0,\mc U_1,\dots\>\)
  of open covers of \(X\) there exists a sequence
  \(\<\mc F_0,\mc F_1,\dots\>\) such that \(\mc F_n\subseteq \mc U_n\),
  \(|\mc F_n|<\omega\), and \(\bigcup_{n<\omega}\mc F_n\) is a cover of \(X\).
\end{definition}

Note that many authors refer to this property as \(S_{fin}(\mc O,\mc O)\)
\cite{MR1378387},
where \(\mc O\) is the collection of open covers of \(X\), and
\(S_{fin}(A,B)\) denotes the selection property such that for each
sequence in \(A^\omega\), there are finite subsets of each entry for which
the union of these subsets belongs in \(B\).

\begin{proposition}
  \(X\) is \(\sigma\)-compact
    \(\Rightarrow\)
  \(X\) is Menger
    \(\Rightarrow\)
  \(X\) is Lindel\"of.
\end{proposition}

None of these implications may be reversed; the irrationals are a simple example
of a Lindel\"of space which is not Menger, and we'll see several examples of
Menger spaces which are not \(\sigma\)-compact.

It will be convenient to consider subsets of \(X\) rather than subsets of
the open covers.

\begin{definition}
  For each cover \(\mc U\) of \(X\), \(S\subseteq X\) is
  \term{\(\mc U\)-finite} if
  there exists a finite subcollection of \(\mc U\) which covers \(S\).
\end{definition}

Of course, a compact space is \(\mc U\)-finite for all open covers \(\mc U\).

\begin{proposition}
  A space \(X\) is Menger if and only if
  for every sequence \(\<\mc U_0,\mc U_1,\dots\>\)
  of open covers of \(X\) there exists a sequence
  \(\<F_0,F_1,\dots\>\) such that \(F_n\subseteq X\), \(F_n\) is
  \(\mc U_n\)-finite, and \(X=\bigcup_{n<\omega}F_n\).
\end{proposition}

This is the characterization we will use in this paper.
A game version of the Menger property is also often considered.

\begin{game}
  Let \(\menGame{X}\) denote the \term{Menger game} with players \(\pl C\), \(\pl F\).
  In round \(n\), \(\pl C\) chooses an open cover \(\mc U_n\), followed by \(\pl F\)
  choosing a \(\mc U_n\)-finite subset \(F_n\) of \(X\).

  \(\pl F\) wins the game if \(X = \bigcup_{n<\omega}F_n\),
  and \(\pl C\) wins otherwise.
\end{game}

As with the Menger property, authors usually characterize this game using
finite subsets \(\mc F_n\)
of \(\mc U_n\) instead (and often refer to it as the selection game
\(G_{fin}(\mc O,\mc O)\)).
This is obviously equivalent in the case of perfect
information, and is also equivalent
in the case of limited information, provided \(\pl F\)
knows \(\mc U_n\) during round \(n\). However, we make this change as we
will investigate \(0\)-Markov strategies which consider no moves of
the opponent, and instead rely only on the current round number.

This game may be used to characterize the Menger property.

\begin{definition}
  If \(\pl A\) has a winning strategy for a game \(G\) (which defeats every
  possible counterattack by her opponent), then we write \(\pl A\win G\).
\end{definition}

\begin{theorem}[Hurewicz \cite{MR1544773}]
  A space \(X\) is Menger if and only if \(\pl C\notwin\menGame{X}\).
\end{theorem}


\section{Limited information strategies}

\begin{definition}
  A \(k\)-tactical strategy for a game \(G\) with moveset \(M\) is a function
  \(\sigma:M^{\leq k}\to M\); intuitively, it is a strategy which only considers
  the previous \(k\) or less moves of the opponent \(t\in M^{\leq k}\), and yields
  the appropriate move \(\sigma(t)\) for the player using it.
  If a winning \(k\)-tactical strategy
  exists for \(\pl P\) in the game \(G\), then we write \(\pl P\ktactwin{k} G\).
\end{definition}

\begin{definition}
  A \(k\)-Markov strategy for a game \(G\) with moveset \(M\) is a function
  \(\sigma:M^{\leq k}\times\omega\to M\); intuitively, it is a strategy which
  only considers the previous \(k\) or less moves of the opponent
  \(t\in M^{\leq k}\) and the round number \(n<\omega\),
  and yields the appropriate move \(\sigma(t,n)\) for the player using it.
  If a winning \(k\)-Markov strategy exists for
  \(\pl P\) in the game \(G\), then we write \(\pl P\kmarkwin{k} G\).
\end{definition}

We will call \(k\)-tactical strategies ``\(k\)-tactics'' and \(k\)-Markov strategies
``\(k\)-marks''. If the \(k\) is omitted then it is assumed that \(k=1\). In
addition, note that some authors refer to tactics as
\term{stationary strategies}.

Tactics will be of interest in a game discussed later; proving
the following is an easy exercise.

\begin{proposition}
  \(X\) is compact if and only if
  \(\pl F \tactwin \menGame{X}\) if and only if
  \(\pl F \ktactwin{k+1} \menGame{X}\) for some \(k<\omega\).
\end{proposition}

Essentially, because \(\pl C\) may repeat the same finite sequence of open covers,
\(\pl F\) needs to be seeded with knowledge of the round number to prevent being
trapped in a loop.

If \(\pl F\)'s memory of \(\pl C\)'s past moves is bounded, then
there is no need to consider more than the two most recent moves. The
intuitive reason is that \(\pl C\) could simply play the same cover repeatedly
until \(\pl F\)'s memory is exhausted, in which case \(\pl F\) would only ever
see the change from one cover to another.

\begin{theorem}
  For each \(k<\omega\), \(F \kmarkwin{(k+2)} \menGame{X}\)
  if and only if \(F \kmarkwin2 \menGame{X}\).
\end{theorem}

\begin{proof}
  Let \(\sigma\) be a winning \((k+2)\)-mark. We define the \(2\)-mark \(\tau\) as
  follows:
    \[
      \tau(\<\mc U\>,0)
        =
      \bigcup_{m<k+2}
        \sigma(\<\underbrace{\mc U,\dots,\mc U}_{m+1}\>,m)
    ,\]
    \[
      \tau(\<\mc U,\mc V\>,n+1)
        =
      \bigcup_{m<k+2}
        \sigma(\<
          \underbrace{\mc U,\dots,\mc U}_{k+1-m},
          \underbrace{\mc V,\dots,\mc V}_{m+1}
        \>,(n+1)(k+2)+m)
    .\]

  Let \(\<\mc U_0,\mc U_1,\dots\>\) be an attack by \(\pl C\) against \(\tau\).
  Then consider the attack
    \[
      \<
        \underbrace{\mc U_0,\dots,\mc U_0}_{k+2},
        \underbrace{\mc U_1,\dots,\mc U_1}_{k+2},
        \dots
      \>
    \]
  by \(\pl C\) against \(\sigma\). Since \(\sigma\) is a winning \((k+2)\)-mark,
    \[
      X
        =
      \bigcup_{m<k+2}
        \sigma(\<\underbrace{\mc U_0,\dots,\mc U_0}_{m+1}\>,m)
      \cup
      \bigcup_{n<\omega}
      \bigcup_{m<k+2}
        \sigma(\<
          \underbrace{\mc U_n,\dots,\mc U_n}_{k+1-m},
          \underbrace{\mc U_{n+1},\dots,\mc U_{n+1}}_{m+1}
        \>,(n+1)(k+2)+m)
    \]
    \[
      =
      \tau(\<\mc U_0\>,0)
      \cup
      \bigcup_{n<\omega}
      \tau(\<\mc U_n,\mc U_{n+1}\>,n+1)
    .\]
  Thus \(\tau\) is a winning \(2\)-mark. Of course, it is trivial to
  define a winning \((k+2)\)-mark from a \(2\)-mark by simply ignoring
  the extra \(k\) moves.
\end{proof}

It's worth noting that the above proof holds for any selection game of
the form \(G_{fin}(A,B)\).

One might wonder if a \(2\)-mark may always be improved to a \(1\)-mark
as well; this is not the case.

\begin{definition}
  For any cardinal \(\kappa\), let \(\onePtLind\kappa=\kappa\cup\{\infty\}\) denote
  the \term{one-point Lindel\"of-ication} of discrete \(\kappa\), where points in
  \(\kappa\) are isolated, and the neighborhoods of \(\infty\) are the co-countable
  sets containing it.
\end{definition}

One may prove directly that
  \(\pl F \notmarkwin \menGame{\onePtLind{\omega_1}}\)
without much effort; however, we postpone stating this until we have
shown that being Markov Menger is equivalent to being \(\sigma\)-compact
in regular spaces. Likewise, we postpone proving
  \(\pl F \kmarkwin2 \menGame{\onePtLind{\omega_1}}\)
in order to consider an equivalent set-theoretic game.

Essentially,
the greatest advantage of a strategy which has knowledge of two or more previous
moves of the opponent, versus only knowledge of the most recent move, is the
ability to react to changes from one round to the next. It's this ability to
react that often allows a player to wield a winning \(2\)-Markov strategy when
a winning \(1\)-Markov strategy does not exist.

\section{Scheepers' countable-finite games}

We now turn to a related game whose \(k\)-tactics were studied by Marion
Scheepers in \cite{MR1129143}.

\begin{game}
  Let \(\schFillStrictGame\kappa\) denote
  \term{Scheepers' strict countable-finite union game}
  with two players \(\pl C\), \(\pl F\). In round \(0\), \(\pl C\) chooses
  \(C_0\in[\kappa]^{\leq\omega}\), followed by \(\pl F\) choosing
  \(F_0\in[\kappa]^{<\omega}\). In round \(n+1\), \(\pl C\) chooses
  \(C_{n+1}\in[\kappa]^{\leq\omega}\) such that \(C_{n+1}\supset C_n\), followed
  by \(\pl F\) choosing \(F_{n+1}\in[\kappa]^{<\omega}\).

  \(\pl F\) wins the game if
  \(\bigcup_{n<\omega} F_n\supseteq\bigcup_{n<\omega} C_n\); otherwise,
  \(\pl C\) wins.
\end{game}

In \(\menGame{\onePtLind\kappa}\), \(\pl C\) essentially chooses a countable set
to not include in her neighborhood of \(\infty\), followed by \(\pl F\) choosing
a finite subset of this complement to cover during each round. Thus,
\(\pl F\) need only be concerned with the \textit{intersection} of the
countable sets chosen by \(\pl C\) in \(\menGame{\onePtLind\kappa}\), rather
than the union as in \(\schFillStrictGame\kappa\).

Another difference between these games:
Scheepers required that \(\pl C\) always choose strictly
growing countable sets. If the goal is to study tactics,
then \(\pl C\) cannot be allowed to trap \(\pl F\) in a loop by
repeating the same
moves. But by eliminating this requirement, the study can then turn to Markov
strategies, bringing the game further in line with the Menger game played upon
\(\onePtLind\kappa\).

We introduce a few games to make the relationship between Scheeper's
\(\schFillStrictGame\kappa\) and \(\menGame{\onePtLind\kappa}\) more precise.

\begin{game}
  Let \(\schFillGame\kappa\) denote the
  \term{Scheepers countable-finite union game} which proceeds analogously
  to \(\schFillStrictGame\kappa\), except that \(\pl C\)'s restriction in round \(n+1\)
  is reduced to \(C_{n+1}\supseteq C_n\).
\end{game}

\begin{game}
  Let \(\schFillInitialGame\kappa\) denote the
  \term{Scheepers countable-finite initial game} which proceeds analogously
  to \(\schFillGame\kappa\), except that \(\pl F\) wins whenever
  \(\bigcup_{n<\omega}F_n\supseteq C_0\).
\end{game}

\begin{game}
  Let \(\schFillIntGame\kappa\) denote the
  \term{Scheepers countable-finite intersection game} which proceeds analogously
  to \(\schFillInitialGame\kappa\), except that \(\pl C\) may choose any
  \(C_n\in[\kappa]^{\leq\omega}\) each round, and \(\pl F\) wins whenever
  \(\bigcup_{n<\omega}F_n\supseteq\bigcap_{n<\omega}C_n\).
\end{game}

% \begin{figure}[h]
% \begin{center}
% \begin{tikzpicture}
%   \matrix (m) [matrix of math nodes,row sep=3em,column sep=1em,minimum width=2em]
%   {
%     \pl F \kmarkwin{k+1}\menGame{\onePtLind\kappa} &
%     \pl F \kmarkwin{k+1}\schFillIntGame\kappa \\
%
%     \pl F \kmarkwin{k+1}\schFillInitialGame\kappa &
%     \pl F \kmarkwin{k+1}\schFillGame\kappa \\
%
%     &
%     \pl F \ktactwin{k+1}\schFillStrictGame\kappa \\
%   };
%   \path[>=latex,->]
%     (m-1-1) edge (m-2-1)
%     (m-1-2) edge (m-2-2)
%     (m-3-2) edge (m-2-2);
%   \path[>=latex,<->]
%     (m-1-1) edge (m-1-2)
%     (m-2-1) edge (m-2-2);
% \end{tikzpicture}
% \end{center}
% \caption{Diagram of Scheepers/Menger game implications}
% \label{GamesDiagram}
% \end{figure}

\begin{figure}[ht]
\begin{center}
\begin{tikzcd}[arrows=Rightarrow]
  &
  \pl F \kmarkwin{k+1} \menGame{\onePtLind\kappa}
  \Leftrightarrow
  \pl F \kmarkwin{k+1} \schFillIntGame{\kappa}
    \arrow[d] \\
  \pl F \ktactwin{k+1} \schFillStrictGame{\kappa}
    \arrow[r] &
  \pl F \kmarkwin{k+1}\schFillGame{\kappa}
  \Leftrightarrow
  \pl F \kmarkwin{k+1} \schFillInitialGame{\kappa}
      \arrow[d] \\
  &
  \pl F \kmarkwin{k+1} \schFillStrictGame{\kappa}
\end{tikzcd}
\end{center}
\caption{Diagram of Scheeper/Menger game implications}
\label{GamesDiagram}
\end{figure}

\begin{theorem}
For any cardinal \(\kappa\geq\omega\) and integer \(k<\omega\),
Figure \ref{GamesDiagram} holds.
\end{theorem}

\begin{proof}
  \(\pl F \kmarkwin{(k+1)}\menGame{\onePtLind\kappa}
    \Rightarrow
  \pl F \kmarkwin{(k+1)}\schFillIntGame\kappa\):
  Let \(\sigma\) be a winning \((k+1)\)-mark for \(\pl F\) in
  \(\menGame{\onePtLind\kappa}\). Let \(\mc U(C)\) (resp. \(\mc U(s)\)) convert each
  countable subset \(C\) of \(\kappa\) (resp. finite sequence \(s\) of such subsets)
  into the open cover \([C]^1\cup\{\onePtLind\kappa\setminus C\}\)
  (resp. finite sequence of such open covers). Then \(\tau\) defined by
    \[
      \tau(s\concat\<C\>,n)
        =
      C\cap\sigma(\mc U(s\concat\<C\>),n)
    \]
  is a winning \((k+1)\)-mark for \(\pl F\) in \(\schFillIntGame\kappa\).

  \(\pl F \kmarkwin{(k+1)}\schFillIntGame\kappa
    \Rightarrow
  \pl F \kmarkwin{(k+1)}\menGame{\onePtLind\kappa}\):
  Let \(\sigma\) be a winning \((k+1)\)-mark for \(\pl F\) in
  \(\schFillIntGame\kappa\). Let \(C(\mc U)\) (resp. \(C(s)\)) convert each open
  cover \(\mc U\) of \(\onePtLind\kappa\) (resp. finite sequence \(s\) of such covers)
  into a countable set \(C\) which is the complement of some neighborhood of
  \(\infty\) in \(\mc U\) (resp. finite sequence of such countable sets).
  Then \(\tau\) defined by
    \[
      \tau(s\concat\<\mc U\>,n)
        =
      (\onePtLind\kappa\setminus C(\mc U))
        \cup
      \sigma(C(s\concat\<\mc U\>),n)
    \]
  is a winning \((k+1)\)-mark for \(\pl F\) in \(\menGame{\onePtLind\kappa}\).

  \(\pl F \kmarkwin{(k+1)}\schFillIntGame\kappa
    \Rightarrow
  \pl F \kmarkwin{(k+1)}\schFillInitialGame\kappa\):
  Let \(\sigma\) be a winning \((k+1)\)-mark for \(\pl F\) in
  \(\schFillIntGame\kappa\). \(\sigma\) is also a winning \((k+1)\)-mark for \(\pl F\)
  in \(\schFillInitialGame\kappa\).

  \(\pl F \kmarkwin{(k+1)}\schFillInitialGame\kappa
    \Rightarrow
  \pl F \kmarkwin{(k+1)}\schFillGame\kappa\): Let \(\sigma\) be a winning
  \((k+1)\)-mark for \(\pl F\) in
  \(\schFillInitialGame\kappa\). For each finite sequence \(s\), let
  \(t\preceq s\) mean \(t\)
  is a final subsequence of \(s\). Then \(\tau\) defined by
    \[
      \tau(s\concat\<C\>,n)
        =
      \bigcup_{t\preceq s, m\leq n}
      \sigma(t\concat\<C\>,m)
    \]
  is a winning \((k+1)\)-mark for \(\pl F\) in \(\schFillGame\kappa\),
  considering that \(\sigma\) ensures that \(C_n\) is covered when
  attacked by \(\<C_n,C_{n+1},\dots\>\) for all \(n<\omega\).

  \(\pl F \kmarkwin{(k+1)}\schFillGame\kappa
    \Rightarrow
  \pl F \kmarkwin{(k+1)}\schFillInitialGame\kappa\):
  Let \(\sigma\) be a winning \((k+1)\)-mark for \(\pl F\) in
  \(\schFillGame\kappa\). \(\sigma\) is also a winning
  \((k+1)\)-mark for \(\pl F\)
  in \(\schFillInitialGame\kappa\).

  \(\pl F \ktactwin{(k+1)}\schFillStrictGame\kappa
    \Rightarrow
  \pl F \kmarkwin{(k+1)}\schFillGame\kappa\):
  Let \(\sigma\) be a winning \((k+1)\)-tactic for \(\pl F\) in
  \(\schFillStrictGame\kappa\). For each countable subset \(C\) of \(\kappa\), let \(C+n\)
  be the union of \(C\) with the \(n\) least ordinals in \(\kappa\setminus C\).
  Then for every valid attack \(\<C_0,C_1,C_2,\dots\>\) by \(\pl C\) in
  \(\schFillGame\kappa\), it follows that \(\<C_0,C_1+1,C_2+2,\dots\>\) is a
  valid attack by \(\pl C\) in \(\schFillStrictGame\kappa\) as
  \(C_{n+1}\supseteq C_n\) ensures that
  \(C_{n+1}+(n+1)\) includes all the \(n\) added elements in \(C_n+n\)
  in addition to at least one more.
  Thus \(\tau\) defined by
    \[
      \tau(\<C_n,\dots,C_{n+i}\>,n+i)
        =
      \sigma(\<C_n+n,\dots,C_{n+i}+(n+i)\>)
    \]
  is a winning \((k+1)\)-mark for \(\pl F\) in \(\schFillGame\kappa\),
  considering that \(\sigma\) ensures that
  \(\bigcup_{n<\omega}(C_n+n)\supseteq\bigcup_{n<\omega}C_n\)
  is covered when attacked by \(\<C_0,C_1+1,C_2+2,\dots\>\).

  \(\pl F \kmarkwin{(k+1)}\schFillGame\kappa
    \Rightarrow
  \pl F \kmarkwin{(k+1)}\schFillStrictGame\kappa\):
  Let \(\sigma\) be a winning \((k+1)\)-mark for \(\pl F\) in
  \(\schFillGame\kappa\). \(\sigma\) is also a winning
  \((k+1)\)-mark for \(\pl F\)
  in \(\schFillStrictGame\kappa\).
\end{proof}

While we have not shown a direct implication between the Menger game and
Scheeper's original countable-finite game, Scheepers introduced the statement
\(\alcompS\kappa\) relating to the almost-compatability of injective functions
from countable subsets of \(\kappa\) into \(\omega\) which may be applied to
both.

\begin{definition}
  For two functions \(f,g\) we say \(f\) is \textbf{almost compatible} with
  \(g\) (\(f\alcomp g\)) if \(|\{x\in\dom(f)\cap\dom(g):f(x)\not=g(x)\}|<\omega\).
\end{definition}

\begin{definition}
  \(\alcompS\kappa\) states that there exist injective functions
  \(f_A:A\to\omega\) for each \(A\in[\kappa]^{\leq\omega}\) such that
  \(f_A\alcomp f_B\) for all \(A,B\in[\kappa]^\omega\).
\end{definition}

Scheepers went on to show that \(\alcompS\kappa\) implies
\(\pl F \ktactwin2 \schFillStrictGame\kappa\). This proof, along with the following
facts, give us inspiration for
finding a winning \(2\)-Markov strategy in the Menger game played on
\(\onePtLind\kappa\).

\begin{theorem}
  \(\alcompS{\omega_1}\) and \(\neg\alcompS{\mf c^+}\)
  are theorems of \(ZFC\).
  \(\alcompS{\mf c}\) is a theorem of \(ZFC+CH\) and consistent with
  \(ZFC+\neg CH\).
\end{theorem}

\begin{proof}
  The construction of an Aronzajn tree in \cite[Theorem 5.9]{MR597342}
  produced a witness for \(\alcompS{\omega_1}\); this of course
  implies \(\alcompS{\mf c}\) under \(CH\).
  \(\neg\alcompS{\mf c^+}\) is shown by a cardinality argument
  in \cite{MR1129143}.
  The consistency result under \(ZFC+\neg CH\)
  is a lemma for the main theorem in \cite{MR1129143}.
\end{proof}

\begin{figure}[ht]
\begin{tikzcd}[arrows=Rightarrow]
  \alcompS\kappa
    \arrow[r]
    \arrow[d] &
  \pl F \kmarkwin2 \menGame{\onePtLind\kappa}
  \Leftrightarrow
  \pl F \kmarkwin2 \schFillIntGame{\kappa}
    \arrow[d] \\
  \pl F \ktactwin2 \schFillStrictGame{\kappa}
    \arrow[r] &
  \pl F \kmarkwin2\schFillGame{\kappa}
  \Leftrightarrow
  \pl F \kmarkwin2 \schFillInitialGame{\kappa}
      \arrow[d] \\
  &
  \pl F \kmarkwin2 \schFillStrictGame{\kappa}
\end{tikzcd}
\caption{Diagram of Scheeper/Menger game implications with \(\alcompS\kappa\)}
\label{GamesDiagram2}
\end{figure}

\begin{theorem}
  \(\alcompS\kappa\) implies the game-theoretic results in
  Figure \ref{GamesDiagram2}.
\end{theorem}

\begin{proof}
  Since \(\alcompS\kappa\Rightarrow\pl F \ktactwin2\schFillStrictGame\kappa\) was
  a main result of \cite{MR1129143}, we need only show that
  \(\alcompS\kappa\Rightarrow \pl F \kmarkwin2\schFillIntGame\kappa\).

  Let \(f_A\) for \(A\in[\kappa]^{\leq\omega}\) witness \(\alcompS\kappa\). We define
  the \(2\)-mark \(\sigma\) as follows:
    \[
      \sigma(\<C\>,0) = \{\alpha\in C: f_C(\alpha) \leq 0 \}
    \]
    \[
      \sigma(\<C,D\>,n+1)
        =
      \{
        \alpha \in C\cap D
      :
        f_D(\alpha) \leq n+1 \text{ or }
        f_C(\alpha)\not=f_D(\alpha)
      \}
    .\]
  For any attack \(\<C_0,C_1,\dots\>\) by \(\pl C\) and
  \(\alpha\in\bigcap_{n<\omega}C_n\), either \(f_{C_n}(\alpha)\) is constant for
  all \(n\), or \(f_{C_n}(\alpha)\not=f_{C_{n+1}}(\alpha)\) for some \(n\);
  either way, \(\alpha\) is covered.
\end{proof}

\begin{corollary}
  \(\pl F \kmarkwin2\menGame{\onePtLind{\omega_1}}\).
\end{corollary}



\section{Menger game derived covering properties}

\begin{figure}[h]
\begin{tikzcd}[arrows=Rightarrow]
  \pl F \markwin \menGame{X}
    \arrow[r]
    \arrow[d,Leftrightarrow] &
  \pl F \kmarkwin2 \menGame{X}
    \arrow[r]
    \arrow[out=150,in=30,not,l]
    \arrow[d,Leftrightarrow] &
  \pl F \win \menGame{X}
    \arrow[r]
    \arrow[out=150,in=30,maybe,l]
    \arrow[d,Leftrightarrow] &
  \pl C \notwin \menGame{X}
    \arrow[out=150,in=30,not,l]
    \arrow[d,Leftrightarrow] \\
  \text{Markov Menger} &
  2\text{-Markov Menger} &
  \text{strategic Menger} &
  \text{Menger}
\end{tikzcd}
\caption{Diagram of covering properties related to the Menger game}
\label{menSpec}
\end{figure}

Limited information strategies for the Menger game naturally define the
spectrum of covering properties shown in Figure \ref{menSpec}. However,
we do not know if the middle two properties are actually distinct.

\begin{question}\label{perfectTo2Mark}
  Does there exist a space \(X\) such that \(\pl F \win \menGame{X}\) but
  \(\pl F \notkmarkwin2 \menGame{X}\)?
\end{question}

We are also interested in non-game-theoretic characterizations of these
covering properties. It has been known for some time that metrizable strategic
Menger spaces are exactly the metrizable \(\sigma\)-compact spaces, shown first
by Telg\'arsky in \cite{MR753073} and later directly by Scheepers in
\cite{MR1273523}. It's Scheepers' technique which gives inspiration to
the following.

In the interest of generality, we will first characterize the Markov Menger
spaces without any separation axioms.

\begin{definition}
  A subset \(Y\) of \(X\) is \term{relatively compact} to \(X\), or
  \(Y\) is \term{compact in} \(X\), if for every open
  cover of \(X\), there exists a finite subcollection which covers \(Y\).
\end{definition}

\begin{definition}
  A subset \(Y\) of \(X\) is \term{precompact} in \(X\) if
  \(\text{cl}_X(Y)\) is compact.
\end{definition}

Some authors use the term
relative compactness to denote precompactness, which causes
no confusion in the context of regular spaces.

\begin{proposition}
  For regular spaces, \(Y\) is relatively compact to \(X\) if and only if
  \(Y\) is precompact in \(X\).
\end{proposition}

\begin{proof}
  Shown by Arhangel'skii in \cite{MR1885790}; a proof is provided for the
  convenience of the reader.

  Let \(\mc U=\{U_y:y\in\text{cl}_X(Y)\}\) where each \(U_y\) is an open
  set in \(X\) containing \(y\); apply regularity to get
  \(y\in V_{y}\subseteq\text{cl}_X(V_{y})\subseteq U_{y}\)
  with \(V_{y}\) open in \(X\).
  Then \(\{X\setminus\text{cl}_X(Y)\}\cup\{V_{y}:y\in \text{cl}_X(Y)\}\)
  is an open cover of \(X\), so
  use relative compactness to choose a finite subset \(F\) of \(\text{cl}_X(Y)\)
  such that \(\{V_{y}:y\in F\}\) is a finite cover of \(Y\). Then since
  \(\bigcup\{\text{cl}_X(V_{y}):y\in F\}\) is a closed set in \(X\)
  containing \(Y\), it contains \(\text{cl}_X(Y)\), and thus
  \(\{U_{y}:y\in F\}\) is a finite subcollection of \(\mc U\) covering
  \(\text{cl}_X(Y)\).

  For the reverse direction, simply take a finite subcover of
  \(\text{cl}_X(Y)\) to obtain a finite subcover of \(Y\).
\end{proof}

\begin{lemma}
  Let \(\sigma(\mc{U}, n)\) be a Markov strategy for \(F\) in
  \(\menGame{X}\), and \(\mc{O}\) collect all open covers of \(X\). Then the
  set
    \[
      R_n = \bigcap_{\mc{U}\in\mc{O}} \sigma(\mc{U},n)
    \]
  is relatively compact to \(X\). If \(\sigma\) is a winning
  Markov startegy, then \(\bigcup_{n<\omega} R_n = X\).
\end{lemma}

\begin{proof}
  First, for every open cover \(\mc U\in\mc{O}\),
  \(R_n\subseteq\sigma(\mc U,n)\) is covered by a finite subcollection of \(\mc U\).

  Suppose that \(x \not\in R_n\) for any \(n<\omega\). Then for each \(n\),
  pick \(\mc{U}_n\in\mc{O}\) such that \(x\not\in \sigma(\mc{U}_n,n)\). Then
  \(\pl C\) may counter \(\sigma\) with the attack \(\<\mc U_0,\mc U_1,\dots\>\).
\end{proof}

\begin{definition}
  A \term{\(\sigma\)-relatively-compact} space is the countable union of
  relatively compact subsets.
\end{definition}

\begin{corollary}
  The following are equivalent:
  \begin{itemize}
    \item \(X\) is \(\sigma\)-relatively-compact
    \item \(\pl F \kmarkwin0 \menGame X\)
    \item \(\pl F \markwin \menGame X\)
  \end{itemize}
\end{corollary}

\begin{proof}
  If \(X=\bigcup_{n<\omega} R_n\) for \(R_n\) relatively compact, then
  \(\sigma(n)=R_n\) is a winning \(0\)-mark, which of course gives a
  winning \(1\)-mark. The previous lemma finishes the proof.
\end{proof}

\begin{corollary}
  Let \(X\) be a regular space. The following are equivalent:
  \begin{itemize}
    \item \(X\) is \(\sigma\)-compact
    \item \(X\) is \(\sigma\)-relatively-compact
    \item \(\pl F \kmarkwin0 \menGame X\)
    \item \(\pl F \markwin \menGame X\)
  \end{itemize}
\end{corollary}

\begin{corollary}
  \(\pl F\notmarkwin\menGame{\onePtLind{\omega_1}}\).
\end{corollary}

Note that for Lindel\"of spaces, metrizability is characterized by regularity
and second-countability.

\begin{lemma}
  Let \(X\) be a second-countable space. \(\pl F\win\menGame{X}\) if and only if
  \(\pl F\markwin\menGame{X}\).
\end{lemma}

\begin{proof}
  Let \(\sigma\) be a strategy for \(\pl F\), and note that
  it's sufficient to consider playthroughs with only basic open covers.

  We proceed by constructing basic open covers indexed by \(\omega^{<\omega}\)
  which will turn out to emulate the behavior of \(\pl C\) well enough
  that \(\pl F\) will be able to define a Markov strategy from a perfect
  information strategy by substituting
  these covers in the place of perfect information.
  So if \(\mc U_t\) is a basic open cover for \(t<s\in\omega^{<\omega}\), and
  \(\mc V\) is any basic open cover, we may choose a finite subcollection
  \(\mc F(s,\mc V)\) of \(\mc V\) such that
  \[
    \sigma(\<\mc U_{s\rest 1},\dots,\mc U_{s},\mc V\>)
      \subseteq
    \bigcup \mc F(s,\mc V)
  .\]

  Note that there are only countably-many finite collections of basic open sets.
  Thus we may choose basic open covers \(\mc U_{s\concat\<n\>}\) for \(n<\omega\)
  such that for any basic open cover \(\mc V\), there exists \(n<\omega\) where
  \(\mc F(s,\mc V)=\mc F(s,\mc U_{s\concat\<n\>})\).

  Let \(t:\omega\to\omega^{<\omega}\) be a bijection. We define the Markov
  strategy \(\tau\) as follows:
  \[
    \tau(\<\mc V\>, n)
      =
    \bigcup \mc F(t(n),\mc V)
  .\]

  Suppose there exists a counter-attack \(\<\mc V_0,\mc V_1,\dots\>\) of
  basic open covers which defeats \(\tau\). Then there exists
  \(f:\omega\to\omega\) such that, letting \(t(m_n)=f\rest n\):
  \[
    \begin{array}{rcl}
    x & \not\in & \tau(\<\mc V_{m_n}\>,m_n) \\
    & = & \bigcup\mc F(f\rest n,\mc V_{m_n}) \\
    & = & \bigcup\mc F(f\rest n,\mc U_{f\rest (n+1)}) \\
    & \supseteq & \sigma(\<\mc U_{f\rest 1},\dots,\mc U_{f\rest(n+1)}\>)
    \end{array}
  .\]

  Thus \(\<\mc U_{f\rest 1}, \mc U_{f\rest 2},\dots\>\) is a successful
  counter-attack by \(\pl C\) against the perfect information strategy \(\sigma\).
\end{proof}

\begin{corollary}
  Let \(X\) be a second-countable space. The following are equivalent:
  \begin{itemize}
    \item \(X\) is \(\sigma\)-relatively-compact
    \item \(F \kmarkwin0 \menGame{X}\)
    \item \(F \markwin \menGame{X}\)
    \item \(F \win \menGame{X}\)
  \end{itemize}
\end{corollary}

\begin{corollary}
  Let \(X\) be a metrizable space. The following are equivalent:
  \begin{itemize}
    \item \(X\) is \(\sigma\)-compact
    \item \(X\) is \(\sigma\)-relatively-compact
    \item \(F \kmarkwin0 \menGame{X}\)
    \item \(F \markwin \menGame{X}\)
    \item \(F \win \menGame{X}\)
  \end{itemize}
\end{corollary}

\begin{proof}
  Each bullet implies \(X\) is Lindel\"of, so \(X\) may be assumed to be
  regular and second-countable.
\end{proof}



\section{Robustly Menger}

To help instigate the topological property \(\pl F\kmarkwin2\menGame{X}\),
we introduce a
slight variant of the Menger game and a related covering property.

\begin{game}
  Let \(\menGame{X,Y}\) denote the \term{Menger subspace game} which proceeds
  analogously to the Menger game, except that \(\pl F\) wins whenever
  \(Y\subseteq\bigcup_{n<\omega} F_n\).
\end{game}

Note of course that \(\menGame{X,X}=\menGame{X}\).

\begin{definition}
  A subset \(Y\) of \(X\) is \term{relatively robustly Menger} if there exist
  functions \(r_{\mc V}:Y\to\omega\)
  for each open cover \(\mc V\) of \(X\) such that
  for all open covers \(\mc U,\mc V\) and numbers \(n<\omega\),
  the following sets are \(\mc V\)-finite:
    \[
      c(\mc V,n)=\{ x\in Y : r_{\mc V}(x)\leq n\}
    ,\]
    \[
      p(\mc U,\mc V,n+1)=\{ x\in Y : n<r_{\mc U}(x)<r_{\mc V}(x)\}
    .\]
\end{definition}

\begin{definition}
  A space \(X\) is \term{robustly Menger} if it is relatively robustly
  Menger to itself.
\end{definition}

\begin{proposition}
  All \(\sigma\)-relatively-compact spaces are robustly Menger.
\end{proposition}

\begin{proof}
  If \(X=\bigcup_{n<\omega}R_n\) for \(R_n\) relatively compact to \(X\),
  then for all \(\mc U\), let \(r_{\mc U}(x)\) be the
  least \(n\) such that \(x\in R_n\). Then \(c(\mc V,n)=\bigcup_{m\leq n}R_m\) and
  \(p(\mc U,\mc V,n+1) = \emptyset\).
\end{proof}

\begin{theorem}
  If \(Y\subseteq X\) is relatively robustly Menger, then
  \(\pl F\kmarkwin2\menGame{X,Y}\).
\end{theorem}

\begin{proof}
  We define the Markov strategy \(\sigma\) as follows.
  Let \(\sigma(\<\mc U\>,0)=c(\mc U,0)\), and let
  \(\sigma(\<\mc U,\mc V\>,n+1)=c(\mc V,n+1)\cup p(\mc U, \mc V,n+1)\).

  For any attack \(\<\mc{U}_0,\mc{U}_1,\dots\>\) by \(\pl C\) and \(x\in Y\),
  one of the following must occur:

  \begin{itemize}
    \item
      \(r_{\mc U_0}(x)=0\) and thus
      \(x\in c(\mc U_0,0)\subseteq \sigma(\<\mc U_0\>,0)\).

    \item
      \(r_{\mc U_0}(x)=N+1\) for some \(N\geq 0\) and:
      \begin{itemize}
        \item
          For all \(n\leq N\),
          \[
            r_{\mc U_{n+1}}(x)\leq N+1
          \]
          and thus
          \(x\in c(\mc U_{N+1},N+1) \subseteq
            \sigma(\<U_N,U_{N+1}\>,N+1)\).
        \item
          For some \(n \leq N\),
          \[ r_{\mc U_n}(x)\leq n \]
          and thus
          \(x\in c(\mc U_{n+1},n+1) \subseteq
            \sigma(\<U_n,U_{n+1}\>,n+1)\).
        \item
          For some \(n \leq N\),
          \[
            n < r_{\mc U_n}(x)\leq N+1 < r_{\mc U_{n+1}}(x)
          \]
         and thus
         \(x\in p(\mc U_n,\mc U_{n+1},n+1) \subseteq
          \sigma(\<U_n,U_{n+1}\>,n+1)\)
       \end{itemize}
  \end{itemize}
\end{proof}

\begin{theorem}
  \(\alcompS\kappa\) implies \(\onePtLind\kappa\) is
  robustly Menger.
\end{theorem}

\begin{proof}
  Let \(f_A\) for \(A\in[\kappa]^{\leq\omega}\) witness \(\alcompS\kappa\) and fix
  \(A(\mc U)\in[\kappa]^{\leq\omega}\) for each open cover \(\mc U\) such that
  \(\onePtLind\kappa\setminus A(\mc U)\) is contained in some element of
  \(\mc U\).
  Then let \(r_{\mc U}(x)=0\) for \(x\in\onePtLind\kappa\setminus A(\mc U)\),
  and \(r_{\mc U}(\alpha)=f_{A(\mc U)}(\alpha)\) for \(\alpha\in A(\mc U)\).

  It follows that
    \[
      c(\mc U,n)
        =
      \left(\onePtLind\kappa\setminus A(\mc U)\right)
        \cup
      \{\alpha\in A(\mc U) : f_{A(\mc U)}(\alpha)\leq n \}
    \]
  is \(\mc U\)-finite, \(\bigcup_{n<\omega}c(\mc U,n)=X\), and
    \[
      p(\mc U,\mc V,n+1)
        =
      \{
        \alpha\in A(\mc U)\cap A(\mc V)
          :
        n < f_{A(\mc U)}(\alpha) < f_{A(\mc V)}(\alpha)
      \}
    \]
  is finite.
\end{proof}

We may also consider common counterexamples (specifically, examples
67 and 63 of \cite{MR1382863}) which are finer
than the usual Euclidean line.

\begin{definition}
  Let \(R_{\mathbb Q}\) be the real line with the topology generated by open
  intervals with or without the rationals removed.
\end{definition}

\begin{example}
  \(R_{\mathbb Q}\) is non-regular and non-\(\sigma\)-compact, but is
  second-countable and \(\sigma\)-relatively-compact.
\end{example}

\begin{proof}
  Compact sets in \(R_{\mathbb Q}\) cannot contain open intervals,
  and thus are nowhere dense in nonmeager \(\mathbb R\),
  so \(R_{\mathbb Q}\) is not \(\sigma\)-compact. The usual base of
  intervals with rational endpoints (with or without rationals removed)
  witnesses second-countability.

  We now will show that \([-n,n]\setminus\mathbb Q\) is relatively compact.
  Let \(\mc U\) be a cover of
  \(R_{\mathbb Q}\) by open intervals with the rationals removed,
  and let \(\mc V\) be the corresponding cover of open intervals.
  There exists a finite subcollection of \(\mc V\) which covers
  \([-n,n]\), so there exists a corresponding subcollection of
  \(\mc U\) which covers \([-n,n]\setminus\mathbb Q\).
  Thus
  \(R_{\mathbb Q}=\mathbb Q\cup\bigcup_{n<\omega}([-n,n]\setminus\mathbb Q)\)
  is a countable union of relatively compact subsets.
\end{proof}

\begin{definition}
  Let \(R_\omega\) be the real line with the topology generated by open
  intervals with countably many points removed.
\end{definition}

\begin{lemma}\label{rOmegaLemma}
  For each compact subset \(K\) of \(\mathbb R\)
  and open cover \(\mc U\) of \(R_\omega\),
  there exists a countable set \(C\) such that \([a,b]\setminus C\)
  is \(\mc U\)-finite.
\end{lemma}

\begin{proof}
  Let \(\mc V\) be the corresponding open cover of \(\mathbb R\) where
  \(\mc U\) may be obtained by removing countable sets from members of
  \(\mc V\). Choose a finite subcollection \(\mc G\) of \(\mc V\) which
  covers \(K\), and then note that the corresponding finite subcollection
  \(\mc F\) of \(\mc U\) covers \([-n,n]\setminus C\) where \(C\) is the
  finite union of the countable sets removed from sets in \(\mc G\)
  to obtain the sets in \(\mc F\).
\end{proof}

\begin{example}
  \(R_\omega\) is non-regular, non-second-countable, and
  non-\(\sigma\)-relatively-compact, but \(\pl F\win\menGame{R_\omega}\).
\end{example}

\begin{proof}
  If \(S\supseteq\{s_n:n<\omega\}\) for distinct \(s_n\),
  then \(U_m=R_\omega\setminus\{s_n:m<n<\omega\}\) yields an
  infinite cover \(\{U_m:m<\omega\}\) with no finite subcollection covering
  \(S\), showing that all relatively compact sets are finite, and \(R_\omega\)
  is not \(\sigma\)-relatively-compact. The set \(R_\omega\setminus\mathbb Q\)
  is open, but the closure of any open subset must contain a rational, so
  this space is not regular. Finally, the space is not second-countable,
  since for any countable collection of open sets \(\{U_n:n<\omega\}\),
  we may choose \(p_n\in U_n\)
  and note \(R_\omega\setminus\{p_n:n<\omega\}\) is an open set not containing
  any \(U_n\).

  Define the winning strategy \(\sigma\) for \(\pl F\) in
  \(\menGame{R_\omega}\) as follows: let
  \(\sigma(\<\mc{U}_0,\dots,\mc{U}_{2n}\>)=[-n,n]\setminus C_n\)
  where \(C_n\) witnesses Lemma \ref{rOmegaLemma} for \([-n,n]\)
  and \(\mc U_{2n}\),
  and let \(\sigma(\<\mc{U}_0,\dots,\mc{U}_{2n+1}\>)=\{c_{i,j} : i,j < n\}\).
\end{proof}

We will soon see that, assuming \(\alcompS{\mf c}\), \(\pl F\) has a winning
\(2\)-Markov strategy for \(\menGame{R_\omega}\) as well.

\begin{proposition}
  If \(X=\bigcup_{i<\omega} X_i\) and \(\pl F \kmarkwin{2} \menGame{X,X_i}\)
  for \(i<\omega\), then \(\pl F \kmarkwin{2} \menGame{X}\)
\end{proposition}

\begin{proof}
  Let \(\sigma_i\) be a \(2\)-Markov strategy
  for \(\pl F\) in \(\menGame{X,X_i}\).

  We define the \(2\)-Markov strategy
  \(\sigma\) for \(\menGame{X}\) as follows:
    \[
      \sigma(\<\mc U\>,0) = \sigma_0(\<\mc U\>,0)
    ,\]
    \[
      \sigma(\<\mc U,\mc V\>,n+1)
        =
      \bigcup_{i,j\leq n}
      \sigma_i(\<\mc V\>,0)
        \cup
      \sigma_i(\<\mc U,\mc V\>,j+1)
    .\]

  Let \(\<\mc U_0,\mc U_1,\dots\>\) be a successful counter-attack by \(\pl C\)
  against \(\sigma\). Then there exists \(x\in X_i\) for
  some \(i<\omega\) such that \(x\) is not contained in
  \[
    \sigma(\<\mc U_0\>,0)
      \cup
    \bigcup_{n<\omega}
    \sigma(\<\mc U_n,\mc U_{n+1}\>,n+1)
  .\]
  It then follows that \(x\) is not contained in
  \[
    \sigma_i(\<\mc U_i\>,0)
      \cup
    \bigcup_{n<\omega}
    \sigma_i(\<\mc U_{i+n},\mc U_{i+n+1}\>,n+1)
  \]
  and \(\<\mc U_i,\mc U_{i+1},\dots\>\) is a successful counter-attack
  by \(\pl C\) against \(\sigma_i\).
\end{proof}

\begin{theorem}
  Assuming \(CH\) or just \(\alcompS{\mf c}\),
  \(\pl F\kmarkwin2\menGame{R_\omega}\).
\end{theorem}

\begin{proof}
  It's sufficient to show that \([0,1]\subseteq R_\omega\) is relatively
  robustly Menger. Let \(f_A\) witness \(\alcompS{\mf c}\)
  for \(A\in[[0,1]]^{\leq\omega}\).
  For each open cover \(\mc U\), let \(A_{\mc U}\) witness Lemma
  \ref{rOmegaLemma} for \([0,1]\) and \(\mc U\).
  Let \(r_{\mc U}(x)=0\) if \(x\in[0,1]\setminus A_{\mc U}\) and
  \(r_{\mc U}(x)=f_{\mc A_{\mc U}}(x)\) otherwise.

  It follows then that
    \[
      c(\mc U,n) = [0,1]\setminus\{x\in A_{\mc U}:f_{\mc A_{\mc U}}(x)>n\}
    \]
  is \(\mc U\)-finite and
    \[
      p(\mc U,\mc V,n+1)
        =
      \{x\in A_{\mc U}\cap A_{\mc V}
        :
      n<f_{\mc A_{\mc U}}(x)<f_{\mc A_{\mc V}}(x)\}
    \]
  is finite.
\end{proof}

The following is left open:

\begin{question}
  Are all \(2\)-Markov Menger spaces
  robustly Menger?
\end{question}

\section*{Acknowlegements}

The author wishes to thank Gary Gruenhage and Alan Dow for their
very useful comments during the writing of this paper.

\bibliographystyle{plain}
\bibliography{../../bibliography}

\end{document}
