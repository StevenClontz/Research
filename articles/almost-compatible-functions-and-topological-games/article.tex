\documentclass{amsart}
\usepackage{amsmath}
\usepackage{amsthm}
\usepackage{amssymb}

\usepackage{tikz}
\usetikzlibrary{matrix}

\usepackage{../../clontzDefinitions}


      \theoremstyle{plain}
      \newtheorem{theorem}{Theorem}
      \newtheorem{lemma}[theorem]{Lemma}
      \newtheorem{corollary}[theorem]{Corollary}
      \newtheorem{proposition}[theorem]{Proposition}
      \newtheorem{conjecture}[theorem]{Conjecture}
      \newtheorem{question}[theorem]{Question}
      \newtheorem{claim}[theorem]{Claim}

      \theoremstyle{definition}
      \newtheorem{definition}[theorem]{Definition}
      \newtheorem{example}[theorem]{Example}
      \newtheorem{game}[theorem]{Game}

      \theoremstyle{remark}
      \newtheorem{remark}[theorem]{Remark}

      \theoremstyle{plain}
      \newtheorem*{theorem*}{Theorem}
      \newtheorem*{lemma*}{Lemma}
      \newtheorem*{corollary*}{Corollary}
      \newtheorem*{proposition*}{Proposition}
      \newtheorem*{conjecture*}{Conjecture}
      \newtheorem*{question*}{Question}
      \newtheorem*{claim*}{Claim}

      \theoremstyle{definition}
      \newtheorem*{definition*}{Definition}
      \newtheorem*{example*}{Example}
      \newtheorem*{game*}{Game}

      \theoremstyle{remark}
      \newtheorem*{remark*}{Remark}

\parskip=0.5em

% dow stuff
          \def\Succ{\mathop{Succ}}
          \def\dom{\mathop{dom}}
          \def\val{\operatorname{val}}
          \def\N{\mathbb N}
          \def\Nstar{\mathbb N^*}



\begin{document}

% \title{Proximal compact spaces are Corson compact\tnoteref{t1}}
% \tnotetext[t1]{2010 Mathematics Subject Classification. 54E15, 54D30, 54A20.}
\title{Almost compatible functions and infinite length games}

% \author[aub]{S.~Clontz\fnref{fn1}}
% \ead{steven.clontz@auburn.edu}
% \author[aub]{G.~Gruenhage\fnref{fn2}}
% \ead{gruengf@auburn.edu}

% \address[aub]{Department of Mathematics, Auburn University,
%  Auburn, AL 36830}


\author{Steven Clontz}
\address{Department of Mathematics and Statistics, UNC Charlotte,
Charlotte, NC 28262}
\email{steven.clontz@gmail.com}
\author{Alan Dow}
\address{Department of Mathematics and Statistics, UNC Charlotte,
Charlotte, NC 28262}
\email{adow@uncc.edu}

\keywords{TODO}

% TODO
\subjclass[2010]{}


\begin{abstract}
  TODO
\end{abstract}


\maketitle

\section{Introduction}

  \begin{definition}
    Two functions \(f,g\) are almost compatible, that is, \(f\sim g\) when
    \(\{a\in\dom f\cap\dom g:f(a)\not=g(a)\}\) is finite.
  \end{definition}

  Marion Scheepers used almost compatible functions in \cite{MR1129143} in order
  to study the existence of limited information strategies on a variation
  of the meager-nowhere dense game he introduced in \cite{MR1183703}.

  \begin{game}
    Let \(\schFillStrictGame\kappa\) denote
    \term{Scheepers' strict countable-finite union game}
    with two players \(\pl C\), \(\pl F\). In round \(0\), \(\pl C\) chooses
    \(C_0\in[\kappa]^{\leq\omega}\), followed by \(\pl F\) choosing
    \(F_0\in[\kappa]^{<\omega}\). In round \(n+1\), \(\pl C\) chooses
    \(C_{n+1}\in[\kappa]^{\leq\omega}\) such that \(C_{n+1}\supset C_n\), followed
    by \(\pl F\) choosing \(F_{n+1}\in[\kappa]^{<\omega}\).

    \(\pl F\) wins the game if
    \(\bigcup_{n<\omega} F_n\supseteq\bigcup_{n<\omega} C_n\); otherwise,
    \(\pl C\) wins.
  \end{game}

  Of course, with perfect information this game is trivial: during round
  \(n\) player \(\pl F\) simply chooses \(n\) ordinals from each of the
  \(n\) countable sets played by \(\pl C\). However, if \(\pl F\) is limited
  to using information from the last \(k\) moves by \(\pl C\) during each
  round, the task becomes more difficult. Call such a strategy a
  \term{\(k\)-tactical strategy} or \term{\(k\)-tactic}; if using the round
  number is allowed, then the strategy is called a \term{\(k\)-Markov strategy}
  or a \term{\(k\)-mark}.

  \begin{definition}
    The statement \(S(\kappa)\) (given as \(S(\kappa,\aleph_0,\omega)\)
    in \cite{MR1129143}) claims that there exist one-to-one functions
    \(f_A:A\to\omega\) for each \(A\in[\kappa]^{\leq\aleph_0}\) such that
    the collection \(\{f_A:A\in[\kappa]^{\leq\aleph_0}\}\) is pairwise
    almost compatible.
  \end{definition}

  In the same paper, Scheepers noted that \(S(\omega_1)\) holds in \(ZFC\),
  and that it's possible to force \(\mf c\) to be arbitrarily large
  while preserving \(S(\mf c)\). However, \(S(\mf c^+)\) always fails.
  This axiom may be applied to obtain a winning \(2\)-tactic for \(\pl F\)
  in the countable-finite game.

  In \cite{clontzMengerGamePreprint}, Clontz related this game to a game
  which may be used to characterize the Menger covering property of a
  topological space.

  \begin{game}
    Let \(\menGame{X}\) denote the \term{Menger game} with players \(\pl C\),
    \(\pl F\).
    In round \(n\), \(\pl C\) chooses an open cover \(\mc U_n\), followed by
    \(\pl F\)
    choosing subset \(F_n\) of \(X\) which may be finitely covered by \(\mc U_n\).

    \(\pl F\) wins the game if \(X = \bigcup_{n<\omega}F_n\),
    and \(\pl C\) wins otherwise.
  \end{game}

  This characterization is slightly different than the typical characterization
  in which the second player first chooses a specific
  finite subcollection \(\mc F_n\) of the cover itself and lets
  \(F_n=\bigcup \mc F_n\), denoted as \(G_{fin}(\mc O,\mc O)\)
  in \cite{MR1378387}.
  However, it's easily seen that these games are equivalent
  for perfect information strategies (so both characterize
  the Menger property), and this characterization is
  more convenient for our concerns.

  \begin{definition}
    Let \(\oneptlind\kappa=\kappa\cup\{\infty\}\) where \(\kappa\) is
    discrete and \(\infty\)'s neighborhoods are the co-countable sets
    containing it.
  \end{definition}

  The relationship between \(\schFillStrictGame\kappa\) and
  \(\menGame{\oneptlind\kappa}\) is strong; in both games \(\pl C\) essentially
  chooses a countable subset of \(\kappa\) followed by \(\pl F\) choosing
  a finite subset of that choice, and it's easy to see the winning perfect
  information strategy for \(\pl F\) in both games.
  In addition, it was shown in
  \cite{clontzMengerGamePreprint} that when \(S(\kappa)\) holds,
  \(\pl F\) has a winning \(2\)-Markov strategy in
  \(\menGame{\oneptlind\kappa}\).

  One source of motivation is to make progress on the following open question:

  \begin{question}
    Does there exist a topological space \(X\) for which
    \(\pl F\win\menGame{X}\) but \(\pl F\notkmarkwin{2}\menGame{X}\)?
    (That is, the second player can win the Menger game on \(X\)
    with perfect information but not with \(2\)-Markov information.)
  \end{question}

  One might hope that \(\oneptlind{(\mf c^+)}\) might answer this question
  in the affirmative as \(S(\mf c^+)\) fails, but we will show that assuming
  \(V=L\), \(\pl F\kmarkwin{2}\menGame{\oneptlind\kappa}\) for all
  cardinals \(\kappa\).

  \section{One-to-one and finite-to-one almost compatible functions}

  We may weaken Scheeper's \(S(\kappa)\) as follows:

  \begin{definition}
    The statement \(S'(\kappa)\) weakens \(S(\kappa)\) be only requiring
    the witness functions \(f_A:A\to\omega\) to be finite-to-one.
  \end{definition}

  The following observation will be convenient.

  \begin{proposition}
    \(S(\kappa)\) and \(S'(\kappa)\) need only be witnessed by
    functions \(\{f_A:A\in\mc S\}\) for some family \(\mc S\)
    cofinal in \([\kappa]^{\leq\aleph_0}\).
  \end{proposition}

  \begin{proof}
    For each \(A\in[\kappa]^{\leq\aleph_0}\) choose \(A'\supseteq A\)
    from \(\mc S\) and let \(g_A=f_{A'}\rest A\).
  \end{proof}

  In the next section we will show that \(S'(\kappa)\) is sufficient
  for the applications to the Scheepers and Menger games.
  In the meantime, we will demonstrate that \(S'(\kappa)\) is strictly
  weaker than \(S(\kappa)\).

  Recall the following.

  \begin{definition}
    A Kurepa family \(\mc K\subseteq[\kappa]^{\aleph_0}\) on \(\kappa\)
    satisfies that
    \(\mc K\rest A=\{K\cap A:K\in\mc K\}\) is countable
    for each \(A\in[\kappa]^{\aleph_0}\).
  \end{definition}

  \begin{theorem}
    \(S'(\kappa)\) holds whenever there exists a cofinal Kurepa family on \(\kappa\).
  \end{theorem}

  \begin{proof}
    Let \(\mc K=\{K_\alpha:\alpha<\theta\}\)
    be a cofinal Kurepa family on \(\kappa\).
    We first define \(f_\alpha:K_\alpha\to\omega\) for each \(\alpha<\theta\).

    Suppose we've already defined pairwise almost compatible finite-to-one functions
    \(\{f_\beta:\beta<\alpha\}\). To define
    \(f_\alpha\), we first recall that \(\mc K\rest K_\alpha\) is countable,
    so we may choose \(\beta_n<\alpha\) for \(n<\omega\) such that
    \(
      \{K_\beta:\beta<\alpha\}\rest K_\alpha \setminus \{\emptyset\}
        =
      \{K_\alpha\cap K_{\beta_n}:n<\omega\}
    \).
    Let \(K_\alpha=\{\delta_{i,j}:i\leq\omega,j<w_i\}\) where
    \(w_i\leq\omega\) for each \(i\leq\omega\),
    \(
      K_\alpha\cap \left(K_{\beta_n}\setminus\bigcup_{m<n}K_{\beta_m}\right)
        =
      \{\delta_{n,j}:j<w_n\}
    \),
    and
    \(
      K_\alpha\setminus\bigcup_{n<\omega}K_{\beta_n}
        =
      \{
        \delta_{\omega,j}:j<w_\omega
      \}
    \).
    Then let \(f_\alpha(\delta_{n,j})=\max(n,f_{\beta_n}(\delta_{n,j}))\) for
    \(n<\omega\) and \(f_\alpha(\delta_{\omega,j})=j\) otherwise.

    We should show that \(f_\alpha\) is finite-to-one. Let \(n<\omega\).
    Since \(f_\alpha(\delta_{m,j})\geq m\), we only consider the finite cases
    where \(m\leq n\). Since each
    \(f_{\beta_m}\) is finite-to-one, \(f_{\beta_m}(\delta_{m,j})\leq n\)
    for only finitely many \(j\). Thus
    \(f_\alpha(\delta_{m,j})=\max(m,f_{\beta_m}(\delta_{m,j}))\) maps to
    \(n\) for only finitely many \(j\).

    We now want to demonstrate that \(f_\alpha\sim f_{\beta_n}\) for all
    \(n<\omega\). Note \(\delta_{m,j}\in K_{\beta_n}\) implies
    \(m\leq n\).
    For \(m=n\), we have
    \(f_\alpha(\delta_{n,j})=\max(n,f_{\beta_n}(\delta_{n,j}))\) which differs
    from \(f_{\beta_n}(\delta_{n,j})\) for only the finitely many \(j\) which
    are mapped below \(n\) by \(f_{\beta_n}\).
    For \(m<n\) and \(\delta_{m,j}\in K_{\beta_n}\),we have
    \(f_\alpha(\delta_{m,j})=\max(m,f_{\beta_m}(\delta_{m,j}))\) which can
    only differ
    from \(f_{\beta_n}(\delta_{m,j})\) for only the finitely many \(j\) which
    are mapped below \(m\) by \(f_{\beta_m}\) or the finitely many \(j\)
    for which the
    almost compatible \(f_{\beta_n}\sim f_{\beta_m}\) differ.

    Finally for any \(\beta<\alpha\), we may conclude \(f_\alpha\sim f_\beta\)
    since there is some \(\beta_n\) with
    \(K_\alpha\cap K_\beta=K_\alpha\cap K_{\beta_n}\),
    \(f_\alpha\sim f_{\beta_n}\), and \(f_{\beta_n}\sim f_\beta\).
  \end{proof}

  We now construct a topology on \(\omega_n\) for each \(n<\omega\) which
  will witness a Kurepa family on \(\aleph_n\).

  \begin{proposition}
    Let \(X\) be a \(T_2\) space with a base of countable and compact
    neighborhoods. Then \(X\) is locally metrizable with
    a base of compact open countable sets.
  \end{proposition}

  \begin{proof}
    For each point \(x\) let \(K\) be a countable and compact neighborhood
    of \(x\), and it follows that it is contained in a countable, open,
    and locally compact neighborhood \(W\) of \(x\), which in turn
    is zero-dimensional and metrizable. So choose \(V\) clopen
    in \(W\) such that \(x\in V\subseteq K\); \(V\) is a compact open
    neighborhood of \(x\) in \(X\).
  \end{proof}

  \begin{definition}
    A topological space is said to be \(\omega\)-bounded if each countable
    subset of the space has compact closure.
  \end{definition}

  \begin{proposition}
    Let \(X\) be a \(T_2\) space with cardinality less than \(\aleph_\omega\)
    which is locally countable and
    \(\omega\)-bounded. Then the closure operation preserves cardinality and
    weight.
  \end{proposition}

  \begin{proof}
    Note that the closure of any countable neighborhood is compact, and
    any Lindel\"of set is countable. This space is locally metrizable
    and thus first-countable, so cardinality and weight coincide for any
    subspace. The result is obvious if \(A\) is countable; otherwise
    let \(A=\{a_\alpha:\alpha<\omega_{n+1}\}\) and since basic neighborhoods
    are countable note any limit point is
    a limit point of \(A_\beta=\{a_\alpha:\alpha<\beta\}\)
    for some \(\beta<\omega_{n+1}\).
    Thus \(\cl A=\bigcup_{\beta<\omega_{n+1}}\cl{A_\beta}\) and
    by induction \(|\cl A|=|A|\).
  \end{proof}

  \begin{lemma}
    Let \(X\) be a \(T_2\) space with cardinality less than \(\aleph_\omega\)
    which is locally countable and locally compact, and such that
    its closure operation preserves cardinalities.
    Then \(X\) has an \(\omega\)-bounded extension \(\tilde X\)
    with the same properties where \(\tilde X\setminus X\) has the same
    cardinality as \(X\).
  \end{lemma}

  \begin{proof}
    We prove this by induction on \(n\).
    If \(n=0\), then we can just use the one-point compactification of two
    copies of \(X\). So suppose \(n>0\) and
    that \(X=\omega_n\) has an appropriate topology. Note that \(X\) has
    a base of countable and compact neighborhoods since the closure operation
    preserves cardinalities.

    For each \(\alpha<\omega_n\), \(\gamma_\alpha\) may be chosen such that
    both the closure of the set \(\alpha\) in \(X\) and a countable
    neighborhood of the point \(\alpha\) are subsets of \(\gamma_\alpha\).
    Note that the set
    \(\{\lambda<\omega_n:\alpha<\lambda\Rightarrow\gamma_\alpha<\lambda\}\)
    is a cub subset of \(\omega_n\) containing a cub subset \(C\)
    of limit ordinals.
    Now for each \(\lambda\in C\), the set \(\lambda\) is
    open as \(\alpha<\lambda\) belongs to the neighborhood
    \(\gamma_\alpha\subsetneq\lambda\). Also, if \(\lambda\) has uncountable
    cofinality, then for \(\beta\geq\lambda\) and any countable neighborhood
    \(U\) of \(\beta\), \(U\cap\lambda=U\cap\alpha\)
    for some \(\alpha<\lambda\); thus \(U\setminus\cl\alpha=U\setminus\lambda\)
    is a neighborhood of \(\beta\), showing that \(\lambda\)
    is clopen.

    Let \(\tilde X=\omega_n\times 2\).
    By induction on \(\lambda \in C\) we will define compatible topologies
    for \(\tilde X_\lambda=\omega_n\times\{0\}\cup\lambda\times\{1\}\)
    such that
    \begin{itemize}
      \item \(\omega_n \times \{0\}\) is an open copy of \(X\),
      \item \(\lambda\times 2\) is open, and when \(\cf\lambda>\omega\) also closed,
      \item the space has a base of countable and compact neighborhoods, and
      \item when \(\lambda\) is a successor,
        for each \(\alpha<\lambda\) the closure of \(\alpha\times 2\) is
        an \(\omega\)-bounded  subset of \(\lambda\times 2\).
    \end{itemize}

    We first consider the case \(n=1\).
    If \(\lambda\) is a limit in \(C\), then
    \(
      \tilde X_\lambda
        =
      \bigcup_{\mu\in C\cap\lambda}\tilde X_\mu
    \)
    satisfies the induction requirements.
    Otherwise we choose an increasing sequence of
    ordinals \(\{\alpha_k : k\in \omega\}\) with limit \(\lambda\)
    such that \(\alpha_0\) is the predecessor of \(\lambda\) in \(C\),
    or \(\alpha_0=0\) if \(\lambda\) is the least element of \(C\).

    The subspace \(\cl\lambda\times\{0\}\cup\alpha_0\times2\) of \(X\)
    is countable and locally compact; therefore it is
    metrizable and zero-dimensional.
    So we may choose increasing sets \(U_k\) for \(k<\omega\) which are
    clopen in this topology and satisfy
    \[
      \cl{\alpha_k\times\{0\}\cup\alpha_0\times2}
        =
      \cl{\alpha_k}\times\{0\}\cup\alpha_0\times2
        \subseteq
      U_k
        \subseteq
      \lambda\times\{0\}\cup\alpha_0\times 2
    \]
    Note that \(U_k\) is also clopen in \(\tilde X_{\alpha_0}\) since it is
    closed in \(\cl\lambda\times\{0\}\cup\alpha_0\times2\) and open in
    \(\lambda\times\{0\}\cup\alpha_0\times 2\).

    We need only describe a base for the points
    \(\<\alpha,1\>\in(\lambda\setminus\alpha_0)\times\{1\}\).
    We do so by letting
    \(\<\alpha,1\>\) be isolated when \(\alpha\not\in\{\alpha_k:k<\omega\}\),
    and giving \(\<\alpha_k,1\>\) the open neighborhoods
    \((U_k\cup((\alpha_k+1)\times\{1\}))\setminus K\) for each compact
    subset \(K\) of \(U_k\cup(\alpha_k\times\{1\})\); that is,
    \(\<\alpha_k,1\>\) is the one point compactifying
    \(U_k\cup(\alpha_k\times\{1\})\).

    The first two requirements of our inductive hypothesis are obviously
    satsified. Note points in \(\lambda\times 2\) are covered by
    the compact countable neighborhood \(U_k\cup((\alpha_k+1)\times\{1\})\)
    for some \(k<\omega\), and for points in
    \((\omega_n\setminus\lambda)\times\{0\}\) we may use a compact
    countable neighborhood from \(X\).
    For the final requirement, note that for \(\alpha<\lambda\), we may
    choose \(\alpha<\alpha_k<\lambda\) and note that
    \(\alpha\times 2\) is contained in the compact
    subset \(U_k\cup((\alpha_k+1)\times\{1\})\) of \(\lambda\times2\).

    For the case \(n>1\), we may assume that the successors in \(C\)
    have uncountable cofinality.
    We again proceed by induction on \(\lambda\in C\).  Again when
    \(\lambda\) is a limit in \(C\),
    \(
      \tilde X_\lambda
        =
      \bigcup_{\mu\in C\cap\lambda}\tilde X_\mu
    \)
    satisfies the given
    requirements; in particular if \(\alpha<\lambda\), then
    \(\alpha<\mu<\lambda\) for some successor \(\mu\in C\) with uncountable
    cofinality. As such, the closure of \(\alpha\times2\) is
    an \(\omega\)-bounded subset of the clopen \(\mu\times 2\) and
    therefore also of \(\lambda\times 2\).
    In case \(\lambda\) is not a limit of \(C\), then \(\lambda\)
    has uncountable cofinality and a predecessor \(\mu\in C\).
    We therefore have that \(\lambda\times\{0\}\) is clopen
    in \(\omega_n\times \{0\}\). Since the cardinality of
    \(\lambda\times \{0\} \cup \mu\times 2\) is less than \(\aleph_n\),
    we may simply apply the induction hypothesis
    to choose an appropriate topology for \(\lambda\times 2\).

    As a result, \(\tilde X=\bigcup_{\lambda\in C}\tilde X_\lambda\)
    is
    \(\omega\)-bounded as any countable set is contained in some
    \(\alpha\times 2\) for \(\alpha<\lambda\in C\).
  \end{proof}

  \begin{theorem}
    For each $n\in \omega$, there is a \(T_2\), locally countable,
    $\omega$-bounded topology on $\omega_n$.
  \end{theorem}

  \begin{proof}
    Apply the previous lemma to \(\omega_n\) with the discrete topology.
  \end{proof}

  \begin{corollary}
    There exists a Kurepa family cofinal in \([\omega_k]^\omega\)
    for each \(k<\omega\).
  \end{corollary}

  \begin{proof}
    We use the family \(\mc K\) of all compact
    open sets in the constructed topology on
    \(\omega_n\) as our witness. Of course, every Lindel\"of set in
    a locally countable space is countable, and the closure of every
    countable set is a compact countable set; thus \(\mc K\)
    is cofinal in \([\omega_n]^\omega\).
    It is Kurepa since for every countable set \(A\), there are only
    countably many distinct compact open sets in \(\cl A\): TODO finish
  \end{proof}

  This is alternatively a corollary of an observation of Todorcevic
  communicated by Dow in \cite{MR1229125}:
  if every Kurepa family of size at most \(\theta\)
  extends to a cofinal Kurepa family, then the same is true of \(\theta^+\).
  So the result follows as
  every Kurepa family \(\mc K\) of size \(\omega\) extends to
  the cofinal Kurepa family \([\bigcup\mc K]^\omega\).

  So we have our desired result.

  \begin{corollary}
    \(S'(\omega_n)\) holds for all \(n<\omega\). Under \(CH\), we have
    both \(S'(\omega_2)\) and \(\neg S(\omega_2)\).
  \end{corollary}









  \newpage

  \section{TODO ALL THIS STUFF NEEDS EDITING STILL}














  As noted in [TODO cite Dow],
  Jensen's one-gap two-cardinal theorem under \(V=L\) [TODO cite] can be used
  to show that there exist cofinal Kurepa families on every cardinal.

  \begin{corollary}[\(V=L\)]
    \(S'(\theta)\) holds for all cardinals.
  \end{corollary}

  In particular, \(S(\omega_2)\) fails under \(CH\), showing the two are
  distinct. Actually, \(CH\) is not required to have \(S(\omega_2)\) fail.


  % TODO: read through Alan's proof and fit to rest of document
        We are going to need a technical lemma (available in Kunen).
        \bigskip

        \begin{lemma}
        Assume that $G\subset \operatorname{Fn}(\omega_2,2)$ is a generic filter,
         and let $\mu\in \omega_2$. Then the final model $V[G]$ can be
         regarded as forcing  with $\operatorname{Fn}(\omega_2\setminus \mu,
         2)$ over the model $V[G_\mu]$.
        In addition, for each $\operatorname{Fn}(\omega_2,2)$-name $\dot A$
        of a subset of $\omega$ (treat as a subset of $\omega\times
        \operatorname{Fn}(\omega_2,2) $),
        there is a canonical name $\dot A(G_\mu)$ where,
        $$\dot A(G_\mu) = \{ (n,p\restriction [\mu,\omega_2))  :
         (n,p)\in \dot A\ \ \mbox{and} \ \ p\restriction \mu\in G_\mu\}$$
        and we get that the valuation of $\dot A(G_\mu)$ by the tail
        of the generic, $G_{\omega_2\setminus \mu}$, is the same as
        the valuation of $\dot A$ by the full generic.
        \end{lemma}


        \begin{theorem}
          If we add $\omega_2$ Cohen reals to a model of CH we get
        that Scheepers' $S(\omega_2)$ (still) fails.
        \end{theorem}

        \begin{proof}
        The forcing poset is $\operatorname{Fn}(\omega_2, 2)$.
        Let $\{ \dot f_{A} : A\in [\omega_2]^\omega\}$ be a family of
        names such that $\dot f_{A}$ is a one-to-one function from $A$ into
        $\omega$. It suffices to only consider sets $A$ from the ground
        model.
        \bigskip

        Put all the lemma  stuff in an elementary submodel $M$ of the universe
        (technically of $H(\kappa)$,  or of $V_\kappa$,
         for some large $\kappa$). Standard methods says that we can assume that
         $|M|=\omega_1 =\mathfrak c$ and that $M^\omega\subset M$ (which means
         that every countable subset of $M$ is a member of $M$).
        \bigskip

        Let $\lambda = M\cap \omega_2$ (same as the supremum of $M\cap
        \omega_2$). Consider the name  $\dot f_{[\lambda,\lambda+\omega)}$.
        What is such a name?  We can assume that it is a set of pairs
         of the form $( (\lambda+k,m), p)$ where $p\in \mathop{Fn}(\omega_2,
         2)$ and, of course, $k,m\in \omega$. This is (almost) equivalent to
         saying
         that $p$ forces that $\dot f_{[\lambda,\lambda+\omega)}(\lambda+k) =
         m$. We don't take all such $p$, in fact for each $k,m$ it is enough
        to take a
         countable set of such $p$ to get an equivalent name
         (Kunen calls it a nice name if we take, for each $k,m$ an
        antichain that is maximal among such conditions).
        Given any such name $\dot f$, let $\operatorname{supp}(\dot f)$
        denote the union of the domains of conditions $p$ appearing in the
        name.



         Also let $Y$ equal $\operatorname{supp}(\dot
         f_{[\lambda,\lambda+\omega)})\setminus \lambda$.
         Let $\delta$ denote the order type of
         $Y$ and let the 2-parameter notation
         $\varphi_{\mu,\lambda}$ be the order-preserving function from
        $\mu\cup Y$ onto  the ordinal $\mu+\delta$.  This lifts canonically to
        an order-preserving bijection $\varphi_{\mu,\lambda}:
        \operatorname{Fn}(\mu\cup Y,2) \mapsto
        \operatorname{Fn}(\mu+\delta,2)$.
        Similarly, we make sense of the name
         $\varphi_{\mu,\lambda}(\dot f_{[\lambda,\lambda+\omega)})$, call it $F_M$.
        Here simply, for each tuple $(~(k,m)~, p)\in \dot
        f_{[\lambda,\lambda+\omega)}$,
         we have that $(~(k,m)~,\varphi_{\mu,\lambda}(p))$ is in $F_M$.
        Again, let $\varphi_{\mu,\lambda}(\dot f_{[\lambda,\lambda+\omega)})$
        be interpreted in the above sense as giving $F_M$ (which is an element
        of $M$).  Note that we do not regard $\delta$ as fixed here, but
        rather simply depending on the $\operatorname{supp}(\dot
        f_{[\lambda,\lambda+\omega)})$ described above. Other values
        replacing $\lambda>\mu$ will result in their own set $Y$
        and canonical map $\varphi_{\mu,\lambda}$; but one thing we do have to
        assume (or arrange) for other values $\alpha$ replacing $\lambda$
        is that $\operatorname{supp}(\dot f_{[\alpha,\alpha+\omega)})$
        intersected with
        $\alpha$ is contained in $\mu$.

        Now the object $F_M$ is an
        element of $M$, and $M$ believes this statement is true:
        $$
        (\forall \beta\in\omega_2)~ (\exists \beta<\lambda\in \omega_2)~~~
        \operatorname{supp}(\dot f_{[\lambda,\lambda+\omega)})
        \cap \lambda\subset \mu \ \ \mbox{and}\ \
        F_M = \varphi_{\mu,\lambda}(\dot f_{[\lambda,\lambda+\omega)})$$


        But now, this  means that,
        not only is there  an $\alpha\in M$,
        $ F_M = \varphi_{\mu,\alpha}(\dot f_{[\alpha,\alpha+\omega)})$
        but also that there is an increasing sequence
         $\{\alpha_\xi : \xi  \in \omega_1\}\subset \lambda$ of
         such $\alpha$'s satisfying that, for each $\xi$
        we have that $\operatorname{supp}(\dot
        f_{[\alpha_\xi,\alpha_\xi+\omega)})$ is contained in $\alpha_{\xi+1}$.

        Choose such a sequence.
         This means that if we let $A = \bigcup_{n>0} [\alpha_n,\alpha_n+\omega)$ we
         have the name $\dot f_A$ in $M$. This then means that all
         the $( (\beta,m) , p)$ appearing in $\dot f_A$ have the property
         that $\mathop{dom}(p)$ is contained in $M$.
        There is, within $M$, a name $\dot g$ satisfying that
         $\dot f_A(\alpha_n+k) = \dot f_{[\alpha_n,\alpha_n+\omega)}(\alpha_n+k)$
        for all $k >\dot g(n)$.

        \bigskip

        We now apply the above Lemma using $\mu = \mu_0$ and we are now
        working in the extension $V[G_\mu]$. We will abuse the notation
        and use $\dot f_{[\alpha_n,\alpha_n+\omega)}$ instead of
          $\dot f_{[\alpha_n,\alpha_n+\omega)}(G_{\mu})$ as defined in
          the Lemma.
        We work for a contradiction. Something special has now happened,
         namely,  the supports of the names $\{
        \dot f_{[\alpha_n, \alpha_n+\omega)} :  0< n<\omega\}$ are pairwise disjoint
        and also disjoint from the support of the name
         $\dot f_{[\lambda,\lambda+\omega)}$ (under the same convention about
         $G_\mu$.  And not only that, these names are pairwise isomorphic (in
         the way that they all map to $F_M$).


        \bigskip

        Since $A$ is disjoint from $[\lambda,\lambda+\omega)$,
        there must be an integer $\ell$
        together with a condition
         $q\in \mathop{Fn}(\omega_2\setminus \mu,2)$ satisfying that for all
         $n>\ell$ , $q$ forces that

        \centerline{
         ``if $k>\dot g(n)$ (since  $\alpha_n+k\in A$) then
         $\dot f_{[\alpha_n,\alpha_n+\omega)}(\alpha_n+k) \neq
        \dot f_{[\lambda,\lambda+\omega)}(\lambda +k)$''.}

        Choose $n$ large enough so that $\mathop{dom}(q) \cap [\alpha_n,
        \mu_{n+1})$ is
        empty.
         Choose $q_1<q\restriction \lambda$ (in $M$) so that
        $$ \varphi_{\mu,\alpha_n}(q_1\restriction \operatorname{supp}(
         \dot f_{[\alpha_n,\alpha_n+\omega)}) =
        \varphi_{\mu,\lambda}(q\restriction \operatorname{supp}(
         \dot f_{[\lambda,\lambda+\omega)}) $$
        and then (again in $M$) choose $q_2 < q_1$
        so that it both forces a value
         $L$ on $\ell+\dot g(n)$
        and subsequently forces a value $m$ on
         $\dot f_{[\alpha_n, \alpha_n+\omega)}(\alpha_n+L+1)$.
        But now, again calculate
        $$ q_3 = \varphi_{\mu,\lambda}^{-1} \circ \varphi_{\mu,\alpha_n}
        (q_2\restriction \operatorname{supp}(\dot
        f_{[\alpha_n,\alpha_n+\omega)}))$$
        and, by the isomorphisms, we have that $q_3$ forces that
         $\dot f_{[\lambda,\lambda+\omega)}(\lambda+L+1) = m$.

        Technically (or with more care)  all of this is taking place in the
        poset $\operatorname{Fn}(\omega_2\setminus \mu,2)$ and this means
        that $q_3$ and $q$ are all compatible with each other.


        Follow
        the bouncing ball: it suffices to consider
         $q(\beta)=e$ and to assume that $q_3(\beta)$ is defined.
        Since $q_3(\beta)$ is defined, we have that
        there is a  $\beta'\in\mathop{dom}(q_2)$ such that $
         \varphi_{\mu,\lambda}(\beta) = \varphi_{\mu,\alpha_n}(\beta')$,
        and that $q_3(\beta) = q_2(\beta')$.
        But, by definition of $q_1$, $\beta'\in\mathop{dom}(q_1)$
        and even  that $q_1(\beta') = q(\beta)$.
         Then, since $q_2<q_1$, we have that $q_2(\beta')=q_1(\beta') =
         q(\beta)$. This completes the circle that $q_3(\beta) = q(\beta)$.
        \bigskip

        Finally, our contradiction is that $q_3\cup q_2\cup q$
        forces that
         $k=L+1$ violates the quoted statement above.
        \end{proof}

  On the other hand, it's also consistent that \(S'(\theta)\) can fail.

  \begin{theorem}
    There's a model where \(S'(\omega_\omega)\)
    fails.
  \end{theorem}


  \begin{proof}
          We will need the model constructed in \cite{MR1045371} in
           which an instance of Chang's conjecture
           $(\aleph_{\omega+1},\aleph_\omega) \rlap{$\rightarrow$}\,{\rightarrow}
           (\aleph_1,\aleph_0)$ is shown to fail.

          We can take as a given (as shown in \cite[Theorem 5]{MR1045371}) that we may
          assume that we have a model $V$
          of GCH in which there are regular limit cardinals $\kappa<\lambda$
          satisfying that
          $(\lambda^{+\omega+1},\lambda^{+\omega})\rlap{$\rightarrow$}\,{\rightarrow}
          (\kappa^{+\omega+1},\kappa^{+\omega})$.


          What this says is that if $L$ is a countable language
          with at least one unary relation symbol $R$ and
           $M$ is a model of $L$  with base set $\lambda^{+\omega+1}$
          in which the interpretation of $R$ has cardinality
           $\lambda^{+\omega}$, then $M$ has an elementary submodel
           $N$ of cardinality $\kappa^{+\omega+1}$ in which
          $R\cap N$ has cardinality $\kappa^{+\omega}$ (of course
           $R\cap N$ is the interpretation of $R$ in $N$ because
          $N\prec M$).

          The interested reader will want to know that it is shown in
          \cite{MR1045371} that if $\kappa$ is a 2-huge cardinal and $j$ is the 2-huge
          embedding with critical point $\kappa$, then with
           $\lambda = j(\kappa)$ one has that
          $(\lambda^{+\omega+1},\lambda^{+\omega})\rlap{$\rightarrow$}\,{\rightarrow}
          (\kappa^{+\omega+1},\kappa^{+\omega})$ holds.
          \bigskip

          Let $\{ h_\xi : \xi \in \lambda^{+\omega+1}\}$
          be a scale in $\Pi\{ \lambda^{+n+1} : n\in \omega\}$
           ordered by the usual mod finite coordinatewise
          ordering. For convenience we may assume that $h_\xi(n) \geq
          \lambda^{+n}$ for all $\xi$ and all $n$.  If $P$ is any poset of
          cardinality less than $\lambda^{+\omega}$, then in the forcing
          extension by $P$,  the sequence
          $\{ h_\xi : \xi \in \lambda^{+\omega+1}\}$  remains cofinal in
           $\Pi\{ \lambda^{+n+1} : n\in \omega\}$.

          \bigskip

          The forcing notion $\mathbb P_0$ is simply the finite condition
          collapse of $\kappa^{+\omega}$, i.e. $\mathbb P_0 = \left(
           \kappa^{+\omega}\right)^{<\omega}$. In the forcing extension by
           $\mathbb P_0$, one now has that the ordinal $\kappa^{+\omega+1}$
           from $V$ is the first uncountable cardinal $\aleph_1$. Then in this
           forcing extension we let $\mathbb P_1$ be the countable condition
           Levy collapse, $Lv(\lambda,\omega_2)$, which collapses all cardinals
           less than $\lambda$ to have cardinality at most $\aleph_1$. The poset
           $\mathbb P_1$ has cardinality $\lambda$. We treat
          $\mathbb P_1$ as containing $\mathbb P_0$
          as a subposet by identifying each $(p_0,1)$ with $p_0$.
          After forcing with $\mathbb P_0*\mathbb P_1$ we will have
          that $\omega_1$ is the ordinal $\left(\kappa^{+\omega+1}\right)^V$,
           $\omega_2$ is the ordinal $\lambda$, and $\omega_\omega$ is the
           ordinal $\left(\lambda^{+\omega}\right)^V$.

          \bigskip

          Now we assume that we have an assignment $\dot f_{\dot A}$ of a
           $\mathbb P_0*\mathbb P_1$-name of a finite-to-one function from
           $\dot A$ into $\omega$ for each
           $\mathbb P_0*\mathbb P_1$-name of a countable subset of
           $\lambda^{+\omega+1}$. We will obtain a contradiction.

          Let $\{ \dot A_\xi : \xi \in \lambda^{+\omega+1}\}$ be an enumeration
          of all the  nice $\mathbb P_0$-names of countable
          subsets of $\lambda^{+\omega}$. For each $\xi\in \lambda^{+\omega+1}$,
           let $\dot f_\xi$ be another notation for $\dot f_{\dot A_\xi}$.
          Since $\mathbb P_0$ forces that $\mathbb P_1$ is countably closed,
          the collection of all nice  $\mathbb P_0$-names will produce all
          the countable sets in the extension by $\mathbb P_0* \mathbb
          P_1$, but $\mathbb P_0*\mathbb P_1$ can introduce new enumerations of
          these names.
          For each $\xi\in \lambda^{+\omega+1}$,
          there is a minimal $\zeta_\xi$ so that $\dot A_{\zeta_\xi}$ is the
          canonical name for the range of $h_\xi$. This means that
           $\dot f_{\zeta_\xi} \circ h_\xi$ is simply the
           $\mathbb P_0 *\mathbb P_1$-name of a finite-to-one function from
           $\omega$ to $\omega$.
            For each $\xi\in \lambda^{+\omega+1}$,
           choose any $p_\xi \in \mathbb P_0*\mathbb P_1$ so that
           there is a nice $\mathbb P_0$-name, $\dot H_\xi$,
           that is forced by
           $p_\xi$  to equal $\dot f_{\zeta_\xi}\circ h_\xi$.
            Choose $\Lambda\subset \lambda^{+\omega+1}$
            of cardinality $\lambda^{+\omega+1}$
            and so that there is a pair $p,\dot H$
            satisfying that $p_\xi = p$ and $\dot H_\xi = \dot H$
            for all $\xi\in \Lambda$. We may assume that $p$
            is in a generic filter $  G$.


          Let $\{ x_\xi : \xi\in \lambda^{+\omega+1}\}$ be any enumeration of
           $H(\lambda^{+\omega+1})$  such that
           $\{ x_\xi : \xi\in \lambda^{+\omega}\}$ is also equal to
           $H(\lambda^{+\omega})$. We choose this enumeration in such a way that
           $x_\xi\in x_\eta$ implies $\xi<\eta$.
          We use relation symbol $R_0$ to code
          (and well order)
           $(H(\lambda^{+\omega+1}), \in)$ as follows: $(\xi,\eta)\in R_0$
          if and only if $x_\xi\in x_\eta$.
          Let $R_1$ be a binary relation on $\kappa^{+\omega}$ so that
          $(\kappa^{+\omega},R_1)$ is isomorphic to $\mathbb P_0$. Let
           $R_2$ be a binary relation on $\lambda$ so that
           $R_2\cap (\kappa^{+\omega}\times \kappa^{+\omega})=R_1$ and
           $(\lambda,R_2)$ is isomorphic to $\mathbb P_0*\mathbb P_1$.
          Let $\psi$ be the poset isomorphism from $\lambda$ to
           $\mathbb P_0*\mathbb P_1$.

          We continue coding. We can code the sequence
           $\{ h_\xi : \xi\in \lambda^{+\omega+1}\}$ as another binary  relation
           $R_3$ on $\lambda^{+\omega+1}$ where $R_3\cap \left(\{\xi\}\times
           \lambda^{+\omega+1}\right) = \{ (\xi,h_\xi(n) ) : n\in \omega\}$
          for each $\xi\in \lambda^{+\omega+1}$. The relation symbol $R_4$ can
          code the sequence $\{ \dot A_\xi : \xi \in \lambda^{+\omega+1}\}$
          where $(\xi, \alpha, \zeta) \in R_4$ if and only if
          $(\check \alpha, \psi(\zeta))$ is in the name $\dot A_\xi$.
           Let $R_5$ code this collection, i.e.
           $(\gamma,n,m,\eta)\in R_5$ if and only if
           $(\check{(n,m)}, \psi(\eta))\in \dot H_\gamma$. Also let $R_6$ code
           (equal)  the set $\Lambda$.
          Finally we use the relation symbol $R_7$ to similarly code the
          sequence $\{ \dot f_\xi  : \xi \in \lambda^{+\omega+1}\}$:
           $(\xi, \alpha, n, \zeta) \in R_7$ if and only if
          $(\check {(\alpha,n)}, \psi(\zeta))$ is in the name $\dot f_\xi$.

          Needless to say, the unary relation symbol $R$
          is interpreted as the set $\lambda^{+\omega}$ for the application of
          $(\lambda^{+\omega+1},\lambda^{+\omega})\rlap{$\rightarrow$}\,{\rightarrow}
          (\kappa^{+\omega+1},\kappa^{+\omega})$.
           Now  we have defined our model
          $M$  of the language $L= \{\in, R, R_0,\ldots, R_7\}$,
           and we choose an elementary submodel
          $N$ witnessing
          $(\lambda^{+\omega+1},\lambda^{+\omega})\rlap{$\rightarrow$}\,{\rightarrow}
          (\kappa^{+\omega+1},\kappa^{+\omega})$. Of course $N$ is really just
          a $\kappa^{+\omega+1}$ sized subset of $\lambda^{+\omega+1}$
          with the additional property that $N\cap \lambda^{+\omega}$ has
          cardinality $\kappa^{+\omega}$.  In the
          forcing extension $N$ has cardinality $\omega_1$ and
           $A= N\cap \lambda^{+\omega}$ is countable.

          We will need the following claim from \cite{MR1045371}

          \begin{claim*} We may assume that $N$ satisfies that
           $N\cap \kappa^{+\omega+1}$ is transitive (i.e. an initial segment).
          \end{claim*}

          \begin{proof}[Proof of Claim]
          Suppose  our originally supplied $N$ fails the conclusion of the
          claim.
          We   know that $\kappa^{+\omega}\in N$,
          (  via $R_1$)
          in which case so is $\kappa^{+\omega+1}$.


          Then set $\beta_0 = \sup(N\cap \kappa^{+\omega+1})$ and consider the
          Skolem closure $\mathop{Hull}(N\cup \beta_0, M)$.  A little informally
          (in that we have to formalize the enumeration of formulas)
          let $\{\varphi_n : n\in \omega\}$ is the enumeration of all formulas
          in the language $L$, and let  $\ell_n$ be the minimal integer such
          that the free variables of $\varphi_n$ are among
           $\{ v_0, \ldots, v_{\ell_n}\}$.
          Then,   for each tuple
          $\langle \xi_1,\ldots, \xi_{\ell_n} \rangle$
          of elements of $\lambda^{+\omega+1}$,
           we define $f_n(\xi_1,\ldots, \xi_{\ell_n})$ to be
           the minimal $\xi_0\in \lambda^{+\omega+1}$ such that
           $M\models \varphi_n(\xi_0,\ldots, \xi_{\ell_n})$.
          If there is no such $\xi_0$,
             in other words if $M\models \lnot\exists x~~  \varphi_n(x,\xi_1,
          \ldots,\xi_{\ell_n})$, then set
           $f_n(\xi_1,\ldots, \xi_{\ell_n})$ to be $0$.
          Now $\mathop{Hull}(N\cup \beta_0,M)$ is just the minimal superset
           $X$
          of $N\cup \beta_0$ that satisfies that
           $f_n[X^{\{1,\ldots,\ell_n\}}] \subset X$ for all $n$. Since this is
           simply a large algebra, we can generate all the terms $t$ of
          the algebraic operations $\{ f_n : n\in \omega\}$.
          It is easily seen that for each $\zeta \in X$, there is a term
          $t(v_1,\ldots, v_m)$ such that $\zeta  = t(\delta_1,\ldots, \delta_m)$
           for some sequence $\langle \delta_1,\ldots, \delta_m\rangle$ with
           each $\delta_i\in  N\cup \beta_0$.
           Assume that
            $\zeta\in \kappa^{+\omega+1}$. By
            re-indexing the variables in the term we can assume
            that there is an $n\leq m$ so that
             $\delta_i  <\beta_0$ for $1\leq i\leq n$
              and $\kappa^{+\omega+1}\leq \delta_i$ for $n<i\leq m$.
              Let $\vec a $ denote
              the tuple $\langle \delta_{n+1},\ldots, \delta_m\rangle$.
            Choose $\eta\in N\cap \kappa^{+\omega+1}$ large enough so that
           $ \{  \delta_1, \ldots, \delta_n\}$ is contained in  $\eta$. Since set-membership
           in $M$ is coded by $R_0$ rather than $\in$ we have to argue a little
           less naturally.
           Consider the set
             $s_0(\eta,\vec a) = \{ t(\gamma_1,\ldots, \gamma_n,\vec a) :
          \{\gamma_1,\cdots,\gamma_n\} \in [\eta]^{\leq n}\}$. Clearly
           $s_0(\eta,\vec a)$ is a member of $H(\lambda^{+\omega+1})$.
           Now define $s_1(\eta,\vec a) $ to be
           $\{ x_\alpha : \alpha\in s_0(\eta, \vec a)\}$,
           and choose the
           unique $\zeta_1\in \lambda^{+\omega+1}$ such
           that $x_{\zeta_1} = s_1(\eta,\vec a)$. We claim that $\zeta_1\in N$.
           Note that $\alpha R_0\zeta_1$ holds if and only if
            $ \alpha \in s_0(\eta,\vec a)$, and therefore
           $$M\models (\forall \alpha)\left[\alpha R_0 \zeta_1 ~~\mbox{iff}~~
           (\exists   \gamma_1\in \eta)\cdots(\exists \gamma_n\in\eta)(\alpha=
           t(\gamma_1,\ldots, \gamma_n,\vec a))\right]~.$$
           By elementarity then we have that $\zeta_1\in N$,
           and by similar reasoning the supremum, $\zeta_0$,
          of $\zeta_1\cap \kappa^{+\omega+1}$ is also in $N$.
             This of course means that $\zeta < \xi_0$.
          \end{proof}

          We use the elementarity of
           $N$  to deduce properties of the families
           $\{ \dot A_\xi : \xi\in N\}$ and $\{ \dot f_\xi : \xi \in
           N\}$.  Actually the collection we are most interested in
           is the family $\{ h_\xi : \xi\in \Lambda\cap N\}$.

           Since  $\mathfrak c < \kappa^{+\omega+1}$ there
            is a function $\langle \varrho_n : n\in \omega\rangle$
            in $\Pi_n \lambda^{+\omega}   $ such that the sequence
             $\{ h_\xi : \xi \in N\}$ is unbounded mod finite
             in $\Pi_n \varrho_n$ (by Shelah's pcf theory).
             This is in Jech somewhere.
             For each $n$, $\rho_n \leq \sup(N\cap \lambda^{+n+2})$.

             Since $\mathbb P_0$ has cardinality less than $|N|=\kappa^{+\omega+1}$,
              the sequence $\{ h_\xi : \xi \in \Lambda\cap N\}$ remains unbounded
              mod finite in $\Pi_n \varrho_n$ (and in
               $\Pi_n (\varrho_n\cap N)$).  Now pass to the extension by
                $G\cap \mathbb P_0$ and let $H$ be the function
                 $\operatorname{val}_{G}(\dot H)$, and we recall
                 that $f_{\zeta_\xi}(h_\xi(n)) = H(n)$ for all $n\in \omega$.
           Now pass to the full extension $V[G]$ and again, since
            $\mathbb P_1$ was forced to be countably closed,
             the family $\{ h_\xi : \xi \in \Lambda\cap N\}$ is still
             unbounded in $\Pi_n (\varrho_n \cap N)$. We let
             $A$ be the countable set $N\cap \lambda^{+\omega}$,
             and for  each
             $\xi\in \Lambda\cap N$, there is an $n_\xi$ such that
              $f_{\xi}(h_\xi(m)) = f_A(h_\xi(m))$ for all $m>n_\xi$.
              There is a single $n$ so that $\Lambda_n
               = \{\xi\in \Lambda\cap N       : n_\xi = n\}$ has cardinality
                $\omega_1 $, and thus
                $\{ h_\xi : \xi\in \Lambda_n\cap N\}$ is also unbounded
                in $\Pi_n (\rho_n\cap N)$.
          This certainly implies that there is an $m>n$
          such that $\{ h_\xi (m) : \xi\in \Lambda_n\cap N\}$ is infinite.
          This completes the proof since  $f_{A}(h_\xi(m)) = H(m)$
          for all $\xi\in \Lambda_n\cap N$.

  \end{proof}

  \begin{question}
    Is \(S'(\theta)\) equivalent to having a Kurepa family on \(\theta\)?
  \end{question}

  \section*{Applications!}

  \begin{figure}[h]
\begin{center}
\begin{tikzpicture}
  \matrix (m) [matrix of math nodes,row sep=3em,column sep=1em,minimum width=2em]
  {
    &
    \pl F \kmarkwin2\menGame{\oneptlind\kappa} &
    \pl F \kmarkwin2\schFillIntGame\kappa \\

    S'(\kappa) &
    \pl F \kmarkwin2\schFillInitialGame\kappa &
    \pl F \kmarkwin2\schFillGame\kappa \\

    &
    &
    \pl F \ktactwin2\schFillStrictGame\kappa \\
  };
  \path[>=latex,->]
    (m-2-1) edge (m-1-2)
    (m-2-1) edge (m-2-2)
    (m-2-1) edge (m-3-3)
    (m-1-2) edge (m-2-2)
    (m-1-3) edge (m-2-3)
    (m-3-3) edge (m-2-3);
  \path[>=latex,<->]
    (m-1-2) edge (m-1-3)
    (m-2-2) edge (m-2-3);
\end{tikzpicture}
\end{center}
\caption{Diagram of Scheeper/Menger game implications with \(S'(\kappa)\)}
\label{GamesDiagram2}
\end{figure}

  \begin{theorem}
    Figure \ref{GamesDiagram2} holds. (Proven in [TODO cite])
    (Actually, TODO double-check that it works with just S', particularly
    the strict game)
  \end{theorem}

  It was left open if these implications can be reversed. The answer is
  consistently no.

  \begin{theorem}
    Let \(\alpha\) be the limit of increasing ordinals \(\beta_n\) for \(n<\omega\).
    If \(\pl F \kmarkwin2\schFillIntGame{\omega_{\beta_n}}\) for all
    \(n<\omega\), then \(\pl F\kmarkwin2\schFillIntGame{\omega_\alpha}\).
  \end{theorem}

  \begin{proof}
    Let \(\sigma_n\) be a winning \(2\)-mark for \(\pl F\) in
    \(\schFillIntGame{\omega_{\beta_n}}\). Define the \(2\)-mark \(\sigma\)
    for \(\pl F\) in \(\schFillIntGame{\omega_\alpha}\) as follows:
    \[
      \sigma(\<C\>,0)
        =
      \sigma_0(\<C\cap\omega_{\beta_0}\>,0)
    \]
    \[
      \sigma(\<C,D\>,n+1)
        =
      \sigma_{n+1}(\<D\cap\omega_{\beta_{n+1}}\>,0)
        \cup
      \bigcup_{m\leq n}
      \sigma_m(\<C\cap\omega_{\beta_m},D\cap\omega_{\beta_m}\>,n-m+1)
    \]

    Let \(\<C_0,C_1,\dots\>\) be an attack by
    \(\pl C\) in \(\schFillIntGame{\omega_\alpha}\), and
    \(\alpha\in\bigcap_{n<\omega}C_n\).
    Choose \(N<\omega\) with \(\alpha<\omega_{\beta_{N+1}}\). Consider the
    attack
    \(\<C_{N+1}\cap\omega_{\beta_{N+1}},C_{N+2}\cap\omega_{\beta_{N+1}},\dots\>\)
    by \(\pl C\) in \(\schFillIntGame{\omega_{\beta_{N+1}}}\). Since
    \(\sigma_{N+1}\) is a winning strategy and
    \(\alpha\in\bigcap_{n<\omega}C_{N+n+1}\cap\omega_{\beta_{N+1}}\), either
    \(\alpha\in\sigma_{N+1}(\<C_{N+1}\cap\omega_{\beta_{N+1}}\>,0)\) and thus
    \(\alpha\in\sigma(\<C_N,C_{N+1}\>,N+1)\), or
    \(
      \alpha
        \in
      \sigma_{N+1}(
        \<C_{N+M+1}\cap\omega_{\beta_{N+1}},C_{N+M+2}\cap\omega_{\beta_{N+1}}\>
        ,M+1
      )
    \)
    for some \(M<\omega\) and thus
    \(
      \alpha
        \in
      \sigma(
        \<C_{N+M+1},C_{N+M+2}\>,
        N+M+2
      )
    \). Thus \(\sigma\) is a winning strategy.
  \end{proof}

  \begin{theorem}
    Let \(\alpha\) be the limit of increasing ordinals \(\beta_n\) for \(n<\omega\).
    If \(\pl F \kmarkwin2\schFillInitialGame{\omega_{\beta_n}}\) for all
    \(n<\omega\), then \(\pl F\kmarkwin2\schFillInitialGame{\omega_\alpha}\).
  \end{theorem}

  \begin{proof}
    Let \(\sigma_n\) be a winning \(2\)-mark for \(\pl F\) in
    \(\schFillInitialGame{\omega_{\beta_n}}\). Define the \(2\)-mark \(\sigma\)
    for \(\pl F\) in \(\schFillInitialGame{\omega_\alpha}\) as follows:
    \[
      \sigma(\<C\>,0)
        =
      \sigma_0(\<C\cap\omega_{\beta_0}\>,0)
    \]
    \[
      \sigma(\<C,D\>,n+1)
        =
      \sigma_{n+1}(\<D\cap\omega_{\beta_{n+1}}\>,0)
        \cup
      \bigcup_{m\leq n}
      \sigma_m(\<C\cap\omega_{\beta_m},D\cap\omega_{\beta_m}\>,n-m+1)
    \]

    Let \(\<C_0,C_1,\dots\>\) be an attack by
    \(\pl C\) in \(\schFillInitialGame{\omega_\alpha}\), and
    \(\alpha\in C_0\).
    Choose \(N<\omega\) with \(\alpha<\omega_{\beta_{N+1}}\). Consider the
    attack
    \(\<C_{N+1}\cap\omega_{\beta_{N+1}},C_{N+2}\cap\omega_{\beta_{N+1}},\dots\>\)
    by \(\pl C\) in \(\schFillInitialGame{\omega_{\beta_{N+1}}}\). Since
    \(\sigma_{N+1}\) is a winning strategy and
    \(\alpha\in C_{N+1}\cap\omega_{\beta_{N+1}}\), either
    \(\alpha\in\sigma_{N+1}(\<C_{N+1}\cap\omega_{\beta_{N+1}}\>,0)\) and thus
    \(\alpha\in\sigma(\<C_N,C_{N+1}\>,N+1)\), or
    \(
      \alpha
        \in
      \sigma_{N+1}(
        \<C_{N+M+1}\cap\omega_{\beta_{N+1}},C_{N+M+2}\cap\omega_{\beta_{N+1}}\>
        ,M+1
      )
    \)
    for some \(M<\omega\) and thus
    \(
      \alpha
        \in
      \sigma(
        \<C_{N+M+1},C_{N+M+2}\>,
        N+M+2
      )
    \). Thus \(\sigma\) is a winning strategy.
  \end{proof}

  \begin{corollary}
    It is consistent that \(S'(\omega_\omega)\) fails,
    but as \(S'(\omega_k)\) holds for all \(k<\omega\), we have
    \(\pl F\kmarkwin2\schFillIntGame{\omega_\omega}\) and
    \(\pl F\kmarkwin2\schFillInitialGame{\omega_\omega}\).
  \end{corollary}

  A tricky topological question: does \(\pl F\win\menGame{X}\) imply
  \(\pl F\kmarkwin2\menGame{X}\)?
  (C showed that )
  Under \(V=L\), we cannot hope to find
  a counterexample using \(X=\oneptlind\kappa\) since
  \(S'(\kappa)\) and thus \(\pl F\kmarkwin2\schFillIntGame{\kappa}\)
  always holds.

  \begin{definition}
    Let \(R_\omega\) be the real numbers with the topology of the usual
    open intervals with countably many elements removed.
  \end{definition}

  \begin{theorem}
    \(\pl F\win\menGame{R_\omega}\).
    If there exists a Kurepa family on the reals, then
    \(\pl F\kmarkwin2\menGame{R_\omega}\).
  \end{theorem}

\bibliographystyle{plain}
\bibliography{../../bibliography}

\end{document}