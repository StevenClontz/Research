\documentclass[11pt]{article}

\pdfpagewidth 8.5in
\pdfpageheight 11in

\setlength\topmargin{0in}
\setlength\headheight{0in}
\setlength\headsep{0.4in}
\setlength\textheight{8in}
\setlength\textwidth{6in}
\setlength\oddsidemargin{0in}
\setlength\evensidemargin{0in}
\setlength\parindent{0.25in}
\setlength\parskip{0.1in} 
 
\usepackage{amssymb}
\usepackage{amsfonts}
\usepackage{amsmath}
\usepackage{mathtools}
\usepackage{amsthm}

\usepackage{fancyhdr}

\usepackage{enumerate}

      \theoremstyle{plain}
      \newtheorem{theorem}{Theorem}
      \newtheorem{lemma}[theorem]{Lemma}
      \newtheorem{corollary}[theorem]{Corollary}
      \newtheorem{proposition}[theorem]{Proposition}
      \newtheorem{conjecture}[theorem]{Conjecture}
      \newtheorem{question}[theorem]{Question}
      
      \theoremstyle{definition}
      \newtheorem{definition}[theorem]{Definition}
      \newtheorem{example}[theorem]{Example}
      \newtheorem{game}[theorem]{Game}
      
      \theoremstyle{remark}
      \newtheorem{remark}[theorem]{Remark}



\pagestyle{fancy}
\renewcommand{\headrulewidth}{0.5pt}
\renewcommand{\footrulewidth}{0pt}
\lfoot{\small \jobname.tex -- Updated on \today}
\chead{\small http://github.com/StevenClontz/Research}
\rfoot{\thepage}
\cfoot{}


% Strategy uparrow shortcuts
\newcommand{\win}{\uparrow}
\newcommand{\prewin}{\uparrow_{\text{pre}}}
\newcommand{\markwin}{\uparrow_{\text{mark}}}
\newcommand{\tactwin}{\uparrow_{\text{tact}}}
\newcommand{\ktactwin}[1]{\uparrow_{#1\text{-tact}}}
\newcommand{\kmarkwin}[1]{\uparrow_{#1\text{-mark}}}
\newcommand{\codewin}{\uparrow_{\text{code}}}
\newcommand{\limitwin}{\uparrow_{\text{limit}}}

\newcommand{\oneptcomp}[1]{#1^*}
\newcommand{\oneptlind}[1]{#1^\dagger}

\newcommand{\congame}[2]{Con_{O,P}(#1,#2)}
\newcommand{\clusgame}[2]{Clus_{O,P}(#1,#2)}

\newcommand{\lfkpgame}[1]{LF_{K,P}(#1)}
\newcommand{\lfklgame}[1]{LF_{K,L}(#1)}

\newcommand{\pfgame}[1]{PF_{F,C}(#1)}

\newcommand{\mengame}[1]{Cov_{C,F}(#1)}
\newcommand{\rothgame}[1]{Cov_{C,S}(#1)}
\newcommand{\altrothgame}[1]{Cov_{P,O}(#1)}

\newcommand{\fillgame}[1]{Fill^{\subseteq}_{M,N}(#1)}
\newcommand{\sfillgame}[1]{Fill^{\subsetneq}_{M,N}(#1)}

\newcommand{\kfillgame}[1]{Fill^{\subseteq}_{C,F}(#1)}
\newcommand{\ksfillgame}[1]{Fill^{\subsetneq}_{C,F}(#1)}

\newcommand{\sigmaprodr}[1]{\Sigma\mathbb{R}^{#1}}
\newcommand{\sigmaprodtwo}[1]{\Sigma2^{#1}}

\newcommand{\concat}{^\frown}
\newcommand{\rest}{\restriction}

\newcommand{\cl}[1]{\overline{#1}}

\newcommand{\pow}[1]{\mc{P}(#1)}

\newcommand{\<}{\langle}
\renewcommand{\>}{\rangle}

\newcommand{\mc}[1]{\mathcal{#1}}

\newcommand{\Lim}{\mathrm{Lim}}
\newcommand{\Suc}{\mathrm{Suc}}

\newcommand{\ds}{\displaystyle}

\newcommand{\rank}{\textrm{rank}}

\newcommand{\scish}{$\sigma$-compactish }

\begin{document}

\begin{example}
Let $L\subseteq \prod_{i=1}^\infty [0,1]$ be the inverse limit space with bonding functions all equal to
\[
  f(x) = \left\{
     \begin{array}{lr}
       2x & : x \leq 0.5 \\
       2-2x & : x \geq 0.5
     \end{array}
   \right.
\]
Then the following hold:
  \begin{enumerate}
    \item The subspaces $C_t = \{\alpha\in L : \alpha(1)=t\}$ are each homeomorphic to the Cantor Set.
    \item All proper subcontinuua are homeomorphic to the unit interval $[0,1]$
    \item All proper subcontinuua are nowhere dense in the space.
  \end{enumerate}
\end{example}

\begin{proof}
For (1), consider that for each number $t\in(0,1)$, there are exactly two preimages under $f$ inside $(0,1)$, giving a natural corespondance with the Cantor tree with branches given by points in the Cantor set $2^\omega$. The arguments for $t=0,1$ are similar.

For (2), we first claim that all basic open sets are of the form $L\cap\prod_{n=1}^\infty B_n$ where $B_n=[0,1]$ for all $n\not=N$. By definition all basic open sets are of the form $L\cap\prod_{n=1}^\infty A_n$ where $A_n=[0,1]$ for all $n\not=N_1,\dots,N_m$. It's easily seen that if $N=N_m$ and $B_N = \bigcap_{i=1}^m (f^{-1})^{(N_m-N_i)}(A_{N_i})$ and $B_n=[0,1]$ otherwise, then $L\cap\prod_{n=1}^\infty A_n = L\cap\prod_{n=1}^\infty B_n$.

Using this fact, we can argue that each proper subcontinuum $K$ is of the form $L \cap \prod_{n=1}^\infty [a_n,b_n]$ where $0\leq a_N < b_N < 1$ for some $N$. 

First note that every maximal basic open set missed by $K$ must be of the form $L\cap\prod_{n=1}^\infty B_n$ here $B_n=[0,1]$ for all $n\not=N$ and $[0,1] \setminus B_N$ is connected, that is, $B_N=[0,a)\cup(b,1]$. (If not, then the disconnection of $[0,1]\setminus B_N$ yields a disconnection for $K$.) Thus $K = \prod_{n=1}^\infty [0,1]\setminus ([0,a_n)\cup(b_n,1]) = \prod_{n=1}^\infty [a_n,b_n]$.

We can see that $a_n<b_n$ always; otherwise the continuum is a single point. Suppose $b_n = 1$ for all $n$, then either $a_n=0$ for all $n$ (contradiction since $K$ is a proper subcontinuum), or $a_N>0$, and $b_{N+1}\leq 1-\frac{a_N}{2}$ (contradition).

Finally, considering $[a_N,b_N]$ where $0\leq a_N<b_n<1$, we note that as only a single sequence $\alpha$ in $K$ may satisfy $\alpha(N)=t$ for each $t\in [a_N,b_N]$, the projection from $K$ onto the $N^{th}$ coordinate is a homeomorphism onto $[a_N,b_N]$.

For (3), we need only observe that for any sequence $\alpha\in K$, an open neighborhood of $\alpha$ would contain infinte sequences $\beta$ such that $\alpha(N)=\beta(N)$.

\end{proof}

\end{document}