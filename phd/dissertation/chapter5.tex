%!TEX root = dissertation.tex
% ^ leave for LaTeXTools build functionality

\chapter{Gruenhage's Locally Finite Games}

A variation of $\gruConGame{X}{x}$, also due to Gruenhage,
may be used to characterize various covering properties, particularly
for locally compact spaces. All spaces are assumed to be $T_1$ in this
chapter.

\section{Characterizations using $\gruKPGame{X}$, $\gruKLGame{X}$}

\begin{game}
  Let $\gruKPGame{X}$ denote the \term{Gruenhage compact/point game}
  with players $\pl K$, $\pl P$. During round $n$, $\pl K$ chooses
  a compact subset $K_n$ of $X$, followed by $\pl P$ choosing a point
  $p_n\in X$ such that $p_n\not\in \bigcup_{m\leq n}K_m$.

  $\pl K$ wins the game if the collection $\{\{p_n\}:n<\omega\}$ is locally
  finite in the space, and $\pl P$ wins otherwise.
\end{game}

This game is often formulated by requiring that the collection
$\{\{p_n\}:n<\omega\}$ be discrete. With the knowledge of at least the latest
move of the opponent, $\pl K$ may guarantee that if $\{\{p_n\}:n<\omega\}$
is locally finite then it is also discrete, since she may require
$p_n\in K_{n+1}$. Thus this formulation is essentially equivalent to the usual
formulation for all existing applications.

We may relate this game to $\gruConGame{X}{x}$ as follows:

\begin{thm}
  If $X$ is locally compact, then for $k<2$:
    \begin{itemize}
      \item $\pl K\win\gruKPGame{X}$ if and only if
        $\pl O\win\gruConGame{\oneptcomp{X}}{\infty}$.
      \item $\pl K\kmarkwin{k}\gruKPGame{X}$ if and only if
        $\pl O\kmarkwin{k}\gruConGame{\oneptcomp{X}}{\infty}$.
      \item $\pl K\ktactwin{k}\gruKPGame{X}$ if and only if
        $\pl O\ktactwin{k}\gruConGame{\oneptcomp{X}}{\infty}$.
    \end{itemize}
\end{thm}

\begin{proof}
  Let $L$ be any of $\nu_0$, $\nu_1$, $\mu_0$, $\mu_1$, or the identity.
  For any sequence $s$ of points in $\oneptcomp X$, let $s'$ be the
  subsequence of non-$\infty$ points in $s$.

  If $\sigma\circ L$ is a winning strategy for $\pl K$ in $\gruKPGame{X}$, let
  $\tau\circ L$ be a strategy for $\pl O$ in $\gruConGame{\oneptcomp X}{\infty}$
  such that $\tau(L(s)) = \oneptcomp X\setminus \sigma(L(s)')$. Then for
  any legal attack $p$ against $\tau$, $p'$ is a legal attack against $\sigma$.
  (The proof of this claim uses the fact that $k<2$.)
  If $p'$ is a finite sequence, then $p$ converges to $\infty$.
  Otherwise, the set $\{\{p'(n)\}:n<\omega\}$ is locally finite in $X$, so
  $\{p'(n):n<\omega\}$ is a closed discrete subset of $X$. Then for every
  neighborhood $U$ of $\infty$, $X\setminus U$ is contained in a compact set,
  so it cannot contain a closed discrete subset. Thus $p'$ and $p$
  converge to $\infty$.

  If $\sigma\circ L$ is a winning strategy for $\pl O$ in
  $\gruConGame{\oneptcomp X}{\infty}$, let $\tau\circ L$ be a strategy for
  $\pl K$ in $\gruKPGame{X}$ such that $\tau(L(s))= X\setminus\sigma(L(s))$.
  Then for any legal attack $p$ against $\tau$, $p$ is a legal attack against
  $\sigma$, so the sequence $p$ converges to $\infty$. For any point $x\in X$
  distinct from the $p(n)$, we may choose a neighborhood $U_x$ of $x$ in $X$
  missing every point in $p(n)$.
  For every $n<\omega$, we may choose a neighborhood $U_{p(n)}$ of $p(n)$ in
  $X$ which misses every distinct $p(m)$. Thus $\{\{p(n)\}:n<\omega\}$ is a
  locally finite collection.
\end{proof}

The reason why $\gruConGame{\oneptcomp X}{\infty}$ and $\gruKPGame{X}$ are
not completely equivalent is due to the fact that $\pl P$ may hide information
from $\pl O$ by playing $\infty$. These moves cannot trivially be ignored
using a $k$-limited information strategy for $k\geq 2$, since $\pl P$ may
ensure that only one ``useful'' move may be seen by $\pl K$ at a time by playing
$k$ $\infty$s between each non-$\infty$.

Applications of limited information strategies for $\pl K$ are already known;
see \cite{MR858337}.

\begin{thm}
  The following are equivalent for a locally compact space $X$:
    \begin{itemize}
      \item $X$ is metacompact
      \item $\pl K \tactwin \gruKPGame{X}$.
    \end{itemize}
\end{thm}

\begin{thm}
  The following are equivalent for a locally compact space $X$:
    \begin{itemize}
      \item $X$ is $\sigma$-metacompact
      \item $\pl K \markwin \gruKPGame{X}$.
    \end{itemize}
\end{thm}

In addition, it's trivial to show the following.

\begin{prop}
  The following are equivalent for any space $X$:
    \begin{itemize}
      \item $X$ is compact
      \item $\pl K \ktactwin{0} \gruKPGame{X}$.
    \end{itemize}
\end{prop}

\begin{proof}
  A $0$-tactic is seeded with zero information about the moves of the opponent
  or the round number, so it must be a constant function valued at $X$.
\end{proof}

A similar game may be considered by allowing the second player to choose
compact sets rather than points, which also provided in \cite{MR858337}
a game-theoretic characterization of a covering property for locally compact
spaces.

\begin{game}
  Let $\gruKLGame{X}$ denote the \term{Gruenhage compact/compact game}
  with players $\pl K$, $\pl L$. This game proceeds analogously to
  $\gruKPGame{X}$, except the second player $\pl L$ chooses comapct sets $L_n$
  missing $\bigcup_{m\leq n}K_n$,
  and $\pl K$ wins if the collection $\{L_n:n<\omega\}$ is locally finite.
\end{game}

As above, this formulation is equivalent to requiring $\{L_n:n<\omega\}$ be
discrete when considering strategies for $\pl K$ which use at least the latest
move of $\pl L$.

\begin{thm}
  The following are equivalent for a locally compact space $X$:
    \begin{itemize}
      \item $X$ is paracompact
      \item $\pl K \win \gruKLGame{X}$.
    \end{itemize}
\end{thm}


\section{Locally compact spaces and predetermined strategies}

As mentioned above, adding knowledge of the round number to a tactic changes the
characterization from metacompact to $\sigma$-metacompact. In fact, the
analogous result holds
for $0$-tactics to $0$-marks, known as predetermined strategies since they
rely only on the round number and not the moves of the opponent.

\begin{thm}
  If $X$ is a locally compact Lindel\"of space, then $\pl K \prewin G_{K,L}(X)$.
\end{thm}

\begin{proof}
  For each $x\in X$, let $U_x$ be an open neighborhood of $x$ with $\cl{U_x}$
  compact. Then as $X$ is Lindel\"of, choose $x_n\in X$ for $n<\omega$ such that
  $\{U_{x_n}:n<\omega\}$ covers $X$. Define the predetermined strategy $\sigma$
  for $\pl K$ by $\sigma(n)=\cl{U_{x_n}}$.

  Let $L:\omega\to \mc K(X)$ legally attack $\sigma$, so
  $L(n)\cap\bigcup_{m\leq n}\sigma(m)=\emptyset$. For each $x\in X$,
  choose $n<\omega$ with $x\in U_{x_n}$. Then $U_{x_n}$ is a neighborhood of
  $x$ which intersects finitely many $L(n)$, so $\{L(n):n<\omega\}$ is
  locally finite.
\end{proof}

\begin{defn}
  A space $X$ is \textbf{hemicompact} if there exist compact sets $K_n$ for
  $n<\omega$ such that every compact set in $X$ is a subset of some $K_n$.
\end{defn}

\begin{thm}
  If $\pl K\prewin \gruKPGame{X}$, then $X$ is hemicompact.
\end{thm}

\begin{proof}
  Let $\sigma$ be a winning predetermined strategy for $\pl K$ in
  $\gruKPGame{X}$. If $C\in \mc K(X)$ is compact, then for each $x\in C$ let
  $U_x$ be an open neighborhood of $x$ which intersects finitely many
  $\sigma(n)$. Thus there exists an open neighborhood of $C$ which intersects
  finitely many $\sigma(n)$.
\end{proof}

\begin{cor}
  The following are equivalent for any locally compact space $X$:
    \begin{itemize}
      \item $X$ is Lindel\"of.
      \item $X$ is $\sigma$-compact.
      \item $X$ is hemicompact.
      \item $\pl K \prewin \gruKPGame{X}$.
      \item $\pl K \prewin \gruKLGame{X}$.
    \end{itemize}
\end{cor}

\section{Compactly generated spaces and predetermined strategies}

\begin{defn}
  A space $X$ is \term{compactly generated} if a set is closed if and
  only if its intersection with every compact set is closed. Such
  spaces are also known as \term{$k$-spaces}.
\end{defn}

All locally compact spaces are $k$-spaces. As will be shown,
the games $\gruKPGame{X}$, $\gruKLGame{X}$ are equivalent for $\pl K$'s
predetermined strategies in Hausdorff $k$-spaces.

\begin{defn}
  A space $X$ is a $k_\omega$-space if there exist compact sets $K_n$ for
  $n<\omega$ such that a set is closed if and
  only if its intersection with every $K_n$ is closed.
\end{defn}

\begin{thm}
  If $X$ is a $k_\omega$-space, then
  $\pl K \prewin \gruKLGame{X}$.
\end{thm}

\begin{proof}
  Let $K_n$ witness that $X$ is a $k_\omega$-space. Define the predetermined
  strategy $\sigma$ for $\pl K$ by $\sigma(n)=K_n$.

  Let $L:\omega\to\mc{K}(X)$ be a legal attack against $\sigma$, and let
  $L_{\omega\setminus n} = \bigcup_{n\leq m<\omega}L(m)$. Then as
    \[
      L_{\omega\setminus n}\cap K_p
        =
      \bigcup_{n\leq m< p}L(m) \cap \sigma(p)
    \]
  is compact for each $p<\omega$, $L_{\omega\setminus n}$ is closed.

  For each $x\in X$, $x\in \sigma(p)$ for some $p$, so
  $x\in X\setminus L_{\omega\setminus p}$ which misses all but finitely
  many $L(n)$, showing that $\{L(n):n<\omega\}$ is locally finite and
  $\sigma$ is a winning predetermined strategy.
\end{proof}

\begin{prop}
  Hemicompact $k$-spaces are $k_\omega$-spaces.
\end{prop}

\begin{proof}
  Let $K_n$ for $n<\omega$ witness hemicompactness.
  If $C\cap K_n$ is closed for each $n<\omega$, then let $K$ be any compact
  set. Since $K\subseteq K_n$ for some $n<\omega$, $C\cap K$ is closed, and
  therefore $C$ is closed.
\end{proof}

As we've already seen that $\pl K \prewin \gruKPGame{X}$ implies
hemicompactness:

\begin{cor}
  The following are equivalent for any $k$-space $X$:
    \begin{itemize}
      \item $X$ is $k_{\omega}$.
      \item $X$ is hemicompact.
      \item $\pl K \prewin \gruKPGame{X}$.
      \item $\pl K \prewin \gruKLGame{X}$.
    \end{itemize}
\end{cor}

\section{Non-equivalence of $\gruKPGame{X}$, $\gruKLGame{X}$}

For $k$-spaces, it has been shown that $\gruKPGame{X}$ and $\gruKLGame{X}$
are equivalent with respect to $\pl K$'s winning predetermined strategies.
Looking at a subspace of the Stone-Cech compactification $\beta\omega$ of
$\omega$ reveals an example for which the predetermined strategies are not
equivalent.

\begin{defn}
  An \term{ultrafilter} on a cardinal $\kappa$ is a maximal filter of non-empty
  subsets of $\kappa$. For each $\alpha\in \kappa$, the ultrafilter
  $\mc F_\alpha$ containing all supersets of $\{\alpha\}$ is called a
  \term{principal ultrafilter}. All
  ultrafilters not of this form are called \term{free ultrafilters}.
\end{defn}

\begin{defn}
  The \term{Stone-Cech compactification} of a cardinal
  $\kappa$ is the space $\beta\kappa$ consisting
  of all ultrafilters on $\kappa$, with open sets of the form
  $U_S=\{\mc F\in\beta\kappa : S\in\mc F\}$ for $S\subseteq \kappa$.
\end{defn}

From these definitions it is easily verified that principal ultrafilters
are isolated, so $\kappa$ with the discrete topology may be viewed as
a dense open subspace of $\beta\kappa$. $\beta\kappa$ is also compact, so
$\pl K\ktactwin{0}\gruKLGame{\beta\kappa}$; of greater interest is the subspace
of $\beta\omega$ consisting of all principal ultrafilters and a single free
ultrafilter $\mc F$, denoted $\omega\cup\{\mc F\}$.

\begin{lem}
  All compact subsets of $\omega\cup\{\mc F\}\subset\beta\omega$ are finite.
  In particular, the difference of compact sets in $\omega\cup\{\mc F\}$
  is compact.
\end{lem}

\begin{proof}
  Let $I=\{n_i:i<\omega\}\cup\{\mc F\}$ be infinite.
  Then $\{U_{\omega\setminus\{n_i:i\geq j\}}:j<\omega\}$ is an open cover of
  $I\cup\{\mc F\}$ with no finite subcover.
\end{proof}

\begin{thm}
  $\pl K\not\prewin \gruKLGame{\omega\cup\{\mc F\}}$ for any free
  ultrafilter $\mc F$.
\end{thm}

\begin{proof}
  Let $\sigma$ be a predetermined strategy for $\pl K$, and define the legal
  counter-attack $H:\omega\to\mc K(X)$ by
  $H(n)=(n\cup\sigma(n+1))\setminus\sigma(n)$. Then for any neighborhood
  $U_S$ of $\mc F$, $S$ is infinite, and since
  $\bigcup_{n<\omega} H(n)\supseteq\omega\setminus\sigma(0)$, $U_S$ meets
  infinitely many of the finite $H(n)$. Thus $\sigma$ is not a winning
  predetermined strategy.
\end{proof}

\begin{thm}
  There exists a free ultrafilter $\mc F$ such that
  $\pl K\prewin \gruKPGame{\omega\cup\{\mc F\}}$.
\end{thm}

\begin{proof}
  Let $\mc F$ be any free ultrafilter, and
  define the predetermined strategy $\sigma$ by
  $\sigma(n)=n^2\cup\{\mc F\}$.

  Consider the set of all legal attacks $A\subseteq\omega^\omega$ by
  $\pl P$ against $\sigma$. For $\{f_i:i\leq m\}\in [A]^{<\omega}$ and
  $m<n<\omega$, each $f_i$ maps only $n$ points into $n^2$, so
  $\bigcup_{i\leq m}\ran{f_i}$ is coinfinite.
  Then $\mc G'=\{\omega\setminus\ran f : f\in A\}$ is contained in a free
  ultrafilter $\mc G$, and if $\mc F=\mc G$, then $\sigma$ is a
  winning predetermined strategy.
\end{proof}

It is not possible to prove that $\pl K\prewin \gruKPGame{\omega\cup\{\mc F\}}$
for arbitrary free ultrafilters in $ZFC$.

\begin{defn}
  A \term{selective ultrafilter} $\mc F$ is a free ultrafilter with the
  property that for every
  partition $\{B_n : n < \omega\}$ of nonempty subsets of $\omega$ such that
  $B_n \not\in \mathcal{F}$ for all $n$, there exists $A \in \mc{F}$ such
  that $|A \cap B_n|=1$ for all $n$.
\end{defn}

\begin{thm}
  CH implies the existence of a selective ultrafilter.
  \cite{MR0080902}
\end{thm}

\begin{thm}
  If $\mc F$ is a selective ultrafilter, then
  $\pl K\not\prewin \gruKPGame{\omega\cup\{\mc F\}}$.
\end{thm}

\begin{proof}
  Let $\sigma$ be a predetermined strategy for $\pl K$ such that
  $\sigma(n)\supset\bigcup_{m<n}\sigma(m)$.
  Then define $B_n=\omega\cap(\sigma(n+1)\setminus\sigma(n))$. Since $B_n$
  is always nonempty finite, $B_n\not\in\mc F$ and there exists $A\in \mc F$
  such that $|A\cap B_n|=1$.

  Define the legal counter-attack $p:\omega\to\omega\cup\{\mc F\}$ by
  $p(n)\in A\cap B_n=A\cap(\sigma(n+1)\setminus\sigma(n))$. Since
  $A=(A\cap\sigma(0))\cup\{p(n):n<\omega\}$, $\{p(n):n<\omega\}\in\mc F$.
  Therefore, every neighborhood of $\mc F$ intersects infinitely many of
  the $p(n)$, and $p$ defeats the predetermined strategy $\sigma$.
\end{proof}

Of particular note is that the author knows of no examples of a
non-$k$-space such that $K\prewin\gruKPGame{X}$.

\begin{ques}
  Does $K\prewin\gruKPGame{X}$ imply $X$ is a $k$-space?
\end{ques}


\section{Tactics and marks for $\gruKPGame{X}$}

While $\pl O\ktactwin{k+1}\gruConGame{\oneptcomp X}{\infty}$ implies
$\pl O\tactwin\gruConGame{\oneptcomp X}{\infty}$, and likewise for Markov
strategies, this result cannot be
immediately extended to $\gruKPGame{X}$. However, this section will
demonstrate a non-trivial example of a space $\mb X$ for which
$\pl K\win\gruKPGame{\mb X}$ but $\pl K\not\kmarkwin{k}\gruKPGame{\mb X}$
for any $k<\omega$.

\begin{thm}
  There exists a compact, zero-dimensional topological space $X$
  such that $X$ has a point-countable cover
  $\mc{U} = \{U_\alpha : \alpha<\omega_1\}$ of
  clopen sets which is not $\sigma$-point-finite.
\end{thm}

\begin{proof}
  (TODO)
\end{proof}

\begin{defn}
  Let $\mb X=(X\times 2^{<\omega})\cup C$ denote a Cantor tree
  of copies of a zero-dimensional, compact space $X$ with a point-countable
  cover $\mc{U} = \{U_\alpha : \alpha<\omega_1\}$ of
  clopen sets which is not $\sigma$-point-finite, along with an uncountable
  subset of the Cantor set
  $C=\{c_\alpha:\alpha<\omega_1\}\in [2^\omega]^{\omega_1}$.
  The topology on $\mb X$ is
  given by decaring $U\times\{s\}$ to be a open
  neighborhood of $\<x,s\>\in X\times 2^{<\omega}$ for each
  open neighborhood $U$ of $x$ in $X$, and declaring
  $B_{\alpha,n}=(U_\alpha\times\{f\rest n: m\leq n<\omega\})\cup\{f\}$ to be
  an clopen neighborhood of $\alpha\in\omega_1$ for each $m<\omega$.
\end{defn}

\begin{defn}
  Let $F\in \omega_1^{<\omega}$ and $m,n<\omega$.
  \[
    K_F = \bigcup_{\alpha \in F} B_{\alpha,0}
  \]
  \[
    z_n:n\to\{0\}
  \]
  \[
    A = \{z_n^\frown \<1\> : n<\omega\}
  \]
  \[
    K_F' = K_F \setminus (X \times A)
  \]
  \[
    L_m = X \times 2^{<m}
  \]
\end{defn}

\begin{lem}
  Then $K_F$, $K_F'$, $L_m$ are compact in $\mb X$.
  Furthermore, every compact set is contained in a union of $K_F'$, $L_m$
  for some $F\in C^{<\omega}$ and $m<\omega$.
\end{lem}

\begin{proof}
  $K_F$ contains $C_F=\{c_\alpha: \alpha\in F\}\subseteq C$, so any
  cover of basic open sets must include $B_{\alpha,n_\alpha}$ for each
  $\alpha\in F$, and the remaining uncovered portion of $K_F$ is a finite
  union of copies of closed subsets of compact $X$. Then $K_F'$ is also
  compact as it is a closed subset of $K_F$, and $L_m$ is compact as it
  is a finite union of copies of compact $X$.

  Let $D$ be compact. Consider the open cover
    \[
      \{
        B_{\alpha,0}
      :
        \alpha<\omega_1
      \}
      \cup
      \{
        X\times\{s\}
      :
        s\in 2^{<\omega}
      \}
    \]
  and note that the finite subcover for $D$ is a refinement of some
  $K_F'$ and $L_m$.
\end{proof}

\begin{thm}
  $\pl K\win\gruKPGame{\mb X}$.
\end{thm}

\begin{proof}
  Since $\{U_\alpha:\alpha<\omega_1\}$ is a point-countable cover,
  for each $x\in X$ let $\alpha_{x,n}<\omega_1$ yield
  ordinals such that $x\in U_{\alpha_{x,n}}$ for $n<\omega$.

  Let $M:\mb X\times \omega\to \mc K(\mb X)$ as follows:
    \[
      M(x,n)
        =
      \left\{
        \begin{array}{lcl}
          K_{\{\alpha_{x',m}:m\leq n\}}
        & : &
          x = \<x',s\> \in X\times 2^{<\omega}
        \\
          K_{\{\alpha\}}
        & : &
          x = c_\alpha \in C
        \end{array}
      \right.
    \]
  and use $M$ to define the strategy $\sigma$ for each $a\in\mb X^{<\omega}$:
    \[
      \sigma(a)
        =
      L_{|a|}
        \cup
      \bigcup_{i\leq |a|}
      M(a(i),|a|)
    \]

  Let $p:\omega\to\mb X$ be a legal attack against $\sigma$. Then as
  $p(n)\not\in L_n$, for each $x=\<x',s\>\in X\times 2^{<\omega}$,
  $X\times\{s\}$ is an open neighborhood of $x$ which contains finitely
  many $p(n)$.

  Then consider $x=c_\alpha$ for some $\alpha<\omega_1$. Then if
  $p(n)=\<x',s\>$ for some $n<\omega$, $x'\in U_\alpha$ or
  $p(n)=c_\alpha$, then
  $\alpha = \alpha_{p(n),N}$ for some $N<\omega$ and $p(m)\not\in B_{\alpha,0}$
  for $n+N<m<\omega$. Otherwise, $p(m)\not\in B_{\alpha,0}$ for any $m<\omega$.
  Either way, $B_{\alpha,0}$ is a neighborhood of $x$ which contains finitely
  many $p(n)$. Therefore, $\sigma$ is a winning strategy.
\end{proof}

We will show that there does not exist any winning $k$-Markov strategy for
$\pl K$ in this game. Knowledge of round number does not assist $\pl K$,
since she may force $\pl P$ to either stay within $C$, or to seed a growing
integer by forcing her to play outside $L_{|s|+1}$ in response to
$\<x,s\>\in X\times 2^{<\omega}$.

\begin{lem}
  If $\pl K\markwin\gruKPGame{\mathbb{X}}$, then
  $\pl K\tactwin\gruKPGame{\mathbb{X}}$.
\end{lem}

\begin{proof}
  Let $\sigma$ be a winning mark for $\pl K$ such that $m\leq n$ and
  $\ran r \subseteq \ran s$ implies $\sigma(r,m)\subseteq\sigma(s,n)$.
  Define $r:\mb X\to\omega$ by
    \[
      r(x)
        =
      \left\{
        \begin{array}{lcl}
          |s|
        & : &
          x = \<x',s\> \in X\times 2^{<\omega}
        \\
          0
        & : &
          x \in C
        \end{array}
      \right.
    \]
  and use $r$ to define the tactic $\tau$ by
    \[
      \tau(\emptyset)
        =
      \sigma(\emptyset,0)
    \]
    \[
      \tau(\<x\>)
        =
      L_{r(x)+1}
        \cup
      \{x\}
        \cup
      \sigma(\<x\>,r(x))
    \]

  Let $p:\omega\to\mb X$ be a legal attack by $\pl P$ against $\tau$.
  If $p(n)\in C$ for $n>N$, then since no $p(n)$ may be repeated,
  $\{\{p(n)\}:N<n<\omega\}$ is a discrete collection, making
  $\{\{p(n)\}:n<\omega\}$ locally finite.

  Otherwise, let $q:\omega\to X\times 2^{<\omega}$ be the subsequence of $p$
  removing points in $C$ with $p(f(i))=q(i)$ (note $f(i)\geq i$).
  It follows that
    \[
      q(0)
        =
      p(f(0))
        \not\in
      \tau(\emptyset)
        \cup
      \bigcup_{m<f(0)}\tau(\<p(m)\>)
        \supseteq
      \tau(\emptyset)
        =
      \sigma(\emptyset,0)
    \]

  Denoting $q(n)=\<x_n',s_n\>$, it's trivial to note that $|s_0|\geq 0$.
  Assuming that $|s_m|\geq m$ for $m\leq n$,
  it then follows that
    \[
      q(n+1)
        =
      p(f(n+1))
        \not\in
      \tau(\emptyset)
        \cup
      \bigcup_{m<f(n+1)}\tau(\<p(m)\>)
    \]
    \[
        \supseteq
      \tau(\emptyset)
        \cup
      \bigcup_{m\leq n}\tau(\<p(f(m))\>)
        =
      \tau(\emptyset)
        \cup
      \bigcup_{m\leq n}\tau(\<q(m)\>)
    \]
    \[
        \supseteq
      \sigma(\emptyset,0)
        \cup
      \bigcup_{m\leq n}\sigma(\<q(m)\>,|s_m|)
        \supseteq
      \sigma(\emptyset,0)
        \cup
      \bigcup_{m\leq n}\sigma(\<q(m)\>,m)
    \]
  and $q(n+1)\not\in\tau(\<q(n)\>)\supseteq L_{|s_n|+1}$ gives
  $|s_{n+1}|\geq|s_n|+1\geq n+1$.
  Thus $q$ is a legal attack on the winning Markov strategy $\sigma$, so
  the collection $\{\{q(n)\}:n<\omega\}$ is locally finite.
  and it follows that $\{\{p(n)\}:n<\omega\}$ is also locally finite.
\end{proof}

\begin{lem}
  If $\pl K\kmarkwin{k+1}\gruKPGame{\mathbb{X}}$, then
  $\pl K\markwin\gruKPGame{\mathbb{X}}$.
\end{lem}

\begin{proof}
  Let $\sigma$ be a winning $k+1$-mark for $\pl K$ such that for
  $s\in[\mb X]^{\leq k+1}$ and $n<\omega$, there exist
  $F(s,n)\in[\omega_1]^{<\omega}$ and $m(s,n)<\omega$ with
  \[
    \sigma(s,n)
      \subseteq
    K_{F(s,n)}'
      \cup
    L_{m(s,n)}
  \]
  Also, it may be assumed that $i<n$ and $\ran r \subseteq \ran s$ implies
  $F(r,i)\subseteq F(s,n)$ and $m(r,i)\leq m(s,n)$, as well as $n\leq m(s,n)$.
\end{proof}


% \begin{prop}
% $K \win \gruKPGame{\mathbb{X}}$.
% \end{prop}

% \begin{thm}(cor of G, game-theoretic proof by me)
% $K\not\tactwin\gruKPGame{\mathbb{X}}$.
% \end{thm}

% \begin{thm}
% $K\not\ktactwin{k}\gruKPGame{\mathbb{X}}$.
% \end{thm}



