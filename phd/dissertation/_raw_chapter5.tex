%!TEX root = dissertation.tex
% ^ leave for LaTeXTools build functionality

\chapter{Locally Finite Games}

Results pertaining to the Locally Finite game related to the W games.


\section{basic results}


\begin{thm}
The following are equivalent for a locally compact space $X$:
    \begin{itemize}
    \item $X$ is paracompact
    \item $K \uparrow G_{K,L}(X)$.
    \end{itemize}
\end{thm}

However, often it is the presence of ``limited information'' strategies which can characterize interesting properties of a space.

\begin{defn}
A \textbf{limited information strategy} for a game is a function whose domain is restricted to less information than all previous moves by the opposing player.
\end{defn}

In the above mentioned paper, Gruenhage used the following limited information strategies to prove some interesting characterizations based on the game $G_{K,P}(X)$.

\begin{defn}
A \textbf{tactical strategy} considers only the most recent move by the opposing player. If Player $Z$ has a winning tactical strategy for a game $G$, this may be denoted $Z \uparrow_{\text{tact}} G$.
\end{defn}

\begin{defn}
A \textbf{Markov strategy} considers only the most recent move by the opposing player and the turn number. If Player $Z$ has a winning Markov strategy for a game $G$, this may be denoted $Z \uparrow_{\text{mark}} G$.
\end{defn}

\begin{thm}
The following are equivalent for a locally compact space $X$:
    \begin{itemize}
    \item $X$ is metacompact
    \item $K \uparrow_{\text{tact}}G_{K,P}(X)$.
    \end{itemize}
\end{thm}

\begin{thm}
The following are equivalent for a locally compact space $X$:
    \begin{itemize}
    \item $X$ is $\sigma$-metacompact
    \item $K \uparrow_{\text{mark}}G_{K,P}(X)$.
    \end{itemize}
\end{thm}

Upon learning these results, one might wonder the consequences of the existence of this type of limited information strategy:

\begin{defn}
A \textbf{predetermined strategy} considers only the turn number. If Player $Z$ has a winning predetermined strategy for a game $G$, this may be denoted $Z \prewin G$.
\end{defn}

Intuitively, if a player is using a predetermined strategy, then that player decides every move he or she will make before the game even begins, ignoring the other player's moves.

Consider the following trivial result:

\begin{defn}
A \textbf{button-mashing strategy} is a constant function. If Player $Z$ has a winning button-mashing strategy for a game $G$, this may be denoted $Z \uparrow_{\text{mash}} G$.
\end{defn}

\begin{prop}
The following are equivalent for any space $X$:
    \begin{itemize}
    \item $X$ is compact
    \item $K \uparrow_{\text{mash}}G_{K,P}(X)$.
    \end{itemize}
\end{prop}

Observing that giving $K$ the added information of turn number to a tactical strategy (making it Markov) changed the characterization of a metacompact space into a $\sigma$-metacompact space, it would be very convenient if adding that same information to a button-mashing strategy (making it predetermined) would similarly change the characterization of a compact space into $\sigma$-compact.

\begin{prop}
If $K \prewin G_{K,P}(X)$, then $X$ is $\sigma$-compact.
\end{prop}

\begin{proof}
Let $K_n$ be the sets given by the winning predetermined strategy. If they did not union to $X$, then the counter play $p_n=p$ for some $p\in X\setminus \bigcup_n K_n$ would defeat the ``winning'' strategy.
\end{proof}

\begin{thm}
If $Y$ is a locally compact, Lindel\"of space, then $K \prewin G_{K,P}(X)$.
\end{thm}

\begin{proof}
Let $\mathcal{K}$ be a collection of compact neighborhoods whose interiors cover $X$. By Lindel\"of, let $\{K_n : n<\omega\}$ be a countable subcollection whose interiors cover $X$. We then define the predetermined strategy $\sigma(n)=\bigcup_{m\leq n} K_n$.

Let $p_n$ give a play by $P$. If $p$ is a cluster point of the $p_n$, then every open set about $p$ contains infinitely many $p_n$. Let $K_N$ be some compact neighborhood in $\{K_n : n<\omega\}$ which covers $p$. Then $K_N$ contains infinitely many $p_n$, which means sometime after round $N$, $P$ played in a set already covered by the strategy $\sigma$, which is an illegal move. Thus $\sigma$ is a winning predetermined strategy.
\end{proof}

\begin{cor}
The following are equivalent for a locally compact space $X$:
    \begin{itemize}
    \item $X$ is $\sigma$-compact
    \item $X$ is Lindel\"of
    \item $K \prewin G_{K,P}(X)$.
    \end{itemize}
\end{cor}

We now turn our attention to an example of a $\sigma$-compact space for which no predetermined strategy exists (which must, of course, not be locally compact). In fact, $P$ will instead have a winning tactical strategy.

\begin{defn}
Let $M=\omega^2\cup\{\infty\}$ denote the \textbf{metric fan space} with the topology generated by the singletons in $\omega^2$ and sets of the form $((\omega\setminus n)\times\omega) \cup \{\infty\}$ for $n<\omega$.
\end{defn}

\begin{prop}
For each compact set $C$ in $M$, there exists a minimal dominating function $f_C$ such that for each $(x,y)\in C\setminus\{\infty\}$, $f(x)> y$.
\end{prop}

\begin{lem}
$P \markwin G_{K,P}(M)$ where $M$ is the metric fan space. (This implies $K\not\uparrow G_{K,P}(M)$.)
\end{lem}

\begin{proof}
Let $P$ respond to the move $C\in K[X]$ by $K$ on round $n$ with the point $p=(n,s_C)$ such that $s_C = \min(\{y<\omega : f_C(n) < y\}$. It is easy to see that either $p_n\rightarrow \infty$, so $P$ has a winning tactical strategy.
\end{proof}

Furthermore, by a theorem due to Eric van Douwen...

\begin{thm}
Every first-countable non-locally countably compact space has the metric fan space $M$ as a closed subspace.
\end{thm}

... we have the following corollary:

\begin{cor}
$P \markwin G_{K,P}(X)$ where $X$ is a first-countable non-locally countably compact space. (This implies $K\not\uparrow G_{K,P}(X)$.)
\end{cor}

(Open question: does $P \tactwin G_{K,P}(X)$?)

\begin{thm}
$P \not\tactwin D_{K,P}(M)$ where $M$ is the metric fan space.
\end{thm}

\begin{proof}
Give $P$ the tactic $\sigma$. Suppose that for all $n<\omega$, there is an upper bound $m<\omega$ so that for each $C \in K[M]$, if $\pi_1(\sigma(C))=n$, then $\pi_2(\sigma(C))<m$. We may then define $f(n)=m$, and let $C_f=\{(x,y):f(x)<y\}\in K[M]$. $\sigma(C_f)$ must show a contradiction if $\sigma$ is legal.

So it follows that there is some $n<\omega$ such that there are compact sets $C_i\in K[M]$ with $\pi_1(\sigma(C_i))=n$ and $\pi_2(\sigma(C_i))<\pi_2(\sigma(C_{i+1}))$. The play $\left<C_0,\sigma(C_0),C_1,\sigma(C_1),\dots\right>$ is a counter to $\sigma$.
\end{proof}

While $K\prewin G_{K,P}(X)$ implies $X$ is $\sigma$-compact, it in fact implies something stronger.

\begin{defn}
A space $X$ is \textbf{hemicompact} if there exists a chain of increasing compact sets $K_0\subseteq K_1 \subseteq \dots$ such that every compact set in $X$ is a subset of some $K_n$.
\end{defn}

\begin{lem}
If $K\prewin G_{K,P}(X)$, then $X$ is hemicompact. Furthermore, any predetermined winning strategy for $K$ witnesses hemicompactness.
\end{lem}

\begin{proof}
Let $\sigma$ be a predetermined strategy for $K$ in the game $G_{K,P}(Y)$ such that there exists a compact set $C$ with $C \not\subseteq \sigma(n)$ for all $n$. On each turn, have $P$ play some $y_n\in C \setminus \sigma(n)$. Then the $y_n$ are an infinite subset of the compact set $C$ and must have a cluster point in $C$, showing $\sigma$ is not a winning strategy.

Thus if $K$ has a winning predetermined strategy, it witnesses that $Y$ is hemicompact.
\end{proof}

In fact, for locally compact spaces, finding winning predetermined strategies for $G_{K,P}(X)$ and $G_{K,L}(X)$ are equivalent problems.

\begin{thm}
The following are equivalent for any locally compact space $X$:
  \begin{itemize}
  \item $X$ is hemicompact.
  \item $K \prewin G_{K,L}(X)$.
  \item $K \prewin G_{K,P}(X)$.
  \end{itemize}
\end{thm}

\begin{proof}
Let $Y$ be hemicompact, witnessed by $K_n=\sigma(n)$. Let $L_0,L_1,\dots$ be a play by $L$ in $G_{K,L}(X)$. Suppose that this play defeats $\sigma$. Then let $x\in X$ be the point such that for all neighborhoods $U$ of $x$, $U$ hits infinite $L_n$. Let $C$ be a compact neighborhood of $x$, which must hit infinite $L_n$. As $K_n$ witnesses hemicompactness, $C \subseteq K_N = \sigma(N)$ for some $N$. But then $C\subset K_N$ intersects infinitely many $L_n$, which shows that the play $L_0,L_1,\dots$ was illegal. Thus $\sigma$ defeats every legal play by $L$ and is thus a winning predetermined strategy for $K$ in $G_{K,L}(X)$.

We conclude by noting that any winning strategy for $G_{K,L}(X)$ is a winning strategy for $G_{K,P}(X)$, and the existence of a winning predetermined strategy for $G_{K,P}(X)$ implies hemicompact by the previous lemma.
\end{proof}

\begin{cor}
The following are equivalent for any locally compact space $X$:
  \begin{itemize}
  \item $X$ is Lindel\"of.
  \item $X$ is $\sigma$-compact.
  \item $X$ is hemicompact.
  \item $K \prewin G_{K,L}(X)$.
  \item $K \prewin G_{K,P}(X)$.
  \end{itemize}
\end{cor}

The compact-point and compact-compact games are also useful in inspecting compactly generated ``k''-spaces.

\begin{defn}
A topological space is called a \textbf{$k$-space} if the following condition is satisfied: \[C \subseteq X \text{ is closed in } X\, \Leftrightarrow \, C\cap K \text{ is closed in } K \text{ for all compact sets } K\in K[X]\]
\end{defn}

\begin{defn}
A topological space is called a \textbf{$k_\omega$-space} if there exist $K_0,K_1,\dots \in K[X]$ that satisfy the following condition: \[C \subseteq X \text{ is closed in } X\, \Leftrightarrow \, C \cap K_n \text{ is closed in } K_n \text{ for all } n\]
\end{defn}

\begin{thm}
The following are equivalent for any Hausdorff $k$-space $X$:
  \begin{itemize}
  \item $X$ is hemicompact.
  \item $X$ is $k_{\omega}$.
  \item $K \prewin G_{K,P}(X)$.
  \end{itemize}
Furthermore, all predetermined strategies for $K$ witness hemicompact and $k_\omega$, and any witness to hemicompact/$k_\omega$ witnesses the other and serves as a predetermined strategy for $K$.
\end{thm}

\begin{proof}
If $X$ is hemicompact, then let it be witnessed by $K_n$. We claim $K_n$ also witnesses $k_\omega$. Note that the forward implication of $k_\omega$ always holds for $T_1$ spaces as $C\cap K_n$ is closed in $X$, and thus in every $K_n$. So assume $C\cap K_n$ is closed in $K_n$ for all $n$. Let $H$ be any compact set. As $X$ is hemicompact, $H\subseteq K_n$ for some $n$. Note $C\cap H = (C\cap K_n)\cap H$. As both $C \cap K_n$ and $H$ are closed in $K_n$, $C\cap H$ is closed in $K_n$, and thus $C\cap H$ is closed in $H$. As $Y$ is $k$ and $C\cap H$ is closed in $H$ for all compact $H$, $C$ is closed, showing the backwards implication.

Now if $Y$ is $k_\omega$, let it be witnessed by $K_n$. Give $K$ the predeterined strategy $\sigma(n)=K_n$ for the game $G_{K,P}(X)$, and let $p_n$ be the result of a legal counter by $P$. Suppose by way of contradiction that $p$ is a cluster point of the $p_n$. Note $p\in \sigma(N)$ for some $N$. $p$ is a cluster point of $\{p_n : n\geq N\}$ but $p\not\in \{p_n : n \geq N\}$. Also, $\{p_n : n \geq N\} \cap \sigma(m)$ is finite for all $m$, and thus closed, so as $\sigma(n)$ witnesses $k_\omega$, $\{p_n : n\geq N\}$ is closed and must contain its cluster point $p$, which is a contradiction. Thus $\sigma$ is a winning predetermined strategy for $K$ in $G_{K,P}(Y)$.

Finally, if $K \prewin G_{K,P}(X)$, $X$ is hemicompact by the previous lemma.
\end{proof}

For $k$-spaces, it turns out that finding winning predetermined strategies for $G_{K,P}(X)$ and $G_{K,L}(X)$ are also equivalent problems.

\begin{thm}
For any hemicompact Hausdorff $k$-space $X$, $K \prewin G_{K,L}(X)$.
\end{thm}

\begin{proof}
Let $X$'s hemicompactness be witnessed by $K_n=\sigma(n)$. Note that this also witnesses $k_\omega$ by the proof of the previous theorem. Let $H_0,H_1,\dots$ be a counter by $H$ for the game $G_{K,L}(X)$ in response to $\sigma$. Suppose by way of contradiction the counter was legal and defeats $\sigma$. Then there is a point $x$ such that every neighborhood of $x$ hits infinitely many of the $H_n$.

Now, $x\in\sigma(N)$ for some $N$, and since the play $H_0,H_1,\dots$ is legal, $x\not\in H_n$ for all $n\geq N$. Consider the set $H_\omega=\bigcup_{n\geq N} H_n$. Note that as the $K_n$ witness $k_\omega$, $H_\omega$ is closed if and only if $H_\omega \cap \sigma(m)$ is closed in $\sigma(m)$ for all $m$. In fact, since every $H_n$ is a subset of some $\sigma(m)$ (by hemicompactness), $H_\omega \cap \sigma(m)$ is a finite union of some $H_n$, and is thus closed in $Y$.

We thus have that $H_\omega$ is a closed set not containing $x$. But since every neighborhood of $x$ intersects $H_\omega$, $x$ is a limit point of the closed set $H_\omega$ and should be included, demonstrating our contradiction. Thus $\sigma$ is a winning predetermined strategy for $K$ in the game $G_{K,L}(X)$.
\end{proof}

\begin{cor}
The following are equivalent for any Hausdorff $k$-space $X$:
  \begin{itemize}
  \item $X$ is hemicompact.
  \item $X$ is $k_{\omega}$.
  \item $K$ has a winning predetermined strategy in $G_{K,L}(X)$.
  \item $K$ has a winning predetermined strategy in $G_{K,P}(X)$.
  \end{itemize}
\end{cor}

It's natural to question whethere there is ever any difference between finding winning predetermined strategies for $G_{K,P}(X)$ and $G_{K,L}(X)$. We now look to a (non-locally compact, non-$k$) Hausdorff space where the distinction arises:

\begin{defn}
Given a set $X$, an ultrafilter on $X$ is a collection $\mathcal{F}\subseteq\mathcal{P}(X)$ such that
    \begin{enumerate}
    \item $\emptyset\not\in \mathcal{F}$
    \item $A,B\in\mathcal{F} \Rightarrow A\cap B \in \mathcal{F}$
    \item $A\in\mathcal{F}$ and $A \subseteq B$ $\Rightarrow$ $B\in\mathcal{F}$
    \item $\forall A \subseteq X(A\in\mathcal{F}$ or $X\setminus A \in \mathcal{F})$
    \end{enumerate}
As a result, ultrafilters which contain a finite set contain only one singleton (and are called \textbf{principal}). Otherwise, ultrafilters which contain no finite sets are called \textbf{free}.
\end{defn}

\begin{defn}
The \textbf{Stone-Cech compactification} $\beta\omega$ of $\omega$ is the collection of ultrafilters on $\omega$. The principal ultrafilters containing a singleton $\{n\}$ are each identified with $n$ itself and are isolated. Free ultrafilters $\mathcal{F}$ are given neighborhoods of the form \[\{\mathcal{G} : \mathcal{G} \text{ is an ultrafilter on } \omega \text{ and } A \in \mathcal{G}\} = A \cup \{\mathcal{G}: \mathcal{G} \text{ is a free ultrafilter on } \omega \text{ and } A \in \mathcal{G}\}\] for each $A\in\mathcal{F}$.

Alternately $\beta\omega=\omega\cup\{\mathcal{F} : \mathcal{F}$ is a free ultrafilter on $\omega\}$ where $\omega$ is discrete and the free ultrafilters have the local base described above.
\end{defn}

\begin{defn}
A \textbf{single-ultrafilter space} is a subset of $\beta\omega$ containing all elements of $\omega$ and a single ultrafilter $\mathcal{F}$.
\end{defn}

\begin{prop}
The compact sets of a single-ultrafilter space are exactly the finite subsets of the space. Thus a single-ultrafilter space is neither locally compact nor $k$.
\end{prop}

Regardless of the ultrafilter chosen, we can see that $K$ has no hope of having a winning predetermined strategy for $G_{K,L}$ played on a single-ultrafilter space.

\begin{prop}
If $X$ is any single-ultrafilter space with the ultrafilter $\mathcal{F}$, then $K\not\prewin G_{K,L}(X)$.
\end{prop}

\begin{proof}
Compact sets are exactly finite sets in this space. Therefore, the difference of any two compact sets is compact.

Give $K$ the predetermined strategy $\sigma(n)$. $H$ counters with \[H_n=(n\cup\sigma(n+1))\setminus\sigma(n)\] on turn $n$. Since any free ultrafilter contains only unbounded sets, every neighborhood $A\cup\{\mathcal{F}\}$ of $\mathcal{F}$ must intersect infinitely many $H_n$, defeating $\sigma$.
\end{proof}

However, while it is consistant that there is an ultrafilter which defies the existance of a predetermined winning strategy for $K$ in $G_{K,P}$...

\begin{prop}
If a selective ultrafilter $\mathcal{F}$ exists (this is independent of ZFC), then $K$ has no winning predetermined strategy in the compact-point game $G_{K,P}(Y)$ for the single selective ultrafilter space $Y=\omega \cup \{\mathcal{F}\}$.
\end{prop}

\begin{proof}
Let $\sigma$ be a predetermined strategy for $K$. By the definition of a selective ultrafilter, for every partition $\{B_n : n < \omega\}$ of subsets of $\omega$ such that $B_n \not\in \mathcal{F}$ for all $n$, there exists $A \in \mathcal{F}$ such that $|A \cap B_n|=1$ for all $n$. So then let \[B_n = \omega \cap \sigma(n) \setminus \sigma(n-1)\]

Note that $B_n$ is finite and thus $B_n \not\in \mathcal{F}$, so there exists $A\in \mathcal{F}$ such that $|A \cap B_n|=1$. Let $p_n$ be the singleton in $A \cap B_{n+1}$, so $\{p_n : n < \omega\}$ is cofinite in $A$, and thus is also a member of $\mathcal{F}$. Thus $p_n$ converges to $\mathcal{F}$, and counters the strategy $\sigma$.
\end{proof}

... in general we can find many ultrafilters for which $K\prewin G_{K,P}$.

\begin{thm}
Let $a_n$ be a sequence such that the sequence $\frac{a_n}{n}$ is unbounded above. Then there is an ultrafilter $\mathcal{F}$ such that $\sigma(n)=(\sum_{m\leq n} a_m )\cup \{\mathcal{F}\}$ is a winning predetermined strategy for $K$ in $G_{K,P}(\omega\cup\{\mathcal{F}\})$.
\end{thm}

\begin{proof}
Let $\mathcal{P}$ be the collection of all legal plays by $P$ against the strategy $\sigma$. Consider a finite collection of plays $P_0,\dots,P_{n-1}\in \mathcal{P}$. As $\frac{a_m}{m}$ is unbounded above, we may find infinitely many $m$ such that $\frac{a_m}{m}>n \Rightarrow mn<a_m$. As the $a_m$ partition $\omega$ such that $P$ may only play at most $m$ points in each part, there are infinitely many parts which are not filled, and thus $\bigcup_{m<n} P_m$ is not cofinite.

It then follows that the closure of $\mathcal{P}$ under finite unions and subsets, along with all finite sets, is an ideal. Its dual filter may then be extended to an ultrafilter $\mathcal{F}$ such that every possible play by $P$ is the complement of some member of $\mathcal{F}$.
\end{proof}

So we can see that there are non-$k$ spaces $X$ for which $K\prewin G_{K,P}(X)$. However, we have found no such spaces for the game $G_{K,L}(X)$. So we conclude with this open question:

\begin{ques}
$K \prewin G_{K,L}(X) \Rightarrow X$ is a $k$-space?
\end{ques}





\section{Cantor space example}



\begin{ex}
Let $X$ be a zero-dimensional, compact L-space (hereditarally Lindeloff and non-separable). It is a fact that there exists a point-countable collection $\mathcal{U}=\{U_\alpha : \alpha<\omega_1\}$ of clopen sets in $X$, and it is also true that any point-finite subcollection of $\mathcal{U}$ is countable. %That is, if we let $A_x = \{\alpha : x \in U_\alpha\}$, then $|A_x|\leq\omega$ for all $x\in X$.

Let $C = \{c_\alpha : \alpha <\omega_1\}$ be any uncountable subset of the Cantor space $2^\omega$. Let $X_s = X \times \{s\}$ for each $s \in 2^{<\omega}$, and $U_{\alpha,s} = U_\alpha \times \{s\}$.

Finally, let \[\mathbb{X} = C \cup \bigcup_{s\in 2^{<\omega}} X_s\] be a tree of $2^{<\omega}$ copies of $X$, and where \[c_\alpha \cup \bigcup_{n < \omega} U_{\alpha, x_\alpha \restriction n}\] is an open set about each $c_\alpha$.
\end{ex}



\begin{defn}
Let $S\in[\omega_1]^{<\omega}$ and $m<\omega$. Define
    \[K_S = \bigcup_{\alpha \in S} \left( c_\alpha \cup \left(\bigcup_{s < c_\alpha} U_{\alpha,s}\right)\right)\]
    \[ A = \{z^\frown \<1\> : z \in 1^{<\omega}\}\]
    \[ K^*_S = K_S \setminus \bigcup_{s \in A} X_s \]
and
    \[L_m = \bigcup_{s \in 2^{<m}} X_s\]
and observe that every compact set is dominated by $K^*_S \cup L_m$ for some $S,m$. Intuitively, $K^*_S$ collects the branches of $U_\alpha$ converging up to $c_\alpha$ for each $\alpha \in S$ while avoiding copies $X_s$ of $X$ for each $s$ in an antichain $A$, and $L_m$ collects the copies $X_s$ of $X$ with $|s| < m$ at the base of the tree.
\end{defn}

\begin{prop}
Without loss of generality, $P$ always plays points in $\bigcup_{s \in 2^{<\omega}} X_s$.
\end{prop}

\begin{prop}
$K \win \gruKPGame{\mathbb{X}}$.
\end{prop}

\begin{proof}
In response to a point $\<x,s\>$, $K$ observes that there are only countably many $\alpha$ such that $U_\alpha \times \{s\}$ contains $\<x,s\>$ (by point-countability of $X$). Enumerate these as $\alpha_n$. $K$ makes a promise that during round $m$, $K$ will forbid some superset of $K_{\{\alpha_n : n\leq m\}}$. Finally, $K$ also always forbids a superset of $L_{|s|+1}$.

Suppose $P$'s moves clustered at some point. Since $K$ forbade $L_{|s|+1}$ during each round, that point must be $c_\alpha$ for some $\alpha$. $P$'s play then must have included a subsequence of points $\<x_0,s_0\>,\<x_1,s_1\>,\<x_2,s_2\>\dots$ such that $x_n \in U_\alpha$ and $s_n \leq s_{n+1} \leq c_\alpha$. However, in response to $\<x_0,s_0\>$, $K$ made a promise to eventually forbid a superset of $K_{\{\alpha\}}$, making every $\<x_n,t_n\>$ illegal after that round.
\end{proof}

\begin{thm}
$K\not\tactwin\gruKPGame{\mathbb{X}}$.
\end{thm}

\begin{proof}
This is actually a corollary of G's theorem in [?]. The following is a direct game-theoretic proof.

Suppose that $\sigma(\<x,s\>)$ was a winning strategy for $K$ and assume
  \[
    \sigma(\<x,s\>) = \bigcup_{|t|\leq |s|} \sigma(\<x,t\>) = \sigma'(x,|s|)
  \]
Thus there exists some $f: \omega_1 \to \omega$ such that $\sigma'(x,f(\alpha))$ covers every neighborhood of $c_\alpha$ for all $x\in U_\alpha$. (If not, $P$ wins by taking the $\alpha$ for which $f$ is not defined, and may always play $\<x,s\>$ in a neighborhood of $c_\alpha$ for which $\sigma'(x,|s|)$ doesn't cover a neighborhood of $c_\alpha$.) Fix $n$ for which $f(\alpha)=n$ for $\alpha$ in an uncountable set $A$.

Since the collection $\{U_\alpha : \alpha \in A\}$ is uncountable, it is not point-finite. Fix $x$ so that $x$ belongs to $U_\alpha$ for all $\alpha$ in an infinite $B \subseteq A$. Finally, consider $\sigma'(x,n)$. For each $\alpha\in B$, $\sigma'(x,f(\alpha))=\sigma'(x,n)$ covers $c_\alpha$. Since $\{c_\alpha : \alpha \in B\}$ is a closed infinite discrete set, we have a contradiction to the compactness of $\sigma'(x,n)$.
\end{proof}



\begin{thm}
$K\not\ktactwin{2}\gruKPGame{\mathbb{X}}$.
\end{thm}

\begin{proof}
Suppose $\sigma(\<x,s\>,\<y,t\>)$ was a winning 2-tactical strategy. We may define $S(x,y,n)\in [\omega_1]^{<\omega}$ (increasing on $n$) and $n<m(x,y,n)<\omega$ such that for each $(x,y)$,
  \[
    \bigcup_{s,t \in 2^{\leq n}} \sigma(\<x,s\>,\<y,t\>) \subseteq
    K^*_{S(x,y,n)} \cup L_{m(x,y,n)}
  \]
and so we assume
  \[
    \sigma(\<x,s\>,\<y,t\>) =
    K^*_{S(x,y,\max(|s|,|t|))} \cup L_{m(x,y,\max(|s|,|t|))}
  \]

Select an arbitrary point $x' \in X$. We define a tactical strategy
  \[
  \tau(x,s) =
  K^*_{S(x,x',m(x,x',|s|)+1)} \cup L_{m(x,x',m(x,x',|s|)+1)}
  \]
We complete the proof by showing $\tau$ is a winning tactical strategy (a contradiction).

Suppose
\[
\<x_0, s_0\>, \<x_1, s_1\>, \<x_2, s_2\>, \dots
\]
successfully countered $\tau$ by clustering at $c\in C$ (the strategy trivially prevents clustering elsewhere). Let $z_n = \<0,\dots,0\>$ with $n$ zeros. We claim
\[
\<x_0, s_0\>, \<x', {z_{m(x_0,x',|s_0|)}}^\frown\<1\>\>, \<x_1, s_1\>, \<x', {z_{m(x_1,x',|s_1|)}}^\frown\<1\>\>,  \<x_2, s_2\>, \<x', {z_{m(x_2,x',|s_2|)}}^\frown\<1\>\>, \dots
\]
is a successful counter to $\sigma$.

We will need the fact that, as $\<x_{i+1},s_{i+1}\>$ was legal against $\tau$:
  \[
    |s_i| <
    m(x_i,x',|s_i|)+1 =
    |{z_{m(x_i,x',|s_i|)}}^\frown\<1\>|
  \]
  \[
    <
    m(x_i,x',m(x_i,x',|s_i|)+1) =
    m(x_i,x',|{z_{m(x_i,x',|s_i|)}}^\frown\<1\>|) \leq
    |s_{i+1}|
  \]

Note that $m(x,y,\max(|s|,|t|))$ is increasing throughout this play of the game versus $\sigma$:
  \[
    m(x_i,x',\max(|s_i|,|{z_{m(x_i,x',|s_i|)}}^\frown\<1\>|))
  \]
  \[
    =
    m(x_i,x',|{z_{m(x_i,x',|s_i|)}}^\frown\<1\>|)
  \]
  \[
    \leq
    |s_{i+1}|
  \]
  \[
    <
    m(x_{i+1},x',|s_{i+1}|)
  \]
  \[
    =
    m(x_{i+1},x',\max(|s_{i+1}|,|{z_{m(x_i,x',|s_i|)}}^\frown\<1\>|))
  \]
  \[
    =
    |{z_{m(x_{i+1},x',|s_{i+1}|)}}|
  \]
  \[
    <
    |{z_{m(x_{i+1},x',|s_{i+1}|)}}^\frown\<1\>|
  \]
  \[
    <
    m(x_{i+1},x',|{z_{m(x_{i+1},x',|s_{i+1}|)}}^\frown\<1\>|)
  \]
  \[
    =
    m(x_{i+1},x',\max(|s_{i+1}|,|{z_{m(x_{i+1},x',|s_{i+1}|)}}^\frown\<1\>|))
  \]

We turn to showing that $\<x', {z_{m(x_{i+1},x',|s_{i+1}|)}}^\frown\<1\>\>$ is always a legal move. Since ${z_{m(x_{i+1},x',|s_{i+1}|)}}^\frown\<1\>$ is on the antichain avoided by any $K^*$, the problem is reduced to showing that this move isn't forbidden by
  \[
  L_{m(x_{i+1},x',\max(|s_{i+1}|,|{z_{m(x_i,x',|s_i|)}}^\frown\<1\>|))}
  \]
which we can see here:
  \[
    m(x_{i+1},x',\max(|s_{i+1}|,|{z_{m(x_i,x',|s_i|)}}^\frown\<1\>|)) =
    m(x_{i+1},x',|s_{i+1}|) <
    |{z_{m(x_{i+1},x',|s_{i+1}|)}}^\frown\<1\>|
  \]

We can conclude by showing that $\<x_{i+1},s_{i+1}\>$ is always a legal move. We can see it avoids
  \[
  L_{m(x_{i},x',\max(|s_{i}|,|{z_{m(x_i,x',|s_i|)}}^\frown\<1\>|))}
  \]
since
  \[
    m(x_{i},x',\max(|s_{i}|,|{z_{m(x_i,x',|s_i|)}}^\frown\<1\>|)) =
    m(x_{i},x',|{z_{m(x_i,x',|s_i|)}}^\frown\<1\>|) \leq
    |s_{i+1}|
  \]

Since $\<x_{i+1},s_{i+1}\>$ was legal against $\tau$, it avoided
  \[
    K^*_{S(x_h,x',m(x_h,x',|s_h|)+1)} =
    K^*_{S(x_h,x',\max(|s_h|,|{z_{m(x_h,x',|s_h|)}}^\frown\<1\>|))}
  \]
for $h\leq i$. And when $h<i$, we see it avoids:
  \[
    K^*_{S(x_{h+1},x',\max(|s_{h+1}|,|{z_{m(x_h,x',|s_h|)}}^\frown\<1\>|))} =
    K^*_{S(x_{h+1},x',|s_{h+1}|)}
  \]
  \[
    \subseteq
    K^*_{S(x_{h+1},x',m(x_{h+1},x',|s_{h+1}|)+1)}
  \]

This concludes the proof.
\end{proof}

\begin{thm}
$K\not\ktactwin{k}\gruKPGame{\mathbb{X}}$.
\end{thm}

\begin{proof}
The proof proceeds in parallel to the proof of $K\not\ktactwin{2}\gruKPGame{\mathbb{X}}$.

Suppose $\sigma(\<x_0,s_0\>,\dots,\<x_{k},s_{k}\>)$ was a winning $(k+1)$-tactical strategy. We may define $S(x_0,\dots,x_{k},n)\in [\omega_1]^{<\omega}$ (increasing on $n$) and $n<m(x_0,\dots,x_{k},n)<\omega$ such that for each $(x_0,\dots,x_{k})$,
  \[
    \bigcup_{s_0,\dots,s_k \in 2^{\leq n}} \sigma(\<x_0,s_0\>,\dots,\<x_{k},s_{k}\>) \subseteq
    K^*_{S(x_0,\dots,x_{k},n)} \cup L_{m(x_0,\dots,x_{k},n)}
  \]
and so we assume
  \[
    \sigma(\<x_0,s_0\>,\dots,\<x_{k},s_{k}\>) =
    K^*_{S(x_0,\dots,x_{k},\max(|s_0|,\dots,|s_k|))} \cup L_{m(x_0,\dots,x_{k},\max(|s_0|,\dots,|s_k|))}
  \]

Select an arbitrary point $x' \in X$. Let $M^0(x,n)=m(x,x',\dots,x',n)$ and $M^{i+1}(x,n)=M^0(x,M^i(x,n)+1)$. We define a tactical strategy
  \[
  \tau(x,s) = K^*_{S(x,x',\dots,x',M^{k-1}(x,|s|)+1)} \cup L_{m(x,x',\dots,x',M^{k-1}(x,|s|)+1)}
  \]
We complete the proof by showing $\tau$ is a winning tactical strategy (a contradiction).

Suppose
\[
\<x_0, s_0\>, \<x_1, s_1\>, \<x_2, s_2\>, \dots
\]
successfully countered $\tau$ by clustering at $c\in C$ (the strategy trivially prevents clustering elsewhere). Let $z_n = \<0,\dots,0\>$ with $n$ zeros. We claim
\[
  \<x_0, s_0\>,
  \<x', {z_{M^0(x_0,|s_0|)}}^\frown\<1\>\>,
  \<x', {z_{M^1(x_0,|s_0|)}}^\frown\<1\>\>,
  \dots,
  \<x', {z_{M^{k-1}(x_0,|s_0|)}}^\frown\<1\>\>,
\]
\[
  \<x_1, s_1\>,
  \<x', {z_{M^0(x_1,|s_1|)}}^\frown\<1\>\>,
  \<x', {z_{M^1(x_1,|s_1|)}}^\frown\<1\>\>,
  \dots,
  \<x', {z_{M^{k-1}(x_1,|s_1|)}}^\frown\<1\>\>,
  \dots
\]
is a successful counter to $\sigma$.

We will need the fact that, as $\<x_{i+1},s_{i+1}\>$ was legal against $\tau$:
  \[
    |s_i| <
    M^0(x_i,|s_i|)+1 =
    |{z_{M^0(x_i,|s_i|)}}^\frown\<1\>| <
    M^0(x_i,M^0(x_i,|s_i|)+1)+1
  \]
  \[
    =
    M^1(x_i,|s_i|)+1 =
    |{z_{M^1(x_i,|s_i|)}}^\frown\<1\>| <
    \dots <
    |{z_{M^{k-1}(x_i,|s_i|)}}^\frown\<1\>|
  \]
  \[
    =
    M^{k-1}(x_i,|s_i|) + 1 <
    m(x_i,x',\dots,x',M^{k-1}(x_i,|s_i|)+1) \leq
    |s_{i+1}|
  \]

Note that $m(x_0,\dots,x_{k},\max(|s_0|,\dots,|s_k|))$ is increasing throughout this play of the game versus $\sigma$:
  \[
    m(x_i,x',\dots,x',\max(|s_i|,|{z_{M^0(x_i,|s_i|)}}^\frown\<1\>|,\dots,|{z_{M^{k-1}(x_i,|s_i|)}}^\frown\<1\>|))
  \]
  \[
    =
    m(x_i,x',\dots,x',|{z_{M^{k-1}(x_i,|s_i|)}}^\frown\<1\>|)
  \]
  \[
    =
    m(x_i,x',\dots,x',M^{k-1}(x_i,|s_i|)+1)
  \]
  \[
    \leq
    |s_{i+1}|
  \]
  \[
    <
    M^0(x_{i+1},|s_{i+1}|)
  \]
  \[
    =
    m(x_{i+1},x',\dots,x',|s_{i+1}|)
  \]
  \[
    =
    m(x_{i+1},x',\dots,x',\max(|s_{i+1}|,|{z_{M^0(x_i,|s_i|)}}^\frown\<1\>|,\dots,|{z_{M^{k-1}(x_i,|s_i|)}}^\frown\<1\>|))
  \]
  \[
    =
    |{z_{m(x_{i+1},x',\dots,x',|s_{i+1}|)}}|
  \]
  \[
    =
    |{z_{M^0(x_{i+1},|s_{i+1}|)}}|
  \]
  \[
    <
    |{z_{M^0(x_{i+1},|s_{i+1}|)}}^\frown\<1\>|
  \]
  \[
    <
    m(x_{i+1},x',\dots,x',|{z_{M^0(x_{i+1},|s_{i+1}|)}}^\frown\<1\>|)
  \]
  \[
    =
    m(x_{i+1},x',\dots,x',\max(|s_{i+1}|,|{z_{M^0(x_{i+1},|s_{i+1}|)}}^\frown\<1\>|,|{z_{M^1(x_{i},|s_{i}|)}}^\frown\<1\>|,\dots,|{z_{M^{k-1}(x_{i},|s_{i}|)}}^\frown\<1\>|))
  \]
  \[
    \vdots
  \]
  \[
    <
    m(x_{i+1},x',\dots,x',\max(|s_{i+1}|,|{z_{M^0(x_{i+1},|s_{i+1}|)}}^\frown\<1\>|,\dots,|{z_{M^{k-1}(x_{i+1},|s_{i+1}|)}}^\frown\<1\>|))
  \]

We turn to showing that $\<x', {z_{M^j(x_{i+1},|s_{i+1}|)}}^\frown\<1\>\>$ is always a legal move. Since ${z_{M^j(x_{i+1},|s_{i+1}|)}}^\frown\<1\>$ is on the antichain avoided by any $K^*$, the problem is reduced to showing that this move isn't forbidden by
  \[
    L_{m(x_{i+1},x',\dots,x',\max(|s_{i+1}|,|{z_{M^0(x_{i+1},|s_{i+1}|)}}^\frown\<1\>|,\dots,|{z_{M^{j-1}(x_{i+1},|s_{i+1}|)}}^\frown\<1\>|,|{z_{M^{j}(x_{i},|s_{i}|)}}^\frown\<1\>|,\dots,|{z_{M^{k}(x_{i},|s_{i}|)}}^\frown\<1\>|))}
  \]
  \[
    =
    L_{m(x_{i+1},x',\dots,x',|{z_{M^{j-1}(x_{i+1},|s_{i+1}|)}}^\frown\<1\>|)}
  \]
which we can see here:
  \[
    m(x_{i+1},x',\dots,x',|{z_{M^{j-1}(x_{i+1},|s_{i+1}|)}}^\frown\<1\>|)
  \]
  \[
    =
    m(x_{i+1},x',\dots,x',M^{j-1}(x_{i+1},|s_{i+1}|)+1)
  \]
  \[
    =
    M^0(x_{i+1},M^{j-1}(x_{i+1},|s_{i+1}|)+1)
  \]
  \[
    =
    M^j(x_{i+1},s_{i+1})
  \]
  \[
    <
    |{z_{M^j(x_{i+1},|s_{i+1}|)}}^\frown\<1\>|
  \]

We can conclude by showing that $\<x_{i+1},s_{i+1}\>$ is always a legal move. We can see it avoids
  \[
  L_{m(x_{i},x',\dots,x',\max(|s_{i}|,|{z_{M^0(x_i,|s_i|)}}^\frown\<1\>|,\dots,|{z_{M^{k-1}(x_i,|s_i|)}}^\frown\<1\>|))}
  \]
since
  \[
    m(x_{i},x',\dots,x',\max(|s_{i}|,|{z_{M^0(x_i,|s_i|)}}^\frown\<1\>|,\dots,|{z_{M^{k-1}(x_i,|s_i|)}}^\frown\<1\>|))
  \]
  \[
    =
    m(x_{i},x',\dots,x',|{z_{M^{k-1}(x_i,|s_i|)}}^\frown\<1\>|)
  \]
  \[
    =
    m(x_{i},x',\dots,x',M^{k-1}(x_i,|s_i|)+1)
  \]
  \[
    \leq
    |s_{i+1}|
  \]



Since $\<x_{i+1},s_{i+1}\>$ was legal against $\tau$, it avoided
  \[
    K^*_{S(x_h,x',\dots,x',M^{k-1}(x_h,|s_h|)+1)}
  \]
  \[
    =
    K^*_{S(x_h,x',\dots,x',\max(|s_h|,|{z_{M^{0}(x_h,|s_h|)}}^\frown\<1\>|,\dots,|{z_{M^{k-1}(x_h,|s_h|)}}^\frown\<1\>|))}
  \]
for $h\leq i$. And when $h<i$, we see it avoids both:
  \[
    K^*_{S(x_{h+1},x',\dots,x',\max(|s_{h+1}|,|{z_{M^0(x_{h+1},|s_{h+1}|)}}^\frown\<1\>|,\dots,|{z_{M^{j-1}(x_{h+1},|s_{h+1}|)}}^\frown\<1\>|,|{z_{M^{j}(x_{h},|s_{h}|)}}^\frown\<1\>|,\dots,|{z_{M^{k}(x_{h},|s_{h}|)}}^\frown\<1\>|))}
  \]
  \[
    =
    K^*_{S(x_{h+1},x',\dots,x',|{z_{M^{j-1}(x_{h+1},|s_{h+1}|)}}^\frown\<1\>|)}
  \]
  \[
    =
    K^*_{S(x_{h+1},x',\dots,x',M^{j-1}(x_{h+1},|s_{h+1}|)+1)}
  \]
  \[
    \subseteq
    K^*_{S(x_{h+1},x',\dots,x',M^{k-1}(x_{h+1},|s_{h+1}|)+1)}
  \]
and:
  \[
    K^*_{S(x_{h+1},x',\dots,x',\max(|s_{h+1}|,|{z_{M^0(x_{h},|s_{h}|)}}^\frown\<1\>|,\dots,|{z_{M^{k}(x_{h},|s_{h}|)}}^\frown\<1\>|))}
  \]
  \[
    =
    K^*_{S(x_{h+1},x',\dots,x',|s_{k+1}|)}
  \]
  \[
    \subseteq
    K^*_{S(x_{h+1},x',\dots,x',M^{k-1}(x_{h+1},|s_{h+1}|)+1)}
  \]


This concludes the proof.
\end{proof}




\section{various examples}

\begin{ex}
If $\mathcal{F}$ is a free ultrafilter on $\omega$, let $L(\mathcal{F})=\omega \cup \{\mathcal{F}\}$ as a subspace of the Stone-Cech compactification $\beta\omega$ be the \textbf{single ultrafilter line}. There is some ultrafilter $\mathcal{F}$ such that $K \prewin \gruKPGame{L(\mathcal{F})}$ and $K \tactwin \gruKPGame{L(\mathcal{F})}$.

($L(\mathcal{F})$ is not compactly generated, and thus not locally compact.)
\end{ex}

\begin{proof}
Let $a_n$ be a sequence such that the sequence $\frac{a_n}{n}$ is unbounded above. Then there is an ultrafilter $\mathcal{F}$ such that $\sigma(n)=(\sum_{m\leq n} a_m )\cup \{\mathcal{F}\}$ is a winning predetermined strategy for $K$ in $\gruKPGame{L(\mathcal{F})}$.

Let $\mathcal{P}$ be the collection of all legal plays by $P$ against the strategy $\sigma$. Consider a finite collection of plays $P_0,\dots,P_{n-1}\in \mathcal{P}$. As $\frac{a_m}{m}$ is unbounded above, we may find infinitely many $m$ such that $\frac{a_m}{m}>n \Rightarrow mn<a_m$. As the $a_m$ partition $\omega$ such that $P$ may only play at most $m$ points in each part, there are infinitely many parts which are not filled, and thus $\bigcup_{m<n} P_m$ is not cofinite.

It then follows that the closure of $\mathcal{P}$ under finite unions and subsets, along with all finite sets, is an ideal. Its dual filter may then be extended to an ultrafilter $\mathcal{F}$ such that every possible play by $P$ is the complement of some member of $\mathcal{F}$, making $\sigma$ a winning predetermined strategy.

A winning tactic can then be easily constructed by using the moves by $P$ as the round number in the predetermined strategy.
\end{proof}

\begin{ex}
Let $T(\mathcal{F}) = 2^{\leq\omega}$ where $2^{<\omega}$ is discrete and for each $c\in 2^\omega$, $\{c \restriction \alpha : \alpha \leq \omega\}$ is homeomorphic to $L(\mathcal{F})$. This is called the \textbf{single ultrafilter tree}. There is some ultrafilter $\mathcal{F}$ such that $K \prewin \gruKPGame{L(\mathcal{F})}$ and $K \tactwin \gruKPGame{L(\mathcal{F})}$.

($T(\mathcal{F})$ is not compactly generated, and thus not locally compact.)
\end{ex}

\begin{proof}
Assume without loss of generality that $P$ does not play points in $2^\omega$.

We use a winning predetermined strategy $\sigma^*(n)$ for $L(\mathcal{F})$ and let $\sigma(n) = \bigcup_{m \in \sigma^*(n)} 2^m$. Note that if $P$ has a counter which converges to some $c\in \,^\omega2$, then $P$ would have a counter within a single branch. Since each branch is homeomorphic to $L(\mathcal{F})$; this is impossible.

A winning tactic can then be easily constructed by using the moves by $P$, taking the level of the tree played upon as the round number in the predetermined strategy.
\end{proof}

\begin{ex}
Let $M = \omega^2 \cup \{\infty\}$ be the \textbf{metric fan} where $\omega^2$ is discrete and $\infty$ has neighborhoods of the form $M \setminus (n\times\omega)$ for any $n<\omega$. Then $K \not\win \gruKPGame{M}$. (In fact, $P \markwin \gruKPGame{M}$.)

($M$ is not locally compact, but is compactly generated.)
\end{ex}

\begin{proof}
For each compact set $C$ in $M$, there exists a minimal dominating function $f_C$ such that for each $(x,y)\in C\setminus\{\infty\}$, $f(x)> y$.

So let $P$ respond to the move $C\in K[X]$ by $K$ on round $n$ with the point $p=(n,s_C)$ such that $s_C = \min(\{y<\omega : f_C(n) < y\}$. It is easy to see that $p_n\rightarrow \infty$, so $P$ has a winning Markov strategy.
\end{proof}

\begin{ex}
Let $S = \omega^2 \cup \{\infty\}$ be the \textbf{sequential fan} where $\omega^2$ is discrete and $\infty$ has neighborhoods of the form $M \setminus \{(x,y) : x<f(y)\}$ for any $f:\omega\to\omega$. Then $K \prewin \gruKPGame{S}$ and $K\tactwin \gruKPGame{S}$.

($S$ is not locally compact, but is compactly generated.)


\end{ex}

\begin{proof}
Let $\sigma(n)=\omega\times(n+1) \cup \{\infty\}$. By defining $f(y)$ to be greater than the $x$-coordinate of all $P$'s plays through round $y$, we see that $M\setminus\{(x,y): x<f(y)\}$ misses every move by $P$, so $P$ cannot converge to $\infty$.

A winning tactic can be easily constructed by using the $y$-coordinate of $P$'s moves as the round number in the predetermined strategy.
\end{proof}
