%!TEX root = dissertation.tex
% ^ leave for LaTeXTools build functionality

\chapter{Toplogical Games and Strategies\\ of Perfect and Limited Information}

The goal of this paper is to explore the applications of limited information
strategies in existing topological games. There are a variety of frameworks
for modeling such games, so we establish one within this chapter which we 
will use for this manuscript.

\section{Games}

Intuitively, the games studied in this paper are two-player games for which
each player takes turns making a choice from a set of possible moves. At
the conclusion of the game, the choices made by both players are examined,
and one of the players is declared the winner of that playthrough.

Games may be modeled mathematically in various ways,
but we will find it convenient to think of them in terms defined by
Gale and Stewart. \cite{gale}

\begin{defn}
  A \term{game} is a tuple $\<M,W\>$ such that 
  $W\subseteq M^{\omega}$. $M$ is set of \term{moves} for
  the game, and $M^{\omega}$ is the set of all possible 
  \term{playthroughs} of the game.

  $W$ is the set of \term{winning playthroughs} or \term{victories} for the 
  first player, and $M^{\omega}\setminus W$ is the set of victories for the 
  second player. ($W$ is often called the \term{payoff set} for the
  first player.)
\end{defn}

Within this model, we may imagine two players $\pl A$ and $\pl B$ playing a 
game which consists of \term{rounds} enumerated for each $n<\omega$.
During round $n$, $\pl A$ chooses $a_n\in M$, followed by $\pl B$ choosing
$b_n\in M$. The playthrough corresponding to those choices would be
the sequence $p = \<a_0,b_0,a_1,b_1,\dots\>$. If $p\in W$, then $\pl A$
is the winner of that playthrough, and if $p\not\in W$, then $\pl B$ is
the winner. (No ties are allowed.)

Rather than explicitly defining $W$, we 
typically define games by declaring the \term{rules} that each player must 
follow and the \term{winning condition} for the first player. 
Then a playthrough is in $W$ if either 
the first player made only \term{legal moves} which observed the game's rules
and the playthrough satisified the winning condition, or the second player
made an \term{illegal move} which contradicted the game's rules.

As an illustration, we could model a game of chess (ignoring stalemates) 
by letting
\[
  M = \{\<p,s\>: p \text{ is a chess piece and } 
                 s \text{ is a space on the board } \}
\]
representing moving a piece $p$ to the space $s$ on the board. Then the
rules of chess restict White from moving pieces which belong to
Black, or moving a piece to an illegal space on the board. 
\footnote{
  In practice, $M$ is often defined as the union of two sets, such
  as white pieces and black pieces in chess. For example,
  the first player may choose open sets in a topology, while the second player
  chooses points within the topological space.
}
The winning condition could then
``inspect'' the resulting positions of pieces on the board after each move 
to see if White attained a
checkmate. This winning condition along with the rules implicitly define the 
set $W$ of winning playthroughs for White.


\subsection{Infinite and Topological Games}

Games never technically end within this model, since 
playthroughs of the game are infinite sequences. However, for all practical
purposes many games end after a finite number of turns.

\begin{defn}
  A game is said to be an \term{finite game} if for every playthrough
  $p\in M^\omega$ there exists a round $n<\omega$ such that
    $ 
      [p\rest n] = \{q\in M^\omega : q\supseteq p\rest n \}
    $
  is a subset of either $W$ or $M^\omega\setminus W$.
\end{defn}

Put another way, a finite game is decided after a finite number
of rounds, after which the game's winner could not change even if further
rounds were played.
Games which are not finite are called \term{infinite games}. 

As an illustration of an infinite game, we may consider a simple example due to 
Baker \cite{baker}. 

\begin{game}
  Let $\bakergame{A}$ denote a game with players $\pl A$ and $\pl B$,
  defined for each subset $A\subset \mathbb{R}$.
  In round $0$, $\pl A$ chooses a number $a_0$, followed by $\pl B$ choosing
  a number $b_0$ such that $a_0<b_0$.
  In round $n+1$, $\pl A$ chooses a number $a_{n+1}$ such that 
  $a_n<a_{n+1}<b_n$, followed by $\pl B$ choosing a number $b_{n+1}$ such that
  $a_{n+1}<b_{n+1}<b_n$.

  $\pl A$ wins the game if the sequence $\<a_n:n<\omega\>$ converges to a 
  point in $A$, and $\pl B$ wins otherwise.
\end{game}

Certainly, $\pl A$ and $\pl B$ will never be in
a position without (infinitely many) legal moves available, and provided that
$A$ is non-trivial, there is a playthrough such that for all $n<\omega$,
the segment $(a_n,b_n)$ intersects both $A$ and $\mathbb{R}\setminus A$. 
Such a playthrough could never be decided in a finite number of moves,
so the winning condition considers the infinite sequence of moves made by the
players and declares a victor at the ``end'' of the game.

\begin{defn}
  A \term{topological game} is a game defined in terms of an arbitrary
  topological space.
\end{defn}

Topological games are usually infinite games for non-trivial spaces.
(The meaning of trivial here depends on the game played.)
One of the earliest examples
of a topological game is the Banach-Mazur game, proposed by Stanislaw Mazur
as Problem 43 in Stefan Banach's Scottish Book (1935). A more comprehensive
history of the Banacy-Mazur and other topological games may be found in 
Telgarsky's survey on the subject \cite{telgarsky}.

The original game was defined for subsets of the real line; however, 
we give a more general definition here.

\begin{game}
  Let $\bmgame{X}$ denote the \term{Banach-Mazur game} with players $\pl E$,
  $\pl N$ defined for each topological space $X$.
  In round $0$, $\pl E$ chooses a nonempty open set $E_0\subseteq X$, followed
  by $\pl N$ choosing a nonempty open subset $N_0\subseteq E_0$.
  In round $n+1$, $\pl E$ chooses a nonempty open subset $E_{n+1}\subseteq N_n$, 
  followed by $\pl N$ choosing a nonempty open subset 
  $N_{n+1}\subseteq E_{n+1}$.

  $\pl E$ wins the game if $\bigcap_{n<\omega} E_n = \emptyset$, 
  and $\pl N$ wins otherwise.
\end{game}

For example, if $X$ is a locally compact Hausdorff space, $\pl N$ can ``force'' 
a win by choosing $N_0$ such that $\cl{N_0}$ is compact, and choosing 
$N_{n+1}$ such that 
$N_{n+1}\subseteq\cl{N_{n+1}}\subseteq O_{n+1}\subseteq N_n$ 
(possible since $N_n$ is a compact Hausdorff $\Rightarrow$ normal space). 
Since $\bigcap_{n<\omega} E_n = \bigcap_{n<\omega} N_n$ is the decreasing 
intersection of compact sets, it cannot be empty.

This concept of when (and how) a player can ``force'' a win in certain
topological games is the focus of this manuscript.



\section{Strategies}

We shall make the notion of forcing a win in a game rigorous by introducing
``strategies'' and ``attacks'' for games.

\begin{defn}
  A \term{strategy} for a game $G=\<M,W\>$ is a function 
  from $M^{<\omega}$ to $M$.
\end{defn}

\begin{defn}
  An \term{attack} for a game $G=\<M,W\>$ is a function 
  from $\omega$ to $M$.
\end{defn}

Intuitively, a strategy is a rule for one of the players on how to play
the game based upon the previous (finite) moves of her opponent, while an 
attack is a fixed strike by an opponent indexed by round number. 


\begin{defn}
  The \term{result} of a game given a strategy $\sigma$ for the first player 
  and an attack $\<a_0,a_1,\dots\>$ by the second player is the playthrough
    \[
      \<
        \sigma(\emptyset), 
        a_0,
        \sigma(\<a_0\>), 
        a_1, 
        \sigma(\<a_0,a_1\>),
        \dots
      \>
    \]
  Likewise, if $\sigma$ is a strategy for the second player, and 
  $\<a_0,a_1,\dots\>$ is an attack by the first player, then the result is
  the playthrough
    \[
      \< 
        a_0,
        \sigma(\<a_0\>), 
        a_1, 
        \sigma(\<a_0,a_1\>),
        \dots
      \>
    \]
\end{defn}

We now may rigorously define the notion of ``forcing'' a win in a game.

\begin{defn}
  A strategy $\sigma$ is a \term{winning strategy} for a player if for
  every attack by the opponent, the result of the game is a victory 
  for that player.

  If a winning strategy exists for player $\pl A$ in the game $G$, then we
  write $\pl A \win G$. Otherwise, we write $\pl A \not\win G$.
\end{defn}

Of course, a strategy $\sigma$ is not a winning strategy for a player if there 
exists some \term{counter-attack} by the opponent for which the result is 
a victory for the opponent. Typically this counter-attack is defined in
terms of the strategy $\sigma$; else, the counter-attack is itself a
winning strategy depending on only the round number (which we will
investigate further in a later section).

\begin{defn}
  A game $G$ with players $\pl A$, $\pl B$ is said to be \term{determined}
  if either $\pl A \win G$ or $\pl B \win G$.
  Otherwise, the game is \term{indetermined}.
\end{defn}

\begin{thm}\cite{gale}
  If the move set $M$ for a game $G=\<M,W\>$ is given the discrete topology
  and $W$ is either an open or closed subset of $M^\omega$ with the
  usual product topology, then $G$ is determined.
\end{thm}

This is actually a special case of the powerful Borel Determinacy theorem 
which states that $G=\<M,W\>$ is determined whenever $W$ is a Borel subset
of $M^\omega$. \cite{martin}

It's an easy corollary that all finite games are determined
($W$ must be clopen). As stated earlier, most topological games are infinite,
and many are indetermined for certain spaces constructed using the Axiom of
Choice.
\footnote{
  The Axiom of Choice is required, as the Axiom of Determinacy stating
  that all Gale-Stewart games are determined is another axiom independent
  of ZF. \cite{MR0140430}
}

\begin{ex}
  Let $B$ be a Bernstein subset of the real line. Then $\bmgame{B}$ is
  undetermined.
\end{ex}

\begin{proof}
  It can be shown that $B$ is a Baire space, and we will soon see that 
  $\pl E\not\win \bmgame{X}$ characterizes Baire spaces. Also, if $\sigma$
  is a winning strategy for $\pl N$ in $\bmgame{Y}$ for $Y\subseteq X$, it can 
  be shown that $Y$ contains a closed uncountable set in $X$. 
  Thus $\pl N\not\win \bmgame{B}$.
\end{proof}

\subsection{Applications of Strategies}

The presence or absence of a winning strategie for a player in a topological
game characterizes a property of the topological space in question.

A classical result follows.

\begin{thm}
$\pl E\not\win \bmgame{X}$ if and only if $X$ is a Baire space. \cite{haworth}
\end{thm}


\subsection{Limited Information Strategies}


\section{Examples of Topological Games}

(TODO: )










