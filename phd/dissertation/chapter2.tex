%!TEX root = dissertation.tex
% ^ leave for LaTeXTools build functionality

\chapter{Toplogical Games and Strategies\\ of Perfect and Limited Information}

The goal of this paper is to explore the applications of limited information
strategies in existing topological games. There are a variety of frameworks
for modeling such games, so we establish one within this chapter which we
will use for this manuscript.

\section{Games}

Intuitively, the games studied in this paper are two-player games for which
each player takes turns making a choice from a set of possible moves. At
the conclusion of the game, the choices made by both players are examined,
and one of the players is declared the winner of that playthrough.

Games may be modeled mathematically in various ways,
but we will find it convenient to think of them in terms defined by
Gale and Stewart. \cite{MR0054922}

\begin{defn}
  A \term{game} is a tuple $\<M,W\>$ such that
  $W\subseteq M^{\omega}$. $M$ is set of \term{moves} for
  the game, and $M^{\omega}$ is the set of all possible
  \term{playthroughs} of the game.

  $W$ is the set of \term{winning playthroughs} or \term{victories} for the
  first player, and $M^{\omega}\setminus W$ is the set of victories for the
  second player. ($W$ is often called the \term{payoff set} for the
  first player.)
\end{defn}

Within this model, we may imagine two players $\pl A$ and $\pl B$ playing a
game which consists of \term{rounds} enumerated for each $n<\omega$.
During round $n$, $\pl A$ chooses $a_n\in M$, followed by $\pl B$ choosing
$b_n\in M$. The playthrough corresponding to those choices would be
the sequence $p = \<a_0,b_0,a_1,b_1,\dots\>$. If $p\in W$, then $\pl A$
is the winner of that playthrough, and if $p\not\in W$, then $\pl B$ is
the winner. Note that no ties are allowed.

Rather than explicitly defining $W$, we
typically define games by declaring the \term{rules} that each player must
follow and the \term{winning condition} for the first player.
Then a playthrough is in $W$ if either
the first player made only \term{legal moves} which observed the game's rules
and the playthrough satisified the winning condition, or the second player
made an \term{illegal move} which contradicted the game's rules.
Often, we will consider \term{legal playthoughs} where both players only
made legal moves, in which case only the winning condition need be considered.

As an illustration, we could model a game of chess (ignoring stalemates)
by letting
\[
  M = \{\<p,s\>: p \text{ is a chess piece and }
                 s \text{ is a space on the board } \}
\]
representing moving a piece $p$ to the space $s$ on the board. Then the
rules of chess restict White from moving pieces which belong to
Black, or moving a piece to an illegal space on the board.\footnote{
  In practice, $M$ is often defined as the union of two sets, such
  as white pieces and black pieces in chess. For example,
  the first player may choose open sets in a topology, while the second player
  chooses points within the topological space.
}
The winning condition could then
``inspect'' the resulting positions of pieces on the board after each move
to see if White attained a
checkmate. This winning condition along with the rules implicitly define the
set $W$ of winning playthroughs for White.


\subsection{Infinite and Topological Games}

Games never technically end within this model, since
playthroughs of the game are infinite sequences. However, for all practical
purposes many games end after a finite number of turns.

\begin{defn}
  A game is said to be an \term{finite game} if for every playthrough
  $p\in M^\omega$ there exists a round $n<\omega$ such that
    $
      [p\rest n] = \{q\in M^\omega : q\supseteq p\rest n \}
    $
  is a subset of either $W$ or $M^\omega\setminus W$.
\end{defn}

Put another way, a finite game is decided after a finite number
of rounds, after which the game's winner could not change even if further
rounds were played.
Games which are not finite are called \term{infinite games}.

As an illustration of an infinite game, we may consider a simple example due to
Baker \cite{noMRbaker}.

\begin{game}
  Let $\bakergame{X}$ denote a game with players $\pl A$ and $\pl B$,
  defined for each subset $A\subset \mathbb{R}$.
  In round $0$, $\pl A$ chooses a number $a_0$, followed by $\pl B$ choosing
  a number $b_0$ such that $a_0<b_0$.
  In round $n+1$, $\pl A$ chooses a number $a_{n+1}$ such that
  $a_n<a_{n+1}<b_n$, followed by $\pl B$ choosing a number $b_{n+1}$ such that
  $a_{n+1}<b_{n+1}<b_n$.

  $\pl A$ wins the game if the sequence $\<a_n:n<\omega\>$ limits to a
  point in $X$, and $\pl B$ wins otherwise.
\end{game}

Certainly, $\pl A$ and $\pl B$ will never be in
a position without (infinitely many) legal moves available, and provided that
$A$ is non-trivial, there is a playthrough such that for all $n<\omega$,
the segment $(a_n,b_n)$ intersects both $A$ and $\mathbb{R}\setminus A$.
Such a playthrough could never be decided in a finite number of moves,
so the winning condition considers the infinite sequence of moves made by the
players and declares a victor at the ``end'' of the game.

\begin{defn}
  A \term{topological game} is a game defined in terms of an arbitrary
  topological space.
\end{defn}

Topological games are usually infinite games, ignoring trivial examples.
One of the earliest examples
of a topological game is the Banach-Mazur game, proposed by Stanislaw Mazur
as Problem 43 in Stefan Banach's Scottish Book (1935). A more comprehensive
history of the Banacy-Mazur and other topological games may be found in
Telgarsky's survey on the subject \cite{MR892457}.

The original game was defined for subsets of the real line; however,
we give a more general definition here.

\begin{game}
  Let $\bmgame{X}$ denote the \term{Banach-Mazur game} with players $\pl E$,
  $\pl N$ defined for each topological space $X$.
  In round $0$, $\pl E$ chooses a nonempty open set $E_0\subseteq X$, followed
  by $\pl N$ choosing a nonempty open subset $N_0\subseteq E_0$.
  In round $n+1$, $\pl E$ chooses a nonempty open subset $E_{n+1}\subseteq N_n$,
  followed by $\pl N$ choosing a nonempty open subset
  $N_{n+1}\subseteq E_{n+1}$.

  $\pl E$ wins the game if $\bigcap_{n<\omega} E_n = \emptyset$,
  and $\pl N$ wins otherwise.
\end{game}

For example, if $X$ is a locally compact Hausdorff space, $\pl N$ can ``force''
a win by choosing $N_0$ such that $\cl{N_0}$ is compact, and choosing
$N_{n+1}$ such that
$N_{n+1}\subseteq\cl{N_{n+1}}\subseteq O_{n+1}\subseteq N_n$
(possible since $N_n$ is a compact Hausdorff $\Rightarrow$ normal space).
Since $\bigcap_{n<\omega} E_n = \bigcap_{n<\omega} N_n$ is the decreasing
intersection of compact sets, it cannot be empty.

This concept of when (and how) a player can ``force'' a win in certain
topological games is the focus of this manuscript.



\section{Strategies}

We shall make the notion of forcing a win in a game rigorous by introducing
``strategies'' and ``attacks'' for games.

\begin{defn}
  A \term{strategy} for a game $G=\<M,W\>$ is a function
  from $M^{<\omega}$ to $M$.
\end{defn}

\begin{defn}
  An \term{attack} for a game $G=\<M,W\>$ is a function
  from $\omega$ to $M$.
\end{defn}

Intuitively, a strategy is a rule for one of the players on how to play
the game based upon the previous (finite) moves of her opponent, while an
attack is a fixed strike by an opponent indexed by round number.


\begin{defn}
  The \term{result} of a game given a strategy $\sigma$ for the first player
  and an attack $\<a_0,a_1,\dots\>$ by the second player is the playthrough
    \[
      \<
        \sigma(\emptyset),
        a_0,
        \sigma(\<a_0\>),
        a_1,
        \sigma(\<a_0,a_1\>),
        \dots
      \>
    \]
  Likewise, if $\sigma$ is a strategy for the second player, and
  $\<a_0,a_1,\dots\>$ is an attack by the first player, then the result is
  the playthrough
    \[
      \<
        a_0,
        \sigma(\<a_0\>),
        a_1,
        \sigma(\<a_0,a_1\>),
        \dots
      \>
    \]
\end{defn}

We now may rigorously define the notion of ``forcing'' a win in a game.

\begin{defn}
  A strategy $\sigma$ is a \term{winning strategy} for a player if for
  every attack by the opponent, the result of the game is a victory
  for that player.

  If a winning strategy exists for a player $\pl A$ in the game $G$, then we
  write $\pl A \win G$. Otherwise, we write $\pl A \notwin G$.
\end{defn}

To show that a winning strategy exists for a player (i.e. $\pl A\win G$),
we typically begin by
defining it and showing that it is \term{legal}: it only yields moves which are
legal according to the rules of the game. Then, we consider an arbitrary
legal attack, and prove that the result of the game is a victory for that
player.

If we wish to show that a winning strategy does not exist for a player
(i.e. $\pl A\notwin G$), we
often consider an arbitrary legal strategy, and use it to define a legal
\term{counter-attack} for the opponent. If we can prove that the result of
the game for that strategy and counter-attack is a victory for the opponent,
then a winning strategy does not exist.

Unlike finite games, is not the case that a winning strategy must exist for
one of the players in an infinite game.

\begin{defn}
  A game $G$ with players $\pl A$, $\pl B$ is said to be \term{determined}
  if either $\pl A \win G$ or $\pl B \win G$.
  Otherwise, the game is \term{undetermined}.
\end{defn}

The Borel Determinacy Theorem states that $G=\<M,W\>$ is determined whenever
$W$ is a Borel subset of $M^\omega$ \cite{MR0403976}. It's an easy corollary that
all finite games are determined; $W$ must be clopen.

However, as stated earlier, most topological games are infinite,
and many are undetermined for certain spaces constructed using the Axiom of
Choice.\footnote{
  These spaces cannot be constructed just only the axioms of ZF. In fact,
  mathematicians have studied an Axiom of Determinacy which declares that
  that all Gale-Stewart games are determined (and implies that the Axiom
  of Choice is false). \cite{MR0140430}
}

Often, we will build new strategies based on others.

\begin{defn}
  A strategy $\tau$ is a \term{strengthening} of another strategy $\sigma$
  for a player if whenever the result of the game for $\sigma$ and an attack
  $a$ by the opponent is a victory for the player, then the result of the game
  for $\tau$ and $a$ is also a victory for the player.
\end{defn}

\begin{prop}
  If $\sigma$ is a winning strategy, and $\tau$ strengthens $\sigma$,
  then $\tau$ is also a winning strategy.
\end{prop}

\subsection{Applications of Strategies}

The power of studying these infinite-length games can be illustrated by
considering the following proposition.

\begin{prop}
  If $X$ is countable, then $\pl B \win \bakergame{X}$.
\end{prop}

\begin{proof}
  Adapted from \cite{noMRbaker}.
  Let $X=\{x_0,x_1,\dots\}$. Let $i(a,b)$ be the least integer such that
  $a<x_{i(x,y)}<b$, if it exists. We define a strategy $\sigma$ for $\pl B$
  such that:
  \begin{itemize}
    \item $\sigma(\<a_0\>)=x_{i(a_0,\infty)}$.
          If $i(a_0,\infty)$ does not exist, then the choice of $\sigma(\<a_0\>)$
          is arbitary, say, $a_0+1$.
    \item $\sigma(\<a_0,...,a_{n+1}\>)=x_{i(a_{n+1},b_n)}$, where
          $b_n=\sigma(\<a_0,\dots,a_n\>)$.
          If $i(a_{n+1},b_n)$ does not exist, then the choice of
          $\sigma(\<a_0,...,a_{n+1}\>)$ is arbitrary, say,
          $
            \sigma(\<a_0,...,a_{n+1}\>)
              =
            \frac{a_{n+1}+b_n}{2}
          $.
  \end{itemize}

  Observe that $\sigma$ is a legal strategy according to the rules of the game
  since $a_0<\sigma(\<a_0\>)$ and $a_{n+1}<\sigma(\<a_0,\dots,a_{n+1}\>)<b_n$.
  We claim this is a winning strategy for $\pl B$. Let $a=\<a_0,a_1,\dots\>$
  be a legal attack by $\pl A$ against $\sigma$: we will show that the
  resulting playthrough is a victory for $\pl B$, that is,
  $\lim_{n\to\infty} a_n\not\in X$. Let $b_n=\sigma(\<a_0,\dots,a_n\>)$.
  Note that
    \[
      a_0<a_1<\dots<\lim_{n\to\infty}a_n<\dots<b_1<b_0
    \]

  If $i(a_0,\infty)$ does not exist, then $a_0$ is greater than every element
  of $X$, and thus
  $\lim_{n\to\infty} a_n \not\in X$. A similar argument follows if some
  $i(a_{n+1},b_n)$ does not exist.

  Otherwise,
    \[
      i(a_0,\infty)<i(a_1,b_0)<i(a_2,b_1)<\dots
    \]
  and for each $i<\omega$, one of the following must hold.
    \begin{itemize}
      \item $i<i(a_0,\infty)$. Then $x_i\leq a_0<\lim_{n\to\infty}a_n$.
      \item $i = i(a_0,\infty)$. Then $x_i=b_0>\lim_{n\to\infty}a_n$.
      \item $i(a_0,\infty)<i<i(a_1,b_0)$.
            Then $x_i\leq a_1<\lim_{n\to\infty}a_n$ or
            $x_i\geq b_0>\lim_{n\to\infty}a_n$.
      \item $i = i(a_{n+1},b_n)$ for some $n<\omega$.
            Then $x_i=b_{n+1}>\lim_{n\to\infty}a_n$.
      \item $i(a_{n+1},b_n)<i<i(a_{n+2},b_{n+1})$ for some $n<\omega$.
            Then $x_i\leq a_{n+2}<\lim_{n\to\infty}a_n$ or
            $x_i\geq b_{n+1}>\lim_{n\to\infty}a_n$.
    \end{itemize}
  In any case, $x_i\not=\lim_{n\to\infty}a_n$, and thus
  $\lim_{n\to\infty}a_n\not\in X$.
\end{proof}

More informally, $\pl B$ can force a win by enumerating the countable set
$X$ and playing every legal choice by the end of the game.
This yields a classical result.

\begin{cor}
  $\mathbb{R}$ is uncountable.
\end{cor}

\begin{proof}
  $\pl A\win \bakergame{\mathbb{R}}$, since $a_n$ must converge to some
  real number. This implies $\pl B\notwin \bakergame{\mathbb{R}}$, and thus
  $\mathbb{R}$ is not countable.
\end{proof}

Infinite games thus provide a rich framework for considering questions in
set theory and topology. In general, the presence or absence of a winning
strategy for a player in a topological game characterizes a property of the
topological space in question.

\begin{thm}
$\pl E\notwin \bmgame{X}$ if and only if $X$ is a Baire space.
\cite{MR0431104}
\end{thm}

\subsection{Limited Information Strategies}

So far we have assumed both players enjoy \term{perfect information}, and
may develop strategies which use all of the previous moves of the opponent
as input.

\begin{defn}
  For a game $G=\<M,W\>$, the \term{$k$-tactical fog-of-war} is the function
  $\nu_k:M^{<\omega}\to M^{\leq k}$ defined by
    \[
      \nu_k(\<m_0,\dots,m_{n-1}\>)=\<m_{n-k},\dots,m_{n-1}\>
    \]
  and the \term{$k$-Markov fog-of-war} is the function
  $\mu_k:M^{<\omega}\to (M^{\leq k}\times\omega)$ defined by
    \[
      \mu_k(\<m_0,\dots,m_{n-1}\>)=\<\<m_{n-k},\dots,m_{n-1}\>,n\>
    \]
\end{defn}

Essentially, these fogs-of-war represent a limited memory: $\nu_k$ filters out
all but the last $k$ moves of the opponent, and $\mu_k$ filters out all but
the last $k$ moves of the opponent and the round number.

We call strategies which do not require full recollection of the opponent's
moves \term{limited information strategies}.

\begin{defn}
  A \term{$k$-tactical strategy} or \term{$k$-tactic} is a function
  $\sigma:M^{\leq k}\to M$ yielding a corresponding strategy
  $\sigma\circ\nu_k:M^{<\omega}\to M$.

  A \term{$k$-Markov strategy} or \term{$k$-mark} is a function
  $\sigma:M^{\leq k}\times\omega\to M$ yielding a corresponding strategy
  $\sigma\circ\mu_k:M^{<\omega}\to M$.
\end{defn}

$k$-tactics and $k$-marks may then only use the last $k$ moves of the opponent,
and in the latter case, also the round number.

The $k$ is usually omitted when $k=1$. A ($1$-)tactic is
called a \term{stationary strategy} by some authors. $0$-tactics are not
usually interesting (such strategies would be constant functions); however,
we will discuss $0$-Markov strategies, called
\term{predetermined strategies} since such a strategy only uses the round
number and does not rely on knowing which moves the opponent will make.
Of course, a limited information strategy $\sigma$ is \term{winning} for a
player if its corresponding strategy $\sigma\circ\nu_k$ or $\sigma\circ\mu_k$
is winning for that player.

\begin{defn}
  If a winning $k$-tactical strategy exists for a player $\pl A$ in the game
  $G$, then we write $\pl A \ktactwin{k} G$. If $k=1$, then $\pl A \tactwin G$.

  If a winning $k$-Markov strategy exists for a player $\pl A$ in the game
  $G$, then we write $\pl A \kmarkwin{k} G$. If $k=1$, then $\pl A \markwin G$,
  and if $k=0$, then $\pl A \prewin G$.
\end{defn}

The existence of a winning limited information strategy can characterize a
stronger property than the property characterized by a perfect information
strategy.

\begin{defn}
  $X$ is an \term{$\alpha$-favorable space}
  when $\pl N \tactwin \bmgame{X}$.
  $X$ is a \term{weakly $\alpha$-favorable space}
  when $\pl N \win \bmgame{X}$.
\end{defn}

\begin{prop}
  $X$ is $\alpha$-favorable
    $\Rightarrow$
  $X$ is weakly $\alpha$-favorable
    $\Rightarrow$
  $X$ is Baire
\end{prop}

Those arrows may not be reversed. A Bernstein subset of the real line is an
example of a Baire space which is not weakly $\alpha$-favorable, and Gabriel
Debs constructed an example of a completely regular space for which $\pl N$
has a winning $2$-tactic, but lacks a winning $1$-tactic. \cite{MR817083}
