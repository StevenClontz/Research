%!TEX root = dissertation.tex
% ^ leave for LaTeXTools build functionality

\chapter{Bell's Game}

A very recent development related to Gruenhage's convergence and
clustering games comes from
Jocelyn Bell, used to study the uniform box product of uniform spaces.



\section{Uniform Spaces}

\begin{defn}
  A \term{uniformity} on a set $X$ is a filter $\mb D$ of subsets of $X^2$,
  known as \term{entourages}, such that $\bigcap\mb D=\Delta=\{\<x,x\>:x\in X\}$
  and, for each entourage $D\in\mb D$:
  \begin{itemize}
    \item There exists $\frac{1}{2}D\in\mb D$ such that
      \[
        \frac{1}{2}D\circ\frac{1}{2}D
          =
        \left\{\<x,z\>:\exists y\in X\left(\<x,y\>,\<y,z\>\in\frac{1}{2}D\right)\right\}
          \subseteq
        D
      \]
    \item $D^{-1}=\{\<y,x\>:\<x,y\>\in D\}\in \mb D$
  \end{itemize}
\end{defn}

A set $X$ with a uniformity is called a \term{uniform space}.
As $\mb D$ is a filter, we also have that $D\cap E\in \mb D$ for all
$E\in\mb D$, and $F\in \mb D$ for all $F\supseteq D$. Note that if $\mb E$ is
a filter base satisfying the conditions for a uniformity, then we say
$\mb E$ is a \term{uniformity base} which may be extended to a uniformity by
closing it under the superset operation.

A uniformity is a generalization of a metric.

\begin{defn}
  For an entourage $D\in\mb D$ and a point $x\in X$, the
  \term{$D$-ball around $x$} is the set $D[x]=\{y:\<x,y\>\in D\}$.
\end{defn}

\begin{defn}
  If $d$ is a metric for the space $X$, then the \term{metric uniformity} for
  $X$ is generated by the uniformity base $\{D_\epsilon:\epsilon>0\}$
  where $D_\epsilon = \{\<x,y\>:d(x,y)<\epsilon\}$.
\end{defn}

Like metrics, uniformities induce natural topological structures.

\begin{defn}
  The \term{uniform topology} for a space $X$ with uniformity $\mb D$ is given
  by letting $U$ be open if for each $x\in U$, there exists $D\in\mathbb{D}$
  such that $D[x]\subseteq U$.
\end{defn}

\begin{thm}
  The uniform topology for a space $X$ with uniformity $\mb D$ is the coarsest
  topology such that for each $x\in X$ and $D\in\mathbb{D}$,
  $D[x]$ is a neighborhood (not necessarily open) of $x$.
\end{thm}

\begin{proof}
  Let $\mc T$ be a topology such that $D[x]$ is a neighborhood
  of $x$ for each $x\in X$, and let $U$ be open in the uniform topology.
  For each $x\in U$, there exists $D_x\in\mathbb{D}$ such that
  $D_x[x]\subseteq U$, and since $D_x[x]$ is a neighborhood of $x$, there
  is $U_x\in \mc T$ such that $x\in U_x\subseteq D_x[x]$. Since
  $U=\bigcup_{x\in U}U_x\in \mc T$, the result follows as $\mc T$ contains the
  uniform topology.
\end{proof}

The uniform topology
for a metric uniformity is simply the usual metric topology, and $D_\epsilon[x]$
is the usual metric $\epsilon$-ball around $x$.

We require a few known results on uniform spaces. See e.g. \cite{MR1039321}
for proofs.

\begin{thm}
  Every uniform topology is $T_{3\frac{1}{2}}$.
\end{thm}

\begin{defn}
  A topology on $X$ is \term{uniformizable} if there exists
  a uniformity whose uniform topology equals the topology on $X$.
\end{defn}

\begin{thm}
  Every $T_{3\frac{1}{2}}$ topology is uniformizable.
\end{thm}

\begin{defn}
  The \term{universal uniformity} for a uniformizable topology is the uniformity
  finer than all uniformities which induce the given topology.
\end{defn}

\begin{thm}
  Every uniformizable topology is induced by its universal uniformity.
\end{thm}

\begin{defn}
  For a uniformizable space $X$, a \term{symmetric open entourage} $D$ is an
  entourage of the universal uniformity such that
  \begin{itemize}
    \item $D$ is open in the product topology on $X^2$
    \item $D$ is symmetric; that is, $D=D^{-1}$.
  \end{itemize}
\end{defn}

\begin{prop}
  If $X$ is a uniformizable space, then for all $x\in X$ and symmetric open
  entourages $D$:
    \[
      x\in \frac{1}{2}D[y]\text{ and } y\in\frac{1}{2}D[z] \Rightarrow x\in D[z]
    \]
  and
    \[
      \frac{1}{2}D[x]\subseteq \cl{\frac{1}{2}D[x]}\subseteq D[x]
    \]
\end{prop}

\begin{thm}
  For every entourage $D$ in the universal uniformity on $X$, there exists a
  symmetric open entourage $U\subseteq D$.
\end{thm}

If $D$ is a symmetric open entourage, then we assume $\frac{1}{2}D$ is
also a symmetric open entourage. Note that if $D$ is an open symmetric
entourage, then $D[x]$ is an open neighborhood of $x$.



\section{Games on uniformizable spaces}

\begin{game}
  Let $\bellUniGame{X, \mb D}$ denote the \term{Bell uniform space game} with
  players $\pl D$, $\pl P$ which
  proceeds as follows for a uniform space $X$ with uniformity $\mb D$. In round
  $0$, $\pl D$ chooses an entourage $D_0$, followed by $\pl P$ choosing a point
  $p_0\in X$. In round $n+1$, $\pl D$ chooses an entourage $D_{n+1}$, followed
  by $\pl P$ choosing a point $p_{n+1}\in D_n[p_n]$.

  $\pl D$ wins in the case that either $\<p_0,p_1,\dots\>$ converges
  with respect to the uniformity $\mb D$,
  or $\bigcap_{n<\omega}D_n[p_n] = \emptyset$.\footnote{
    A sequence $p$ converges to $x$ with respect to a uniformity $\mb D$
    if for every entourage $D\in\mb D$, there exists $N<\omega$ such that
    $p(n)\in D[x]$ for $n\geq N$.
  } $\pl P$ wins otherwise.
\end{game}

This original formulation was called the ``proximal game'' in \cite{MR3239205}.
The game was used to prove that the
$\sigma$-product of spaces for which $\pl D$ has a winning strategy is
collectionwise normal, as well as to show the collectionwise normality
of certain uniform box products.

\begin{defn}
  A uniformizable space $X$ is \term{proximal} if it has a compatible uniformity
  $\mb D$ such that $\pl D\win\bellUniGame{X,\mb D}$.
\end{defn}

\begin{prop}
  Every metric space is proximal.
\end{prop}

Considering the universal uniformity, we may
consider purely topological variations of this game.

\begin{game}
  Let $\bellConHardGame{X}$ denote the \term{hard Bell convergence game} with
  players $\pl D$, $\pl P$ which proceeds as follows for
  a uniformizable space $X$. In round $0$,
  $\pl D$ chooses a symmetric open entourage $D_0$, followed by $\pl P$
  choosing a point $p_0\in X$. In round $n+1$, $\pl D$ chooses a symmetric
  open entourage $D_{n+1}$, followed
  by $\pl P$ choosing a point $p_{n+1}\in D_n[p_n]$.

  $\pl D$ wins in the case that either $\<p_0,p_1,\dots\>$ converges in $X$,
  or $\bigcap_{n<\omega}D_n[p_n] = \emptyset$. $\pl P$ wins otherwise.
\end{game}

\begin{game}
  Let $\bellConGame{X}$ denote the \term{Bell convergence game} with players $\pl D$,
  $\pl P$ which proceeds analogously to $\bellConHardGame{X}$, except for the
  following. Let $E_n=\bigcap_{m\leq n}D_n$, where $D_n$ is the entourage played
  by $\pl D$ in round $n$.
  Then $\pl P$ must ensure that $p_{n+1}\in E_n[p_n]$,
  and $\pl D$ wins when either $\<p_0,p_1,\dots\>$ converges in $X$
  or $\bigcap_{n<\omega}E_n[p_n] = \emptyset$.
\end{game}

\begin{thm}
  $\pl D\win\bellConHardGame{X}$ if and only if
  $\pl D\win\bellConGame{X}$ if and only if
  $X$ is proximal.
\end{thm}

\begin{proof}
  If $\pl D\win\bellConHardGame{X}$, then we immediately see that
  $\pl D\win\bellConGame{X}$. If $\sigma$ is a winning strategy for $\pl D$
  in $\bellConGame{X}$, then $\tau$ defined by
  $\tau(s)=\bigcap_{t\leq s}\sigma(t)$ is easily seen to be a winning strategy
  for $\pl D$ in $\bellConHardGame{X}$.

  If $\pl D\win\bellConHardGame{X}$, then $\pl D\win\bellUniGame{X,\mb D}$
  where $\mb D$ is the universal uniformity, showing $X$ is proximal.
  Finally, if $X$ is proximal, then there exists a winning strategy $\sigma$
  for $\bellUniGame{X,\mb D}$ where $\mb D$ is a uniformity inducing the
  topology on $X$. Let $\tau$ be defined by $\tau(s)=O(\sigma(s))$ where
  $O(D)$ is an open symmetric entourage which is a subset of $D$; $\tau$
  is easily seen to be a winning strategy for $\pl D$ in $\bellConHardGame{X}$.
\end{proof}

The hard and normal variants $\bellConHardGame{X}$, $\bellConGame{X}$ mirror the
hard and normal variants of Gruenhage's convergence game. In fact, Bell
showed a strong connection in \cite{MR3239205}.

\begin{thm}
  If $X$ is proximal, then $X$ is a $W$-space.
\end{thm}

We similarly may consider clustering versions of the Bell games.
(TODO: will I actually use this game?)

\begin{defn}
  Let $\bellClusGame{X}$ ($\bellClusHardGame{X}$) denote the
  \term{Bell clustering game} (\term{hard Bell clustering game}) which
  proceeds analogously to $\bellConGame{X}$ ($\bellConHardGame{X}$), except
  that $\pl D$ need only ensure that $\<p_0,p_1,\dots\>$ clusters
  in $X$ in order to win.
\end{defn}

\begin{defn}
  A uniformizable space $X$ is \term{almost proximal} if
  $\pl D \win \bellClusGame{X}$.
\end{defn}

Or we eliminate the secondary winning condition to create a stronger
variant of the game.

\begin{defn}
  Let $\bellAbsConGame{X}$ denote the
  \term{absolute Bell convergence game} which
  proceeds analogously to $\bellConGame{X}$, except
  that $\pl D$ must always ensure that $\<p_0,p_1,\dots\>$ converges
  in $X$ in order to win.
\end{defn}

\begin{defn}
  A uniformizable space $X$ is \term{absolutely proximal} if
  $\pl D \win \bellAbsConGame{X}$.
\end{defn}

Proximal spaces have strong preservation properties, demonstrated in
\cite{MR3239205}.

\begin{thm}
  Every closed subspace of a proximal space is proximal.
\end{thm}

\begin{thm}
  Every $\Sigma$-product of proximal spaces is proximal.
\end{thm}

Gruenhage and the author showed in \cite{MR3227201} that Bell's game
yields an internal characterization of Corson compact spaces, answering
a question of Nyikos in \cite{nyikosProximalPreprint}.



\section{Characterizing Corson compactness}

\begin{defn}
  A compact space is \term{Corson compact} if it is homeomorphic to a compact
  subset of a $\Sigma$-product of real lines.
\end{defn}

Nyikos observed the following in \cite{nyikosProximalPreprint}.

\begin{prop}
  Corson compact spaces are proximal.
\end{prop}

\begin{proof}
  The real line is proximal, and closed subsets of $\Sigma$-products of proximal
  spaces are proximal.
\end{proof}

Following this observation, Nyikos asked if Bell's game actually provided an
internal characterization of Corson compactness. The author, working with
Gary Gruenhage, answered this question in the affirmative in \cite{MR3227201}.

To see this, we require a result of Gruenhage which gives another game
characterization of Corson compactness, found in \cite{MR752278}.

\begin{thm}
  A compact space is Corson compact if and only if
  $\pl O \win\gruConGame{X^2}{\Delta}$.
\end{thm}

Thus our desired result will follow if we may show that a winning strategy
in $\bellConGame{X}$ may be used to construct a winning strategy in
$\gruConGame{X^2}{\Delta}$. However, due to the secondary winning condition
for $\pl D$ in $\bellConGame{X}$ ($\bigcap_{n<\omega}E_n[p_n]=\emptyset$),
it will be more convenient if we may use a winning strategy for $\pl D$
in $\bellAbsConGame{X}$ instead.

\begin{defn}
  A uniformizable space $X$ is \term{uniformly locally compact} if there
  exists an open symmetric entourage $D$ such that $\cl{D[x]}$ is compact
  for all $x$.
\end{defn}

Of course, any compact uniformizable space is uniformly locally compact.
However, note that a space may be locally compact without being uniformly
locally compact.

\begin{thm}
  A uniformizable space is uniformly locally compact if and only if it is
  locally compact and paracompact. (See e.g. \cite{MR2040233}.)
\end{thm}

As an example, $\omega_1$ with the linear order topology is locally compact,
but not paracompact.

\begin{thm}
  If $X$ is a uniformly locally compact space, then
  $\pl D \win \bellConGame{X}$ if and only if
  $\pl D \win \bellAbsConGame{X}$.
\end{thm}

\begin{proof}
  Let $L$ be an open symmetric entourage such that $\cl{L[x]}$ is compact
  for all $x$.
  Let $\sigma$ be a strategy witnessing $\pl D\win \bellConGame{X}$.
  Without loss of generality, we may assume such that
  $\sigma(t)\subseteq L$
  (so $\cl{\sigma(t)[x]}\subseteq\cl{L[x]}$ is compact), and that
  $t \supseteq s$ implies $\sigma(t)\subseteq \frac{1}{4}\sigma(s)$.

  Let $\tau(t)=\frac{1}{2}\sigma(t)$. If $p$ attacks $\tau$
  in $\bellAbsConGame{X}$, then
    \[
      p(n+1)
        \in
      \tau(p\rest n)[p(n)]
        =
      \frac{1}{2}\sigma(p\rest n)[p(n)]
    \]

  \noindent and for

    \[
      x
        \in
      \cl{\sigma(p\rest (n+1))[p(n+1)]}
        \subseteq
      \cl{\frac{1}{4}\sigma(p\rest n)[p(n+1)]}
        \subseteq
      \frac{1}{2}\sigma(p\rest n)[p(n+1)]
    \]

  \noindent we can conclude $x\in\sigma(p\rest n)[p(n)]$. Thus

    \[
      \sigma(p\rest (n+1))[p(n+1)]
        \subseteq
      \cl{\sigma(p\rest (n+1))[p(n+1)]}
        \subseteq
      \sigma(p\rest n)[p(n)]
    \]

  Finally, note that since $\tau$ yields subsets of $\sigma$, $p$ also attacks
  the winning strategy $\sigma$ in $\bellConGame{X}$, but since the intersection
  of a descending chain of nonempty compact sets is nonempty, we have

    \[
      \bigcap_{n<\omega} \sigma(p\rest n)[p(n)]
        =
      \bigcap_{n<\omega} \cl{\sigma(p\rest n)[p(n)]}
        \not=
      \emptyset
    \]

  We conclude that $p$ converges.
\end{proof}


\begin{thm}
  If $\pl D\win\bellAbsConGame{X}$, then $\pl O\win\gruConGame{X}{H}$ for
  all compact $H$ in $X$.
\end{thm}


\begin{proof}
  Let $\sigma$ be a winning strategy for $\pl D$ in $\bellAbsConGame{X}$
  game such that $p\supsetneq q$ implies
  $\sigma(p)\subseteq \frac{1}{4}\sigma(q)$.
  For any sequence $t$, let $o_t=\{\<n,t(2n+1)\>:2n+1\in\dom(t)\}$
  be the subsequence of $t$ consisting of its odd-indexed terms.
  We proceed by constructing a winning strategy for $\pl O$ in
  $\gruClusGame{X}{H}$. Since $\pl O\win\gruClusGame{X}{H}$
  if and only if $\pl O\win\gruConGame{X}{H}$, the result will follow.

%   \bigskip

  We begin by defining a tree $T(\emptyset)$, during which we will
  define a number $m_\emptyset<\omega$ and points
  $h_{\emptyset,i},h_{\emptyset,i,j}$ for $i,j<m_\emptyset$
  which yield an open set
    \[
      \bigcup_{i,j<m_\emptyset}
        \frac{1}{4}\sigma(\<h_{\emptyset,i}\>)[h_{\emptyset,i,j}]
      =
      \bigcup_{\emptyset\concat\<i,h_{\emptyset,i},j\>\in\max(T(\emptyset))}
        \frac{1}{4}\sigma(o_\emptyset\concat\<h_{\emptyset,i}\>)[h_{\emptyset,i,j}]
    \]
  containing $H$. $\pl O$ will use this as the inital move in her winning
  strategy for $\gruClusGame{X}{H}$.

  \begin{itemize}
    \item
    Choose $m_\emptyset<\omega$, $h_{\emptyset,i}\in H$ for $i<m_\emptyset$, and
    $h_{\emptyset,i,j}\in H\cap\cl{\frac{1}{4}\sigma(\emptyset)[h_{\emptyset,i}]}$
    for $i,j<m_\emptyset$ such that
      \[
        \left\{\frac{1}{4}\sigma(\emptyset)[h_{\emptyset,i}]:i<m_\emptyset\right\}
      \]
    is a cover for $H$ and such that for each $i<m_\emptyset$
      \[
        \left\{\frac{1}{4}\sigma(\<h_{\emptyset,i}\>)[h_{\emptyset,i,j}]:j<m_\emptyset\right\}
      \]
    is a cover for $H\cap\cl{\frac{1}{4}\sigma(\emptyset)[h_{\emptyset,i}]}$.
    \item
    Let $\<i,h_{\emptyset,i},j\>$ and its initial segments be in
    $T(\emptyset)$ for $i,j<m_\emptyset$.
  \end{itemize}

  It follows that
    \[
      \left\{
        \frac{1}{4}\sigma(o_\emptyset\concat\<h_{\emptyset,i}\>)[h_{\emptyset,i,j}]
      :
        \emptyset\concat\<i,h_{\emptyset,i},j\>\in\max(T(a))
      \right\}
    \]
  covers $H$.

% \bigskip

  Now suppose that $a$ is a partial attack by $\pl P$ in $\gruClusGame{X}{H}$
  for which we have defined a tree $T(a)$.
  We will define $T(a\concat\<x\>)\supseteq T(a)$ for each $x\in X$, and
  whenever $x\in \frac{1}{4}\sigma(o_s\concat\<h_{s,i}\>)[h_{s,i,j}]$ for
  some $s\concat\<i,h_{s,i},j\>\in\max(T(a))$, we will set
  $t=s\concat\<i,h_{s,i},j,x\>$ and define a number $m_t<\omega$ and points
  $h_{t,k},h_{t,k,l}$ for $k,l<m_t$ to yield an open set
    \[
      \bigcup_{t\concat\<k,h_{t,k},l\>\in\max(T(a\concat\<x\>))}
        \frac{1}{4}\sigma(o_t\concat\<h_{t,k}\>)[h_{t,k,l}]
    \]
  which will be the next open neighborhood of $H$ in $\pl O$'s winning strategy
  for $\gruClusGame{X}{H}$.

  \begin{itemize}
    \item
    We will extend the nodes $s\concat\<i,h_{s,i},j\>\in\max(T(a))$ such
    that $x\in \frac{1}{4}\sigma(o_s\concat\<h_{s,i}\>)[h_{s,i,j}]$.
    For each such $s\concat\<i,h_{s,i},j\>$, set $t=s\concat\<i,h_{s,i},j,x\>$.
    \item
    Note that whenever $o_s\concat\<h_{s,i}\>$ is a legal partial attack
    against $\sigma$, then
      \[
        x
          \in
        \frac{1}{4}\sigma(o_s\concat\<h_{s,i}\>)[h_{s,i,j}]
          \subseteq
        \frac{1}{4}\sigma(o_s)[h_{s,i,j}]
      \]
    and
      \[
        h_{s,i,j}
          \in
        \cl{\frac{1}{4}\sigma(o_s)[h_{s,i}]}
          \subseteq
        \frac{1}{2}\sigma(o_s)[h_{s,i}]
      \]
    implies
      \[
        x
          \in
        \sigma(o_s)[h_{s,i}]
      \]
    and thus $o_s\concat\<h_{s,i},x\>=o_t$ is also a legal partial attack
    against $\sigma$.
    \item
    Choose $m_t<\omega$,
    $h_{t,k}\in H\cap \cl{\frac{1}{4}\sigma(o_s\concat\<h_{s,i}\>)[h_{s,i,j}]}$
    for $k<m_t$, and
    $h_{t,k,l}\in H\cap\cl{\frac{1}{4}\sigma(o_t)[h_{t,k}]}$
    for $k,l<m_t$ such that
      \[
        \left\{\frac{1}{4}\sigma(o_t)[h_{t,k}]:k<m_t\right\}
      \]
    is a cover for
    $H\cap \cl{\frac{1}{4}\sigma(o_s\concat\<h_{s,i}\>)[h_{s,i,j}]}$
    and such that for each $k<m_t$
      \[
        \left\{\frac{1}{4}\sigma(o_t\concat\<h_{t,k}\>)[h_{t,k,l}]:l<m_t\right\}
      \]
    is a cover for $H\cap\cl{\frac{1}{4}\sigma(o_t)[h_{t,k}]}$.
    \item
    Note that whenever $o_t$ is a legal partial attack against $\sigma$, then
      \[
        h_{t,k}
          \in
        \cl{\frac{1}{4}\sigma(o_s\concat\<h_{s,i}\>)[h_{s,i,j}]}
          \subseteq
        \frac{1}{2}\sigma(o_s\concat\<h_{s,i}\>)[h_{s,i,j}]
      \]
    and
      \[
        x
          \in
        \frac{1}{4}\sigma(o_s\concat\<h_{s,i}\>)[h_{s,i,j}]
      \]
    implies
      \[
        h_{t,k}
          \in
        \sigma(o_s\concat\<h_{s,i}\>)[x]
      \]
    and thus $o_t\concat\<h_{t,k}\>$ is a legal partial attack against $\sigma$.
    \item
    Include all initial segments of $t\concat\<k,h_{t,k},l\>$ in
    $T(a\concat\<x\>)$ for $k,l<m_t$.
  \end{itemize}

%     \bigskip

 This completes the construction of  $ T(a\concat\<x\>)$. Note that since
    \[
      \left\{
        \frac{1}{4}\sigma(o_s\concat\<h_{s,i}\>)[h_{s,i,j}]
      :
        s\concat\<i,h_{s,i},j\>\in\max(T(a))
      \right\}
    \]
  covers $H$, then since
    \[
      \left\{
        \frac{1}{4}\sigma(o_t\concat\<h_{t,k}\>)[h_{t,k,l}]
      :
        s\concat\<i,h_{s,i},j,x,k,h_{t,k},l\>
          \in
        \max(T(a\concat\<x\>))\setminus\max(T(a))
      \right\}
    \]
  covers $H\cap \frac{1}{4}\sigma(o_s\concat\<h_{s,i}\>)[h_{s,i,j}]$, we have that
    \[
      \left\{
        \frac{1}{4}\sigma(o_t\concat\<h_{t,k}\>)[h_{t,k,l}]
      :
        t\concat\<k,h_{t,k},l\>\in\max(T(a\concat\<x\>))
      \right\}
    \]
  covers $H$.

  We define a strategy $\tau$ for $\pl O$ in $\gruClusGame{X}{H}$ such
  that:
  \[
    \tau(a)
      =
    \bigcup_{s\concat\<i,h_{s,i},j\>\in\max(T(a))}
    \frac{1}{4}\sigma(o_s\concat\<h_{s,i}\>)[h_{s,i,j}]
  \]

  If $p$ is a legal attack by $\pl P$ against $\tau$, then
  let $\ds T(p)=\bigcup_{n<\omega} T(p\rest n)$. We note $T(p)$ is an infinite
  tree (for each $n<\omega$, a node of $T(p\rest n)$ was extended)
  with finite levels:
    \begin{itemize}
      \item $\emptyset$ has exactly $m_\emptyset$ successors $\<i\>$.
      \item $s\concat\<i\>$ has exactly one successor $s\concat\<i,h_{s,i}\>$
      \item $s\concat\<i,h_{s,i}\>$ has exactly $m_s$ successors
            $s\concat\<i,h_{s,i},j\>$
      \item $s\concat\<i,h_{s,i},j\>$ has either no successors or exactly
            one successor $s\concat\<i,h_{s,i},j,x\>$
      \item $t=s\concat\<i,h_{s,i},j,x\>$ has exactly $m_t$ successors
            $t\concat\<k\>$
    \end{itemize}

  Hence $T(p)$ has an infinite branch
    \[
      q'=\<i_0,h_0,j_0,x_0,i_1,h_1,j_1,x_1,\dots\>
    \]
  Let $q=o_{q'}=\<h_0,x_0,h_1,x_1,\dots\>$. Note that by the construction of
  $T(p)$, $q$ is a legal attack on the winning strategy $\sigma$ in
  $\bellAbsConGame{X}$, so it must converge. Since every other term of $q$ is
  in $H$, it must converge to $H$. Then since $o_q$ is a subsequence of $p$,
  $p$ must cluster at $H$.
\end{proof}


\begin{cor}
  A compact space is Corson compact if and only if
  $\pl D \win\bellConGame{X}$.
\end{cor}

\begin{proof}
  If $\pl D \win\bellConGame{X}$, then $\pl D \win\bellConGame{X^2}$.
  As $X^2$ is (uniformly locally) compact, $\pl D \win\bellAbsConGame{X^2}$.
  Thus $\pl O\win\gruConGame{X^2}{\Delta}$, showing that $X$ is Corson compact.
\end{proof}



\section{Limited information results}

Many of the results shown in \cite{MR3239205} and \cite{nyikosProximalPreprint}
have limited information analogs.

\begin{thm}
For all $x\in X$:
  \begin{itemize}
    \item
      $\pl D\ktactwin{2k} \bellConGame{X} \Rightarrow \pl O \ktactwin{k} \gruConGame{X}{x}$
    \item
      $\pl D\kmarkwin{2k} \bellConGame{X} \Rightarrow \pl O \kmarkwin{k} \gruConGame{X}{x}$
    \item
      $\pl D\win \bellConGame{X} \Rightarrow \pl O \win \gruConGame{X}{x}$
  \end{itemize}
\end{thm}

\begin{proof}
The perfect-information result was originally shown by Bell.

Let $\sigma\circ \nu_{2k}$ witness $\pl D \ktactwin{2k} \bellConGame{X}$,
where $\nu_k$ is the $k$-tactical fog-of-war which removes all information
except the final $k$ moves of the opponent.

We define the $k$-tactical strategy $\tau$ for $\gruConGame{X}{x}$ such that
  \[
    \tau\circ \nu_k(t)
      =
    \sigma\circ \nu_{2k}(\<x,t(0),\dots,x,t(|t|-1)\>)[x]
      \cap
    \sigma\circ \nu_{2k}(\<x,t(0),\dots,x,t(|t|-1),x\>)[x]
  \]

Let $p$ attack $\tau$. Consider the attack $q$ against the winning strategy
$\sigma\circ \nu_{2k}$ such that $q(2n)=x$ and $q(2n+1)=p(n)$, and let
$D_n=\sigma\circ \nu_{2k}(q\rest n)$ and $E_n=\bigcap_{m\leq n}D_m$.

Certainly, $x\in E_{2n}[x]= E_{2n}[q(2n)]$ for any $n<\omega$.
Note also for any $n<\omega$ that
    \[
      p(n) \in
      \bigcap_{m\leq n}\tau\circ \nu_k(p\rest m)
    \]
    \[
      =
      \bigcap_{m\leq n}\left(
        \sigma\circ \nu_{2k}(\<x,p(0),\dots,x,p(m-1)\>)[x]\cap
        \sigma\circ \nu_{2k}(\<x,p(0),\dots,x,p(m-1),x\>)[x]
      \right)
    \]
    \[
      =
      \bigcap_{m\leq n}\left(
        D_{2m}[x]\cap
        D_{2m+1}[x]
      \right) =
      \bigcap_{m\leq 2n+1} D_m[x]=E_{2n+1}[x]
    \]
so by the symmetry of $E_{2n+1}$, $x\in E_{2n+1}[p(n)]= E_{2n+1}[q(2n+1)]$.
Thus $x\in \bigcap_{n<\omega} E_n[q(n)]\not=\emptyset$, and since $\sigma$ is
a winning $2k$-tactic, the attack $q$ converges. Since $q(2n)=x$, $q$ must
converge to $x$. Thus its subsequence $p$ converges to $x$, and $\tau$ is a
winning $k$-tactic for $\pl O$ in $\gruConGame{X}{x}$.

The proofs of the other two implications follow analogously by
replacing $\nu_i$ with the $i$-Markov fog-of-war $\mu_i$ or the identity
function, repsectively.
\end{proof}

\begin{thm}
  Let $X\cup\{\infty\}$ be a uniformizable space such that $X$ is discrete,
  and $k<\omega$. Then
  \begin{itemize}
    \item
      $
        \pl O\ktactwin{k} \gruConGame{X\cup\{\infty\}}{\infty}
          \Leftrightarrow
        \pl D \ktactwin{k} \bellConGame{X\cup\{\infty\}}
      $
    \item
      $
        \pl O\kmarkwin{k} \gruConGame{X\cup\{\infty\}}{\infty}
          \Leftrightarrow
        \pl D \kmarkwin{k} \bellConGame{X\cup\{\infty\}}
      $
    \item
      $
        \pl O\win \gruConGame{X\cup\{\infty\}}{\infty}
          \Leftrightarrow
        \pl D \win \bellConGame{X\cup\{\infty\}}
      $
  \end{itemize}
\end{thm}

\begin{proof}
  The perfect-information result is due to Nyikos.

  The right-to-left implications have already been shown.
  For any open neighborhood $U$ of $\infty$,
  $D(U)=\Delta\cup U^2$ is an open symmetric entourage of $X$.

  Let $\sigma\circ\nu_k$ witness
  $\pl D \ktactwin{k}\gruConGame{X\cap\{\infty\}}{\infty}$. We then define
  the $k$-tactic $\tau$ such that
    \[
      \tau\circ\nu_k(t) = D(\sigma\circ\nu_k(t))
    \]

  Let $p$ attack $\tau$ such that
  $\bigcap_{n<\omega}\tau\circ\nu_k(p\rest n)[p(n)]\not=\emptyset$.

  If $\infty\in\bigcap_{n<\omega}\tau\circ\nu_k(p\rest n)[p(n)]$, it follows
  that $p$
  is an attack on $\sigma\circ\nu_k$. Since $\sigma\circ\nu_k$ is a winning
  strategy, it follows
  that $q$ and its subsequence $p$ must coverge to $\infty$.

  Otherwise, $\infty\not\in\tau\circ\nu_k(p\rest N)[p(N)]$ for some $N<\omega$,
  and then
  $\tau\circ\nu_k(p\rest N)[p(N)]=\{p(N)\}$ implies $p\to p(N)$.

  Thus $\tau$ is a winning $k$-tactic. The other implications may
  be proven analogously by
  replacing $\nu_i$ with the $i$-Markov fog-of-war $\mu_i$ or the identity
  function, repsectively.
\end{proof}

\begin{prop} For any $x\in X$ and $k<\omega$,
  \begin{itemize}
    \item
      $
      \pl O\ktactwin{k+1}\gruConGame{X}{x}
        \Leftrightarrow
      \pl O\tactwin\gruConGame{X}{x}
      $
    \item
      $
      \pl O\kmarkwin{k+1}\gruConGame{X}{x}
        \Leftrightarrow
      \pl O\markwin\gruConGame{X}{x}
      $
  \end{itemize}
\end{prop}


\begin{proof}
  If $\sigma$ witnesses $\pl O\ktactwin{k+1}\gruConGame{X}{x}$,
  let $\tau(\emptyset)=\sigma(\emptyset)$ and
    \[
      \tau(\<q\>)
        =
      \bigcap_{i< k}
      \sigma(\<\underbrace{x,\dots,x}_{k-i},q,\underbrace{x,\dots,x}_{i+1}\>)
    \]

  Then $\tau$ is easily verified to be a winning tactic, and
  the proof for the second part is analogous.
\end{proof}

\begin{cor}
  Let $X\cup\{\infty\}$ be a uniformizable space such that $X$ is discrete,
  and $k<\omega$. Then
  \begin{itemize}
    \item
      $
        \pl D\ktactwin{k+1}\bellConGame{X\cup\{\infty\}}
          \Leftrightarrow
        O\tactwin\bellConGame{X\cup\{\infty\}}
      $
    \item
      $
        \pl D\kmarkwin{k+1}\bellConGame{X\cup\{\infty\}}
          \Leftrightarrow
        O\markwin\bellConGame{X\cup\{\infty\}}
      $
  \end{itemize}
\end{cor}

\begin{prop} For any uniform space $X$ and $k<\omega$,
  \begin{itemize}
    \item
      $
        \pl D\ktactwin{k+2}\bellConGame{X}
          \Leftrightarrow
        \pl D\ktactwin{2}\bellConGame{X}
      $
    \item
      $
        \pl D\kmarkwin{k+2}\bellConGame{X}
          \Leftrightarrow
        \pl D\kmarkwin{2}\bellConGame{X}
      $
  \end{itemize}
\end{prop}

\begin{proof}
  If $\sigma$ witnesses $\pl O\ktactwin{k}\bellConGame{X}$,
  let $\tau(\emptyset)=\sigma(\emptyset)$ and
    \[
      \tau(\<q\>)
        =
      \bigcap_{i<k+2}
      \sigma(\<\underbrace{q,\dots,q}_{i+1}\>)
    \]
    \[
      \tau(\<q,q'\>)
        =
      \bigcap_{i<k+2}
      \sigma(\<\underbrace{q,\dots,q}_{k+1-i},\underbrace{q',\dots,q'}_{i+1}\>)
    \]

  Then $\tau$ is easily verified to be a winning strategy, and
  the proof for the second part is analogous.
\end{proof}

\begin{prop}
  $\pl D \prewin \bellConGame{X}$ implies $X$ is first countable.
\end{prop}

\begin{proof}
  Result follows as $\pl O \prewin \gruConGame{X}{x}$ for all $x\in X$.
\end{proof}

% \begin{defn}
%   Scattered Eberlein compact spaces are known as \term{strong Eberlein
%   compact} spaces.
% \end{defn}

% \begin{thm}[folklore]
%   Scattered compact first-countable spaces are metrizable.
% \end{thm}

% \begin{cor}
%   If $X$ is scattered compact and $\pl D \prewin\bellConGame{X}$,
%   then $X$ is metrizable.
% \end{cor}


\begin{thm}
  If $H$ is a closed subset of $X$, then
    $
      \pl D \limitwin \bellConGame{X}
        \Rightarrow
      \pl D \limitwin \bellConGame{H}
    $
  where $\limitwin$ is any of $\win$, $\ktactwin{k}$, or $\kmarkwin{k}$.
\end{thm}

% \begin{proof}
%   Let $\sigma\circ L$ witness $\pl D \limitwin \bellConGame{X}$. We define
%   $\tau\circ L$ for $\pl D$ in $\bellConGame{H}$ as follows:
%     \[
%       \tau\circ L(p\rest n)
%         =
%       \sigma\circ L(p\rest n)
%         \cap
%       H^2
%     \]

%   Let $p$ attack $\tau\circ L$. $p$ also attacks the winning strategy
%   $\sigma\circ L$, so either
%     \[
%       \bigcap_{n<\omega}
%       \left(\bigcap_{m\leq n}\tau\circ L(p\rest n)\right)
%       [p(n)]
%         \subseteq
%       \bigcap_{n<\omega}
%       \left(\bigcap_{m\leq n}\sigma\circ L(p\rest n)\right)
%       [p(n)]
%         =
%       \emptyset
%     \]
%   or $p$ converges in $X$, and thus converges in $H$.
% \end{proof}


% \begin{thm}
%   If $\pl D\limitwin \bellConGame{X_i}$ for $i<\omega$, then
%   $\pl D\limitwin \bellConGame{\prod_{i<\omega}X_i}$, where $\limitwin$ is either
%   $\win$ or $\kmarkwin{k}$.
% \end{thm}

% (TODO: I expect I should be able to do some clever things
% assuming $S(\kappa,\omega,\omega)$ to get a similar result for sigma
% products of dimension $\kappa$.)

% \begin{lem}
%   $\pl O\prewin\gruClusGame{X}{S}$ if and only if $\pl O\prewin\gruConGame{X}{S}$.
% \end{lem}

% \begin{thm}
%   For any predetermined absolutely proximal space $X$,
%   $\pl O\prewin \gruConGame{X}{H}$ for all compact $H\subseteq X$.
% \end{thm}


% \begin{ex}
%   Let $X=I\times 2$ be the Alexandrov double interval. Then
%   $\pl D \not\prewin \bellConGame{X}$, but $\pl D \markwin \bellConGame{X}$.
% \end{ex}

% \begin{thm}
%   For any uniformly locally compact space $X$,
%       $\pl D\prewin \bellConGame{X} \Leftrightarrow \pl D\prewin \bellClusGame{X}$
% \end{thm}

% \begin{prop}
% If $\pl D \prewin \bellConGame{X}$, then $X$ has a $G_\delta$ diagonal.
% \end{prop}


% \begin{ex}
% The Sorgenfrey line $S$ has a $G_\delta$ diagonal but $\pl P\win\bellConGame{S}$.
% \end{ex}

% \begin{cor}
%   For $X$ with uniformity $\mathbb{D}$ inducing the compact Hausdorff topology
%   $\tau$, the following are equivalent:
%     \begin{enumerate}[(a)]
%       \item $\pl D \prewin \bellConGame{X}$
%       \item $\pl D \prewin \bellClusGame{X}$
%       \item $X$ has a $G_\delta$ diagonal
%       \item $\mathbb{D}$ is metrizable
%       \item $\tau$ is metrizable
%     \end{enumerate}
% \end{cor}


% \begin{thm}
%   A uniformly locally compact space with a $G_\delta$ diagonal is metrizable.
% \end{thm}

% \begin{cor}
%   If $X$ is uniformly locally compact, then $\pl D \prewin \bellConGame{X}$
%   implies $X$'s topology is metrizable.
% \end{cor}

% \begin{ex}
%   Let $R$ be the Michael Line. Then $\pl P\win \bellConGame{X}$.
% \end{ex}

% \begin{proof}
%   During round $0$, $\pl P$ may choose $m(0)=0$ and $p(0)=1$,
%   and during round $n+1$,
%   $\pl P$ may choose $m(n+1)>m(n)$ and
%   $p(n+1)=p(n)+\frac{1}{10^{m(n+1)}}$ such that $p$ is a legal attack.

%   It follows that $p$ ``converges'' to
%   $x=\sum_{n<\omega}\frac{1}{10^{m(n)}}$, except $x$ is an irrational
%   number composed of $1$s separated by strings of $0$s of strictly
%   increasing size.
% \end{proof}

% \begin{ex}
%   Let $\omega_1$ be given a ladder topology:
%     \begin{itemize}
%       \item All successor ordinals are isolated.
%       \item Strictly increasing sequences (ladders) $L_\alpha:\omega\to\alpha$
%             are defined for each limit ordinal $\alpha$ such that $L_\alpha$ converges to $\alpha$ in the order topology, and each limit
%             $\alpha$ is given neighborhoods of the form
%             $L(\alpha,m)=\{\alpha\}\cup\{L_\alpha(n):n\geq m\}$.
%       \item $\omega_1=\bigcup_{\alpha\in\omega_1^L} L(\alpha,0)$
%     \end{itemize}

%   Let
%     \[
%       A(\alpha,n)
%         =
%       [L(\alpha,0)\setminus L(\alpha,n)]^1
%         \cup
%       \{\oneptcomp\omega_1 \setminus (L(\alpha,0)\setminus L(\alpha,n))\}
%     \]
%     \[
%       B(\alpha)
%         =
%       \{L(\alpha,0), \oneptcomp\omega_1 \setminus L(\alpha,0)\}
%     \]

%   Finite refinements of $A(\alpha,n)$ and $B(\alpha)$ give partitions
%   witnessing a uniformization of the ladder topology.

%   Then $\bellConGame{\oneptcomp\omega_1}$ is indetermined.
% \end{ex}

% (TODO: finish proof)