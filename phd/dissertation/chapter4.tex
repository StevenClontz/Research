%!TEX root = dissertation.tex
% ^ leave for LaTeXTools build functionality

\chapter{Bell's Convergence Games}

A very recent development related to Gruenhage's convergence and
clustering games comes from
Jocelyn Bell.



\section{A Game on Uniform Spaces}

\begin{defn}
  A \term{uniformity} on a set $X$ is a filter $\mb D$ of subsets of $X^2$,
  known as \term{$\mb D$-entourages}, such that
  $\bigcap\mb D=\Delta=\{\<x,x\>:x\in X\}$ and, for each entourage $D\in\mb D$:
  \begin{itemize}
    \item There exists ${E}\in\mb D$ such that
      \[
        {E}\circ{E}
          =
        \left\{\<x,z\>:\exists y\in X\left(\<x,y\>,\<y,z\>\in{E}\right)\right\}
          \subseteq
        D
      \]
    \item $D^{-1}=\{\<y,x\>:\<x,y\>\in D\}\in \mb D$
  \end{itemize}
\end{defn}

A set $X$ with a uniformity is called a \term{uniform space}.
As $\mb D$ is a filter, we also have that $D\cap E\in \mb D$ for all
$E\in\mb D$, and $F\in \mb D$ for all $F\supseteq D$. Note that if $\mb E$ is
a filter base satisfying the conditions for a uniformity, then we say
$\mb E$ is a \term{uniformity base} which may be extended to a uniformity by
closing it under the superset operation.

A uniformity is a generalization of a metric.

\begin{defn}
  For a $\mb D$-entourage $D$ and a point $x\in X$, the
  \term{$D$-ball around $x$} is the set $D[x]=\{y:\<x,y\>\in D\}$.
\end{defn}

\begin{defn}
  If $d$ is a metric for the space $X$, then the \term{metric uniformity} for
  $X$ is generated by the uniformity base $\{D_\epsilon:\epsilon>0\}$
  where $D_\epsilon = \{\<x,y\>:d(x,y)<\epsilon\}$.
\end{defn}


Bell first introduced what she called the ``proximal game'' in \cite{MR3239205}.
This game was used to prove that the
$\Sigma$-product of spaces for which $\pl D$ has a winning strategy is
collectionwise normal, as well as to show the collectionwise normality
of certain uniform box products.

\begin{game}
  Let $\bellUniGame{X, \mb D}$ denote the \term{Bell uniform space game}
  with players $\pl D$, $\pl P$ which
  proceeds as follows for a uniform space $X$ with uniformity $\mb D$. In round
  $0$, $\pl D$ chooses a $\mb D$-entourage $D_0$, followed by $\pl P$ choosing a
  point $p_0\in X$. In round $n+1$, $\pl D$ chooses a $\mb D$-entourage
  $D_{n+1}$, followed by $\pl P$ choosing a point $p_{n+1}\in D_n[p_n]$.

  $\pl D$ wins in the case that either $\<p_0,p_1,\dots\>$ converges
  with respect to the uniformity $\mb D$ (there exists $N<\omega$ such that
  $p(n)\in D[x]$ for $n\geq N$), or $\bigcap_{n<\omega}D_n[p_n] = \emptyset$.
\end{game}

\section{Topologizing Bell's Game}

Like metrics, uniformities induce natural topological structures.

\begin{defn}
  The \term{uniform topology} induced by a uniformity $\mb D$ on $X$
  declares $U$ open if for each $x\in U$, there exists $D\in\mathbb{D}$
  such that $D[x]\subseteq U$.
\end{defn}

\begin{thm}
  The \term{uniform topology} induced by a uniformity $\mb D$ on $X$ is the
  coarsest topology such that for each $x\in X$ and $D\in\mathbb{D}$,
  $D[x]$ is a neighborhood (not necessarily open) of $x$.
\end{thm}

\begin{proof}
  Let $\mc T$ be a topology such that $D[x]$ is a neighborhood
  of $x$ for each $x\in X$, and let $U$ be open in the uniform topology.
  For each $x\in U$, there exists $D_x\in\mathbb{D}$ such that
  $D_x[x]\subseteq U$, and since $D_x[x]$ is a neighborhood of $x$, there
  is $U_x\in \mc T$ such that $x\in U_x\subseteq D_x[x]$. Thus
  $\mc T$ contains the uniform topology.
\end{proof}

The uniform topology for a metric uniformity is simply the usual metric
topology, and $D_\epsilon[x]$ is the usual metric $\epsilon$-ball around $x$.

\begin{defn}
  A topological space $X$ is \term{uniformizable} if there exists
  a uniformity which induces the given topology on $X$.
\end{defn}

Bell's game may thus be used to characterize the structure of uniformizable
spaces.

\begin{defn}
  A uniformizable space $X$ is \term{proximal} if there exists a compatible
  uniformity $\mb D$ such that $\pl D\win\bellUniGame{X,\mb D}$.
\end{defn}

However, since the focus of this manuscript is on topological spaces,
it will be useful to recharacterize the proximal property in terms of a
topological game. This may be attained by considering a few known results on
uniform spaces. See e.g. \cite{MR2048350} for proofs.

\begin{thm}
  Every uniform topology is $T_{3\frac{1}{2}}$, and
  every $T_{3\frac{1}{2}}$ topology is uniformizable.
\end{thm}

\begin{thm}
  The union of all uniformities which induce a particular topology is itself
  a uniformity and induces the same topology.
\end{thm}

\begin{defn}
  The \term{universal uniformity} for a uniformizable topology is the uniformity
  finer than all uniformities which induce the given topology.
\end{defn}

\begin{defn}
  For a uniformizable space $X$, a \term{universal entourage} $D$ is a
  $\mb D$-entourage of the universal uniformity $\mb D$.
\end{defn}

\begin{thm}
  For every uniformizable space, if $D$ is a neighborhood of the diagonal
  $\Delta$  such that there exist neighborhoods $D_n$ of $\Delta$ with
  $D\supseteq D_0$ and $D_n \supseteq D_{n+1}\circ D_{n+1}$, then $D$ is a
  universal entourage.
\end{thm}

\begin{thm}
  Every neighborhood of the diagonal is a universal entourage for
  paracompact uniformizable spaces.
\end{thm}

\begin{defn}
  An \term{open symmetric $\mb D$-entourage} $D$ is a $\mb D$-entourage
  which is open in the product topology induced by $\mb D$ and where
  $D=D^{-1}$.
\end{defn}

\begin{thm}
  For every $\mb D$-entourage $D$, there exists an open symmetric
  $\mb D$-entourage $U\subseteq D$.
\end{thm}

We will simply use the word \term{entourage} to refer to open symmetric
universal entourages. Note that if $D$ is an
entourage, then $D[x]$ is an open neighborhood of $x$.

\begin{defn}
  For every entourage $D$, let $\frac{1}{2^n}D$ denote entourages for $n<\omega$
  such that $\frac{1}{1}D=D$ and
  $\frac{1}{2^{n+1}}D\circ\frac{1}{2^{n+1}}D\subseteq\frac{1}{2^n}D$.
\end{defn}

The proof of the following is routine.

\begin{prop}
  If $X$ is a uniformizable space, then for all $x\in X$ and
  entourages $D$:
    \[
      x\in \frac{1}{2}D[y]\text{ and } y\in\frac{1}{2}D[z] \Rightarrow x\in D[z]
    \]
  and
    \[
      \frac{1}{2}D[x]\subseteq \cl{\frac{1}{2}D[x]}\subseteq D[x]
    \]
\end{prop}


The natural adaptation of Bell's game simply replaces the $\mb D$-entourages
of the uniform space with the (open symmetric universal) entourages of
a uniformizable space.

\begin{game}
  Let $\bellConHardGame{X}$ denote the \term{hard Bell convergence game} with
  players $\pl D$, $\pl P$ which proceeds as follows for
  a uniformizable space $X$. In round $0$,
  $\pl D$ chooses an entourage $D_0$, followed by $\pl P$
  choosing a point $p_0\in X$. In round $n+1$, $\pl D$ chooses an entourage
  $D_{n+1}$, followed by $\pl P$ choosing a point $p_{n+1}\in D_n[p_n]$.

  $\pl D$ wins in the case that either $\<p_0,p_1,\dots\>$ converges in $X$,
  or $\bigcap_{n<\omega}D_n[p_n] = \emptyset$. $\pl P$ wins otherwise.
\end{game}

Like $\gruConGameHard{X}{x}$, $\pl D$ may choose to intersect her
plays with her previous plays given perfect information. Since this cannot
be guaranteed with limited information, a simpler variation for $\pl D$
may also be considered, which will be the focus of this chapter.

\begin{game}
  Let $\bellConGame{X}$ denote the \term{Bell convergence game} with players
  $\pl D$, $\pl P$ which proceeds analogously to $\bellConHardGame{X}$, except
  for the following. Let $E_n=\bigcap_{m\leq n}D_n$, where $D_n$ is the
  entourage played by $\pl D$ in round $n$.
  Then $\pl P$ must ensure that $p_{n+1}\in E_n[p_n]$,
  and $\pl D$ wins when either $\<p_0,p_1,\dots\>$ converges in $X$
  or $\bigcap_{n<\omega}E_n[p_n] = \emptyset$.
\end{game}

These games are all essentially equivalent with respect to perfect
information for $\pl D$.

\begin{thm}
  $\pl D\win\bellConHardGame{X}$ if and only if
  $\pl D\win\bellConGame{X}$ if and only if
  $X$ is proximal.
\end{thm}

\begin{proof}
  If $\pl D\win\bellConHardGame{X}$, then we immediately see that
  $\pl D\win\bellConGame{X}$. If $\sigma$ is a winning strategy for $\pl D$
  in $\bellConGame{X}$, then $\tau$ defined by
  $\tau(s)=\bigcap_{t\leq s}\sigma(t)$ is easily seen to be a winning strategy
  for $\pl D$ in $\bellConHardGame{X}$.

  If $\pl D\win\bellConHardGame{X}$, then $\pl D\win\bellUniGame{X,\mb D}$
  where $\mb D$ is the universal uniformity, showing $X$ is proximal.
  Finally, if $X$ is proximal, then there exists a winning strategy $\sigma$
  for $\bellUniGame{X,\mb D}$ where $\mb D$ is a uniformity inducing the
  topology on $X$. Then a winning strategy for $\pl D$ in $\bellConHardGame{X}$
  may be constructed by converting every $\mb D$-entourage into a smaller
  open symmetric universal entourage.
\end{proof}

Bell showed the following results in \cite{MR3239205}.

\begin{thm}
  If $X$ is metrizable, then $\pl D \prewin \bellConGame{X}$.
\end{thm}

\begin{thm}
  If $\pl D \win \bellConGame{X}$, then
  $\pl O \win \gruConGame{X}{x}$ for all $x\in X$.
  Thus proximal spaces are $W$ spaces.
\end{thm}

\begin{thm}
  Proximal spaces are collectionwise normal.
\end{thm}

\begin{thm}
  Every closed subspace of a proximal space is proximal.
\end{thm}

\begin{thm}
  Every $\Sigma$-product of proximal spaces is proximal.
\end{thm}

Since the empty intersection winning condition of her game can obfuscate things,
Bell suggested that $\pl D$ ``absolutely wins'' her uniform space game
if the points played by $\pl P$ always converge, inspiring the following game.

\begin{defn}
  Let $\bellAbsConGame{X}$ denote the
  \term{absolute Bell convergence game} which
  proceeds analogously to $\bellConGame{X}$, except
  that $\pl D$ must always ensure that $\<p_0,p_1,\dots\>$ converges
  in $X$ in order to win.
\end{defn}

\begin{defn}
  A uniformizable space $X$ is \term{absolutely proximal} if
  $\pl D \win \bellAbsConGame{X}$.
\end{defn}



\section{Characterizing Corson compactness}

Gruenhage and the author showed in \cite{MR3227201} that Bell's uniform space
game yields an internal characterization of Corson compact spaces, answering
a question of Nyikos in \cite{nyikosProximalPreprint}. We include a proof
of this result using the topological version of the game instead.

\begin{defn}
  A compact space is \term{Corson compact} if it is homeomorphic to a compact
  subset of a $\Sigma$-product of real lines.
\end{defn}

Nyikos observed the following in \cite{nyikosProximalPreprint}.

\begin{prop}
  Corson compact spaces are proximal.
\end{prop}

\begin{proof}
  The real line is metrizable and thus proximal, and closed subsets of
  $\Sigma$-products of proximal spaces are proximal.
\end{proof}

A result of Gruenhage
\cite{MR752278} gives a useful game characterization of Corson compactness.

\begin{thm}
  A compact space is Corson compact if and only if
  $\pl O \win\gruConGame{X^2}{\Delta}$.
\end{thm}

Thus our desired result will follow if we may show that a winning strategy
in $\bellConGame{X}$ may be used to construct a winning strategy in
$\gruConGame{X^2}{\Delta}$. However, due to the secondary winning condition
for $\pl D$ in $\bellConGame{X}$,
it will be more convenient if we may use a winning strategy for $\pl D$
in $\bellAbsConGame{X}$ instead.

\begin{defn}
  A uniformizable space $X$ is \term{uniformly locally compact} if there
  exists an entourage $D$ such that $\cl{D[x]}$ is compact
  for all $x$.
\end{defn}

Of course, any compact uniformizable space is uniformly locally compact.
However, note that a space may be locally compact without being uniformly
locally compact.

\begin{thm}
  A uniformizable space is uniformly locally compact if and only if it is
  locally compact and paracompact. \cite{MR2040233}
\end{thm}

As an example, $\omega_1$ with the linear order topology is locally compact,
but not paracompact or uniformly locally compact.

\begin{thm}
  If $X$ is a uniformly locally compact space, then
  $\pl D \win \bellConGame{X}$ if and only if
  $\pl D \win \bellAbsConGame{X}$.
\end{thm}

\begin{proof}
  Let $L$ be an entourage such that $\cl{L[x]}$ is compact
  for all $x$.
  Let $\sigma$ be a strategy witnessing $\pl D\win \bellConGame{X}$.
  Without loss of generality, we may assume $\sigma(t)\subseteq L$ and that
  $t \supseteq s$ implies $\sigma(t)\subseteq \frac{1}{4}\sigma(s)$.
  Note then that $\cl{\sigma(t)[x]}\subseteq\cl{L[x]}$ is compact.

  Let $\tau(t)=\frac{1}{2}\sigma(t)$. If $p$ attacks $\tau$
  in $\bellAbsConGame{X}$, then
    \[
      p(n+1)
        \in
      \tau(p\rest n)[p(n)]
        =
      \frac{1}{2}\sigma(p\rest n)[p(n)]
    \]

  \noindent and for

    \[
      x
        \in
      \cl{\sigma(p\rest (n+1))[p(n+1)]}
        \subseteq
      \cl{\frac{1}{4}\sigma(p\rest n)[p(n+1)]}
        \subseteq
      \frac{1}{2}\sigma(p\rest n)[p(n+1)]
    \]

  \noindent we can conclude $x\in\sigma(p\rest n)[p(n)]$. Thus

    \[
      \sigma(p\rest (n+1))[p(n+1)]
        \subseteq
      \cl{\sigma(p\rest (n+1))[p(n+1)]}
        \subseteq
      \sigma(p\rest n)[p(n)]
    \]

  Finally, note that since $\tau$ yields subsets of $\sigma$, $p$ also attacks
  the winning strategy $\sigma$ in $\bellConGame{X}$, but since the intersection
  of a descending chain of nonempty compact sets is nonempty, we have

    \[
      \bigcap_{n<\omega} \sigma(p\rest n)[p(n)]
        =
      \bigcap_{n<\omega} \cl{\sigma(p\rest n)[p(n)]}
        \not=
      \emptyset
    \]

  We conclude that $p$ converges.
\end{proof}


\begin{lem}
  If $\pl D\win\bellAbsConGame{X}$, then $\pl O\win\gruConGame{X}{H}$ for
  all compact $H$ in $X$.
\end{lem}


\begin{proof}
  Let $\sigma$ be a winning strategy for $\pl D$ in $\bellAbsConGame{X}$
  game such that $p\supsetneq q$ implies
  $\sigma(p)\subseteq \frac{1}{4}\sigma(q)$.
  For any sequence $t$, let $o_t=\{\<n,t(2n+1)\>:2n+1\in\dom(t)\}$
  be the subsequence of $t$ consisting of its odd-indexed terms.
  We proceed by constructing a winning strategy for $\pl O$ in
  $\gruClusGame{X}{H}$. Since $\pl O\win\gruClusGame{X}{H}$
  if and only if $\pl O\win\gruConGame{X}{H}$, the result will follow.

%   \bigskip

  We begin by defining a tree $T(\emptyset)$, during which we will
  define a number $m_\emptyset<\omega$ and points
  $h_{\emptyset,i},h_{\emptyset,i,j}$ for $i,j<m_\emptyset$
  which yield an open set
    \[
      \bigcup_{i,j<m_\emptyset}
        \frac{1}{4}\sigma(\<h_{\emptyset,i}\>)[h_{\emptyset,i,j}]
      =
      \bigcup_{\emptyset\concat\<i,h_{\emptyset,i},j\>\in\max(T(\emptyset))}
        \frac{1}{4}\sigma(o_\emptyset\concat\<h_{\emptyset,i}\>)[h_{\emptyset,i,j}]
    \]
  containing $H$. $\pl O$ will use this as the inital move in her winning
  strategy for $\gruClusGame{X}{H}$.

  \begin{itemize}
    \item
    Choose $m_\emptyset<\omega$, $h_{\emptyset,i}\in H$ for $i<m_\emptyset$, and
    $h_{\emptyset,i,j}\in H\cap\cl{\frac{1}{4}\sigma(\emptyset)[h_{\emptyset,i}]}$
    for $i,j<m_\emptyset$ such that
      \[
        \left\{\frac{1}{4}\sigma(\emptyset)[h_{\emptyset,i}]:i<m_\emptyset\right\}
      \]
    is a cover for $H$ and such that for each $i<m_\emptyset$
      \[
        \left\{\frac{1}{4}\sigma(\<h_{\emptyset,i}\>)[h_{\emptyset,i,j}]:j<m_\emptyset\right\}
      \]
    is a cover for $H\cap\cl{\frac{1}{4}\sigma(\emptyset)[h_{\emptyset,i}]}$.
    \item
    Let $\<i,h_{\emptyset,i},j\>$ and its initial segments be in
    $T(\emptyset)$ for $i,j<m_\emptyset$.
  \end{itemize}

  It follows that
    \[
      \left\{
        \frac{1}{4}\sigma(o_\emptyset\concat\<h_{\emptyset,i}\>)[h_{\emptyset,i,j}]
      :
        \emptyset\concat\<i,h_{\emptyset,i},j\>\in\max(T(a))
      \right\}
    \]
  covers $H$.

% \bigskip

  Now suppose that $a$ is a partial attack by $\pl P$ in $\gruClusGame{X}{H}$
  for which we have defined a tree $T(a)$.
  We will define $T(a\concat\<x\>)\supseteq T(a)$ for each $x\in X$, and
  whenever $x\in \frac{1}{4}\sigma(o_s\concat\<h_{s,i}\>)[h_{s,i,j}]$ for
  some $s\concat\<i,h_{s,i},j\>\in\max(T(a))$, we will set
  $t=s\concat\<i,h_{s,i},j,x\>$ and define a number $m_t<\omega$ and points
  $h_{t,k},h_{t,k,l}$ for $k,l<m_t$ to yield an open set
    \[
      \bigcup_{t\concat\<k,h_{t,k},l\>\in\max(T(a\concat\<x\>))}
        \frac{1}{4}\sigma(o_t\concat\<h_{t,k}\>)[h_{t,k,l}]
    \]
  which will be the next open neighborhood of $H$ in $\pl O$'s winning strategy
  for $\gruClusGame{X}{H}$.

  \begin{itemize}
    \item
    We will extend the nodes $s\concat\<i,h_{s,i},j\>\in\max(T(a))$ such
    that $x\in \frac{1}{4}\sigma(o_s\concat\<h_{s,i}\>)[h_{s,i,j}]$.
    For each such $s\concat\<i,h_{s,i},j\>$, set $t=s\concat\<i,h_{s,i},j,x\>$.
    \item
    Note that whenever $o_s\concat\<h_{s,i}\>$ is a legal partial attack
    against $\sigma$, then
      \[
        x
          \in
        \frac{1}{4}\sigma(o_s\concat\<h_{s,i}\>)[h_{s,i,j}]
          \subseteq
        \frac{1}{4}\sigma(o_s)[h_{s,i,j}]
      \]
    and
      \[
        h_{s,i,j}
          \in
        \cl{\frac{1}{4}\sigma(o_s)[h_{s,i}]}
          \subseteq
        \frac{1}{2}\sigma(o_s)[h_{s,i}]
      \]
    implies
      \[
        x
          \in
        \sigma(o_s)[h_{s,i}]
      \]
    and thus $o_s\concat\<h_{s,i},x\>=o_t$ is also a legal partial attack
    against $\sigma$.
    \item
    Choose $m_t<\omega$,
    $h_{t,k}\in H\cap \cl{\frac{1}{4}\sigma(o_s\concat\<h_{s,i}\>)[h_{s,i,j}]}$
    for $k<m_t$, and
    $h_{t,k,l}\in H\cap\cl{\frac{1}{4}\sigma(o_t)[h_{t,k}]}$
    for $k,l<m_t$ such that
      \[
        \left\{\frac{1}{4}\sigma(o_t)[h_{t,k}]:k<m_t\right\}
      \]
    is a cover for
    $H\cap \cl{\frac{1}{4}\sigma(o_s\concat\<h_{s,i}\>)[h_{s,i,j}]}$
    and such that for each $k<m_t$
      \[
        \left\{\frac{1}{4}\sigma(o_t\concat\<h_{t,k}\>)[h_{t,k,l}]:l<m_t\right\}
      \]
    is a cover for $H\cap\cl{\frac{1}{4}\sigma(o_t)[h_{t,k}]}$.
    \item
    Note that whenever $o_t$ is a legal partial attack against $\sigma$, then
      \[
        h_{t,k}
          \in
        \cl{\frac{1}{4}\sigma(o_s\concat\<h_{s,i}\>)[h_{s,i,j}]}
          \subseteq
        \frac{1}{2}\sigma(o_s\concat\<h_{s,i}\>)[h_{s,i,j}]
      \]
    and
      \[
        x
          \in
        \frac{1}{4}\sigma(o_s\concat\<h_{s,i}\>)[h_{s,i,j}]
      \]
    implies
      \[
        h_{t,k}
          \in
        \sigma(o_s\concat\<h_{s,i}\>)[x]
      \]
    and thus $o_t\concat\<h_{t,k}\>$ is a legal partial attack against $\sigma$.
    \item
    Include all initial segments of $t\concat\<k,h_{t,k},l\>$ in
    $T(a\concat\<x\>)$ for $k,l<m_t$.
  \end{itemize}

%     \bigskip

 This completes the construction of  $ T(a\concat\<x\>)$. Note that since
    \[
      \left\{
        \frac{1}{4}\sigma(o_s\concat\<h_{s,i}\>)[h_{s,i,j}]
      :
        s\concat\<i,h_{s,i},j\>\in\max(T(a))
      \right\}
    \]
  covers $H$, then since
    \[
      \left\{
        \frac{1}{4}\sigma(o_t\concat\<h_{t,k}\>)[h_{t,k,l}]
      :
        s\concat\<i,h_{s,i},j,x,k,h_{t,k},l\>
          \in
        \max(T(a\concat\<x\>))\setminus\max(T(a))
      \right\}
    \]
  covers $H\cap \frac{1}{4}\sigma(o_s\concat\<h_{s,i}\>)[h_{s,i,j}]$, we have that
    \[
      \left\{
        \frac{1}{4}\sigma(o_t\concat\<h_{t,k}\>)[h_{t,k,l}]
      :
        t\concat\<k,h_{t,k},l\>\in\max(T(a\concat\<x\>))
      \right\}
    \]
  covers $H$.

  We define a strategy $\tau$ for $\pl O$ in $\gruClusGame{X}{H}$ such
  that:
  \[
    \tau(a)
      =
    \bigcup_{s\concat\<i,h_{s,i},j\>\in\max(T(a))}
    \frac{1}{4}\sigma(o_s\concat\<h_{s,i}\>)[h_{s,i,j}]
  \]

  If $p$ is a legal attack by $\pl P$ against $\tau$, then
  let $\ds T(p)=\bigcup_{n<\omega} T(p\rest n)$. We note $T(p)$ is an infinite
  tree (for each $n<\omega$, a node of $T(p\rest n)$ was extended)
  with finite levels:
    \begin{itemize}
      \item $\emptyset$ has exactly $m_\emptyset$ successors $\<i\>$.
      \item $s\concat\<i\>$ has exactly one successor $s\concat\<i,h_{s,i}\>$
      \item $s\concat\<i,h_{s,i}\>$ has exactly $m_s$ successors
            $s\concat\<i,h_{s,i},j\>$
      \item $s\concat\<i,h_{s,i},j\>$ has either no successors or exactly
            one successor $s\concat\<i,h_{s,i},j,x\>$
      \item $t=s\concat\<i,h_{s,i},j,x\>$ has exactly $m_t$ successors
            $t\concat\<k\>$
    \end{itemize}

  Hence $T(p)$ has an infinite branch
    \[
      q'=\<i_0,h_0,j_0,x_0,i_1,h_1,j_1,x_1,\dots\>
    \]
  Let $q=o_{q'}=\<h_0,x_0,h_1,x_1,\dots\>$. Note that by the construction of
  $T(p)$, $q$ is a legal attack on the winning strategy $\sigma$ in
  $\bellAbsConGame{X}$, so it must converge. Since every other term of $q$ is
  in $H$, it must converge to $H$. Then since $o_q$ is a subsequence of $p$,
  $p$ must cluster at $H$.
\end{proof}


\begin{thm}
  A compact space is Corson compact if and only if
  $\pl D \win\bellConGame{X}$.
\end{thm}

\begin{proof}
  If $\pl D \win\bellConGame{X}$, then $\pl D \win\bellConGame{X^2}$.
  As $X^2$ is (uniformly locally) compact, $\pl D \win\bellAbsConGame{X^2}$.
  Thus $\pl O\win\gruConGame{X^2}{\Delta}$, showing that $X$ is Corson compact.
\end{proof}



\section{Limited information results}

One may generalize many of the results originally shown by Bell \cite{MR3239205}
and Nyikos \cite{nyikosProximalPreprint} by considering limited information
strategies.

\begin{thm}
Let $k<\omega$. For all $x\in X$:
  \begin{itemize}
    \item
      $
        \pl D\ktactwin{2k} \bellConGame{X}
          \Rightarrow
        \pl O \ktactwin{k} \gruConGame{X}{x}
      $
    \item
      $
        \pl D\kmarkwin{2k} \bellConGame{X}
          \Rightarrow
        \pl O \kmarkwin{k} \gruConGame{X}{x}
      $
    \item
      $
        \pl D\win \bellConGame{X}
          \Rightarrow
        \pl O \win \gruConGame{X}{x}
      $
  \end{itemize}
\end{thm}

\begin{proof}
The perfect-information result was originally shown by Bell.
Let $L_k$ represent either the $k$-tactical fog-of-war $\nu_k$,
the $k$-Markov fog-of-war $\mu_k$, or the identity $id$.

Let $\sigma\circ L_{2k}$ be a winning strategy for $\pl D$ in $\bellConGame{X}$.
We define the strategy $\tau\circ L_k$ for $\pl O$ in $\gruConGame{X}{x}$ such
that
  \[
    \tau\circ {L}_k(t)
      =
    \sigma\circ {L}_{2k} \Big(\<x,t(0),\dots,x,t(|t|-1)\>\Big)[x]
      \cap
    \sigma\circ {L}_{2k} \Big(\<x,t(0),\dots,x,t(|t|-1),x\>\Big)[x]
  \]

Let $p$ attack $\tau$ such that $p(n)\in\bigcap_{m\leq n}\tau\circ L_k(t)$.
Consider the attack $q$ against the winning strategy
$\sigma\circ {L}_{2k}$ such that $q(2n)=x$ and $q(2n+1)=p(n)$.
Let $D_n=\sigma\circ {L}_{2k}(q\rest n)$ and $E_n=\bigcap_{m\leq n}D_m$.

Certainly, $x\in E_{2n}[x]= E_{2n}[q(2n)]$ for any $n<\omega$.
Note also for any $n<\omega$ that
    \[
      p(n) \in
      \bigcap_{m\leq n}\tau\circ {L}_k(p\rest m)
    \]
    \[
      =
      \bigcap_{m\leq n}\left(
        \sigma\circ {L}_{2k}(\<x,p(0),\dots,x,p(m-1)\>)[x]\cap
        \sigma\circ {L}_{2k}(\<x,p(0),\dots,x,p(m-1),x\>)[x]
      \right)
    \]
    \[
      =
      \bigcap_{m\leq n}\left(
        D_{2m}[x]\cap
        D_{2m+1}[x]
      \right) =
      \bigcap_{m\leq 2n+1} D_m[x]=E_{2n+1}[x]
    \]
so by the symmetry of $E_{2n+1}$, $x\in E_{2n+1}[p(n)]= E_{2n+1}[q(2n+1)]$.
Also, $q(2n+1)=p(n)\in E_{2n}[x]=E_{2n}[q(2n)]$ and
$p(n)\in E_{2n+1}[x] \Rightarrow q(2n+2)=x\in E_{2n+1}[p(n)]=E_{2n+1}[q(2n+1)]$,
making $q$ a legal attack.

Then as $x\in \bigcap_{n<\omega} E_n[q(n)]\not=\emptyset$, and $\sigma$ is
a winning strategy, the attack $q$ converges. Since $q(2n)=x$, $q$ must
converge to $x$. Thus its subsequence $p$ converges to $x$, and $\tau\circ L_k$
is a winning strategy for $\pl O$ in $\gruConGame{X}{x}$.
\end{proof}

\begin{thm}
  Let $X\cup\{\infty\}$ be a $T_1$ space with points in $X$ isolated
  (and therefore a uniformizable space),
  and $k<\omega$. Then
  \begin{itemize}
    \item
      $
        \pl O\ktactwin{k} \gruConGame{X\cup\{\infty\}}{\infty}
          \Leftrightarrow
        \pl D \ktactwin{k} \bellConGame{X\cup\{\infty\}}
      $
    \item
      $
        \pl O\kmarkwin{k} \gruConGame{X\cup\{\infty\}}{\infty}
          \Leftrightarrow
        \pl D \kmarkwin{k} \bellConGame{X\cup\{\infty\}}
      $
    \item
      $
        \pl O\win \gruConGame{X\cup\{\infty\}}{\infty}
          \Leftrightarrow
        \pl D \win \bellConGame{X\cup\{\infty\}}
      $
  \end{itemize}
\end{thm}

\begin{proof}
  The perfect-information result is due to Nyikos.
  Let $L_k$ represent either the $k$-tactical fog-of-war $\nu_k$,
  the $k$-Markov fog-of-war $\mu_k$, or the identity $id$.

  The right-to-left implications have already been shown.
  For any open neighborhood $U$ of $\infty$,
  $D(U)=\Delta\cup U^2$ is an entourage of $X$.

  Let $\sigma\circ{L}_k$ be a winning strategy for
  $\gruConGame{X\cap\{\infty\}}{\infty}$.
  We then define the strategy $\tau\circ L_k$ such that
    \[
      \tau\circ{L}_k(t) = D(\sigma\circ{L}_k(t))
    \]

  Let $p$ attack $\tau$ such that
  $\bigcap_{n<\omega}\tau\circ{L}_k(p\rest n)[p(n)]\not=\emptyset$.

  If $\infty\in\bigcap_{n<\omega}\tau\circ{L}_k(p\rest n)[p(n)]$, it follows
  that $p$ is a legal attack on $\sigma\circ{L}_k$. Since $\sigma\circ{L}_k$
  is a winning strategy, it follows that $p\to\infty$.

  Otherwise, $\infty\not\in\tau\circ{L}_k(p\rest N)[p(N)]$ for some $N<\omega$,
  and then
  $\tau\circ{L}_k(p\rest N)[p(N)]=\{p(N)\}$ implies $p\to p(N)$.

  Thus $\tau\circ L_k$ is a winning strategy for
  $\gruConGame{X\cup\{\infty\}}{\infty}$.
\end{proof}

\begin{cor}
  Let $X\cup\{\infty\}$ be a uniformizable space such that $X$ is discrete,
  and $k<\omega$. Then
  \begin{itemize}
    \item
      $
        \pl D\ktactwin{k+1}\bellConGame{X\cup\{\infty\}}
          \Leftrightarrow
        \pl D\tactwin\bellConGame{X\cup\{\infty\}}
      $
    \item
      $
        \pl D\kmarkwin{k+1}\bellConGame{X\cup\{\infty\}}
          \Leftrightarrow
        \pl D\markwin\bellConGame{X\cup\{\infty\}}
      $
  \end{itemize}
\end{cor}

\begin{proof}
  The equivalencies hold for $\gruConGame{X}{x}$.
\end{proof}

A close result may be obtained for arbitrary uniformizable spaces.

\begin{prop} For any uniformizable space $X$ and $k<\omega$,
  \begin{itemize}
    \item
      $
        \pl D\ktactwin{k+2}\bellConGame{X}
          \Leftrightarrow
        \pl D\ktactwin{2}\bellConGame{X}
      $
    \item
      $
        \pl D\kmarkwin{k+2}\bellConGame{X}
          \Leftrightarrow
        \pl D\kmarkwin{2}\bellConGame{X}
      $
  \end{itemize}
\end{prop}

\begin{proof}
  If $\sigma$ is a winning $k+2$-tactic, then define $\tau$ by
    \[
      \tau (s)
        =
      \bigcap_{t\in \ran{s}^{<k+2}}
      \sigma(t)
    \]

  If $\sigma$ is a winning $k+2$-mark, then define $\tau$ by
    \[
      \tau (s,n)
        =
      \bigcap_{t\in \ran{s}^{<k+2} \atop {m\leq(k+2)n}}
      \sigma(t,m)
    \]

  In either case, the proof that $\tau$ is a winning limited information
  strategy is routine.
\end{proof}


\begin{thm}
  Let $X$ be a uniformizable space and $H$ be a closed subset of $X$.
  If $k<\omega$, then
  \begin{itemize}
    \item
      $
        \pl D \ktactwin{k} \bellConGame{X}
          \Rightarrow
        \pl D \ktactwin{k} \bellConGame{H}
      $
    \item
      $
        \pl D \kmarkwin{k} \bellConGame{X}
          \Rightarrow
        \pl D \kmarkwin{k} \bellConGame{H}
      $
    \item
      $
        \pl D \win \bellConGame{X}
          \Rightarrow
        \pl D \win \bellConGame{H}
      $
  \end{itemize}
\end{thm}

\begin{proof}
  The perfect-information result was originally shown by Bell.
  Let $L_k$ represent either the $k$-tactical fog-of-war $\nu_k$,
  the $k$-Markov fog-of-war $\mu_k$, or the identity $id$.

  Let $\sigma\circ L_k$ be a winning strategy for $\pl D$ in $\bellConGame{X}$.
  We define the strategy $\tau\circ L_k$ for $\pl D$ in $\bellConGame{H}$ as
  follows:
    \[
      \tau\circ L(p\rest n)
        =
      \sigma\circ L(p\rest n)
        \cap
      H^2
    \]

  Let $p$ attack $\tau\circ L_k$. $p$ also attacks the winning strategy
  $\sigma\circ L_k$, so either
    \[
      \bigcap_{n<\omega}
      \left(\bigcap_{m\leq n}\tau\circ L(p\rest n)\right)
      [p(n)]
        \subseteq
      \bigcap_{n<\omega}
      \left(\bigcap_{m\leq n}\sigma\circ L(p\rest n)\right)
      [p(n)]
        =
      \emptyset
    \]
  or $p:\omega\to H$ converges in $X$, and thus converges in $H$ as $H$ is
  closed.
\end{proof}

Bell showed the following to obtain a result of Mary Ellen Rudin \cite{MR716576}
and S. P. Gulko \cite{MR0461410} as a corollary.

\begin{thm}
  A $\Sigma$-product of proximal spaces is proximal.
\end{thm}

\begin{cor}
  A $\Sigma$-product of metrizable spaces is collectionwise normal.
\end{cor}

A sketch of the proof: $\pl D$ may use the winning strategies for the
proximal spaces coordinate-wise to ensure that either every coordinate
converges, or one coordinate intersects to the empty set. But in order to do
this throughout the entire game, perfect information is used in a non-trival
way to remember all coordinates for which $\pl P$ played a point with
non-zero value in that coordinate. But for countable products, this memory
may be replaced with knowledge of the round number.

\begin{lem}
  Let $D_\alpha$ be an entourage of $X_\alpha$ for $\alpha<\kappa$. Then
  $P_\alpha(D_\alpha)=\{\<x,y\>:\<x(\alpha),y(\alpha)\>\in D_\alpha\}$
  is an entourage of $\prod_{\alpha<\kappa} X_\alpha$.
\end{lem}

\begin{proof}
  Let $D_\alpha\supseteq U_0$ and $U_{n+1}\circ U_{n+1}\subseteq U_n$ for
  neighborhoods $U_n$ of the diagonal of $X_\alpha$. Then note that
  $P_\alpha(U_{n})$ is a neighborhood of the diagonal of
  $\prod_{\alpha<\kappa} X_\alpha$. Thus as
  $P_\alpha(D_\alpha) \supseteq P_\alpha(U_0)$ and
  $P_\alpha(U_{n+1})\circ P_\alpha(U_{n+1})\subseteq P_\alpha(U_{n})$,
  we conclude $P_\alpha(D_\alpha)$ is an entourage.
\end{proof}

\begin{thm}
  If $\pl D\kmarkwin{k} \bellConGame{X_i}$ for $i<\omega$, then
  $\pl D\kmarkwin{k} \bellConGame{\prod_{i<\omega}X_i}$.
\end{thm}

\begin{proof}
  Let $\sigma_i$ be a winning $k$-mark for $\pl D$ in $\bellConGame{X_i}$.
  For $s\in \left(\prod_{i<\omega}X_i\right)^{\leq \omega}$,
  let $s_i \in X_i^{\leq \omega}$ such that $s(j)(i)=s_i(j)$ for $j\in\dom(s)$.
  Recall that $\nu_k$ removes all but the last $k$ elements of a finite
  sequence, and for an countably infinite sequence $p$,
  let $p\rest^k n=\nu_k(p\rest n)$.

  Define the $k$-mark $\tau$ for $\pl D$ in $\bellConGame{\prod_{i<\omega}X_i}$
  by
    \[
      \tau(s,n)
        =
      \bigcap_{i\leq n}
      P_i(\sigma_i(s_i,n))
    \]
  and let $p\in\left(\prod_{i<\omega}X_i\right)^\omega$ be a legal attack
  against $\tau$, so the following must hold for $m,n<\omega$:
    \[
      p(n+1)
        \in
      \left(\bigcap_{j\leq n}\tau(p\rest^k j,j)\right)[p(n)]
        =
      \left(\bigcap_{i\leq j\leq n}P_i(\sigma_i(p_i\rest^k j,j))\right)[p(n)]
    \]
    \[
      p_m(m+n+1)
        \in
      \left(\bigcap_{j\leq n}\sigma_m(p_m\rest^k(m+j),m+j)\right)[p_m(m+n)]
    \]

  For $m<\omega$, attack $\sigma_m$ with $q_m\in X_m^\omega$ defined by
  $q_m(n)=p_m(m+n)$. Note that
    \[
      q_m(n+1) = p_m(m+n+1)
        \in
      \left(\bigcap_{j\leq n}\sigma_m(p_m\rest^k(m+j),m+j)\right)[p_m(m+n)]
    \]
    \[
        \subseteq
      \left(\bigcap_{j\leq n}\sigma_m(q_m\rest^k(j),j)\right)[q_m(n)]
    \]
  so $q_m$ is a legal attack.

  If for some $m<\omega$,
    \[
      \emptyset
        =
      \left(\bigcap_{n<\omega} \sigma_m(q_m\rest^k n,n)\right)[q_m(n)]
        \supseteq
      \left(\bigcap_{n<\omega} \sigma_m(p_m\rest^k(m+n),n)\right)[p_m(m+n)]
    \]
    \[
        \supseteq
      \left(\bigcap_{n<\omega} \sigma_m(p_m\rest^k(m+n),m+n)\right)[p_m(m+n)]
    \]
    \[
        =
      \left(\bigcap_{m\leq n<\omega} \sigma_m(p_m\rest^k n,n)\right)[p_m(n)]
    \]
  then
    \[
      \bigcap_{n<\omega} \left(
        \bigcap_{j\leq n} \tau(p\rest^k j,j)
      \right)[p(n)]
        =
      \bigcap_{n<\omega} \left(
        \bigcap_{i\leq j\leq n} P_i(\sigma_i(p_m\rest^k j,j))
      \right)[p(n)]
    \]
    \[
        \subseteq
      \bigcap_{m\leq n<\omega} \left(
        \bigcap_{m\leq j\leq n} P_m(\sigma_m(p_m\rest^k j,j))
      \right)[p(n)]
        \subseteq
      \bigcap_{m\leq n<\omega} \left(
        \bigcap_{m\leq n} P_m(\sigma_m(p_m\rest^k n,n))
      \right)[p(n)]
    \]
    \[
        =
      \bigcap_{m\leq n<\omega} \left(
        \bigcap_{m\leq n} P_m(\sigma_m(p_m\rest^k n,n)[p_m(n)])
      \right)
        =
      \bigcap_{m\leq n<\omega}
      P_m(\sigma_m(p_m\rest^k n,n)[p_m(n)])
    \]
    \[
        =
      P_m \left(
        \bigcap_{m\leq n<\omega}\sigma_m(p_m\rest^k n,n)[p_m(n)]
      \right)
        =
      P_m(\emptyset)
        =
      \emptyset
    \]

  Otherwise for all $m<\omega$, $q_m$ converges to some $x_m\in X_m$,
  then $p_m$ converges to $x_m\in X_m$ and thus $p$ converges. This
  shows that $\tau$ is a winning $k$-mark.
\end{proof}