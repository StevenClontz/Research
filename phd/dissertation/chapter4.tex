%!TEX root = dissertation.tex
% ^ leave for LaTeXTools build functionality

\chapter{The Proximal Game}

placeholder

% Results pertaining to Bell's proximal game for uniform spaces.



% \begin{thm}
% For all $x\in X$:
%   \begin{itemize}
%     \item
%       $\pl D\win \proxgame{X} \Rightarrow \pl O \win \congame{X}{x}$
%     \item
%       $\pl D\ktactwin{2k} \proxgame{X} \Rightarrow \pl O \ktactwin{k} \congame{X}{x}$
%     \item
%       $\pl D\kmarkwin{2k} \proxgame{X} \Rightarrow \pl O \kmarkwin{k} \congame{X}{x}$
%   \end{itemize}
% \end{thm}

% \begin{thm}
%   Let $X\cup\{\infty\}$ be a uniformizable space such that $X$ is discrete. Then
%   \begin{itemize}
%     \item
%       $\pl O\win \congame{X\cup\{\infty\}}{\infty} \Leftrightarrow \pl D \win \proxgame{X\cup\{\infty\}}$
%     \item
%       $\pl O\ktactwin{k} \congame{X\cup\{\infty\}}{\infty} \Leftrightarrow \pl D \ktactwin{k} \proxgame{X\cup\{\infty\}}$
%     \item
%       $\pl O\kmarkwin{k} \congame{X\cup\{\infty\}}{\infty} \Leftrightarrow \pl D \kmarkwin{k} \proxgame{X\cup\{\infty\}}$
%   \end{itemize}
% \end{thm}

% \begin{prop} For any $x\in X$ and $k\geq 1$,
%   \begin{itemize}
%     \item
%       $\pl O\ktactwin{k}\congame{X}{x} \Leftrightarrow \pl O\tactwin\congame{X}{x}$
%     \item
%       $\pl O\kmarkwin{k}\congame{X}{x} \Leftrightarrow \pl O\markwin\congame{X}{x}$
%   \end{itemize}
% \end{prop}

% \begin{cor}
%   Let $X\cup\{\infty\}$ be a uniformizable space such that $X$ is discrete, and $k\geq 1$. Then
%   \begin{itemize}
%     \item
%       $\pl D\ktactwin{k}\proxgame{X\cup\{\infty\}} \Leftrightarrow O\tactwin\proxgame{X\cup\{\infty\}}$
%     \item
%       $\pl D\kmarkwin{k}\proxgame{X\cup\{\infty\}} \Leftrightarrow O\markwin\proxgame{X\cup\{\infty\}}$
%   \end{itemize}
% \end{cor}

% \begin{prop} For any uniform space $X$,
%   \begin{itemize}
%     \item
%       $\pl D\ktactwin{k}\proxgame{X} \Leftrightarrow \pl D\ktactwin{2}\proxgame{X}$
%     \item
%       $\pl D\kmarkwin{k}\proxgame{X} \Leftrightarrow \pl D\kmarkwin{2}\proxgame{X}$
%   \end{itemize}
% \end{prop}

% \begin{thm}
%   For any uniformly locally compact space $X$,
%       $\pl D\win \proxgame{X} \Leftrightarrow \pl D\win \aproxgame{X}$
% \end{thm}

% \begin{thm}
%   For any uniformly locally compact proximal space $X$, $\pl O\win \congame{X}{H}$ for all compact $H\subseteq X$.
% \end{thm}

% \begin{cor}
%   A compact uniform space $X$ is Corson compact if and only if it is proximal.
% \end{cor}

% \begin{thm}
%   $\pl O \prewin \congame{X}{H}$ if and only if there exists a countable base
%   around $H$.
% \end{thm}


% \begin{cor}
%   $X$ is first countable if and only if $\pl O \prewin \congame{X}{x}$ for
%   all $x\in X$
% \end{cor}

% \begin{cor}
%   $\pl D \prewin \proxgame{X}$ implies $X$ is first countable.
% \end{cor}

% \begin{cor}
%   If $X$ is scattered compact and $\pl O \prewin\congame{X}{x}$ for all
%   $x\in X$ (or $\pl D \prewin\proxgame{X}$), then $X$ is metrizable.
% \end{cor}


% \begin{thm}
%   If $H$ is a closed subset of $X$, then
%     $
%       \pl D \limitwin \proxgame{X}
%         \Rightarrow
%       \pl D \limitwin \proxgame{H}
%     $
%   where $\limitwin$ is any of $\win$, $\ktactwin{k}$, or $\kmarkwin{k}$.
% \end{thm}


% \begin{thm}
%   If $\pl D\limitwin \proxgame{X_i}$ for $i<\omega$, then
%   $\pl D\limitwin \proxgame{\prod_{i<\omega}X_i}$, where $\limitwin$ is either
%   $\win$ or $\kmarkwin{k}$.
% \end{thm}

% (TODO: I expect I should be able to do some clever things
% assuming $S(\kappa,\omega,\omega)$ to get a similar result for sigma
% products of dimension $\kappa$.)

% \begin{lem}
%   $\pl O\prewin\clusgame{X}{S}$ if and only if $\pl O\prewin\congame{X}{S}$.
% \end{lem}

% \begin{thm}
%   For any predetermined absolutely proximal space $X$,
%   $\pl O\prewin \congame{X}{H}$ for all compact $H\subseteq X$.
% \end{thm}


% \begin{ex}
%   Let $X=I\times 2$ be the Alexandrov double interval. Then
%   $\pl D \not\prewin \proxgame{X}$, but $\pl D \markwin \proxgame{X}$.
% \end{ex}

% \begin{thm}
%   For any uniformly locally compact space $X$,
%       $\pl D\prewin \proxgame{X} \Leftrightarrow \pl D\prewin \aproxgame{X}$
% \end{thm}

% \begin{prop}
% If $\pl D \prewin \proxgame{X}$, then $X$ has a $G_\delta$ diagonal.
% \end{prop}


% \begin{ex}
% The Sorgenfrey line $S$ has a $G_\delta$ diagonal but $\pl P\win\proxgame{S}$.
% \end{ex}

% \begin{cor}
%   For $X$ with uniformity $\mathbb{D}$ inducing the compact Hausdorff topology
%   $\tau$, the following are equivalent:
%     \begin{enumerate}[(a)]
%       \item $\pl D \prewin \proxgame{X}$
%       \item $\pl D \prewin \aproxgame{X}$
%       \item $X$ has a $G_\delta$ diagonal
%       \item $\mathbb{D}$ is metrizable
%       \item $\tau$ is metrizable
%     \end{enumerate}
% \end{cor}


% \begin{thm}
%   A uniformly locally compact space with a $G_\delta$ diagonal is metrizable.
% \end{thm}

% \begin{cor}
%   If $X$ is uniformly locally compact, then $\pl D \prewin \proxgame{X}$
%   implies $X$'s topology is metrizable.
% \end{cor}

% \begin{ex}
%   Let $R$ be the Michael Line. Then $\pl P\win \proxgame{X}$.
% \end{ex}

% \begin{proof}
%   During round $0$, $\pl P$ may choose $m(0)=0$ and $p(0)=1$,
%   and during round $n+1$,
%   $\pl P$ may choose $m(n+1)>m(n)$ and
%   $p(n+1)=p(n)+\frac{1}{10^{m(n+1)}}$ such that $p$ is a legal attack.

%   It follows that $p$ ``converges'' to
%   $x=\sum_{n<\omega}\frac{1}{10^{m(n)}}$, except $x$ is an irrational
%   number composed of $1$s separated by strings of $0$s of strictly
%   increasing size.
% \end{proof}

% \begin{ex}
%   Let $\omega_1$ be given a ladder topology:
%     \begin{itemize}
%       \item All successor ordinals are isolated.
%       \item Strictly increasing sequences (ladders) $L_\alpha:\omega\to\alpha$
%             are defined for each limit ordinal $\alpha$ such that $L_\alpha$ converges to $\alpha$ in the order topology, and each limit
%             $\alpha$ is given neighborhoods of the form
%             $L(\alpha,m)=\{\alpha\}\cup\{L_\alpha(n):n\geq m\}$.
%       \item $\omega_1=\bigcup_{\alpha\in\omega_1^L} L(\alpha,0)$
%     \end{itemize}

%   Let
%     \[
%       A(\alpha,n)
%         =
%       [L(\alpha,0)\setminus L(\alpha,n)]^1
%         \cup
%       \{\oneptcomp\omega_1 \setminus (L(\alpha,0)\setminus L(\alpha,n))\}
%     \]
%     \[
%       B(\alpha)
%         =
%       \{L(\alpha,0), \oneptcomp\omega_1 \setminus L(\alpha,0)\}
%     \]

%   Finite refinements of $A(\alpha,n)$ and $B(\alpha)$ give partitions
%   witnessing a uniformization of the ladder topology.

%   Then $\proxgame{\oneptcomp\omega_1}$ is indetermined.
% \end{ex}

% (TODO: finish proof)