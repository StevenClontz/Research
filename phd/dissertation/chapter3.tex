%!TEX root = dissertation.tex
% ^ leave for LaTeXTools build functionality

\chapter{$W$ convergence and clustering games}

We begin by investigating a game due to Gary Gruenhage.

\begin{game}
  Let $\congame{X}{S}$ denote the \term{$W$-convergence game} with
  players $\pl O$, $\pl P$, for a topological space $X$ and $S\subseteq X$.

  In round $n$, $\pl O$ chooses an open neighborhood $O_n\supseteq S$, followed
  by $\pl P$ choosing a point $x_n\in \bigcap_{m\leq n}O_m$.

  $\pl O$ wins the game if the points $x_n$ converge to the set $S$; that is,
  for every open neighborhood $U\supseteq S$, $x_n\in U$ for
  all but finite $n<\omega$.

  If $S=\{x\}$ then we write $\congamehard{X}{x}$ for short.
\end{game}

(TODO: Any reason for the ``W''?)

Gruenhage defined this game in his doctoral dissertation to define a class
of spaces generalizing first-countability. \cite{MR0413049}

\begin{defn}
  The spaces $X$ for which $\pl O \win \congamehard{X}{x}$ for all $x\in X$ are
  called \term{$W$-spaces}.
\end{defn}

In fact, using limited information strategies, one may characterize the
first-countable spaces using this game.

\begin{prop}
  $X$ is first countable
    if and only if
  $\pl O \prewin \congamehard{X}{x}$ for all $x\in X$.
\end{prop}

\begin{proof}
  The forward implication shows that all $W$ spaces are first-countable spaces,
  and was proven in \cite{MR0413049}: if $\{U_n:n<\omega\}$ is a countable
  base at $x$, let $\sigma(n)=\bigcap_{m\leq n} U_m$. $\sigma$ is easily seen
  to be a winning predetermined strategy.

  If $X$ is not first countable at some $x$, let $\sigma$ be a
  predetermined strategy for $\pl O$ in $\congamehard{X}{x}$. There exists
  an open neighborhood $U$ such that $U$ is not a subset of any $\sigma(n)$
  (otherwise $\{\sigma(n):n<\omega\}$ would be a countable base at $x$).
  Let $x_n$ be an element of $\sigma(n)\setminus U$ for all $n<\omega$.
  Then $\<x_0,x_1,\dots\>$ is a winning counter-attack to $\sigma$ for $\pl P$,
  so $\pl O$ lacks a winning predetermined strategy.
\end{proof}

At first glance, the difficulty of $\congame{X}{S}$ could be increased for
$\pl O$ by only restricting the choices for $\pl P$ to be within the most
recent open set played by $\pl O$, rather than all the previously played
open sets.

\begin{defn}
  Let $\congame{X}{S}$ denote the \term{hard $W$-convergence game} which proceeds
  as $\congamehard{X}{S}$, except that $\pl P$ need only play within $O_n$
  rather than $\bigcap_{m\leq n}O_n$.
\end{defn}

Of course, this is normally easily circumvented.

\begin{prop}
  $\pl O \limitwin \congame{X}{S}$
    if and only if
  $\pl O \limitwin \congamehard{X}{S}$,
  where $\limitwin$ is either $\win$ or $\prewin$.
\end{prop}

\begin{proof}
  The backwards implication is immediate.

  For the forward implication, let $\sigma$ be a winning predetermined
  (perfect information) strategy, and $\lambda$ be $\mu_0$ (the identity).

  We define a new predetermined (perfect information) strategy $\tau$ by
    \[
      \tau\circ\lambda(\<x_0,\dots,x_{n-1}\>)
        =
      \bigcap_{m\leq n}\sigma\circ\lambda(\<x_0,\dots,x_{m-1}\>)
    \]
  so that each move by $\pl O$ according to $\tau$ is the intersection of
  $\pl O$'s previous moves. Then any attack against $\tau$ is an attack
  against $\sigma$, and since $\sigma$ is a winning strategy, so is $\tau$.
\end{proof}

It's important to note that $\tau$ is only well-defined in the case
$\lambda=\mu_0$ because the value of
$\sigma\circ\lambda(\<x_0,\dots,x_{m-1}\>)$ does not rely on knowledge of the
points $\<x_0,\dots,x_{m-1}\>$ for any $m\leq n$.
The proof would be invalid if $\lambda$ was required to be, say,
$\nu_{k+1}$, since the value of
$\sigma\circ\nu_{k+1}(\<x_0,\dots,x_k\>)=\sigma(\<x_0,\dots,x_k\>)$
could not be uniquely determined from
$\nu_{k+1}(\<x_0,\dots,x_{k+1}\>)=\<x_1,\dots,x_{k+1}\>$.

Due to the equivalency of the ``hard'' and ``normal'' variations of the
convergence game in the perfect information case, most authors use them
interchangibly. However, it is easy to find spaces for which the games are
not equivalent when considering $k+1$-tactics and $k+1$-marks.

In addition to the $W$-convergence games, we will also investigate
``clustering'' analogs to both variations.

\begin{game}
  Let $\clusgame{X}{S}$ (resp. $\clusgamehard{X}{S}$) be a variation of
  $\congame{X}{S}$ (resp. $\congamehard{X}{S}$) such that $x_n$ need only
  cluster at $S$, that is, for every open neighborhood $U$ of $S$, $x_n\in U$
  for infinitely many $n<\omega$.
\end{game}

Gruenhage noted that the clustering game is
perfect-information equivalent to the convergence game for $\pl O$. This
can easily be extended for some limited information cases as well.

\begin{prop}
  $\pl O \limitwin \congame{X}{S}$
    if and only if
  $\pl O \limitwin \clusgame{X}{S}$
  where $\limitwin$ is any of $\win$, $\prewin$, $\tactwin$, or $\markwin$.
\end{prop}

\begin{proof}
  For the perfect information case we refer to \cite{MR0413049}.

  In the predetermined (resp. tactical) case, suppose that $\sigma$ is a
  winning predetermined (resp. tactical) strategy for $\pl O$ in
  $\clusgame{X}{S}$. Let $p$ be a legal attack against $\sigma$, and $q$ be a
  subsequence of $p$. It's easily seen that $q$ is also a legal attack against
  $\sigma$, so $q$ clusters at $S$. Since every subsequence of $p$ clusters
  at $S$, $p$ converges to $S$, and $\sigma$ is a winning predetermined
  (resp. tactical) strategy for $\pl O$ in $\congame{X}{S}$ as well.

  In the final case, note that any Mark\"ov strategy $\sigma$ for $\pl O$ may be
  strengthened by setting $\sigma'(x,n)=\bigcap_{m\leq n}\sigma(x,m)$.
  So, suppose that $\sigma$ is a winning Mark\"ov strategy for $\pl O$ in
  $\clusgame{X}{S}$ such that $\sigma(x,m)\supseteq\sigma(x,n)$ for all
  $m\leq n$.

  Let $p$ be a legal attack against $\sigma$, and $q$ be a subsequence of $p$.
  For $m<\omega$, there exists $f(m)\leq m$ such that $q(m)=p(f(m))$. It follows
  that
    \[
      q(n+1)
        \in
      \bigcap_{m\leq n}\sigma(q(m),m+1)
        =
      \bigcap_{m\leq n}\sigma(p(f(m)),m+1)
    \]
    \[
        \subseteq
      \bigcap_{m\leq n}\sigma(p(f(m)),f(m)+1)
        \subseteq
      \bigcap_{m\leq f(n)}\sigma(p(m),m+1)
    \]
  so $q$ is also a legal attack against $\sigma$. Since $\sigma$ is a winning
  strategy, $q$ clusters at $S$, and since every subsequence of $p$ clusters
  at $S$, $p$ must converge to $S$. Thus $\sigma$ is also a winning Mark\"ov
  strategy for $\pl O$ in $\congame{X}{S}$ as well.
\end{proof}

(TODO: Maybe $k+2$ tacts/marks as well, but not as obvious if so.)



\section{Fort spaces}

Gruenhage suggested the one-point-compactification of a discrete space as
an example of a $W$-space which is not first-countable.

\begin{defn}
  A \term{Fort space} $\oneptcomp\kappa=\kappa\cup\{\infty\}$ is defined
  for each cardinal $\kappa$. Its subspace $\kappa$ is discrete, and the
  neighborhoods of $\infty$ are of the form $\oneptcomp\kappa\setminus F$
  for each $F\in[\kappa]^{<\omega}$.
\end{defn}

\begin{thm}
$O\not\ktactwin{k}\clusgame{\oneptcomp{\omega_1}}{\infty}$.
\end{thm}

\begin{thm}
$O\markwin \clusgame{\oneptcomp{\omega_1}}{\infty}$.
\end{thm}

\begin{thm}(Nyikos)
$O\not\markwin \congame{\oneptcomp{\omega_1}}{\infty}$.
\end{thm}

\begin{thm}
$O\not\kmarkwin{k}\clusgame{\oneptcomp{\kappa}}{\infty}$ for $\kappa>\omega_1$.
\end{thm}

(TODO: It's feasible that $k$-limit $\Leftrightarrow$ $1$-limit.)

\section{Sigma-products}

\begin{thm}
  Let $\textrm{cf}([\kappa]^{\leq\omega})=\kappa$.
  Then $F \codewin PF_{F,C}(\kappa)$.
\end{thm}

\begin{thm}
  Let $\kappa$ be the limit of cardinals $\kappa_n$ such that
  $\textrm{cf}([\kappa_n]^{\leq\omega},\subseteq)=\kappa_n$.
  Then $F \codewin \pfgame{\kappa}$.
\end{thm}

\begin{thm}
  $F\codewin\pfgame{\kappa}$ for all cardinals $\kappa$.
\end{thm}

\begin{cor}
  $O\codewin\congame{\sigmaprodr{\kappa}}{\vec{0}}$
  for all cardinals $\kappa$.
\end{cor}
