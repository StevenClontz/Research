%!TEX root = dissertation.tex
% ^ leave for LaTeXTools build functionality

\chapter{$W$ convergence and clustering games}

We begin by investigating a game due to Gary Gruenhage.

\begin{game}
  Let $\congame{X}{S}$ denote the \term{$W$-convergence game} with
  players $\pl O$, $\pl P$, for a topological space $X$ and $S\subseteq X$.

  In round $n$, $\pl O$ chooses an open neighborhood $O_n\supseteq S$, followed
  by $\pl P$ choosing a point $x_n\in \bigcap_{m\leq n}O_m$.

  $\pl O$ wins the game if the points $x_n$ converge to the set $S$; that is,
  for every open neighborhood $U\supseteq S$, $x_n\in U$ for
  all but finite $n<\omega$.

  If $S=\{x\}$ then we write $\congame{X}{x}$ for short.
\end{game}

(TODO: Any reason for the ``W''?)

Gruenhage defined this game in his doctoral dissertation to define a class
of spaces generalizing first-countability. \cite{MR0413049}

\begin{defn}
  The spaces $X$ for which $\pl O \win \congamehard{X}{x}$ for all $x\in X$ are
  called \term{$W$-spaces}.
\end{defn}

In fact, using limited information strategies, one may characterize the
first-countable spaces using this game.

\begin{prop}
  $X$ is first countable
    if and only if
  $\pl O \prewin \congamehard{X}{x}$ for all $x\in X$.
\end{prop}

\begin{proof}
  The forward implication shows that all $W$ spaces are first-countable spaces,
  and was proven in \cite{MR0413049}: if $\{U_n:n<\omega\}$ is a countable
  base at $x$, let $\sigma(n)=\bigcap_{m\leq n} U_m$. $\sigma$ is easily seen
  to be a winning predetermined strategy.

  If $X$ is not first countable at some $x$, let $\sigma$ be a
  predetermined strategy for $\pl O$ in $\congamehard{X}{x}$. There exists
  an open neighborhood $U$ of $x$ which does not contain any $\sigma(n)$
  (otherwise $\{\sigma(n):n<\omega\}$ would be a countable base at $x$).
  Let $x_n$ be an element of $\sigma(n)\setminus U$ for all $n<\omega$.
  Then $\<x_0,x_1,\dots\>$ is a winning counter-attack to $\sigma$ for $\pl P$,
  so $\pl O$ lacks a winning predetermined strategy.
\end{proof}

At first glance, the difficulty of $\congame{X}{S}$ could be increased for
$\pl O$ by only restricting the choices for $\pl P$ to be within the most
recent open set played by $\pl O$, rather than all the previously played
open sets.

\begin{defn}
  Let $\congamehard{X}{S}$ denote the \term{hard $W$-convergence game} which
  proceeds as $\congame{X}{S}$, except that $\pl P$ need only choose
  $x_n\in O_n$ rather than $x_n\in \bigcap_{m\leq n}O_m$ during each round.
\end{defn}

This seemingly more difficult game for $\pl O$ is Gruenhage's original
formulation. But with perfect information, there is no real difference for
$\pl O$.

\begin{prop}\label{propNormalHardCon}
  $\pl O \limitwin \congame{X}{S}$
    if and only if
  $\pl O \limitwin \congamehard{X}{S}$,
  where $\limitwin$ is either $\win$ or $\prewin$.
\end{prop}

\begin{proof}
  The backwards implication is immediate.

  For the forward implication, let $\sigma$ be a winning predetermined
  (perfect information) strategy, and $\lambda$ be the $0$-Mark\"ov
  fog-of-war $\mu_0$ (the identity).

  We define a new predetermined (perfect information) strategy $\tau$ by
    \[
      \tau\circ\lambda(\<x_0,\dots,x_{n-1}\>)
        =
      \bigcap_{m\leq n}\sigma\circ\lambda(\<x_0,\dots,x_{m-1}\>)
    \]
  so that each move by $\pl O$ according to $\tau\circ\lambda$ is the
  intersection of $\pl O$'s previous moves. Then any attack against
  $\tau\circ\lambda$ is an attack against $\sigma\circ\lambda$, and since
  $\sigma\circ\lambda$ is a winning strategy, so is $\tau\circ\lambda$.
\end{proof}

Put another way, $\tau(n)=\bigcap_{m\leq n}\sigma(m)$ in the predetermined
case, and
$\tau(\<x_0,\dots,x_{n-1}\>)=\bigcap_{m\leq n}\sigma(\<x_0,\dots,x_{m-1}\>)$
in the perfect information case.
The original proof would have been invalid if $\lambda$ was required to be, say,
the tactical fog-of-war $\nu_1$, since the value of $\pl O$'s own second move
$\sigma\circ\nu_1(\<x_0\>)=\sigma(\<x_0\>)$
could not be determined from the information she has during round
$2$: $\nu_1(\<x_0,x_1\>)=\<x_1\>$.

Due to the equivalency of the ``hard'' and ``normal'' variations of the
convergence game in the perfect information case, many authors use them
interchangibly. However, it is possible to find spaces for which the games are
not equivalent when considering $k+1$-tactics and $k+1$-marks, as we will
soon see.

In addition to the $W$-convergence games, we will also investigate
``clustering'' analogs to both variations.

\begin{game}
  Let $\clusgame{X}{S}$ ($\clusgamehard{X}{S}$) be a variation of
  $\congame{X}{S}$ ($\congamehard{X}{S}$) such that $x_n$ need only
  cluster at $S$, that is, for every open neighborhood $U$ of $S$, $x_n\in U$
  for infinitely many $n<\omega$.
\end{game}

This variation seems to make $\pl O$'s job easier, but Gruenhage noted that
the clustering game is perfect-information equivalent to the convergence game
for $\pl O$. This can easily be extended for some limited information cases
as well.

\begin{prop}
  $\pl O \limitwin \congame{X}{S}$
    if and only if
  $\pl O \limitwin \clusgame{X}{S}$
  where $\limitwin$ is any of $\win$, $\prewin$, $\tactwin$, or $\markwin$.
\end{prop}

\begin{proof}
  For the perfect information case we refer to \cite{MR0413049}.

  In the predetermined (resp. tactical) case, suppose that $\sigma$ is a
  winning predetermined (resp. tactical) strategy for $\pl O$ in
  $\clusgame{X}{S}$. Let $p$ be a legal attack against $\sigma$, and $q$ be a
  subsequence of $p$. It's easily seen that $q$ is also a legal attack against
  $\sigma$, so $q$ clusters at $S$. Since every subsequence of $p$ clusters
  at $S$, $p$ converges to $S$, and $\sigma$ is a winning predetermined
  (resp. tactical) strategy for $\pl O$ in $\congame{X}{S}$ as well.

  In the final case, note that any Mark\"ov strategy $\sigma'$ for $\pl O$ may
  be strengthened to $\sigma$ defined by
  $\sigma(x,n)=\bigcap_{m\leq n}\sigma'(x,m)$.
  So, suppose that $\sigma$ is a winning Mark\"ov strategy for $\pl O$ in
  $\clusgame{X}{S}$ such that $\sigma(x,m)\supseteq\sigma(x,n)$ for all
  $m\leq n$.

  Let $p$ be a legal attack against $\sigma$, and $q$ be a subsequence of $p$.
  For $m<\omega$, there exists $f(m)\geq m$ such that $q(m)=p(f(m))$. It follows
  that
    $
      q(0)=p(f(0))
        \in
      \sigma(\emptyset,0)\cap \bigcap_{m\leq f(0)}\sigma(\<p(m)\>,m)
        \subseteq
      \sigma(\emptyset,0)
    $
  and
    \[
      \begin{array}{rcl}
        q(n+1)=p(f(n+1)) &
          \in &
        \displaystyle
        \sigma(\emptyset,0)\cap \bigcap_{m<f(n+1)}\sigma(\<p(m)\>,m+1) \\ &
          \subseteq &
        \displaystyle
        \sigma(\emptyset,0)\cap \bigcap_{m<n+1}\sigma(\<p(f(m))\>,f(m)+1) \\ &
          = &
        \displaystyle
        \sigma(\emptyset,0)\cap \bigcap_{m<n+1}\sigma(\<q(m)\>,f(m)+1) \\ &
          \subseteq &
        \displaystyle
        \sigma(\emptyset,0)\cap \bigcap_{m<n+1}\sigma(\<q(m)\>,m+1)
      \end{array}
    \]
  so $q$ is also a legal attack against $\sigma$. Since $\sigma$ is a winning
  strategy, $q$ clusters at $S$, and since every subsequence of $p$ clusters
  at $S$, $p$ must converge to $S$. Thus $\sigma$ is also a winning Mark\"ov
  strategy for $\pl O$ in $\congame{X}{S}$ as well.
\end{proof}

(TODO: Maybe $k+2$ tacts/marks as well, but not as obvious if so.)

(TODO: It's feasible that $k$-limit $\Leftrightarrow$ $1$-limit for
one variation or another.)



\section{Fort spaces}

In his original paper, Gruenhage suggested the one-point-compactification of a
discrete space as an example of a $W$-space which is not first-countable.

\begin{defn}
  A \term{Fort space} $\oneptcomp\kappa=\kappa\cup\{\infty\}$ is defined
  for each cardinal $\kappa$. Its subspace $\kappa$ is discrete, and the
  neighborhoods of $\infty$ are of the form $\oneptcomp\kappa\setminus F$
  for each $F\in[\kappa]^{<\omega}$.
\end{defn}

\begin{prop}\label{propFortTact}
  $\pl O \tactwin \congame{\oneptcomp\kappa}{\infty}$ for all cardinals $\kappa$
\end{prop}

\begin{proof}
  Let $\sigma(\emptyset)=\sigma(\<\infty\>)=\oneptcomp\kappa$ and
  $\sigma(\<\alpha\>)=\oneptcomp\kappa\setminus\{\alpha\}$. Any legal attack
  against the tactic $\sigma$ could not repeat non-$\infty$ points, so
  it must converge to $\infty$.
\end{proof}

\begin{cor}
  $\pl O \win \congamehard{\oneptcomp\kappa}{\infty}$ for all cardinals $\kappa$
\end{cor}

\begin{proof}
  Propositions \ref{propNormalHardCon} and \ref{propFortTact}.
\end{proof}

Since it's trivial to show that $\pl O\prewin\congame{\oneptcomp\kappa}$ if and
only if $\kappa\leq\omega$, this closes the question on limited information
strategies for $\congame{\oneptcomp\kappa}{\infty}$. However, limited
information analysis of the harder $\congamehard{\oneptcomp\kappa}{\infty}$
is more interesting.

Peter Nyikos noted Proposition \ref{propFortTact} and the following in
\cite{MR1031771}.

\begin{thm}
  $\pl O \not\markwin \congamehard{\oneptcomp\omega_1}{\infty}$.
\end{thm}

This actually can be generalized to any $k$-Mark\"ov strategy with just
a little more bookkeeping.

\begin{thm}
  $\pl O \not\kmarkwin{k} \congamehard{\oneptcomp\omega_1}{\infty}$.
\end{thm}

\begin{proof}
  Let $\sigma$ be a $k$-mark for $\pl O$. Since the set
    \[
      D_\sigma
        =
      \bigcap_{n<\omega,s\in \omega^{\leq k}}
      \sigma(s,n)
    \]
  is co-countable, we may choose $\alpha_\sigma\in D_\sigma\cap\omega_1$.
  Thus, we may choose $n_0<n_1<\dots<\omega$ such that
    \[
      \<n_0,\dots,n_{k-1},\alpha_\sigma,
        n_k,\dots,n_{2k-1},\alpha_\sigma,
        \dots\>
    \]
  is a legal counterattack, which fails to converge to $\infty$ since
  $\alpha_\sigma$ is repeated infinitely often.
\end{proof}

However, while the clustering and convergence variants are equivalent
for Mark\"ov strategies in the ``normal'' version of the $W$ game, they are
\textit{not} equivalent in the ``hard'' version.

\begin{thm}
  $\pl O\markwin \clusgamehard{\oneptcomp\omega_1}{\infty}$.
\end{thm}

\begin{proof}
  For each $\alpha<\omega_1$ let $A_\alpha=\<A_\alpha(0),A_\alpha(1),\dots\>$
  be a countable sequence of finite sets such that
  $A_\alpha(n)\subset A_\alpha(n+1)$ and
  $\bigcup_{n<\omega}A_\alpha(n)=\alpha+1$.

  We define the Mark\"ov strategy $\sigma$ by setting
    \[
      \sigma(\emptyset,0) = \sigma(\<\infty\>,n) = \oneptcomp\omega_1
    \]
  and for all $\alpha<\omega_1$ setting
    \[
      \sigma(\<\alpha\>,n) = \oneptcomp\omega_1 \setminus A_\alpha(n)
    \]

  Note that for any $\alpha_0<\dots<\alpha_{k-1}$, there is some $n<\omega$
  such that
  $\{\alpha_0,\dots,\alpha_{k-1}\}\subseteq A_{\alpha_i}(n)$ for all $i<k$.
  Thus for any legal attack $p$ against $\sigma$, the range of $p$ cannot
  be finite. Since the range of $p$ is infinite, every open neighborhood of
  $\infty$ contains infinitely many points of $p$, so $p$ clusters at $\infty$.
\end{proof}

However, knowledge of the round number is critical.

\begin{thm}
  $O\not\ktactwin{k}\clusgamehard{\oneptcomp\omega_1}{\infty}$.
\end{thm}

\begin{proof}
  Let $\sigma$ be a $k$-tactic for $\pl O$ in
  $\clusgame{\oneptcomp\omega_1}{\infty}$. By the closing-up lemma, the set
    \[
      C_\sigma
        =
      \{
        \alpha<\omega_1
          :
        s\in\alpha^{\leq k}
          \Rightarrow
        \oneptcomp\omega_1\setminus\sigma(s)
        \subset \alpha
      \}
    \]
  is closed and unbounded. Let $a_\sigma:\omega_1\to C_\sigma$ be an order
  isomorphism.

  Choose $n_0<\dots<n_{k-1}<\omega$ such that for each $i<k$:
    \[
      a_\sigma(n_i)
        \in
      \sigma(
        \<a_\sigma(n_0),\dots,a_\sigma(n_{i-1}),
          a_\sigma(\omega+i),\dots,a_\sigma(\omega+k-1)\>
      )
    \]

  Finally, observe that the legal counterattack
    \[
      \<a_\sigma(n_0),\dots,a_\sigma(n_{k-1}),
        a_\sigma(\omega),\dots,a_\sigma(\omega+k-1),
        a_\sigma(n_0),\dots,a_\sigma(n_{k-1}),
        a_\sigma(\omega),\dots,a_\sigma(\omega+k-1),
        \dots\>
    \]
  has a range outside the open neighborhood
    \[
      \oneptcomp\omega_1
        \setminus
      \{a_\sigma(n_0),\dots,a_\sigma(n_{k-1}),
        a_\sigma(\omega),\dots,a_\sigma(\omega+k-1)\}
    \]
  of $\infty$. Thus $\sigma$ is not a winning $k$-tactic.
\end{proof}

Once the discrete space is larger than $\omega_1$, knowing the round number
is not sufficient to construct a limited information strategy, due to a
similar argument.

\begin{thm}
  $O\not\kmarkwin{k}\clusgame{\oneptcomp\omega_2}{\infty}$.
\end{thm}

\begin{proof}
  Let $\sigma$ be a $k$-mark for $\pl O$ in
  $\clusgame{\oneptcomp\omega_2}{\infty}$. By the closing-up lemma, the set
    \[
      C_\sigma
        =
      \{
        \alpha<\omega_2
          :
        s\in\alpha^{<\omega}
          \Rightarrow
        \oneptcomp\omega_2\setminus\sigma\circ\mu_k(s)
        \subset \alpha
      \}
    \]
  is closed and unbounded. Let $a_\sigma:\omega_2\to C_\sigma$ be
  an order isomorphism.

  Choose $\beta_0<\dots<\beta_{k-1}<\omega_1$ such that for each $i<k$:
    \[
      a_\sigma(\beta_i)
        \in
      \bigcup_{n<\omega}
      \sigma(
        \<a_\sigma(\beta_0),\dots,a_\sigma(\beta_{i-1}),
          a_\sigma(\omega_1+i),\dots,a_\sigma(\omega_1+k-1)\>,
      n)
    \]

  Finally, observe that the legal counterattack
    \[
      \<a_\sigma(\beta_0),\dots,a_\sigma(\beta_{k-1}),
        a_\sigma(\omega_1),\dots,a_\sigma(\omega_1+k-1),
        a_\sigma(\beta_0),\dots,a_\sigma(\beta_{k-1}),
        a_\sigma(\omega_1),\dots,a_\sigma(\omega_1+k-1),
        \dots\>
    \]
  has a range outside the open neighborhood
    \[
      \oneptcomp\omega_2
        \setminus
      \{a_\sigma(\beta_0),\dots,a_\sigma(\beta_{k-1}),
        a_\sigma(\omega_1),\dots,a_\sigma(\omega_1+k-1)\}
    \]
  of $\infty$. Thus $\sigma$ is not a winning $k$-mark.
\end{proof}




\section{TODO: Sigma-products}

\begin{thm}
  Let $\textrm{cf}([\kappa]^{\leq\omega})=\kappa$.
  Then $F \codewin PF_{F,C}(\kappa)$.
\end{thm}

\begin{thm}
  Let $\kappa$ be the limit of cardinals $\kappa_n$ such that
  $\textrm{cf}([\kappa_n]^{\leq\omega},\subseteq)=\kappa_n$.
  Then $F \codewin \pfgame{\kappa}$.
\end{thm}

\begin{thm}
  $F\codewin\pfgame{\kappa}$ for all cardinals $\kappa$.
\end{thm}

\begin{cor}
  $O\codewin\congame{\sigmaprodr{\kappa}}{\vec{0}}$
  for all cardinals $\kappa$.
\end{cor}
