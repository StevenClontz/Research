%!TEX root = dissertation.tex
% ^ leave for LaTeXTools build functionality

\chapter{Convergence/clustering games}

Results related to Gruenhage's ``W''-convergence game and variants.

\section{Basic game}

\begin{definition} 
Gruenhage's open-point convergence game $\congame{X}{x}$ has $O$ choosing nested open sets and $P$ choosing a point within the last chosen open set by $O$. $O$ wins if the points chosen by $P$ converge to $x$.
\end{definition}

TODO: This version of the game (where $\pl P$ need not play within all 
previously chosen open sets) is more difficult for $\pl O$. Need to 
differientiate these two versions somehow...

\begin{definition}
The one-point compactification of a space $X$ is $\oneptcomp{X}$, where neighborhoods of points in $X$ are the same as they were originally, and neighborhoods of $\infty$ are sets $\oneptcomp{X}\setminus K$ for compact $K$. If $X$ is discrete then neighborhoods of $\infty$ are cofinite sets containing $\infty$, and the game is equivalent to $O$ choosing finite ``forbidden'' sets and $P$ choosing points not forbidden by $O$.
\end{definition}

\begin{proposition}
$O\codewin \congame{\oneptcomp{\kappa}}{\infty}$ for all cardinals $\kappa$.
\end{proposition}

\begin{proof}
Use $F(N,p)=N\cup\{p\}$.
\end{proof}

\begin{proposition}
$O\prewin \congame{\oneptcomp{\omega}}{\infty}$.
\end{proposition}

\begin{proof}
Use $F(n)=n$.
\end{proof}


\begin{proposition}
$O\not\prewin \clusgame{\oneptcomp{\kappa}}{\infty}$ for $\kappa\geq\omega_1$.
\end{proposition}

\begin{proof}
Let $F(n)$ be $O$'s predetermined forbidding strategy, let $\alpha\in\kappa\setminus\bigcup_{n<\omega}F(n)$, and have $P$ counter with $\left<\alpha,\alpha,\dots\right>$.
\end{proof}


\begin{proposition}
$O\tactwin\congame{\oneptcomp{\omega}}{\infty}$.
\end{proposition}

\begin{proof}
Use $F(n)=n+1$.
\end{proof}

\begin{theorem}
If $\kappa$ is a regular uncountable cardinal, for every function $f:[\kappa]^{<\omega}\to[\kappa]^{<\omega}$ the set $C_f = \{\alpha < \kappa : S\in[\alpha]^{<\omega} \Rightarrow f(S)\in[\alpha]^{<\omega}\}$ is club.
\end{theorem}

\begin{proof}
First assume $\alpha_0<\alpha_1<\dots\in C_f$. It is easily seen that $\sup(\alpha_n)\in C_f$, showing $C_f$ is closed.

Now assume $\gamma_0\in C_f$. Let $\gamma_{n+1}>\gamma_n$ be the least ordinal such that if $S\in[\gamma_n+1]^{<\omega}$ then $f(S)\in[\gamma_{n+1}]^{<\omega}$. We claim $\gamma_\omega = \sup(\gamma_n)\in C_f$. Let $S\in[\gamma_\omega]^{<\omega}$. Then $S\in[\gamma_n+1]^{<\omega}$ for some $n$, and thus $f(S)\in[\gamma_{n+1}]^{<\omega}\subset[\gamma_\omega]^{<\omega}$. Therefore $C_f$ is unbounded.
\end{proof}

We may thus assume, for the purposes of countering a tactical or Markov strategy, that the strategy is \textbf{downward} on some regular uncountable cardinal.

\begin{theorem}
$O\not\ktactwin{k}\clusgame{\oneptcomp{\kappa}}{\infty}$ for $\kappa\geq\omega_1$.
\end{theorem}

\begin{proof}
Let $F:[\kappa]^{\leq k} \to [\kappa]^{<\omega}$ be a forbidding strategy by $O$ against $P$ which is downward on $\omega_1$. We define $n_i$ for $0\leq i < k$ to be a natural number such that \[ n_i \in \omega \setminus(F(n_0,\dots,n_{i-1}) \cup F(n_0,\dots,n_{i-1},\omega+i,\dots,\omega+k-1))\] and note that\[\left<n_0,n_1,\dots,n_{k-1},\omega,\omega+1,\dots,\omega+k-1,n_0,n_1,\dots,n_{k-1},\omega,\omega+1,\dots,\omega+k-1,\dots\right>\] counters $F$.
\end{proof}

\begin{theorem}
$O\markwin \clusgame{\oneptcomp{\omega_1}}{\infty}$.
\end{theorem}

\begin{proof}
For $\alpha<\omega_1$ let $A_{\alpha,n}$ be a sequence of finite sets such that $A_{\alpha,n}\subset A_{\alpha,n+1}$ and $\bigcup_{n<\omega}A_{\alpha,n}=\alpha+1$.

Give $O$ the Markov forbidding strategy $F(n,\alpha)=A_{\alpha,n}$. To observe that any legal play by $P$ against the strategy $F$ has infinite range, we observe that for any $\alpha_0<\dots<\alpha_{k-1}$, there is some round $n$ such that $\{\alpha_0,\dots,\alpha_{k-1}\}\subseteq \bigcup_{0\leq i < k}F(n,\alpha_i)$, and thus $P$ cannot legally play any of these ordinals until another ordinal is played.
\end{proof}

Peter J. Nyikos has shown the following:

\begin{theorem}
$O\not\markwin Con_{O,P}(\omega_1\cup\{\infty\},\infty)$.
\end{theorem}

which improves to:

\begin{theorem}
$O\not\kmarkwin{k}Clus_{O,P}(\kappa\cup\{\infty\},\infty)$ for $\kappa>\omega_1$.
\end{theorem}

\begin{proof}
Let $F:\omega \times [\kappa]^{\leq k} \to [\kappa]^{<\omega}$ be a forbidding strategy by $O$ against $P$ which is downward on $\omega_2$. We define $\alpha_i$ for $0\leq i < k$ to be a countable ordinal such that \[ \alpha_i \in \omega_1 \setminus\bigcup_{n<\omega}\left(F(n,\{\alpha_0,\dots,\alpha_{i-1}\}) \cup F(n,\{\alpha_0,\dots,\alpha_{i-1},\omega_1+i,\dots,\omega_1+k-1\})\right)\] and note that \[\left<\alpha_0,\alpha_1,\dots,\alpha_{k-1},\omega_1,\omega_1+1,\dots,\omega_1+k-1,\alpha_0,\alpha_1,\dots,\alpha_{k-1},\omega_1,\omega_1+1,\dots,\omega_1+k-1,\dots\right>\] counters $F$.
\end{proof}

\section{$Con_{O,P}(X,x)$ for Sigma Product}

\begin{proposition}
$O\uparrow\congame{\sigmaprodr{\kappa}}{\vec{0}}$.
\end{proposition}

\begin{proof}
For $s\in\sigmaprodr{\kappa}$ let $C(s)=\{\alpha<\kappa:s(\alpha)\not=0\}$ denote the countable nonzero coordinates of $s$. Let $\Phi:[\kappa]^{\leq\omega}\times\omega\to[\kappa]^{<\omega}$ be such that $\Phi(C,n)\subseteq\Phi(C,n+1)$ and $\bigcup_{n<\omega}\Phi(C,n)=C$.

If $\tau$ is the usual topology on $\mathbb{R}$, let $\sigma_\alpha:(\sigmaprodr{\kappa})^{<\omega}\to\tau$ be such that
\[
U_\alpha(s_0,\dots,s_{n-1})=\left\{
\begin{array}{l}
(-\frac{1}{n},\frac{1}{n}) \text{ if } \alpha\in\bigcup_{i<n}\Phi(C(s_i),n) \\
\mathbb{R} \text{ otherwise}
\end{array}
\right.
\]

Finally, give $O$ the winning strategy $\sigma(s_0,\dots,s_{n-1})=\sigmaprodr{\kappa}\cap\prod_{\alpha<\kappa}U_\alpha(s_0,\dots,s_{n-1})$.
\end{proof}

\begin{theorem}
For all cardinals $\kappa\leq 2^\omega$, $O\codewin\congame{\sigmaprodr{\kappa}}{\vec{0}}$.
\end{theorem}

\begin{proof}
Note that $|\Sigma\mathbb{R}^\kappa| \leq 2^\omega = |\mathbb{R}|$. Define the following:

    \begin{itemize}
    \item Encode every $S \in (\Sigma\mathbb{R}^{\kappa})^{<\omega}$ as a real number $0<r(S)<1$. 
    \item Let $\gamma(U)$ be the function which, for basic open sets $U=\sigmaprodr{\kappa}\cap\prod_{\alpha<\kappa}U_\alpha$ where for all $\alpha<\kappa$ either $U_\alpha=\mathbb{R}$ or $(-\frac{1}{r},\frac{1}{r})$, returns $\lfloor r \rfloor$.
    \item Let $n(U)$ be the number of non-$\mathbb{R}$ components of a basic open set $U$.
    \item Let $\psi(U,s)=r^{-1}(\gamma(U))^\frown\left<y\right>$.
    \item For $s\in\sigmaprodr{\kappa}$ let $C(s)=\{\alpha<\kappa:s(\alpha)\not=0\}$ denote the countable nonzero coordinates of $s$.
    \item Let $\Phi:[\kappa]^{\leq\omega}\times\omega\to[\kappa]^{<\omega}$ be such that $\Phi(C,n)\subseteq\Phi(C,n+1)$ and $\bigcup_{n<\omega}\Phi(C,n)=C$.
    \item For each $\alpha<\kappa$, define the interval $\sigma_\alpha(U,s)$ about $0$ as follows:
        \begin{itemize}
        \item If $\alpha\leq n(U)$ or $\alpha\in\bigcup_{s\in\psi(U,s)}\Phi(C(s),n(U))$ then $\sigma_\alpha(U,s)=(-\frac{1}{n(U)+r(\psi(U,s))},-\frac{1}{n(U)+r(\psi(U,s))})$.
        \item Otherwise, $\sigma_\alpha(U,s)=\mathbb{R}$.
        \end{itemize}
    \end{itemize}

It follows that $\sigma(U,s)=\sigmaprodr{\kappa}\cap\prod_{\alpha<\kappa} \sigma_\alpha(U,s)$ is a winning coding strategy.
\end{proof}

\begin{theorem}
Let $\kappa$ be a cardinal such that there exists a function $f:\kappa\to[\kappa]^{\leq \omega}$ where for every $W\in[\kappa]^{\leq\omega}$ there exists $\alpha_W<\kappa$ with $W\subseteq f(\alpha_W)$. (That is, $\textrm{cf}([\kappa]^{\leq\omega})=\kappa$.) Then $F \codewin PF_{F,C}(\kappa)$ and $O \codewin \congame{\sigmaprodr{\kappa}}{\vec{0}}$.
\end{theorem}

\begin{proof}
Let $W\restriction n \in [\kappa]^n$ be a subset of $W\in[\kappa]^\omega$ such that $W\restriction n \subset W \restriction (n+1)$ and $\bigcup_{n<\omega} W\restriction n = W$.

Define \[\sigma(N,W) = N \cup (|N|+1) \cup \{\alpha_W\} \cup \bigcup_{\alpha \in N} f(\alpha) \restriction |N|\]

Consider the play $\left<\emptyset,W_0,N_1,W_1,N_2,W_2,\dots\right>$ with $F$ following the strategy $\sigma$. Let $\gamma \in W_i$, and note $\gamma \in f(\alpha_{W_i})$ (and $\gamma \in f(\alpha_{W_i})\restriction |N_n|$ for sufficiently large $n$). \[N_{i+1} = \sigma(N_i,W_i) \supseteq \{\alpha_{W_i}\}\] and thus \[N_{n+1} = \sigma(N_n,W_n) \supseteq \bigcup_{\alpha \in N_n} f(\alpha) \restriction |N_n| \supseteq \bigcup_{\alpha \in N_{i+1}} f(\alpha) \restriction |N_n| \supseteq f(\alpha_{W_i}) \restriction |N_n|\] showing $\gamma \in N_{n+1}$. Since $\gamma$ is forbidden in round $n+1$, $\gamma$ appears in finitely many sets chosen by $C$.

We turn our attention to $Con_{O,P}(\Sigma\mathbb{R}^\kappa)$. We define the winning strategy $\tau(U,p)$ for $O$ as follows: let $N(U)$ be the non-$\mathbb{R}$ coordinates in the basic open set $U$ and $W(p)$ be the non-$0$ coordinates in $p$. Then $\tau(U,p) = \left(\prod_{\alpha<\kappa} U_\alpha\right) \cap \Sigma\mathbb{R}^\kappa$ where if $\alpha \in \sigma(N(U),W(p))$ then $U_\alpha = (-\frac{1}{|N(U)|},\frac{1}{|N(U)|})$ and $U_\alpha=\mathbb{R}$ otherwise.

Consider the play $\left<\emptyset,p_0,U_1,p_1,U_2,p_2,\dots\right>$ with $O$ following the strategy $\tau$. Observe that $N(\tau(U,p))=\sigma(N(U),W(p))$. Thus $p_i(\gamma)\not=0$ is equivalent to $\gamma \in W(p_i)$, and by the above argument, for sufficiently large $n$, $\gamma \in \sigma(N(U_n),W(p_n))$. Therefore from round $n$ onward the $\gamma$-coordinates of points chosen by $P$ must lay in $(-\frac{1}{|N(U)|},\frac{1}{|N(U)|})$ and converge to $0$.
\end{proof}

\begin{theorem}
Let $\kappa$ be the limit of cardinals $\kappa_n$ such that $\textrm{cf}([\kappa_n]^{\leq\omega},\subseteq)=\kappa_n$. Then $F \codewin PF_{F,C}(\kappa)$ and $O \codewin \congame{\sigmaprodr{\kappa}}{\vec{0}}$.
\end{theorem}

\begin{proof}
Let $f_n:\kappa_n\to[\kappa_n]^{\leq\omega}$ be such that for every $W\in[\kappa_n]^{\leq\omega}$ there exists $\alpha_{W,n}<\kappa_n$ such that $f_n(\alpha_{W,n})\supseteq W$.

Define \[\sigma(N,W)=N\cup(|N|+1)\cup\{\alpha_{W\cap \kappa_{|N|},|N|}\}\cup\bigcup_{n\leq|N|}\bigcup_{\alpha\in N} f_n(\alpha)\restriction |N|\] We claim that $\sigma$ is a winning coding strategy.

Consider the play $\left<N_0,W_0,N_1,W_1,\dots\right>$ where $O$ follows the strategy $\sigma$. For $\sigma$ to be a winning strategy for $F \codewin PF_{F,C}(\kappa)$, it must follow that for each $\gamma\in\bigcup_{i<\omega}W_i$, $\gamma$ is forbidden by some $\sigma(N_j,W_j)$. 

Let $\gamma\in W_i\cap \kappa_{|N_i|}$. For all $j>i$, $\alpha_{W\cap\kappa_{N_i},|N_i|}\in N_j$. Also, $\gamma \in f_{N_i}(\alpha_{W\cap\kappa_{N_i},|N_i|})\restriction|N_j|$ for some sufficiently large $j$. So we observe that $\gamma \in \bigcup_{n\leq|N_j|}\bigcup_{\alpha\in N_j} f_n(\alpha)\restriction |N_j|\subseteq \sigma(N_j,W_j)$.

We turn our attention to $\congame{\sigmaprodr{\kappa}}{\vec{0}}$. We define the winning strategy $\tau(U,p)$ for $O$ as follows: let $N(U)$ be the non-$\mathbb{R}$ coordinates in the basic open set $U$ and $W(p)$ be the non-$0$ coordinates in $p$. Then $\tau(U,p) =\Sigma\mathbb{R}^\kappa \cap \prod_{\alpha<\kappa} U_\alpha$ where if $\alpha \in \sigma(N(U),W(p))$ then $U_\alpha = (-\frac{1}{|N(U)|},\frac{1}{|N(U)|})$ and $U_\alpha=\mathbb{R}$ otherwise.

Consider the play $\left<\sigmaprodr{\kappa},p_0,U_1,p_1,U_2,p_2,\dots\right>$ with $O$ following the strategy $\tau$. Observe that $N(\tau(U,p))=\sigma(N(U),W(p))$. Thus $p_i(\gamma)\not=0$ is equivalent to $\gamma \in W(p_i)$, and by the above argument, for sufficiently large $n$, $\gamma \in \sigma(N(U_n),W(p_n))$. Therefore from round $n$ onward the $\gamma$-coordinates of points chosen by $P$ must lay in $(-\frac{1}{|N(U)|},\frac{1}{|N(U)|})$ and converge to $0$.
\end{proof}

\begin{theorem}
$F\codewin\pfgame{\kappa}$ for all cardinals $\kappa$.
\end{theorem}

\begin{proof}
Let $\kappa$ be the limit of cardinals $\kappa_n$ such that $F\codewin\pfgame{\kappa_n}$ using the strategy $\sigma_n(N,W)$ such that for $M\subseteq N$, $\sigma_n(M,W)\subseteq\sigma_n(N,W)$. Define \[\sigma(N,W)=(|N|+1)\cup\bigcup_{n<|N|}\sigma_n(N\cap\kappa_n,W\cap\kappa_n)\]

Let $\left<N_0,W_0,N_1,W_1,\dots\right>$ be a legal play of the game with $N_{i+1}=\sigma(N_i,W_i)$. Suppose $\gamma\in W_i$ for infinitely-many $i$. $\gamma\in\kappa_n$ for some $n$, so observe the play $\left<M_0,W_0\cap\kappa_n,M_1,W_1\cap\kappa_n,\dots\right>$ with $M_0=N_0\cap\kappa_n$ and $M_{i+1}=\sigma_n(M_i,W_i\cap\kappa_n)\subseteq\sigma_n(N_i\cap\kappa_n,W_i\cap\kappa_n)$ which is $\subseteq\sigma(N_i,W_i)=N_{i+1}$ for sufficiently large $i$.

Since $\sigma_n$ is a winning strategy, $\gamma\in M_{m+1}\subseteq N_{m+1}$ for sufficiently large $i$, making $\left<N_0,W_0,N_1,W_1,\dots\right>$ illegal, contradiction.

Now suppose $F\codewin\pfgame{\kappa}$. For each $\alpha<\kappa^+$, let $\sigma_\alpha(N,W)$ be a winning coding strategy for $\pfgame{\alpha}$ such that for $M\subseteq N$, $\sigma_\alpha(M,W)\subseteq\sigma_\alpha(N,W)$. We define the following strategy for $F$ in $\pfgame{\kappa^+}$: \[\sigma(N,W)=N\cup\bigcup_{\alpha\in N}\sigma_{\alpha+1}(N\cap(\alpha+1),W\cap(\alpha+1))\]

Let $\left<N_0,W_0,N_1,W_1,\dots\right>$ be a legal play of the game with $N_{i+1}=\sigma(N_i,W_i)$. Suppose $\gamma\in W_i$ for infinitely-many $i$. Observe the play $\left<M_0,W_0\cap(\gamma+1),M_1,W_1\cap(\gamma+1),\dots\right>$ with $M_0=N_0\cap(\gamma+1)$ and $M_{i+1}=\sigma_{\gamma+1}(M_i,W_i\cap(\gamma+1))\subseteq\sigma_{\gamma+1}(N_i\cap(\gamma+1),W_i\cap(\gamma+1))$.

Since $\sigma_{\gamma+1}$ is a winning strategy, $\gamma\in M_{m+1}\subseteq\sigma_{\gamma+1}(N_i\cap(\gamma+1),W_i\cap(\gamma+1))\subseteq\sigma(N_i,W_i)=N_{i+1}$ for some sufficiently large $m$, making $\left<N_0,W_0,N_1,W_1,\dots\right>$ illegal, contradiction.
\end{proof}

\begin{corollary}
$O\codewin\congame{\sigmaprodr{\kappa}}{\vec{0}}$ for all cardinals $\kappa$.
\end{corollary}

\begin{proof}
Let $\tau(N,W)$ be the winning coding strategy for $F$ in $\pfgame{\kappa}$, $N(U)\in[\kappa]^{<\omega}$ represent the non-$\mathbb{R}$ coordinates of a basic open set $U$ of $\sigmaprodr{\kappa}$, and $W(p)\in[\kappa]^{\leq\omega}$ represent the non-$0$ coordinates of a point $p$ in $\sigmaprodr{\kappa}$. For each $\alpha<\kappa$, let \[\sigma_\alpha(U,p)=\left\{\begin{array}{ll}(-\frac{1}{|N(U)|},\frac{1}{|N(U)|})&\text{if }\alpha\in\tau(N(U),W(p))\\\mathbb{R}&\text{otherwise}\end{array}\right.\] and $\sigma(U,p)=\sigmaprodr{\kappa}\cap\prod_{\alpha<\kappa}\sigma_\alpha(U,p)$.
\end{proof}