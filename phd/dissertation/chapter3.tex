%!TEX root = dissertation.tex
% ^ leave for LaTeXTools build functionality

\chapter{Gruenhage's Convergence and Clustering Games}

We begin by investigating a game due to Gary Gruenhage.

\begin{game}
  Let $\gruConGame{X}{S}$ denote the \term{$W$-convergence game} with
  players $\pl O$, $\pl P$, for a topological space $X$ and $S\subseteq X$.

  In round $n$, $\pl O$ chooses an open neighborhood $O_n\supseteq S$, followed
  by $\pl P$ choosing a point $x_n\in \bigcap_{m\leq n}O_m$.

  $\pl O$ wins the game if the points $x_n$ converge to the set $S$; that is,
  for every open neighborhood $U\supseteq S$, $x_n\in U$ for
  all but finite $n<\omega$.

  If $S=\{x\}$ then we write $\gruConGame{X}{x}$ for short.
\end{game}

The ``W'' in the name merely refers to $\pl O$'s goal: to ``win'' the game.
Gruenhage defined this game in his doctoral dissertation to define a class
of spaces generalizing first-countability. \cite{MR0413049}

\begin{defn}
  The spaces $X$ for which $\pl O \win \gruConGame{X}{x}$ for all $x\in X$ are
  called \term{$W$-spaces}.
\end{defn}

In fact, using limited information strategies, one may characterize the
first-countable spaces using this game.

\begin{prop}
  $X$ is first countable
    if and only if
  $\pl O \prewin \gruConGame{X}{x}$ for all $x\in X$.
\end{prop}

\begin{proof}
  The forward implication shows that all first-countable spaces are $W$ spaces,
  and was proven in \cite{MR0413049}: if $\{U_n:n<\omega\}$ is a countable
  base at $x$, let $\sigma(n)=\bigcap_{m\leq n} U_m$. $\sigma$ is easily seen
  to be a winning predetermined strategy.

  If $X$ is not first countable at some $x$, let $\sigma$ be a
  predetermined strategy for $\pl O$ in $\gruConGame{X}{x}$. There exists
  an open neighborhood $U$ of $x$ which does not contain any
  $\bigcap_{m\leq n}\sigma(m)$ (otherwise
  $\{\bigcap_{m\leq n}\sigma(m):n<\omega\}$ would be a countable base at $x$).
  Let $x_n$ be an element of $\bigcap_{m\leq n}\sigma(m)\setminus U$ for all
  $n<\omega$. Then $\<x_0,x_1,\dots\>$ is a winning counter-attack to $\sigma$
  for $\pl P$, so $\pl O$ lacks a winning predetermined strategy.
\end{proof}

At first glance, the difficulty of $\gruConGame{X}{S}$ could be increased for
$\pl O$ by only restricting the choices for $\pl P$ to be within the most
recent open set played by $\pl O$, rather than all the previously played
open sets.

\begin{defn}
  Let $\gruConGameHard{X}{S}$ denote the \term{hard $W$-convergence game} which
  proceeds as $\gruConGame{X}{S}$, except that $\pl P$ need only choose
  $x_n\in O_n$ rather than $x_n\in \bigcap_{m\leq n}O_m$ during each round.
\end{defn}

This seemingly more difficult game for $\pl O$ is Gruenhage's original
formulation. But with perfect information, there is no real difference for
$\pl O$.

\begin{prop}\label{propNormalHardCon}
  $\pl O \limitwin \gruConGame{X}{S}$
    if and only if
  $\pl O \limitwin \gruConGameHard{X}{S}$,
  where $\limitwin$ is either $\win$ or $\prewin$.
\end{prop}

\begin{proof}
  The backwards implication is immediate.

  For the forward implication, let $\sigma$ be a winning predetermined
  (perfect information) strategy, and $\lambda$ be the $0$-Markov
  fog-of-war $\mu_0$ (the identity).

  We define a new predetermined (perfect information) strategy $\tau$ by
    \[
      \tau\circ\lambda(\<x_0,\dots,x_{n-1}\>)
        =
      \bigcap_{m\leq n}\sigma\circ\lambda(\<x_0,\dots,x_{m-1}\>)
    \]
  so that each move by $\pl O$ according to $\tau\circ\lambda$ is the
  intersection of $\pl O$'s previous moves. Then any attack against
  $\tau\circ\lambda$ is an attack against $\sigma\circ\lambda$, and since
  $\sigma\circ\lambda$ is a winning strategy, so is $\tau\circ\lambda$.
\end{proof}

Put more simply, $\tau(n)=\bigcap_{m\leq n}\sigma(m)$ in the predetermined
case, and
$\tau(\<x_0,\dots,x_{n-1}\>)=\bigcap_{m\leq n}\sigma(\<x_0,\dots,x_{m-1}\>)$
in the perfect information case.
The original proof would have been invalid if $\lambda$ was required to be, say,
the tactical fog-of-war $\nu_1$, since the value of $\pl O$'s own round $1$ move
$\sigma\circ\nu_1(\<x_0\>)=\sigma(\<x_0\>)$
could not be determined from the information she has during round
$2$: $\nu_1(\<x_0,x_1\>)=\<x_1\>$.

Due to the equivalency of the ``hard'' and ``normal'' variations of the
convergence game in the perfect information case, many authors use them
interchangibly. However, it is possible to find spaces for which the games are
not equivalent when considering $k+1$-tactics and $k+1$-marks, as we will
soon see.

In addition to the $W$-convergence games, we will also investigate
``clustering'' analogs to both variations.

\begin{game}
  Let $\gruClusGame{X}{S}$ ($\gruClusGameHard{X}{S}$) be a variation of
  $\gruConGame{X}{S}$ ($\gruConGameHard{X}{S}$) such that $x_n$ need only
  cluster at $S$, that is, for every open neighborhood $U$ of $S$, $x_n\in U$
  for infinitely many $n<\omega$.
\end{game}

This variation seems to make $\pl O$'s job easier, but Gruenhage noted that
the clustering game is perfect-information equivalent to the convergence game
for $\pl O$. This can easily be extended for some limited information cases
as well.

\begin{prop}
  $\pl O \limitwin \gruConGame{X}{S}$
    if and only if
  $\pl O \limitwin \gruClusGame{X}{S}$
  where $\limitwin$ is any of $\win$, $\prewin$, $\tactwin$, or $\markwin$.
\end{prop}

\begin{proof}
  For the perfect information case we refer to \cite{MR0413049}.

  In the predetermined (resp. tactical) case, suppose that $\sigma$ is a
  winning predetermined (resp. tactical) strategy for $\pl O$ in
  $\gruClusGame{X}{S}$. Let $p$ be a legal attack against $\sigma$, and $q$ be a
  subsequence of $p$. It's easily seen that $q$ is also a legal attack against
  $\sigma$, so $q$ clusters at $S$. Since every subsequence of $p$ clusters
  at $S$, $p$ converges to $S$, and $\sigma$ is a winning predetermined
  (resp. tactical) strategy for $\pl O$ in $\gruConGame{X}{S}$ as well.

  In the final case, note that any Markov strategy $\sigma'$ for $\pl O$ may
  be strengthened to $\sigma$ defined by
  $\sigma(x,n)=\bigcap_{m\leq n}\sigma'(x,m)$.
  So, suppose that $\sigma$ is a winning Markov strategy for $\pl O$ in
  $\gruClusGame{X}{S}$ such that $\sigma(x,m)\supseteq\sigma(x,n)$ for all
  $m\leq n$.

  Let $p$ be a legal attack against $\sigma$, and $q$ be a subsequence of $p$.
  For $m<\omega$, there exists $f(m)\geq m$ such that $q(m)=p(f(m))$. It follows
  that
    $
      q(0)=p(f(0))
        \in
      \sigma(\emptyset,0)\cap \bigcap_{m\leq f(0)}\sigma(\<p(m)\>,m)
        \subseteq
      \sigma(\emptyset,0)
    $
  and
    \[
      \begin{array}{rcl}
        q(n+1)=p(f(n+1)) &
          \in &
        \displaystyle
        \sigma(\emptyset,0)\cap \bigcap_{m<f(n+1)}\sigma(\<p(m)\>,m+1) \\ &
          \subseteq &
        \displaystyle
        \sigma(\emptyset,0)\cap \bigcap_{m<n+1}\sigma(\<p(f(m))\>,f(m)+1) \\ &
          = &
        \displaystyle
        \sigma(\emptyset,0)\cap \bigcap_{m<n+1}\sigma(\<q(m)\>,f(m)+1) \\ &
          \subseteq &
        \displaystyle
        \sigma(\emptyset,0)\cap \bigcap_{m<n+1}\sigma(\<q(m)\>,m+1)
      \end{array}
    \]
  so $q$ is also a legal attack against $\sigma$. Since $\sigma$ is a winning
  strategy, $q$ clusters at $S$, and since every subsequence of $p$ clusters
  at $S$, $p$ must converge to $S$. Thus $\sigma$ is also a winning Markov
  strategy for $\pl O$ in $\gruConGame{X}{S}$ as well.
\end{proof}

\begin{prop} For any $x\in X$ and $k<\omega$,
  \begin{itemize}
    \item
      $
      \pl O\ktactwin{k+1}\gruConGame{X}{x}
        \Leftrightarrow
      \pl O\tactwin\gruConGame{X}{x}
      $
    \item
      $
      \pl O\kmarkwin{k+1}\gruConGame{X}{x}
        \Leftrightarrow
      \pl O\markwin\gruConGame{X}{x}
      $
  \end{itemize}
\end{prop}


\begin{proof}
  If $\sigma$ witnesses $\pl O\ktactwin{k+1}\gruConGame{X}{x}$,
  let $\tau(\emptyset)=\sigma(\emptyset)$ and
    \[
      \tau(\<q\>)
        =
      \bigcap_{i< k}
      \sigma(\<\underbrace{x,\dots,x}_{k-i},q,\underbrace{x,\dots,x}_{i+1}\>)
    \]

  Then $\tau$ is easily verified to be a winning tactic, and
  the proof for the second part is analogous.
\end{proof}

Two types of questions emerge from these results.

\begin{ques}
  Does $\pl O \ktactwin{2} \gruClusGame{X}{S}$ imply
  $\pl O \ktactwin{2} \gruConGame{X}{S}$? What about for $\kmarkwin{2}$?
\end{ques}

\begin{ques}
  Could $\pl O \ktactwin{k+1} \gruConGame{X}{S}$ actually imply
  $\pl O \tactwin \gruConGame{X}{S}$? What about for $\gruClusGame{X}{S}$?
\end{ques}



\section{Fort spaces}

In his original paper, Gruenhage suggested the one-point-compactification of a
discrete space as an example of a $W$-space which is not first-countable.

\begin{defn}
  A \term{Fort space} $\oneptcomp\kappa=\kappa\cup\{\infty\}$ is defined
  for each cardinal $\kappa$. Its subspace $\kappa$ is discrete, and the
  neighborhoods of $\infty$ are of the form $\oneptcomp\kappa\setminus F$
  for each $F\in[\kappa]^{<\omega}$.
\end{defn}

\begin{prop}\label{propFortTact}
  $\pl O \tactwin \gruConGame{\oneptcomp\kappa}{\infty}$ for all cardinals $\kappa$
\end{prop}

\begin{proof}
  Let $\sigma(\emptyset)=\sigma(\<\infty\>)=\oneptcomp\kappa$ and
  $\sigma(\<\alpha\>)=\oneptcomp\kappa\setminus\{\alpha\}$. Any legal attack
  against the tactic $\sigma$ could not repeat non-$\infty$ points, so
  it must converge to $\infty$.
\end{proof}

\begin{cor}
  $\pl O \win \gruConGameHard{\oneptcomp\kappa}{\infty}$ for all cardinals $\kappa$
\end{cor}

\begin{proof}
  Propositions \ref{propNormalHardCon} and \ref{propFortTact}.
\end{proof}

Since it's trivial to show that
$\pl O\prewin\gruConGame{\oneptcomp\kappa}{\infty}$ if and
only if $\kappa\leq\omega$, this closes the question on limited information
strategies for $\gruConGame{\oneptcomp\kappa}{\infty}$. However, limited
information analysis of the harder $\gruConGameHard{\oneptcomp\kappa}{\infty}$
is more interesting.

Peter Nyikos noted Proposition \ref{propFortTact} and the following in
\cite{MR1031771}.

\begin{thm}
  $\pl O \not\markwin \gruConGameHard{\oneptcomp\omega_1}{\infty}$.
\end{thm}

This actually can be generalized to any $k$-Markov strategy with just
a little more bookkeeping.

\begin{thm}
  $\pl O \not\kmarkwin{k} \gruConGameHard{\oneptcomp\omega_1}{\infty}$.
\end{thm}

\begin{proof}
  Let $\sigma$ be a $k$-mark for $\pl O$. Since the set
    \[
      D_\sigma
        =
      \bigcap_{n<\omega,s\in \omega^{\leq k}}
      \sigma(s,n)
    \]
  is co-countable, we may choose $\alpha_\sigma\in D_\sigma\cap\omega_1$.
  Thus, we may choose $n_0<n_1<\dots<\omega$ such that
    \[
      \<n_0,\dots,n_{k-1},\alpha_\sigma,
        n_k,\dots,n_{2k-1},\alpha_\sigma,
        \dots\>
    \]
  is a legal counterattack, which fails to converge to $\infty$ since
  $\alpha_\sigma$ is repeated infinitely often.
\end{proof}

However, while the clustering and convergence variants are equivalent
for Markov strategies in the ``normal'' version of the $W$ game, they are
\textit{not} equivalent in the ``hard'' version.

\begin{thm}
  $\pl O\markwin \gruClusGameHard{\oneptcomp\omega_1}{\infty}$.
\end{thm}

\begin{proof}
  For each $\alpha<\omega_1$ let $A_\alpha=\<A_\alpha(0),A_\alpha(1),\dots\>$
  be a countable sequence of finite sets such that
  $A_\alpha(n)\subset A_\alpha(n+1)$ and
  $\bigcup_{n<\omega}A_\alpha(n)=\alpha+1$.

  We define the Markov strategy $\sigma$ by setting
    \[
      \sigma(\emptyset,0) = \sigma(\<\infty\>,n) = \oneptcomp\omega_1
    \]
  and for all $\alpha<\omega_1$ setting
    \[
      \sigma(\<\alpha\>,n) = \oneptcomp\omega_1 \setminus A_\alpha(n)
    \]

  Note that for any $\alpha_0<\dots<\alpha_{k-1}$, there is some $n<\omega$
  such that
  $\{\alpha_0,\dots,\alpha_{k-1}\}\subseteq A_{\alpha_i}(n)$ for all $i<k$.
  Thus for any legal attack $p$ against $\sigma$, the range of $p$ cannot
  be finite. Since the range of $p$ is infinite, every open neighborhood of
  $\infty$ contains infinitely many points of $p$, so $p$ clusters at $\infty$.
\end{proof}

However, knowledge of the round number is critical.

\begin{thm}
  $O\not\ktactwin{k}\gruClusGameHard{\oneptcomp\omega_1}{\infty}$.
\end{thm}

\begin{proof}
  Let $\sigma$ be a $k$-tactic for $\pl O$ in
  $\gruClusGameHard{\oneptcomp\omega_1}{\infty}$. By the closing-up lemma, the set
    \[
      C_\sigma
        =
      \{
        \alpha<\omega_1
          :
        s\in\alpha^{\leq k}
          \Rightarrow
        \oneptcomp\omega_1\setminus\sigma(s)
        \subset \alpha
      \}
    \]
  is closed and unbounded. Let $a_\sigma:\omega_1\to C_\sigma$ be an order
  isomorphism.

  Choose $n_0<\dots<n_{k-1}<\omega$ such that for each $i<k$:
    \[
      a_\sigma(n_i)
        \in
      \sigma(
        \<a_\sigma(n_0),\dots,a_\sigma(n_{i-1}),
          a_\sigma(\omega+i),\dots,a_\sigma(\omega+k-1)\>
      )
    \]

  Finally, observe that the legal counterattack
    \[
      \<a_\sigma(n_0),\dots,a_\sigma(n_{k-1}),
        a_\sigma(\omega),\dots,a_\sigma(\omega+k-1),
        a_\sigma(n_0),\dots,a_\sigma(n_{k-1}),
        a_\sigma(\omega),\dots,a_\sigma(\omega+k-1),
        \dots\>
    \]
  has a range outside the open neighborhood %FIXME, say has finite range.
    \[
      \oneptcomp\omega_1
        \setminus
      \{a_\sigma(n_0),\dots,a_\sigma(n_{k-1}),
        a_\sigma(\omega),\dots,a_\sigma(\omega+k-1)\}
    \]
  of $\infty$. Thus $\sigma$ is not a winning $k$-tactic.
\end{proof}

Once the discrete space is larger than $\omega_1$, knowing the round number
is not sufficient to construct a limited information strategy, due to a
similar argument.

\begin{thm}
  $O\not\kmarkwin{k}\gruClusGameHard{\oneptcomp\omega_2}{\infty}$.
\end{thm}

\begin{proof}
  Let $\sigma$ be a $k$-mark for $\pl O$ in
  $\gruClusGame{\oneptcomp\omega_2}{\infty}$. By the closing-up lemma, the set
    \[
      C_\sigma
        =
      \{
        \alpha<\omega_2
          :
        s\in\alpha^{<\omega}
          \Rightarrow
        \oneptcomp\omega_2\setminus\sigma\circ\mu_k(s)
        \subset \alpha
      \}
    \]
  is closed and unbounded. (Recall that $\mu_k$ is the
  $k$-Markov fog-of-war which turns perfect information into the last $k$
  moves and the round number.) Let $a_\sigma:\omega_2\to C_\sigma$ be
  an order isomorphism.

  Choose $\beta_0<\dots<\beta_{k-1}<\omega_1$ such that for each $i<k$:
    \[
      a_\sigma(\beta_i)
        \in
      \bigcap_{n<\omega}
      \sigma(
        \<a_\sigma(\beta_0),\dots,a_\sigma(\beta_{i-1}),
          a_\sigma(\omega_1+i),\dots,a_\sigma(\omega_1+k-1)\>,
      n)
    \]

  Finally, observe that the legal counterattack
    \[
      \<a_\sigma(\beta_0),\dots,a_\sigma(\beta_{k-1}),
        a_\sigma(\omega_1),\dots,a_\sigma(\omega_1+k-1),
        a_\sigma(\beta_0),\dots,a_\sigma(\beta_{k-1}),
        a_\sigma(\omega_1),\dots,a_\sigma(\omega_1+k-1),
        \dots\>
    \]
  has a range outside the open neighborhood %FIXME, say has finite range.
    \[
      \oneptcomp\omega_2
        \setminus
      \{a_\sigma(\beta_0),\dots,a_\sigma(\beta_{k-1}),
        a_\sigma(\omega_1),\dots,a_\sigma(\omega_1+k-1)\}
    \]
  of $\infty$. Thus $\sigma$ is not a winning $k$-mark.
\end{proof}



\section{Sigma-products}

Knowing the status of $W$-games in simpler spaces yields insight to larger
spaces.

\begin{prop}
  Suppose $S\subseteq Y\subseteq X$, $\limitwin$ is any of $\win$,
  $\ktactwin{k}$, or $\kmarkwin{k}$, and $G(X,S)$ is any of $\gruConGame{X}{S}$,
  $\gruConGameHard{X}{S}$, $\gruClusGame{X}{S}$, or $\gruClusGameHard{X}{S}$.

  Then $\pl O \limitwin G(X,S)$ implies $\pl O \limitwin G(Y,S)$.
\end{prop}

\begin{proof}
  Simply intersect the output of the winning strategy in $G(X,S)$ with $Y$.
\end{proof}

A natural superspace of a Fort space is the sigma-product of a discrete
cardinal.

\begin{defn}
  Let $\Sigma_{y} \prod_{\alpha<\kappa}X_\alpha$ be a \term{sigma-product} of
  $X$ with dimension $\kappa$ for each $y\in \prod_{\alpha<\kappa}X_\alpha$,
  the subset of the usual Tychonoff product space $\prod_{\alpha<\kappa}X_\alpha$
  such that $x\in \Sigma_y \prod_{\alpha<\kappa}X_\alpha$
  if and only if $\{\alpha<\kappa : x(\alpha)\not=y(\alpha)\}$ is countable.

  Unless otherwise specified, we assume $0\in X_\alpha$ for all $\alpha<\kappa$
  and $y=\vec 0$, and write $\Sigma \prod_{\alpha<\kappa}X_\alpha$.
\end{defn}

\begin{prop}
  $\oneptcomp\kappa$ is homeomorphic to the space
  \[
    \{
    x\in \Sigma 2^\kappa
      :
    x(\alpha)=0 \text{ for all but one } \alpha<\kappa
    \}
  \]
\end{prop}

\begin{proof}
  Map $\alpha<\kappa$ to $x_\alpha$ such that
  \[
    x_\alpha(\beta) =
    \left\{
      \begin{array}{ll}
        0 & \beta\not=\alpha \\
        1 & \beta=\alpha
      \end{array}
    \right.
  \]
  and map $\infty$ to the zero vector $\vec0$.
\end{proof}

\begin{cor}
  $\pl O\not\ktactwin{k}\gruClusGameHard{\SigmaProdR{\omega_1}}{\vec0}$,
  $\pl O\not\kmarkwin{k}\gruConGameHard{\SigmaProdR{\omega_1}}{\vec0}$, and
  $\pl O\not\kmarkwin{k}\gruClusGameHard{\SigmaProdR{\omega_2}}{\vec0}$.
\end{cor}

While this closes the question on tactics and marks for high dimensional
sigma- (and Tychonoff-) products of the real line, there is another type of
limited information strategy to investigate.

\begin{defn}
  For a game $G=\<M,W\>$ and \term{coding strategy} or \term{code}
  $\sigma:M^2\to M$, the \term{$\sigma$-coding fog-of-war}
  $\gamma_\sigma: M^{<\omega}\to M^{\leq2}$ is the function defined such that
    \[
      \gamma_\sigma(\emptyset) = \emptyset
    \]
  and
    \[
      \gamma_\sigma(s\concat\<x\>) = \<\sigma\circ\gamma_\sigma(s),x\>
    \]

  For a coding strategy $\sigma$, its coresponding strategy is
  $\sigma\circ\gamma_\sigma$. For a game $G$, if $\sigma\circ\gamma_\sigma$
  is a winning strategy for $\pl A$, then $\sigma$ is a winning coding
  strategy and we write $\pl A \codewin G$.
\end{defn}

Intuitively, a $\sigma$-coding fog-of-war converts perfect information of the
game into the last moves of both the player and her opponent, so a player has
a winning coding strategy when she only needs to know the move of her opponent
and her own last move. The term ``coding'' comes from the fact
that a player may encode information about the history of the game into
her own moves, and use this encoded information in later rounds.

As an example, the existence of a winning coding strategy is necessary for the
second player to force a win the Banach-Mazur game.

\begin{thm}\label{bmcode}
  $\pl N\win\bmgame{X}$ if and only if $\pl N\codewin\bmgame{X}$
  \cite{MR817083} \cite{MR831181}.
\end{thm}

We are interested in whether the same holds for $W$ games.

The hard and normal versions of the $W$ games are all equivalent with regards
to coding strategies since $\pl O$
may always ensure her new move is a subset of her previous move. For
Fort spaces, the question is immediately closed.

\begin{prop}
  $\pl O \codewin \gruConGame{\oneptcomp\kappa}{\infty}$.
\end{prop}

\begin{proof}
  Let $\sigma(\emptyset)=\oneptcomp\kappa$,
  $\sigma(\<U,\alpha\>)=U\setminus\{\alpha\}$ for $\alpha<\kappa$,
  and $\sigma(\<U,\infty\>)=U$. $\pl P$
  cannot legally repeat non-$\infty$ points of the set, so her points converge
  to $\infty$.
\end{proof}

This trick does not simply extend to the $\SigmaProdR\kappa$ case, however.
An open set may only restrict finitely many coordinates of the product,
and a point in $\SigmaProdR\kappa$ may have countably infinite non-zero
coordinates. Thus, information about the previous non-zero coordinates cannot be
directly encoded into the open set.

Circumventing this takes a bit of extra machinery. We proceed by defining
a simpler infinite game for each cardinal $\kappa$.

\begin{game}
  Let $\cloPFGame\kappa$ denote the \term{point-finite game} with players
  $\pl F$, $\pl C$ for each cardinal $\kappa$.

  In round $n$, $\pl F$ chooses $F_n\in[\kappa]^{<\omega}$, followed by
  $\pl C$ choosing $C_n\in[\kappa\setminus\bigcup_{m\leq n}F_m]^{\leq\omega}$.

  $\pl F$ wins the game if the collection $\{C_n:n<\omega\}$ is a point-finite
  cover of its union $\bigcup_{n<\omega} C_n$, that is, each point in
  $\bigcup_{n<\omega} C_n$ is in $C_n$ only for finitely many $n<\omega$.
\end{game}

This game has a strong resemblance to a game defined by Scheepers in
\cite{MR1129143} in relationship to the Banach-Mazur game and studied
specifically with finite and countable sets in \cite{MR1183703}. Scheeper's
game and the results pertaining to it aren't of use here; however, they will be
referenced in a later chapter in studying a different topological game.

This game of finite and countable sets is directly applicable to the $W$
games played upon the sigma-product of real lines.

\begin{lem}
  $\pl F \codewin \cloPFGame\kappa$ implies
  $\pl O \codewin \gruConGame{\SigmaProdR\kappa}{\vec0}$.
\end{lem}

\begin{proof}
  Let $\sigma$ be a winning coding strategy for $\pl F$ in $\cloPFGame\kappa$
  such that $\sigma(\emptyset)\supset\emptyset$ and $\sigma(F,C)\supset F$.

  For $F\in[\kappa]^{<\omega}$ and $\epsilon>0$ let $U(F,\epsilon)$ be the
  basic open set in $\mathbb{R}^\kappa$ such that each projection is of the form
  \[
    \pi_\alpha(U(F,\epsilon)) =
    \left\{
      \begin{array}{ll}
        (-\epsilon,\epsilon) & \alpha\in F\\
        \mathbb{R} & \alpha\not\in F
      \end{array}
    \right.
  \]
  Note that $F\supset\emptyset$ and $\epsilon$ are uniquely identifible given
  $U(F,\epsilon)\cap\SigmaProdR\kappa$.

  For each point $x\in\SigmaProdR\kappa$ and $\epsilon> 0$, let
  $C_\epsilon(x)\in[\kappa]^{\leq\omega}$
  such that $\alpha\in C_\epsilon(\alpha)$ if and only if
  $|x(\alpha)|\geq\epsilon$.

  We define the coding strategy $\tau$ for $\pl O$ in
  $\gruConGame{\SigmaProdR\kappa}{\vec0}$ as follows:
  \[
    \tau(\emptyset)
      =
    U(\sigma(\emptyset),1) \cap \SigmaProdR\kappa
  \]
  \[
    \tau(\<U(F,\epsilon)\cap\SigmaProdR\kappa,x\>)
      =
    U\left(\sigma(\<F,C_{\epsilon}(x)\>),\frac{\epsilon}{2}\right)
      \cap
    \SigmaProdR\kappa
  \]

  Let $\<a_0,a_1,a_2,\dots\>$ be a legal attack by $\pl P$ against $\tau$. It
  then follows that
    \[
      b=\<C_{1}(a_0),C_{1/2}(a_1),C_{1/4}(a_2),\dots\>
    \]
  is a
  legal attack by $\pl C$ against $\sigma$. Since $\sigma$ is a winning
  strategy, each ordinal in $\bigcup_{n<\omega} C_{2^{-n}}(a_n)$ is in
  $C_{2^{-n}}(a_n)$ only for finitely many $n<\omega$. Thus for every coordinate
  $\alpha<\kappa$ it follows that there exists some $n_\alpha<\omega$ such that
  $a_{n}(\alpha)\leq 2^{-n}$ for $n\geq n_\alpha$. We conclude $a_n\to\vec0$,
  showing that $\tau$ is a winning strategy.
\end{proof}

This lemma simplifies our notation in proving the main result. Intuitively, we
aim to show that when $\kappa$ has cofinality $\omega$, $\pl F$ can split up
the game among $\omega$-many smaller cardinals converging to $\kappa$, and
when $\kappa$ has a larger cofinality, $\pl F$ may exploit the fact that $\pl C$
may only play within some ordinal smaller than $\kappa$.

\begin{thm}
  $\pl F \codewin \cloPFGame{\kappa}$ for all cardinals $\kappa$.
\end{thm}

\begin{proof}
  For each cardinal $\kappa$ and $\lambda<\kappa$,
  assume $\sigma_\lambda$ is a winning strategy for $\pl F$ in $\cloPFGame\lambda$
  such that $\sigma_\lambda(\emptyset)\supset\emptyset$ and
  $\sigma_\lambda(\<F,C\>)\supset F$.

  In the case that $\cf\kappa=\omega$, let $\<\kappa_0,\kappa_1,\dots\>$ be
  an increasing sequence of cardinals limiting to $\kappa$. Then we define
  the coding strategy $\sigma$ for $\pl F$ as follows:
    \[
      \sigma(\emptyset) = \sigma_{\kappa_0}(\emptyset)
    \]
    \[
      \sigma(\<F,C\>)
        =
      \bigcup_{n\leq|F|}
      \sigma_{\kappa_n}(\<F\cap \kappa_n,C\cap \kappa_n\>)
    \]

  Then for each legal attack $a=\<a(0),a(1),\dots\>$ by $\pl C$ against $\sigma$
  and each $n<\omega$, the sequence
  $b_n=\<a(n)\cap \kappa_n,a(n+1)\cap \kappa_n,\dots\>$
  is a legal attack by $\pl C$ against the winning coding strategy
  $\sigma_{\kappa_n}$. It follows then that $\{a(i+n)\cap \kappa_n:i<\omega\}$
  is a point-finite cover of $\bigcup_{i<\omega}a(i+n)\cap \kappa_n$. We
  conclude that $\{a(i):i<\omega\}$ is a point-finite cover of
  $\bigcup_{i<\omega}a(i)$ and $\sigma$ is a winning strategy.

  It remains to consider the case where $\cf\kappa>\omega$. Note that now, for
  each $C\in[\kappa]^{\leq\omega}$, $C$ is bounded above in $\kappa$. So we
  define the coding strategy $\sigma$ for $\pl F$ as follows:
    \[
      \sigma(\emptyset) = \emptyset
    \]
    \[
      \sigma(\<F,C\>)
        =
      \{\sup(C)\}
        \cup
      \bigcup_{\alpha\in F}
      \sigma_{\alpha+1}(\<F\cap(\alpha+1),C\cap(\alpha+1)\>)
    \]

  Then for each legal attack $a=\<a(0),a(1),\dots\>$ by $\pl C$ against $\sigma$
  and each $n<\omega$, the sequence
  $b_n=\<a(n)\cap (\sup(a(n))+1),a(n+1)\cap (\sup(a(n))+1),\dots\>$
  is a legal attack by $\pl C$ against the winning coding strategy
  $\sigma_{\sup(a(n))+1}$. It follows then that
  $\{a(i+n)\cap (\sup(a(n))+1):i<\omega\}$
  is a point-finite cover of $\bigcup_{i<\omega}a(i+n)\cap(\sup(a(n))+1)$. We
  conclude that $\{a(i):i<\omega\}$ is a point-finite cover of
  $\bigcup_{i<\omega}a(i)$ and $\sigma$ is a winning strategy.
\end{proof}

\begin{cor}
  $\pl O\codewin\gruConGame{\SigmaProdR{\kappa}}{\vec{0}}$
  for all cardinals $\kappa$.
\end{cor}

This leaves open the question analogous to Theorem \ref{bmcode}.

\begin{ques}
  Does $\pl O \win\gruConGame{X}{x}$ imply $\pl O \codewin\gruConGame{X}{x}$?
\end{ques}
