%!TEX root = dissertation.tex
% ^ leave for LaTeXTools build functionality

\chapter{The Menger Game}

In 1924 Karl Menger introduced a covering property generalizing
$\sigma$-compactness \cite{custom31879423}.

\begin{defn}
  A space $X$ is Menger if for every sequence of open covers of $X$
  $\<\mc U_0,\mc U_1,\dots\>$ there exists a sequence
  $\<\mc F_0,\mc F_1,\dots\>$ such that $\mc F_n\subseteq \mc U_n$,
  $|\mc F_n|<\omega$, and $\bigcup_{n<\omega}\mc F_n$ is a cover of $X$.
\end{defn}

\begin{prop}
  $X$ is $\sigma$-compact
    $\Rightarrow$
  $X$ is Menger
    $\Rightarrow$
  $X$ is Lindel\"of.
\end{prop}

None of these implications may be reversed; the irrationals are a simple example
of a Lindel\"of space which is not Menger, and we'll see several examples of
Menger spaces which are not $\sigma$-compact.

It can be shown via a non-trivial
proof that the following game can be used to characterize the Menger property.

\begin{game}
  Let $\mengame{X}$ denote the \term{Menger game} with players $\pl C$, $\pl F$.
  In round $n$, $\pl C$ chooses an open cover $\mc C_n$, followed by $\pl F$
  choosing a finite subcollection $\mc F_n\subseteq \mc C_n$.

  $\pl F$ wins the game if $\bigcup_{n<\omega}\mc F_n$ is a cover for the space
  $X$, and $\pl C$ wins otherwise.
\end{game}

\begin{thm}
  A space $X$ is Menger if and only if $\pl C\not\win\mengame{X}$
  \cite{MR1544773}.
\end{thm}

\section{Mark\"ov strategies}

To the author's knowledge, no other direct work has been done on limited
information strategies pertaining to the Menger game, although as we'll see
there are results which can be sharpened when considering them.
However, we immediately see that tactics are not of any real interest.

\begin{prop}
  $X$ is compact if and only if
  $\pl F \tactwin \mengame{X}$ if and only if
  $\pl F \ktactwin{(k+1)} \mengame{X}$.
\end{prop}

\begin{proof}
  For any open cover $\mc U$, attack the winning $k$-tactic with
  $\<\mc U,\mc U,\dots\>$ to produce a finite open subcover.
\end{proof}

Essentially, because $\pl C$ may repeat the same finite sequence of open covers,
$\pl F$ needs to be seeded with knowledge of the round number to prevent being
trapped in a loop.

If $\pl F$'s memory of $\pl C$'s past moves is bounded, then
there is no need to consider more than the two most recent moves. The
intuitive reason is that $\pl C$ could simply play the same cover repeatedly
until $\pl F$'s memory is exhausted, in which case $\pl F$ would only ever
see the change from one cover to another.

\begin{thm}
  $F \kmarkwin{2} \mengame{X}$ if and only if $F \kmarkwin{(k+2)} \mengame{X}$.
\end{thm}

\begin{proof}
  Let $\sigma$ be a winning $(k+2)$-mark. We define the $2$-mark $\tau$ as
  follows:
    \[
      \tau(\<\mc U\>,0)
        =
      \bigcup_{m<k+2}
        \sigma(\<\underbrace{\mc U,\dots,\mc U}_{m+1}\>,m)
    \]
    \[
      \tau(\<\mc U,\mc V\>,n+1)
        =
      \bigcup_{m<k+2}
        \sigma(\<
          \underbrace{\mc U,\dots,\mc U}_{k+1-m},
          \underbrace{\mc V,\dots,\mc V}_{m+1}
        \>,(n+1)(k+2)+m)
    \]

  Let $\<\mc U_0,\mc U_1,\dots\>$ be an attack by $\pl C$ against $\tau$.
  Then consider the attack
    \[
      \<
        \underbrace{\mc U_0,\dots,\mc U_0}_{k+2},
        \underbrace{\mc U_1,\dots,\mc U_1}_{k+2},
        \dots
      \>
    \]
  by $\pl C$ against $\sigma$. Since $\sigma$ is a winning $(k+2)$-mark, the
  collection
    \[
      \bigcup_{m<k+2}
        \sigma(\<\underbrace{\mc U_0,\dots,\mc U_0}_{m+1}\>,m)
      \cup
      \bigcup_{n<\omega}
      \bigcup_{m<k+2}
        \sigma(\<
          \underbrace{\mc U_n,\dots,\mc U_n}_{k+1-m},
          \underbrace{\mc U_{n+1},\dots,\mc U_{n+1}}_{m+1}
        \>,(n+1)(k+2)+m)
    \]
    \[
      =
      \tau(\<\mc U_0\>,0)
      \cup
      \bigcup_{n<\omega}
      \tau(\<\mc U_n,\mc U_{n+1}\>,n+1)
    \]
  covers the space, and $\tau$ is a winning $2$-mark.
\end{proof}

A natural question arises: is there an example of a space $X$ for which
$\pl F\kmarkwin{2}\mengame{X}$ but $\pl F\not\markwin\mengame{X}$? Fortunately,
perhaps the simplest example of a Lindel\"of non-$\sigma$-compact
space has this property.

\begin{defn}
  For any cardinal $\kappa$, let $\oneptlind\kappa=\kappa\cup\{\infty\}$ denote
  the \term{one-point Lindel\"of-ication} of discrete $\kappa$, where points in
  $\kappa$ are isolated, and the neighborhoods of $\infty$ are the co-countable
  sets containing it.
\end{defn}

We require a techincal lemma.

\begin{lem}
  For each function
  $\tau:\omega_1\times\omega \rightarrow [\omega_1]^{<\omega}$,
  there exists a sequence $\alpha\in\omega_1^\omega$
  such that $\{\tau(\alpha(n),n): n<\omega\}$ is not a cover for
  $\bigcap_{n<\omega}\alpha(n)$.
\end{lem}

\begin{proof}
  Suggested by the author's major professor: let
    \[
      P_n =
      \{
        \beta : \forall \beta <\alpha < \omega_1 ( \beta \in \tau(\alpha, n))
      \}
    \]
  Observe that since
  $P_n \subseteq \tau(\sup(P_n)+1, n)$, each $P_n$ is finite.

  Choose $\beta \in\omega_1\setminus \bigcup_{n<\omega} P_n$. Then for each
  $n<\omega$, pick $\alpha(n)>\beta$ such that
  $\beta \not\in \tau(\alpha(n),n)$.
\end{proof}

\begin{thm}
  $F\not\markwin\mengame{\oneptlind{\omega_1}}$.
\end{thm}

\begin{proof}
  Let $\sigma$ be a Mark\"ov strategy for $\pl F$. For each $\alpha<\omega_1$,
  let $\mc U_\alpha$ be the open cover
  $\{\{\beta\}:\beta<\alpha\}\cup\{\oneptlind{\omega_1}\setminus\alpha\}$ of
  $\oneptlind{\omega_1}$. Then define
  $\tau:\omega_1\times\omega \rightarrow [\omega_1]^{<\omega}$ so that
  $\tau(\alpha,n)=\bigcup(\sigma(\<\mc U_\alpha\>,n)\cap[\omega_1]^1)$.

  Use the preceding lemma to generate $\alpha\in\omega_1^\omega$, and attack
  $\sigma$ with $\<\mc U_{\alpha(0)}, \mc U_{\alpha(1)},\dots\>$.
  Since $\{\tau(\alpha(n),n): n<\omega\}$ is not a cover for
  $\bigcap_{n<\omega}\alpha(n)$, it follows that
  $\bigcup_{n<\omega}\sigma(\<\mc U_{\alpha(n)}\>,n)$ is not a cover for
  $\oneptlind{\omega_1}$.
\end{proof}

The greatest advantage of a strategy which has knowledge of two or more previous
moves of the opponent, versus only knowledge of the most recent move, is the
ability to react to changes from one round to the next. It's this ability to
react that will give $\pl F$ her winning $2$-Mark\"ov strategy in the Menger
game on $\oneptlind\omega_1$.

\begin{defn}
  For two functions $f,g$ we say $f$ is \textbf{$\mu$-almost compatible} with
  $g$ ($f\alcomp_\mu g$) if $|\{x\in\dom(f)\cap\dom(g):f(x)\not=g(x)\}|<\mu$.
  If $\mu=\omega$ then we say $f,g$ are \textbf{almost compatible}
  ($f\alcomp g$).
\end{defn}

For inspiration, we turn to a game whose $n$-tactics were studied by Marion
Scheepers in \cite{MR1129143} which has similar goals to the Menger game when
played upon $\oneptlind\kappa$.

\begin{game}
  Let $\ksfillgame\kappa$ denote the \term{strict countable/finite filling game}
  with two players $\pl C$, $\pl F$. In round $0$, $\pl C$ chooses
  $C_0\in[\kappa]^{\leq\omega}$, followed by $\pl F$ choosing
  $F_0\in[\kappa]^{<\omega}$. In round $n+1$, $\pl C$ chooses
  $C_{n+1}\in[\kappa]^{\leq\omega}$ such that $C_{n+1}\supset C_n$, followed
  by $\pl F$ choosing $F_{n+1}\in[\kappa]^{<\omega}$.

  $\pl F$ wins the game if
  $\bigcup_{n<\omega} F_n\supseteq\bigcup_{n<\omega} C_n$; otherwise, $\pl C$
  wins.
\end{game}

Essentially, in $\mengame{\oneptlind\kappa}$ $\pl C$ chooses a countable set
to not include in her neighborhood of $\infty$, followed by $\pl F$ choosing
finitely many of them to cover each round. The games are not quite equivalent,
however. In the Menger game, $\pl F$ only has to cover the \textit{intersection}
of the countable sets chosen by $\pl C$ rather than the union, since she may
always choose to cover the neighborhood of $\infty$ chosen by $\pl C$ each
round.

In addition, Scheepers required that $\pl C$ always choose strictly growing
countable sets. The reasoning is clear: if the goal is to study tactics, then
$\pl C$ cannot be allowed to trap $\pl F$ in a loop by repeating the same moves.
But by eliminating this requirement, the study can then turn to Mark\"ov
srategies, bringing the game further in line with the Menger game played upon
$\oneptlind\kappa$.

\begin{game}
  Let $\kfillgame\kappa$ denote the
  \term{(nonstrict) countable/finite filling game} which proceeds analogously
  to $\ksfillgame\kappa$, except that $\pl C$ need only ensure that
  $C_{n+1}\supseteq C_n$.
\end{game}

In studying these filling games, Scheepers introduced the statement
$\alcompS\kappa$ relating to the almost-compatability of functions
from countable subsets of $\kappa$ into $\omega$.

\begin{defn}
  $\alcompS\kappa$ states that there exist injective functions
  $f_A:A\to\lambda$ for each $A\in[\kappa]^\omega$ such that $f_A\alcomp f_B$
  for all $A,B\in[\kappa]^\omega$.
  \footnote{
    In \cite{MR1129143}, $\alcompS\kappa$ is restricted to the case that
    $A\subset B$, but from $f_A\alcomp f_{A\cap B}$ and
    $f_B\alcomp f_{A\cap B}$ it follows then that $f_A\alcomp f_B$ for any
    $A,B$.
  }
\end{defn}

Scheepers went on to show that $\alcompS\kappa$ implies
$\pl F \ktactwin2 \ksfillgame\kappa$. It can shown similarly that
$\alcompS\kappa$ implies $\pl F \kmarkwin2 \kfillgame\kappa$, either directly
or by using the winning $2$-tactic in $\ksfillgame\kappa$ to construct a
winning $2$-mark in $\kfillgame\kappa$. These proofs, along with the following
fact, give us inspiration for
finding a winning $2$-Mark\"ov strategy in the Menger game.

\begin{thm}
  $\alcompS{\omega_1}$ and $\neg\alcompS{(2^\omega)^+}$ are theorems of $ZFC$.
  $\alcompS{2^\omega}$ is a theorem of $ZFC+CH$ and consistent with
  $ZFC+\neg CH$.
\end{thm}

\begin{proof}
  For $\alcompS{\omega_1}$, look at pg. 70 of \cite{MR597342}; this of course
  shows $\alcompS{2^\omega}$ under $CH$.
  $\neg\alcompS{(2^\omega)^+}$ can be shown with a simple cardinality argument.
  The consistency result under $ZFC+\neg CH$
  is a lemma for the main theorem in \cite{MR1129143}.
\end{proof}

\begin{thm}
  $\pl F \kmarkwin2 \mengame{\oneptlind\omega_1}$.
\end{thm}

\begin{proof}
  The result follows from $S(\omega_1,\omega,\omega)$; however, we delay the
  details of the proof pending the introduction of a related covering property.
\end{proof}




%   \begin{thm}
%     A space $X$ is $\sigma$-(relatively compact) if and only if $F \markwin \mengame{X}$.
%   \end{thm}

%   \begin{cor}
%     For regular spaces $X$, the following are equivalent:
%       \begin{enumerate}[(a)]
%         \item $X$ is $\sigma$-compact
%         \item $X$ is $\sigma$-(relatively compact)
%         \item $F \markwin \mengame{X}$
%       \end{enumerate}
%   \end{cor}

%   \begin{thm}
%     For second-countable $X$, the following are equivalent:
%       \begin{enumerate}[(a)]
%         \item $X$ is $\sigma$-(relatively compact)
%         \item $F \win \mengame{X}$
%         \item $F\markwin \mengame{X}$
%       \end{enumerate}
%   \end{thm}

%   \begin{cor}(Telgarsky)
%     For metric spaces $X$, the following are equivalent:
%       \begin{enumerate}[(a)]
%         \item $X$ is $\sigma$-compact
%         \item $X$ is $\sigma$-(relatively compact)
%         \item $F \win \mengame{X}$
%         \item $F \markwin \mengame{X}$
%       \end{enumerate}
%   \end{cor}

%   \begin{ex}
%   Let $R$ be given the topology from example 63 from Counterexamples in Topology, the topology generated by open intervals with countable sets removed. This space is a non-regular example where $F \win \mengame{R}$, but $F \not\markwin \mengame{R}$, that is, $R$ is not $\sigma$-(relatively compact).
%   \end{ex}

%   \begin{ex}
%   Let $R$ be given the topology from example 67 from Counterexamples in Topology, the topology generated by open intervals with or without the rationals removed. This space is non-regular, and non-$\sigma$-compact, but is second-countable and $\sigma$-(relatively compact).
%   \end{ex}

%   \begin{defn}
%     Let $\mc U$ be a cover of $X$. We say $C\subseteq X$ is $\mc U$-compact if there exists a finite subcover of $\mc U$ which covers $C$.

%     We say $X$ is \scish~if there exist functions $r_{\mc V}:X\to\omega$ for each open cover $\mc V$ of $X$ such that both of the following sets are $\mc V$-compact for all open covers $\mc U$, $\mc V$ and $n<\omega$:
%       \[
%         c(\mc V,n)=\{ x\in X : r_{\mc V}(x)\leq n\}
%       \]
%       \[
%         p(\mc U,\mc V)=\{ x\in X : 0<r_{\mc U}(x)<r_{\mc V}(x)\}
%       \]
%   \end{defn}

%   \begin{defn}
%     For two functions $f,g$ we say $f$ is \textbf{$\mu$-almost compatible} with $g$ ($f\alcomp_\mu g$) if $|\{x\in\dom(f)\cap\dom(g):f(x)\not=g(x)\}|<\mu$. If $\mu=\omega$ then we say $f,g$ are \textbf{almost compatible} ($f\alcomp g$).
%   \end{defn}

%   \begin{ex}
%     The one-point Lindel\"ofication of the uncountable discrete space, $\oneptlind{\omega_1}$, is \scish.
%   \end{ex}

%   \begin{thm}
%     If $X$ is \scish, then $F \kmarkwin{2}\mengame{X}$.
%   \end{thm}

%   \begin{cor}
%     $F\kmarkwin{2}\mengame{\oneptlind{\omega_1}}$
%   \end{cor}

%   \begin{prop}
%     $\neg S(\kappa,\omega,\omega)$ for $\kappa>2^\omega$
%   \end{prop}

%   \begin{thm}
%     $\alcompS\kappa$ implies $\oneptlind{\kappa}$ is \scish.
%   \end{thm}

%   \begin{cor}
%     $\alcompS\kappa$ implies $F\kmarkwin{2}\mengame{\oneptlind{\kappa}}$.
%   \end{cor}

%   \begin{thm}
%     $S(\kappa,\omega,\omega)+(\kappa=2^\omega)$ is consistent with ZFC for any cardinal $\kappa$ with $cf(\kappa)>\omega$.
%   \end{thm}

%   \begin{cor}
%     For each $\kappa$, $F\kmarkwin{2}\mengame{\oneptlind{\kappa}}$ is consistent with ZFC.
%   \end{cor}



%   \section{Alster property}

%   Besides various limited information characterizations of $\mengame{X}$, there are other interesting covering properties between $\sigma$-(relatively compact) and Menger.

%   \begin{prop}
%     Every ample cover of a regular space $X$ is really ample.
%   \end{prop}

%   \begin{prop}
%     Every regular relatively Alster space is Alster.
%   \end{prop}

%   \begin{thm}(Aurichi, Tall)
%     $X$ $\sigma$-compact $\Rightarrow$ $X$ Alster $\Rightarrow$ $X$ Menger
%   \end{thm}

%   \begin{prop}
%     $X$ $\sigma$-(relatively compact) $\Rightarrow$ $X$ relatively Alster $\Rightarrow$ $X$ Menger
%   \end{prop}

%   \begin{ex}
%     Let the real numbers $R$ be given the topology generated by open intervals with countable sets removed. $R$ is not relatively Alster and $F\win\mengame{R}$. If $S(2^\omega,\omega,\omega)$ holds, then $F\kmarkwin{2}\mengame{R}$.
%   \end{ex}







%   \section{Filling Games}

%   \begin{defn}
%     The \textbf{filling game} $\fillgame{J}$ on an ideal $J$ proceeds as follows: player $M$ chooses $M_0 \in \<J\>$, the $\sigma$-completion of $J$, in the initial round, followed by $N$ choosing $N_0\in J$. In round $n+1$, player $M$ chooses $M_{n+1}$ where $M_n\subseteq M_{n+1}\in\<J\>$, and player $N$ replies with $N_{n+1}\in J$. Player $N$ wins the game if $\bigcup_{n<\omega} N_n \supseteq \bigcup_{n<\omega} M_n$. (The sets in $J$ and $\<J\>$ are thought of as nowhere-dense and meager sets, respectively.)

%     The \textbf{strict filling game} $\sfillgame{J}$ proceeds analogously, with the added requirement that $M_n\subsetneq M_{n+1}$. This game has been studied by Scheepers.
%   \end{defn}

%   \begin{thm}
%     $N\ktactwin{2}\sfillgame{J} \Rightarrow N\kmarkwin{2}\fillgame{J}$
%   \end{thm}

%   \begin{ex}
%     There is a free ideal $J$ such that $N\not\ktactwin{2}\sfillgame{J}$ but $N\kmarkwin{2}\fillgame{J}$.
%   \end{ex}








%   \section{Rothberger property}

%   \begin{thm}(Pawlikowski)
%     $X$ is Rothberger if and only if $C\not\win\rothgame{X}$.
%   \end{thm}


%   \begin{thm}
%     The following are equivalent for compact $T_2$ $X$:
%       \begin{enumerate}[(a)]
%         \item $X$ is Rothberger
%         \item $X$ is scattered
%         \item $S\win\rothgame{X}$
%         \item $C\not\win\rothgame{X}$
%       \end{enumerate}
%   \end{thm}

%   \begin{thm}(Galvin)
%   $\altrothgame{X}$ is ``perfect information equivalent'' to $\rothgame{X}$. That is:

%     \begin{itemize}
%       \item $P\win\altrothgame{X}$ if and only if $S\win\rothgame{X}$
%       \item $O\win\altrothgame{X}$ if and only if $C\win\rothgame{X}$.
%     \end{itemize}
%   \end{thm}

%   \begin{thm}
%     \begin{itemize}
%       \item $P\prewin\altrothgame{X}$ if and only if $S\markwin\rothgame{X}$
%       \item $O\markwin\altrothgame{X}$ if and only if $C\prewin\rothgame{X}$.
%     \end{itemize}
%   \end{thm}

%   \begin{thm}
%     For any space $X$, the following are equivalent:
%     \begin{itemize}
%       \item $S\markwin\rothgame{X}$
%       \item $P\prewin\altrothgame{X}$
%       \item $X$ is almost countable
%     \end{itemize}
%   \end{thm}

%   \begin{thm}
%     For any $T_1$ space $X$, the following are equivalent:
%     \begin{itemize}
%       \item $S\markwin\rothgame{X}$
%       \item $P\prewin\altrothgame{X}$
%       \item $X$ is almost countable
%       \item $|X|\leq\omega$
%     \end{itemize}
%   \end{thm}

%   \begin{ex}
%     Let $X=\omega_1\cup\{\infty\}$ be a ``weak Lindel\"ofication'' of discrete $\omega_1$ such that open neighborhoods of $\infty$ contain $\omega_1\setminus\omega$. This space is $T_0$ but not $T_1$, and note that $S\markwin\rothgame{X}$ and $|X|>\omega$.
%   \end{ex}

%   \begin{thm}
%     The following are equivalent for points-$G_\delta$ $X$:
%       \begin{enumerate}[(a)]
%         \item $S\win\rothgame{X}$
%         \item $P\win\altrothgame{X}$
%         \item $S\kmarkwin{k}\rothgame{X}$ for some $k\geq 1$
%         \item $P\kmarkwin{k}\altrothgame{X}$ for some $k\geq 1$
%         \item $S\markwin\rothgame{X}$
%         \item $P\prewin\altrothgame{X}$
%         \item $X$ is almost countable
%         \item $|X|\leq\omega$
%       \end{enumerate}
%   \end{thm}

%   \begin{cor}
%     The following are equivalent for compact points-$G_\delta$ $X$:
%       \begin{enumerate}[(a)]
%         \item $S\win\rothgame{X}$
%         \item $P\win\altrothgame{X}$
%         \item $S\kmarkwin{k}\rothgame{X}$ for some $k\geq 1$
%         \item $P\kmarkwin{k}\altrothgame{X}$ for some $k\geq 1$
%         \item $S\markwin\rothgame{X}$
%         \item $P\prewin\altrothgame{X}$
%         \item $X$ is almost countable
%         \item $|X|\leq\omega$
%         \item $C\not\win\rothgame{X}$
%         \item $O\not\win\altrothgame{X}$
%         \item $X$ is Rothberger
%         \item $X$ is scattered
%       \end{enumerate}
%   \end{cor}

%   \begin{defn}
%     The game $\recallgame{m}{\kappa}$ proceeds as follows: during round $0$, player $F$ chooses $F_0\in[\kappa]^m$, followed by player $S$ choosing $x_0\in F_0\cup\{\infty\}$. During round $n+1$, $F$ chooses $F_{n+1}\in[\kappa]^{m^{n+2}}$ such that $F_{n+1}\supset F_n$, followed by $S$ choosing $x_{n+1}\in F_{n+1}\cup\{\infty\}$.

%     $S$ wins the game if $\{x_n : n<\omega\}\supseteq F_0\cup\{\infty\}$, and $F$ wins otherwise.
%   \end{defn}

%   \begin{prop}
%     $S\limitwin\rothgame{\oneptlind{\kappa}} \Rightarrow S\limitwin\recallgame{m}{\kappa}$
%   \end{prop}

%   \begin{prop}
%     $S\kmarkwin{k}\recallgame{m}{\kappa} \Leftrightarrow S\ktactwin{k}\recallgame{m}{\kappa}$
%   \end{prop}