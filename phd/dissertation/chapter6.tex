%!TEX root = dissertation.tex
% ^ leave for LaTeXTools build functionality

\chapter{The Menger Game}

In 1924 Karl Menger introduced a covering property generalizing
$\sigma$-compactness \cite{custom31879423}.

\begin{defn}
  A space $X$ is Menger if for every sequence $\<\mc U_0,\mc U_1,\dots\>$
  of open covers of $X$ there exists a sequence
  $\<\mc F_0,\mc F_1,\dots\>$ such that $\mc F_n\subseteq \mc U_n$,
  $|\mc F_n|<\omega$, and $\bigcup_{n<\omega}\mc F_n$ is a cover of $X$.
\end{defn}

\begin{prop}
  $X$ is $\sigma$-compact
    $\Rightarrow$
  $X$ is Menger
    $\Rightarrow$
  $X$ is Lindel\"of.
\end{prop}

None of these implications may be reversed; the irrationals are a simple example
of a Lindel\"of space which is not Menger, and we'll see several examples of
Menger spaces which are not $\sigma$-compact.

It can be shown via a non-trivial
proof that the following game can be used to characterize the Menger property.

\begin{game}
  Let $\mengame{X}$ denote the \term{Menger game} with players $\pl C$, $\pl F$.
  In round $n$, $\pl C$ chooses an open cover $\mc C_n$, followed by $\pl F$
  choosing a finite refinement $\mc F_n$ of $\mc C_n$.

  $\pl F$ wins the game if $\bigcup_{n<\omega}\mc F_n$ is a cover for the space
  $X$, and $\pl C$ wins otherwise.
\end{game}

\begin{thm}
  A space $X$ is Menger if and only if $\pl C\not\win\mengame{X}$
  \cite{MR1544773}.
\end{thm}

\section{Markov strategies}

To the author's knowledge, no other direct work has been done on limited
information strategies pertaining to the Menger game, although as we'll see
there are results which can be sharpened when considering them.
However, we immediately see that tactics are not of any real interest.

\begin{prop}
  $X$ is compact if and only if
  $\pl F \tactwin \mengame{X}$ if and only if
  $\pl F \ktactwin{(k+1)} \mengame{X}$.
\end{prop}

\begin{proof}
  For any open cover $\mc U$, attack the winning $k$-tactic with
  $\<\mc U,\mc U,\dots\>$ to produce a finite refinement of which covers $X$.
\end{proof}

Essentially, because $\pl C$ may repeat the same finite sequence of open covers,
$\pl F$ needs to be seeded with knowledge of the round number to prevent being
trapped in a loop.

If $\pl F$'s memory of $\pl C$'s past moves is bounded, then
there is no need to consider more than the two most recent moves. The
intuitive reason is that $\pl C$ could simply play the same cover repeatedly
until $\pl F$'s memory is exhausted, in which case $\pl F$ would only ever
see the change from one cover to another.

\begin{thm}
  $F \kmarkwin{2} \mengame{X}$ if and only if $F \kmarkwin{(k+2)} \mengame{X}$.
\end{thm}

\begin{proof}
  Let $\sigma$ be a winning $(k+2)$-mark. We define the $2$-mark $\tau$ as
  follows:
    \[
      \tau(\<\mc U\>,0)
        =
      \bigcup_{m<k+2}
        \sigma(\<\underbrace{\mc U,\dots,\mc U}_{m+1}\>,m)
    \]
    \[
      \tau(\<\mc U,\mc V\>,n+1)
        =
      \bigcup_{m<k+2}
        \sigma(\<
          \underbrace{\mc U,\dots,\mc U}_{k+1-m},
          \underbrace{\mc V,\dots,\mc V}_{m+1}
        \>,(n+1)(k+2)+m)
    \]

  Let $\<\mc U_0,\mc U_1,\dots\>$ be an attack by $\pl C$ against $\tau$.
  Then consider the attack
    \[
      \<
        \underbrace{\mc U_0,\dots,\mc U_0}_{k+2},
        \underbrace{\mc U_1,\dots,\mc U_1}_{k+2},
        \dots
      \>
    \]
  by $\pl C$ against $\sigma$. Since $\sigma$ is a winning $(k+2)$-mark, the
  collection
    \[
      \bigcup_{m<k+2}
        \sigma(\<\underbrace{\mc U_0,\dots,\mc U_0}_{m+1}\>,m)
      \cup
      \bigcup_{n<\omega}
      \bigcup_{m<k+2}
        \sigma(\<
          \underbrace{\mc U_n,\dots,\mc U_n}_{k+1-m},
          \underbrace{\mc U_{n+1},\dots,\mc U_{n+1}}_{m+1}
        \>,(n+1)(k+2)+m)
    \]
    \[
      =
      \tau(\<\mc U_0\>,0)
      \cup
      \bigcup_{n<\omega}
      \tau(\<\mc U_n,\mc U_{n+1}\>,n+1)
    \]
  covers the space, and $\tau$ is a winning $2$-mark.
\end{proof}

A natural question arises: is there an example of a space $X$ for which
$\pl F\kmarkwin{2}\mengame{X}$ but $\pl F\not\markwin\mengame{X}$? Fortunately,
perhaps the simplest example of a Lindel\"of non-$\sigma$-compact
space has this property.

\begin{defn}
  For any cardinal $\kappa$, let $\oneptlind\kappa=\kappa\cup\{\infty\}$ denote
  the \term{one-point Lindel\"of-ication} of discrete $\kappa$, where points in
  $\kappa$ are isolated, and the neighborhoods of $\infty$ are the co-countable
  sets containing it.
\end{defn}

\begin{thm}
  $F\not\markwin\mengame{\oneptlind{\omega_1}}$.
\end{thm}

\begin{proof}
  This result will later follow from the fact that
  $\oneptlind\omega_1$ is not a $\sigma$-compact space (all its compact subsets
  are finite).

  Let $\sigma$ be a Markov strategy for $\pl F$. For each $\alpha<\omega_1$,
  let $\mc U_\alpha$ be the open cover
  $\{\{\beta\}:\beta<\alpha\}\cup\{\oneptlind{\omega_1}\setminus\alpha\}$ of
  $\oneptlind{\omega_1}$. Then define
  $\tau:\omega_1\times\omega \rightarrow [\omega_1]^{<\omega}$ so that
  $\tau(\alpha,n)=\bigcup(\sigma(\<\mc U_\alpha\>,n)\cap[\omega_1]^1)$.

  It can be shown that there exists a sequence $\alpha\in\omega_1^\omega$
  such that $\{\tau(\alpha(n),n): n<\omega\}$ is not a cover for
  $\bigcap_{n<\omega}\alpha(n)$. So $\pl C$ may attack
  $\sigma$ with $\<\mc U_{\alpha(0)}, \mc U_{\alpha(1)},\dots\>$,
  and it follows that
  $\bigcup_{n<\omega}\sigma(\<\mc U_{\alpha(n)}\>,n)$ is not a cover for
  $\oneptlind{\omega_1}$.
\end{proof}

The greatest advantage of a strategy which has knowledge of two or more previous
moves of the opponent, versus only knowledge of the most recent move, is the
ability to react to changes from one round to the next. It's this ability to
react that will give $\pl F$ her winning $2$-Markov strategy in the Menger
game on $\oneptlind\omega_1$.

For inspiration, we turn to a game whose $n$-tactics were studied by Marion
Scheepers in \cite{MR1129143} which has similar goals to the Menger game when
played upon $\oneptlind\kappa$.

\begin{game}
  Let $\fillgameS\kappa$ denote the \term{strict union filling game}
  with two players $\pl C$, $\pl F$. In round $0$, $\pl C$ chooses
  $C_0\in[\kappa]^{\leq\omega}$, followed by $\pl F$ choosing
  $F_0\in[\kappa]^{<\omega}$. In round $n+1$, $\pl C$ chooses
  $C_{n+1}\in[\kappa]^{\leq\omega}$ such that $C_{n+1}\supset C_n$, followed
  by $\pl F$ choosing $F_{n+1}\in[\kappa]^{<\omega}$.

  $\pl F$ wins the game if
  $\bigcup_{n<\omega} F_n\supseteq\bigcup_{n<\omega} C_n$; otherwise, $\pl C$
  wins.
\end{game}

In $\mengame{\oneptlind\kappa}$, $\pl C$ essentially chooses a countable set
to not include in her neighborhood of $\infty$, followed by $\pl F$ choosing
finitely many of them to cover each round. However,
$\pl F$ need only be concerned with the \textit{intersection} of the
countable sets essentially chosen $\pl C$ rather than the union as in
$\fillgameS\kappa$, since she may always choose to cover the neighborhood of
$\infty$ chosen by $\pl C$ each round.

In addition, Scheepers required that $\pl C$ always choose strictly growing
countable sets. The reasoning is clear: if the goal is to study tactics, then
$\pl C$ cannot be allowed to trap $\pl F$ in a loop by repeating the same moves.
But by eliminating this requirement, the study can then turn to Markov
srategies, bringing the game further in line with the Menger game played upon
$\oneptlind\kappa$.

We introduce a few games to show a relationship between Scheeper's
$\fillgameS\kappa$ and $\mengame{\oneptlind\kappa}$.

\begin{game}
  Let $\fillgame\kappa$ denote the
  \term{union filling game} which proceeds analogously
  to $\fillgameS\kappa$, except that $\pl C$'s restriction in round $n+1$
  is reduced to $C_{n+1}\supseteq C_n$.
\end{game}

\begin{game}
  Let $\fillgameOne\kappa$ denote the
  \term{initial filling game} which proceeds analogously
  to $\fillgame\kappa$, except that $\pl F$ wins whenever
  $\bigcup_{n<\omega}F_n\supseteq C_0$.
\end{game}

\begin{game}
  Let $\fillgameInt\kappa$ denote the
  \term{intersection filling game} which proceeds analogously
  to $\fillgameOne\kappa$, except that $\pl C$ may choose any
  $C_n\in[\kappa]^{\leq\omega}$ each round, and $\pl F$ wins whenever
  $\bigcup_{n<\omega}F_n\supseteq\bigcap_{n<\omega}C_n$.
\end{game}

\begin{thm}
For any cardinal $\kappa>\omega$ and integer $k<\omega$,
Figure \ref{fillingGamesDiagram} holds.
\end{thm}

\begin{proof}
  $\pl F \kmarkwin{k}\mengame{\oneptlind\kappa}
    \Rightarrow
  \pl F \kmarkwin{k}\fillgameInt\kappa$:
  Let $\sigma$ be a winning $k$-mark for $\pl F$ in
  $\mengame{\oneptlind\kappa}$. Let $\mc U(C)$ (resp. $\mc U(s)$) convert each
  countable subset $C$ of $\kappa$ (resp. finite sequence $s$ of such subsets)
  into the open cover $[C]^1\cup\{\oneptlind\kappa\setminus C\}$
  (resp. finite sequence of such open covers). Then $\tau$ defined by
    \[
      \tau(s\concat\<C\>,n)
        =
      \bigcup(\sigma(\mc U(s\concat\<C\>),n)\cap[\kappa]^1)
    \]
  is a winning $k$-mark for $\pl F$ in $\fillgameInt\kappa$.

  $\pl F \kmarkwin{k}\fillgameInt\kappa
    \Rightarrow
  \pl F \kmarkwin{k}\mengame{\oneptlind\kappa}$:
  Let $\sigma$ be a winning $k$-mark for $\pl F$ in
  $\fillgameInt\kappa$. Let $C(\mc U)$ (resp. $C(s)$) convert each basic open
  cover $\mc U$ of $\oneptlind\kappa$ (resp. finite sequence $s$ of such covers)
  into a countable set $C$ which is the complement of some neighborhood of
  $\infty$ in $\mc U$ (resp. finite sequence of such countable sets).
  Then $\tau$ defined by
    \[
      \tau(s\concat\<\mc U\>,n)
        =
      [\sigma(C(s\concat\<\mc U)\>),n)\cap C(\mc U)]^1
        \cup
      \{\oneptlind\kappa\setminus C(\mc U)\}
    \]
  is a winning $k$-mark for $\pl F$ in $\mengame{\oneptlind\kappa}$.

  $\pl F \kmarkwin{k}\fillgameInt\kappa
    \Rightarrow
  \pl F \kmarkwin{k}\fillgameOne\kappa$:
  Let $\sigma$ be a winning $k$-mark for $\pl F$ in
  $\fillgameInt\kappa$. $\sigma$ is also a winning $k$-mark for $\pl F$
  in $\fillgameOne\kappa$.

  $\pl F \kmarkwin{k}\fillgameOne\kappa
    \Rightarrow
  \pl F \kmarkwin{k}\fillgame\kappa$: Let $\sigma$ be a winning $k$-mark for $\pl F$ in
  $\fillgameOne\kappa$. For each finite sequence $s$, let $t\preceq s$ mean $t$
  is a final subsequence of $s$. Then $\tau$ defined by
    \[
      \tau(s\concat\<C\>,n)
        =
      \bigcup_{t\preceq s, m\leq n}
      \sigma(t\concat\<C\>,m)
    \]
  is a winning $k$-mark for $\pl F$ in $\fillgame\kappa$.

  $\pl F \kmarkwin{k}\fillgame\kappa
    \Rightarrow
  \pl F \kmarkwin{k}\fillgameOne\kappa$:
  Let $\sigma$ be a winning $k$-mark for $\pl F$ in
  $\fillgame\kappa$. $\sigma$ is also a winning $k$-mark for $\pl F$
  in $\fillgameOne\kappa$.

  $\pl F \ktactwin{k}\fillgameS\kappa
    \Rightarrow
  \pl F \kmarkwin{k}\fillgame\kappa$:
  Let $\sigma$ be a winning $k$-tactic for $\pl F$ in
  $\fillgameS\kappa$. For each countable subset $C$ of $\kappa$, let $C+n$
  be the union of $C$ with the $n$ least ordinals in $\kappa\setminus C$.
  Then $\tau$ defined by
    \[
      \tau(\<C_0,\dots,C_i\>,n)
        =
      \sigma(\<C_0+(n-i),\dots,C_i+n\>)
    \]
  is a winning $k$-mark for $\pl F$ in $\fillgame\kappa$.
\end{proof}

While we have not proven a direct implication between the Menger game and
Scheeper's original filling game, Scheepers introduced the statement
$\alcompS\kappa$ relating to the almost-compatability of functions
from countable subsets of $\kappa$ into $\omega$ which may be applied to
both.

\begin{defn}
  For two functions $f,g$ we say $f$ is \textbf{$\mu$-almost compatible} with
  $g$ ($f\alcomp_\mu g$) if $|\{x\in\dom(f)\cap\dom(g):f(x)\not=g(x)\}|<\mu$.
  If $\mu=\omega$ then we say $f,g$ are \textbf{almost compatible}
  ($f\alcomp g$).
\end{defn}

\begin{defn}
  $\alcompS\kappa$ states that there exist functions
  $f_A:A\to\omega$ for each $A\in[\kappa]^{\leq\omega}$ such that
  $|\{\alpha\in A:f_A(\alpha)\leq n\}|<\omega$ for all $n<\omega$ and
  $f_A\alcomp f_B$ for all $A,B\in[\kappa]^\omega$.
  \footnote{
  This is equivalent to the original characterization given in
  \cite{MR1129143}: there exist injections $g_A:A\to\omega$
  such that $g_A\alcomp g_B$ for all $A,B\in[\kappa]^\omega$ and $A\subset B$.
  }
\end{defn}

Scheepers went on to show that $\alcompS\kappa$ implies
$\pl F \ktactwin2 \fillgameS\kappa$. This proof, along with the following
facts, give us inspiration for
finding a winning $2$-Markov strategy in the Menger game.

\begin{thm}
  $\alcompS{\omega_1}$ and $\kappa>2^\omega\Rightarrow\neg\alcompS\kappa$
  are theorems of $ZFC$.
  $\alcompS{2^\omega}$ is a theorem of $ZFC+CH$ and consistent with
  $ZFC+\neg CH$.
\end{thm}

\begin{proof}
  For $\alcompS{\omega_1}$, look at pg. 70 of \cite{MR597342}; this of course
  implies $\alcompS{2^\omega}$ under $CH$.
  $\neg\alcompS{(2^\omega)^+}$ is shown by a cardinality argument
  in \cite{MR1129143}.
  The consistency result under $ZFC+\neg CH$
  is a lemma for the main theorem in \cite{MR1129143}.
\end{proof}

\begin{thm}
  $\alcompS\kappa$ implies the game-theoretic results in
  Figure \ref{fillingGamesDiagram2}.
\end{thm}

\begin{proof}
  Since $\alcompS\kappa\Rightarrow\pl F \ktactwin2\fillgameS\kappa$ was
  a main result of \cite{MR1129143}, it is sufficient to show that
  $\alcompS\kappa\Rightarrow \pl F \kmarkwin2\fillgameInt\kappa$.

  Let $f_A$ for $A\in[\kappa]^{\leq\omega}$ witness $\alcompS\kappa$. We define
  the $2$-mark $\sigma$ as follows:
    \[
      \sigma(\<A\>,0) = \{\alpha\in A: f_A(\alpha) \leq 0 \}
    \]
    \[
      \sigma(\<A,B\>,n+1)
        =
      \{
        \alpha \in A\cap B
      :
        f_B(\alpha) \leq n+1 \text{ or }
        f_A(\alpha)\not=f_B(\alpha)
      \}
    \]
  For any attack $\<A_0,A_1,\dots\>$ by $\pl C$ and
  $\alpha\in\bigcap_{n<\omega}A_n$, either $f_{A_n}(\alpha)$ is constant for
  all $n$, or $f_{A_n}(\alpha)\not=f_{A_{n+1}}(\alpha)$ for some $n$;
  either way, $\alpha$ is covered.
\end{proof}

\begin{cor}
  $\pl F \kmarkwin2\mengame{\oneptlind\omega_1}$.
\end{cor}



\section{Menger game derived covering properties}

Limited information strategies for the Menger game naturally define a spectrum
of covering properties, see Figure \ref{menSpec}. However,
we do not know if the middle two properties are actually distinct.



\begin{ques}\label{perfectTo2Mark}
  Does there exist a space $X$ such that $\pl F \win \mengame{X}$ but
  $\pl F \not\kmarkwin2 \mengame{X}$?
\end{ques}

We are also interested in non-game-theoretic characterizations of these
covering properties. It has been known for some time that for metrizable spaces,
winning Menger spaces are exactly the $\sigma$-compact spaces, shown first
by Telgarksy in \cite{MR753073} and later directly by Scheepers in
\cite{MR1273523}.

In the interest of generality, we will first characterize the Markov Menger
spaces without any separation axioms.

\begin{defn}
  A subset $Y$ of $X$ is \term{relatively compact} to $X$ if for every open
  cover of $X$, there exists a finite subcollection (equivalently, refinement)
  which covers $Y$.
\end{defn}

For example, any bounded subset of Euclidean space is relatively compact
whether it is closed or not. Actually, relative compactness can be thought
of as an analogue of boundedness for regular spaces.

\begin{prop}
  For regular spaces, $Y$ is relatively compact to $X$ if and only if
  $\cl Y$ is compact in $X$.
  \footnote{
    It should be noted that some authors define relative compactness in
    this way, but such a definition creates pathological implications for
    non-regular spaces. For example, the singleton containing
    the particular point of an infinite space with the particular point topology
    would not be relatively compact since its closure is not compact, even
    though it is finite.
  }
\end{prop}

\begin{proof}
  For any space, all subsets any compact set are relatively compact.

  Assume $Y$ is relatively compact, let $\mc U$ be an open cover of $\cl Y$,
  and define $x\in V_x\subseteq \cl{V_x}\subseteq U_x \in \mc U$ for each
  $x\in X$. Then if we take a subcollection $\mc F = \{W_{x_i} : i<n\}$ covering
  $Y$ by relative compactness, then $\{U_{x_i}:i<n\}$ is a finite subcollection
  of $\mc U$ covering $\cl Y$, showing compactness.
\end{proof}

We now begin the process of factoring out Scheeper's proof to reveal the
limited information implications at work.

\begin{lem}
  Let $\sigma(\mc{U}, n)$ be a Markov strategy for $F$ in
  $\mengame{X}$, and $\mathfrak{C}$ collect all open covers of $X$. Then for
    \[
      R_n = \bigcap_{\mc{U}\in\mathfrak{C}} \bigcup\sigma(\mc{U},n)
    \]
  it follows that $R_n$ is relatively compact to $X$. If $\sigma$ is a winning
  Markov startegy, then $\bigcup_{n<\omega} R_n = X$.
\end{lem}

\begin{proof}
  First, for every open cover $\mc U\in\mathfrak C$, $\sigma(\mc{U},n)$
  witnesses relative compactness for $R_n$.

  Suppose that $x \not\in R_n$ for any $n<\omega$. Then for each $n$,
  pick $\mc{U}_n\in\mathfrak{C}$ such that $x\not\in \bigcup\sigma(\mc{U}_n,n)$.
  $\pl C$ may counter $\sigma$ with the attack $\<\mc U_0,\mc U_1,\dots\>$.
\end{proof}

\begin{defn}
  A \term{$\sigma$-relatively-compact} space is the countable union of
  relatively compact subsets.
\end{defn}

\begin{cor}
  A space $X$ is $\sigma$-relatively-compact if and only if
  $F \markwin \mengame{X}$.
\end{cor}

\begin{cor}
  Let $X$ be a regular space. The following are equivalent:
  \begin{itemize}
    \item $X$ is $\sigma$-compact
    \item $X$ is $\sigma$-relatively-compact
    \item $F \markwin \mengame{X}$
  \end{itemize}
\end{cor}

For Lindel\"of spaces, metrizability is characterized by regularity and
second-countability, the latter of which was essentially used by Scheepers in
this way:

\begin{lem}
  Let $X$ be a second-countable space. $\pl F\win\mengame{X}$ if and only if
  $\pl F\markwin\mengame{X}$.
\end{lem}

\begin{proof}
  Let $\sigma$ be a strategy for $\pl F$, and let $\mathfrak{C}$ be the
  collection of all covers of $X$ using open sets from some countable
  base.

  Note that there are only
  countably many finite collections of basic open sets. So for
  $s\in\omega^{<\omega}$, we may define $\mc U_{s\concat\<n\>}\in\mathfrak C$
  such that for each basic open cover $\mc U$, there is some $n<\omega$ where
  \[
    \sigma(\mc U_{s\rest 1},\dots,\mc U_{s},\mc U)
      =
    \sigma(\mc U_{s\rest 1},\dots,\mc U_{s},\mc U_{s\concat\<n\>})
  \]

  Let $t:\omega\to\omega^{<\omega}$ be a bijection. We define the Mark\"ov
  strategy $\tau$ as follows:
  \[
    \tau(\mc U,n) = \sigma(\mc U_{t(n)\rest 1},\dots,\mc U_{t(n)},\mc U)
  \]

  Suppose there exists a counter-attack $\<\mc V_0,\mc V_1,\dots\>$ which
  defeats $\tau$. Then there exists
  $f:\omega\to\omega$ such that:
  \[
    \begin{array}{rcl}
    x & \not\in & \bigcup \tau(\mc V_{t^{-1}(f\rest n)},t^{-1}(f\rest n)) \\
    & = & \bigcup\sigma(\mc U_{f\rest 1},\dots,\mc U_{f\rest n},\mc V_{t^{-1}(f\rest n)}) \\
    & = & \bigcup\sigma(\mc U_{f\rest 1},\dots,\mc U_{f\rest n},\mc U_{f\rest (n+1)})
    \end{array}
  \]

  Thus $\<\mc U_{f\rest 1}, \mc U_{f\rest 2},\dots\>$ is a successful
  counter-attack by $\pl C$ against the perfect information strategy $\sigma$.
\end{proof}

\begin{cor}
  Let $X$ be a second-countable space. The following are equivalent:
  \begin{itemize}
    \item $X$ is $\sigma$-relatively-compact
    \item $F \markwin \mengame{X}$
    \item $F \win \mengame{X}$
  \end{itemize}
\end{cor}

\begin{cor}
  Let $X$ be a metrizable space. The following are equivalent:
  \begin{itemize}
    \item $X$ is $\sigma$-compact
    \item $X$ is $\sigma$-relatively-compact
    \item $F \markwin \mengame{X}$
    \item $F \win \mengame{X}$
  \end{itemize}
\end{cor}

\begin{proof}
  Each property implies Lindel\"of, so $X$ may be assumed to be
  regular and second-countable.
\end{proof}

\section{Almost-$\sigma$-relatively-compact}

In an attempt to characterize $\pl F\kmarkwin2\mengame{X}$, we introduce the
following covering property.

\begin{defn}
  For each cover $\mc U$ of $X$, $S\subseteq X$ is \term{$\mc U$-compact} if
  there exists a finite subcollection of $\mc U$ which covers $C$.

  A space $X$ is \term{almost-$\sigma$-relatively-compact} if there exist
  functions $r_{\mc V}:X\to\omega$ for each open cover $\mc V$ of $X$ such that
  for all open covers $\mc U$, open covers $\mc V$, and numbers $n<\omega$,
  both of the following sets are $\mc V$-compact:
    \[
      c(\mc V,n)=\{ x\in X : r_{\mc V}(x)\leq n\}
    \]
    \[
      p(\mc U,\mc V,n+1)=\{ x\in X : n<r_{\mc U}(x)<r_{\mc V}(x)\}
    \]
  and $X = \bigcup_{n<\omega} c(\mc V,n)$ for all open covers $\mc V$.
\end{defn}

\begin{prop}
  All $\sigma$-relatively-compact spaces are almost-$\sigma$-relatively-compact.
\end{prop}

\begin{proof}
  If $X=\bigcup_{n<\omega}R_n$, then for all $\mc U$, let $r_{\mc U}(x)$ be the
  least $n$ such that $x\in R_n$. Then $c(\mc V,n)=\bigcup_{m\leq n}R_m$ and
  $p(\mc U,\mc V) = \emptyset$.
\end{proof}

\begin{thm}
  If $X$ is almost-$\sigma$-relatively-compact, then
  $\pl F\kmarkwin2\mengame X$.
\end{thm}

\begin{proof}
  We define the Markov strategy $\sigma$ as follows.
  Let $\sigma(\<\mc U\>,0)$ cover $c(\mc U,0)$, and let
  $\sigma(\<\mc U,\mc V\>,n+1)$ cover both $c(\mc V,n+1)$ and
  $p(\mc U, \mc V,n+1)$.

  For any attack $\<\mc{U}_0,\mc{U}_1,\dots\>$ by $\pl C$ and $x\in X$,
  one of the following must occur:

  \begin{itemize}
    \item
      $r_{\mc U_0}(x)=0$ and thus
      $x\in c(\mc U_0,0)\subseteq \bigcup \sigma(\<\mc U_0\>,0)$.

    \item
      $r_{\mc U_0}(x)=N+1$ for some $N\geq 0$ and:
      \begin{itemize}
        \item
          For all $n\leq N$,
          \[
            r_{\mc U_{n+1}}(x)\leq N+1
          \]
          and thus
          $x\in c(\mc U_{N+1},N+1) \subseteq
            \bigcup \sigma(\<U_N,U_{N+1}\>,N+1)$.
        \item
          For some $n \leq N$,
          \[ r_{\mc U_n}(x)\leq n \]
          and thus
          $x\in c(\mc U_{n+1},n+1) \subseteq
            \bigcup \sigma(\<U_n,U_{n+1}\>,n+1)$.
        \item
          For some $n \leq N$,
          \[
            n < r_{\mc U_n}(x)\leq N+1 < r_{\mc U_{n+1}}(x)
          \]
         and thus
         $x\in p(\mc U_n,\mc U_{n+1},n+1) \subseteq
          \bigcup \sigma(\<U_n,U_{n+1}\>,n+1)$
       \end{itemize}
  \end{itemize}
\end{proof}

\begin{thm}
  $\alcompS\kappa$ implies $\oneptlind\kappa$ is
  almost-$\sigma$-relatively-compact, and thus
  $\pl F\kmarkwin2\mengame{\oneptlind\kappa}$.
\end{thm}

\begin{proof}
  Let $f_A$ for $A\in[\kappa]^{\leq\omega}$ witness $\alcompS\kappa$ and fix
  $A(\mc U)\in[\kappa]^{\leq\omega}$ for each open cover $\mc U$ such that
  $\oneptlind\kappa\setminus A(\mc U)$ is contained in some element of
  $\mc U$.
  Then let $r_{\mc U}(x)=0$ for $x\in\oneptlind\kappa\setminus A(\mc U)$,
  and $r_{\mc U}(\alpha)=f_{A(\mc U)}(\alpha)$ for $\alpha\in A(\mc U)$.

  It follows that
    \[
      c(\mc U,n)
        =
      \left(\oneptlind\kappa\setminus A(\mc U)\right)
        \cup
      \{\alpha\in A(\mc U) : f_{A(\mc U)}(\alpha)\leq n \}
    \]
  is $\mc U$-compact, $\bigcup_{n<\omega}c(\mc U,n)=X$, and
    \[
      p(\mc U,\mc V,n+1)
        =
      \{
        \alpha\in A(\mc U)\cap A(\mc V)
          :
        n < f_{A(\mc U)}(\alpha) < f_{A(\mc V)}(\alpha)
      \}
    \]
  is finite and thus $\mc U$-compact.
\end{proof}

We may also consider common (non-regular) counterexamples which are finer
than the usual Euclidean line.

\begin{defn}
  Let $R_{\mathbb Q}$ be the real line with the topology generated by open
  intervals with or without the rationals removed.
\end{defn}

\begin{thm}
  $R_{\mathbb Q}$ is non-regular and non-$\sigma$-compact, but is
  second-countable and $\sigma$-relatively-compact.
\end{thm}

\begin{proof}
  Compact sets in $R_{\mathbb Q}$ can be shown to not contain open intervals,
  and thus are nowhere dense in nonmeager $\mathbb R$,
  so $R_{\mathbb Q}$ is not $\sigma$-compact. The usual base of
  intervals with rational endpoints (with or without rationals removed)
  witnesses second-countability.

  To see that $R_{\mathbb Q}$ is $\sigma$-relatively compact, consider
  $[a,b]\setminus\mathbb{Q}$. Let $\mc U$ be a cover of $R_{\mathbb Q}$, and
  let $\mc U'$ fill in the missing rationals for any open set in $\mc U$.
  There is a finite subcover $\mc V'\subseteq \mc U'$ for $[a,b]$ since
  $\mc U'$ contains open sets from the Euclidean topology. Let
  $\mc V= \{V\setminus\mathbb{Q}:V\in V'\}$: this is a finite refinment of
  $\mc U$ covering $[a,b]\setminus\mathbb{Q}$, so $[a,b]\setminus\mathbb{Q}$
  is relatively compact. It follows then that $R_{\mathbb Q}\setminus\mathbb Q$
  is $\sigma$-relatively-compact, and since $\mathbb Q$ is countable,
  $R_{\mathbb Q}$ is $\sigma$-relatively-compact. Non-regularity follows since
  regular and $\sigma$-relatively-compact implies $\sigma$-compact.
\end{proof}

\begin{defn}
  Let $R_\omega$ be the real line with the topology generated by open
  intervals with countably many points removed.
\end{defn}

\begin{thm}
  $R_\omega$ is non-regular, non-second-countable, and
  non-$\sigma$-relatively-compact, but $\pl F\win\mengame{R_\omega}$.
\end{thm}

\begin{proof}
  The closure of any open set is its closure in the usual Euclidean topology,
  so $R_\omega$ is not regular. If $S\supseteq\{s_n:n<\omega\}$ for $s_n$
  discrete, then $U_m=R_\omega\setminus\{s_n:m<n<\omega\}$ yields an
  infinite cover $\{U_m:m<\omega\}$ with no finite subcollection covering $S$,
  showing that all relatively compact sets are finite, and $R_\omega$ is not
  $\sigma$-relatively-compact.

  Define the winning strategy $\sigma$ for $\pl F$ in $\mengame{R_\omega}$ as
  follows: let $\sigma(\mc{U}_0,\dots,\mc{U}_{2n})$ be a finite subcover for
  $[-n,n]\setminus C_n$ for some countable $C_n=\{c_{n,m}: m<\omega\}$,
  and let $\sigma(\mc{U}_0,\dots,\mc{U}_{2n+1})$ be a finite subcover for
  $\{c_{i,j} : i,j < n\}$. Non-second-countable follows since
  second-countable and $\pl F\win\mengame{X}$ implies
  $\sigma$-relatively-compact.
\end{proof}

% TODO: Either show 2mark, or prove otherwise

% TODO: \omega_1^\sharp

% TODO: Alster









%%% FIGURES %%%




\begin{figure}[p!]
\begin{center}
\begin{tikzpicture}
  \matrix (m) [matrix of math nodes,row sep=3em,column sep=1em,minimum width=2em]
  {
    \pl F \kmarkwin{k}\mengame{\oneptlind\kappa} &
    \pl F \kmarkwin{k}\fillgameInt\kappa \\

    \pl F \kmarkwin{k}\fillgameOne\kappa &
    \pl F \kmarkwin{k}\fillgame\kappa \\

    &
    \pl F \ktactwin{k}\fillgameS\kappa \\
  };
  \path[>=latex,->]
    (m-1-1) edge (m-2-1)
    (m-1-2) edge (m-2-2)
    (m-3-2) edge (m-2-2);
  \path[>=latex,<->]
    (m-1-1) edge (m-1-2)
    (m-2-1) edge (m-2-2);
\end{tikzpicture}
\end{center}
\caption{Diagram of Filling/Menger game implications}
\label{fillingGamesDiagram}
\end{figure}

\begin{figure}[p!]
\begin{center}
\begin{tikzpicture}
  \matrix (m) [matrix of math nodes,row sep=3em,column sep=1em,minimum width=2em]
  {
    &
    \pl F \kmarkwin2\mengame{\oneptlind\kappa} &
    \pl F \kmarkwin2\fillgameInt\kappa \\

    \alcompS\kappa &
    \pl F \kmarkwin2\fillgameOne\kappa &
    \pl F \kmarkwin2\fillgame\kappa \\

    &
    &
    \pl F \ktactwin2\fillgameS\kappa \\
  };
  \path[>=latex,->]
    (m-2-1) edge (m-1-2)
    (m-2-1) edge (m-2-2)
    (m-2-1) edge (m-3-3)
    (m-1-2) edge (m-2-2)
    (m-1-3) edge (m-2-3)
    (m-3-3) edge (m-2-3);
  \path[>=latex,<->]
    (m-1-2) edge (m-1-3)
    (m-2-2) edge (m-2-3);
\end{tikzpicture}
\end{center}
\caption{Diagram of Filling/Menger game implications with $\alcompS\kappa$}
\label{fillingGamesDiagram2}
\end{figure}

\begin{figure}[p!]
\begin{center}
\begin{tikzpicture}
  \matrix (m) [matrix of math nodes,row sep=3em,column sep=1em,minimum width=2em]
  {
     \pl F \markwin \mengame{X} &
     \pl F \kmarkwin2 \mengame{X} &
     \pl F \win \mengame{X} &
     \pl C \not\win \mengame{X} \\

     \text{Markov Menger} &
     2\text{-Markov Menger} &
     \text{Perfect Menger} &
     \text{Menger} \\
  };
  \path[>=latex,<->]
    (m-1-1) edge (m-2-1)
    (m-1-2) edge (m-2-2)
    (m-1-3) edge (m-2-3)
    (m-1-4) edge (m-2-4);
  \path[>=latex,->]
    (m-1-1) edge (m-1-2)
    (m-2-1) edge (m-2-2)
    (m-1-2) edge (m-1-3)
    (m-2-2) edge (m-2-3)
    (m-1-3) edge (m-1-4)
    (m-2-3) edge (m-2-4);
  \path[>=latex,->,bend right=30, dotted]
    (m-1-2) edge node {$ \not $} (m-1-1)
    (m-2-2) edge node {$ \not $} (m-2-1)
    (m-1-4) edge node {$ \not $} (m-1-3)
    (m-2-4) edge node {$ \not $} (m-2-3);
  \path[>=latex,->,bend right=30, dotted]
    (m-1-3) edge node {$ ? $} (m-1-2)
    (m-2-3) edge node {$ ? $} (m-2-2);
\end{tikzpicture}
\end{center}
\caption{Diagram of covering properties related to the Menger game}
\label{menSpec}
\end{figure}


%   \section{Alster property}

%   Besides various limited information characterizations of $\mengame{X}$, there are other interesting covering properties between $\sigma$-(relatively compact) and Menger.

%   \begin{prop}
%     Every ample cover of a regular space $X$ is really ample.
%   \end{prop}

%   \begin{prop}
%     Every regular relatively Alster space is Alster.
%   \end{prop}

%   \begin{thm}(Aurichi, Tall)
%     $X$ $\sigma$-compact $\Rightarrow$ $X$ Alster $\Rightarrow$ $X$ Menger
%   \end{thm}

%   \begin{prop}
%     $X$ $\sigma$-(relatively compact) $\Rightarrow$ $X$ relatively Alster $\Rightarrow$ $X$ Menger
%   \end{prop}

%   \begin{ex}
%     Let the real numbers $R$ be given the topology generated by open intervals with countable sets removed. $R$ is not relatively Alster and $F\win\mengame{R}$. If $S(2^\omega,\omega,\omega)$ holds, then $F\kmarkwin{2}\mengame{R}$.
%   \end{ex}







%   \section{Filling Games}

%   \begin{defn}
%     The \textbf{filling game} $\fillgame{J}$ on an ideal $J$ proceeds as follows: player $M$ chooses $M_0 \in \<J\>$, the $\sigma$-completion of $J$, in the initial round, followed by $N$ choosing $N_0\in J$. In round $n+1$, player $M$ chooses $M_{n+1}$ where $M_n\subseteq M_{n+1}\in\<J\>$, and player $N$ replies with $N_{n+1}\in J$. Player $N$ wins the game if $\bigcup_{n<\omega} N_n \supseteq \bigcup_{n<\omega} M_n$. (The sets in $J$ and $\<J\>$ are thought of as nowhere-dense and meager sets, respectively.)

%     The \textbf{strict filling game} $\sfillgame{J}$ proceeds analogously, with the added requirement that $M_n\subsetneq M_{n+1}$. This game has been studied by Scheepers.
%   \end{defn}

%   \begin{thm}
%     $N\ktactwin{2}\sfillgame{J} \Rightarrow N\kmarkwin{2}\fillgame{J}$
%   \end{thm}

%   \begin{ex}
%     There is a free ideal $J$ such that $N\not\ktactwin{2}\sfillgame{J}$ but $N\kmarkwin{2}\fillgame{J}$.
%   \end{ex}








%   \section{Rothberger property}

%   \begin{thm}(Pawlikowski)
%     $X$ is Rothberger if and only if $C\not\win\rothgame{X}$.
%   \end{thm}


%   \begin{thm}
%     The following are equivalent for compact $T_2$ $X$:
%       \begin{enumerate}[(a)]
%         \item $X$ is Rothberger
%         \item $X$ is scattered
%         \item $S\win\rothgame{X}$
%         \item $C\not\win\rothgame{X}$
%       \end{enumerate}
%   \end{thm}

%   \begin{thm}(Galvin)
%   $\altrothgame{X}$ is ``perfect information equivalent'' to $\rothgame{X}$. That is:

%     \begin{itemize}
%       \item $P\win\altrothgame{X}$ if and only if $S\win\rothgame{X}$
%       \item $O\win\altrothgame{X}$ if and only if $C\win\rothgame{X}$.
%     \end{itemize}
%   \end{thm}

%   \begin{thm}
%     \begin{itemize}
%       \item $P\prewin\altrothgame{X}$ if and only if $S\markwin\rothgame{X}$
%       \item $O\markwin\altrothgame{X}$ if and only if $C\prewin\rothgame{X}$.
%     \end{itemize}
%   \end{thm}

%   \begin{thm}
%     For any space $X$, the following are equivalent:
%     \begin{itemize}
%       \item $S\markwin\rothgame{X}$
%       \item $P\prewin\altrothgame{X}$
%       \item $X$ is almost countable
%     \end{itemize}
%   \end{thm}

%   \begin{thm}
%     For any $T_1$ space $X$, the following are equivalent:
%     \begin{itemize}
%       \item $S\markwin\rothgame{X}$
%       \item $P\prewin\altrothgame{X}$
%       \item $X$ is almost countable
%       \item $|X|\leq\omega$
%     \end{itemize}
%   \end{thm}

%   \begin{ex}
%     Let $X=\omega_1\cup\{\infty\}$ be a ``weak Lindel\"ofication'' of discrete $\omega_1$ such that open neighborhoods of $\infty$ contain $\omega_1\setminus\omega$. This space is $T_0$ but not $T_1$, and note that $S\markwin\rothgame{X}$ and $|X|>\omega$.
%   \end{ex}

%   \begin{thm}
%     The following are equivalent for points-$G_\delta$ $X$:
%       \begin{enumerate}[(a)]
%         \item $S\win\rothgame{X}$
%         \item $P\win\altrothgame{X}$
%         \item $S\kmarkwin{k}\rothgame{X}$ for some $k\geq 1$
%         \item $P\kmarkwin{k}\altrothgame{X}$ for some $k\geq 1$
%         \item $S\markwin\rothgame{X}$
%         \item $P\prewin\altrothgame{X}$
%         \item $X$ is almost countable
%         \item $|X|\leq\omega$
%       \end{enumerate}
%   \end{thm}

%   \begin{cor}
%     The following are equivalent for compact points-$G_\delta$ $X$:
%       \begin{enumerate}[(a)]
%         \item $S\win\rothgame{X}$
%         \item $P\win\altrothgame{X}$
%         \item $S\kmarkwin{k}\rothgame{X}$ for some $k\geq 1$
%         \item $P\kmarkwin{k}\altrothgame{X}$ for some $k\geq 1$
%         \item $S\markwin\rothgame{X}$
%         \item $P\prewin\altrothgame{X}$
%         \item $X$ is almost countable
%         \item $|X|\leq\omega$
%         \item $C\not\win\rothgame{X}$
%         \item $O\not\win\altrothgame{X}$
%         \item $X$ is Rothberger
%         \item $X$ is scattered
%       \end{enumerate}
%   \end{cor}

%   \begin{defn}
%     The game $\recallgame{m}{\kappa}$ proceeds as follows: during round $0$, player $F$ chooses $F_0\in[\kappa]^m$, followed by player $S$ choosing $x_0\in F_0\cup\{\infty\}$. During round $n+1$, $F$ chooses $F_{n+1}\in[\kappa]^{m^{n+2}}$ such that $F_{n+1}\supset F_n$, followed by $S$ choosing $x_{n+1}\in F_{n+1}\cup\{\infty\}$.

%     $S$ wins the game if $\{x_n : n<\omega\}\supseteq F_0\cup\{\infty\}$, and $F$ wins otherwise.
%   \end{defn}

%   \begin{prop}
%     $S\limitwin\rothgame{\oneptlind{\kappa}} \Rightarrow S\limitwin\recallgame{m}{\kappa}$
%   \end{prop}

%   \begin{prop}
%     $S\kmarkwin{k}\recallgame{m}{\kappa} \Leftrightarrow S\ktactwin{k}\recallgame{m}{\kappa}$
%   \end{prop}