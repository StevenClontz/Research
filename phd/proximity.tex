\documentclass[11pt]{article}

\pdfpagewidth 8.5in
\pdfpageheight 11in

\setlength\topmargin{0in}
\setlength\headheight{0in}
\setlength\headsep{0.4in}
\setlength\textheight{8in}
\setlength\textwidth{6in}
\setlength\oddsidemargin{0in}
\setlength\evensidemargin{0in}
\setlength\parindent{0.25in}
\setlength\parskip{0.1in} 
 
\usepackage{amssymb}
\usepackage{amsfonts}
\usepackage{amsmath}
\usepackage{mathtools}
\usepackage{amsthm}

\usepackage{fancyhdr}

\usepackage{enumerate}

      \theoremstyle{plain}
      \newtheorem{theorem}{Theorem}
      \newtheorem{lemma}[theorem]{Lemma}
      \newtheorem{corollary}[theorem]{Corollary}
      \newtheorem{proposition}[theorem]{Proposition}
      \newtheorem{conjecture}[theorem]{Conjecture}
      \newtheorem{question}[theorem]{Question}
      
      \theoremstyle{definition}
      \newtheorem{definition}[theorem]{Definition}
      \newtheorem{example}[theorem]{Example}
      \newtheorem{game}[theorem]{Game}
      
      \theoremstyle{remark}
      \newtheorem{remark}[theorem]{Remark}



\pagestyle{fancy}
\renewcommand{\headrulewidth}{0.5pt}
\renewcommand{\footrulewidth}{0pt}
\lfoot{\small \jobname.tex -- Updated on \today}
\chead{\small http://github.com/StevenClontz/Research}
\rfoot{\thepage}
\cfoot{}


% Strategy uparrow shortcuts
\newcommand{\win}{\uparrow}
\newcommand{\prewin}{\uparrow_{\text{pre}}}
\newcommand{\markwin}{\uparrow_{\text{mark}}}
\newcommand{\tactwin}{\uparrow_{\text{tact}}}
\newcommand{\ktactwin}[1]{\uparrow_{#1\text{-tact}}}
\newcommand{\kmarkwin}[1]{\uparrow_{#1\text{-mark}}}
\newcommand{\codewin}{\uparrow_{\text{code}}}
\newcommand{\limitwin}{\uparrow_{\text{limit}}}

\newcommand{\oneptcomp}[1]{#1^*}
\newcommand{\oneptlind}[1]{#1^\dagger}

\newcommand{\congame}[2]{Con_{O,P}(#1,#2)}
\newcommand{\clusgame}[2]{Clus_{O,P}(#1,#2)}

\newcommand{\lfkpgame}[1]{LF_{K,P}(#1)}
\newcommand{\lfklgame}[1]{LF_{K,L}(#1)}

\newcommand{\pfgame}[1]{PF_{F,C}(#1)}

\newcommand{\mengame}[1]{Cov_{C,F}(#1)}
\newcommand{\rothgame}[1]{Cov_{C,S}(#1)}
\newcommand{\altrothgame}[1]{Cov_{P,O}(#1)}

\newcommand{\fillgame}[1]{Fill^{\subseteq}_{M,N}(#1)}
\newcommand{\sfillgame}[1]{Fill^{\subsetneq}_{M,N}(#1)}

\newcommand{\kfillgame}[1]{Fill^{\subseteq}_{C,F}(#1)}
\newcommand{\ksfillgame}[1]{Fill^{\subsetneq}_{C,F}(#1)}

\newcommand{\sigmaprodr}[1]{\Sigma\mathbb{R}^{#1}}
\newcommand{\sigmaprodtwo}[1]{\Sigma2^{#1}}

\newcommand{\concat}{^\frown}
\newcommand{\rest}{\restriction}

\newcommand{\cl}[1]{\overline{#1}}

\newcommand{\pow}[1]{\mc{P}(#1)}

\newcommand{\<}{\langle}
\renewcommand{\>}{\rangle}

\newcommand{\mc}[1]{\mathcal{#1}}

\newcommand{\Lim}{\mathrm{Lim}}
\newcommand{\Suc}{\mathrm{Suc}}

\newcommand{\ds}{\displaystyle}

\newcommand{\rank}{\textrm{rank}}

\newcommand{\scish}{$\sigma$-compactish }

\begin{document}

\begin{definition}
  A \textbf{uniform space} $\<X,\mc D\>$ is a set $X$ paired with a filter $\mc D$ (called its \textbf{uniformity}) of relations (called \textbf{entourages}) on $X$ such that for each entourage $D\in \mc D$:
    \begin{itemize}
      \item $D$ is reflexive, i.e., the diagonal $\Delta\subseteq D$.
      \item Its inverse $D^{-1}=\{\<y,x\>:\<x,y\>\in D\}\in \mc D$.
      \item There exists $\frac{1}{2}D\in\mc D$ such that 
        \[
          2(\frac{1}{2}D)=\frac{1}{2}D\circ\frac{1}{2}D=\{\<x,z\>:\exists y(\<x,y\>,\<y,z\>\in\frac{1}{2}D)\}\subseteq D
        \]
    \end{itemize}
  Note that since $\mc D$ is a filter, for each $D\in\mc D$, the symmetric relation $D\cap D^{-1}\in \mc D$.
\end{definition}

\begin{proposition}
For each $D\in\mc D$ and $n<\omega$ there exists $\frac{1}{2^{n+1}}D\in\mc D$ such that 
  \[2(\frac{1}{2^{n+1}}D)=\frac{1}{2^{n+1}}D\circ \frac{1}{2^{n+1}}D\subseteq \frac{1}{2^{n}}D\]
and if $2E\subseteq \frac{1}{2^{n}}D$, then $E\subseteq \frac{1}{2^{n+1}}D$.
\end{proposition}

\begin{definition}
  For an entourage $D\in\mc D$, let $D[x]=\{y:(x,y)\in D\}$ be the \textbf{$D$-neighborhood} of $x$. The \text{uniform topology} for a uniform space $\<X,\mc D\>$ is generated by the base $\{D[x]:x\in X,D\in\mc D\}$.
\end{definition}

\begin{theorem}
  A space $X$ is uniformizable (its topology is the uniform topology for some uniformity) if and only if $X$ is completely regular ($T_{3\frac{1}{2}}$).
\end{theorem}

\begin{proposition}
  If $X$ is a uniform space, then for all $x\in X$ and symmetric entourages $D$:
    \[
      x\in \frac{1}{2}D[y]\text{ and } y\in\frac{1}{2}D[z] \Rightarrow x\in D[z]
    \]
  and
    \[
      \frac{1}{2}D[x]\subseteq \cl{\frac{1}{2}D[x]}\subseteq D[x]
    \]
\end{proposition}

\begin{proof}
  The first is by definition of $\frac{1}{2}D$.

  If $z\in \cl{\frac{1}{2}D[x]}$, it follows that there is $y\in \frac{1}{2}D[x]\cap\frac{1}{2}D[z]$ since $\frac{1}{2}D[z]$ is an open neighborhood of $z$. Thus $(x,z)\in D \Rightarrow z\in D[x] \Rightarrow \cl{\frac{1}{2}D[x]}\subseteq D[x]$.
\end{proof}

\begin{definition}
  For a uniform space $X$, Bell's proximity game proceeds as follows. 

  In round $0$, $\pl D$ chooses an entourage $D_0$, followed by $\pl P$ choosing a point $p_0\in X$. 

  In round $n+1$, $\pl D$ chooses an entourage $D_{n+1}\subseteq D_n$, followed by $\pl P$ choosing a point $p_{n+1}\in 4D_n[p_n]$.

  Player $\pl D$ wins if either $\bigcap_{n<\omega} 4D_n[p_n] = \emptyset$ or $\<p_0,p_1,\dots\>$ converges.
\end{definition}

\begin{definition}
  For a uniform space $X$, the simplified proximal game $\proxgame{X}$ can be defined as follows:

  In round $0$, $\pl D$ chooses a symmetric entourage $D_0$, followed by $\pl P$ choosing a point $p_0\in X$. 

  In round $n+1$, $\pl D$ chooses a symmetric entourage $D_{n+1}$, followed by $\pl P$ choosing a point $p_{n+1}\in\left(\bigcap_{m\leq n}D_m\right)[p_n]$.

  Player $\pl D$ wins if either $\bigcap_{n<\omega}\left(\bigcap_{m\leq n} D_m\right)[p_n]=\emptyset$ or $\<p_0,p_1,\dots\>$ converges.
\end{definition}

\begin{theorem}
  $\pl D$ has a winning perfect-information strategy in Bell's game if and only if $\pl D\win\proxgame{X}$.
\end{theorem}

\begin{proof}
  Let $\sigma$ be a winning perfect information strategy for $\pl D$ in Bell's game. We define a perfect information strategy $\tau$ in the simplified game to yield symmetric entourages $\tau(p\rest n)=\sigma(p\rest n)\cap(\sigma(p\rest n))^{-1}$ for all partial attacks $p\rest n$. Note that $\tau(p\rest n)=\bigcap_{m\leq n}\tau(p\rest m)$.

  If $p$ attacks $\tau$ in the simplified game, $p(n+1)\in\left(\bigcap_{m\leq n}\tau(p\rest m)\right)[p(n)]=\tau(p\rest n)[p(n)]\subseteq\sigma(p\rest n)[p(n)]\subseteq 4\sigma(p\rest n)[p(n)]$, so $p$ attacks $\sigma$ in Bell's game. Thus either $p$ converges, or 
    \[
      \emptyset
        =
      \bigcap_{n<\omega} 4\sigma(p\rest n)[p(n)]
        \supseteq 
      \bigcap_{n<\omega}\tau(p\rest n)[p(n)]
        =
      \bigcap_{n<\omega}\left(\bigcap_{m\leq n}\tau(p\rest n)\right)[p(n)]
    \]

  For the other direction, let $\sigma$ be a winning perfect information strategy for $\pl D$ in the simplified game such that $\sigma(p\rest n)=\bigcap_{m\leq n}\sigma(p\rest m)$. Define the perfect information strategy $\tau$ in Bell's Game such that $4\tau(p\rest n)\subseteq \sigma(p\rest n)$ and $\tau(p\rest n)=\bigcap_{m\leq n}\tau(p\rest m)$ for all partial attacks $p\rest n$.

  If $p$ attacks $\tau$ in Bell's game, $p(n)\in 4\tau(p\rest n)\subseteq\sigma(p\rest n)=\bigcap_{m\leq n}\sigma(p\rest m)$, so $p$ attacks $\sigma$ in the simplified game. Thus either $p$ converges, or
    \[
      \emptyset
        =
      \bigcap_{n<\omega}\left(\bigcap_{m\leq n}\sigma(p\rest n)\right)[p(n)]
        =
      \bigcap_{n<\omega}\sigma(p\rest n)[p(n)]
        \supseteq
      \bigcap_{n<\omega}4\tau(p\rest n)[p(n)]
        \supseteq
      \bigcap_{n<\omega}\tau(p\rest n)[p(n)]
    \]
\end{proof}

\begin{proposition}
$\pl P$ has a winning perfect-information strategy in Bell's game if and only if $\pl P \win \proxgame{X}$.
\end{proposition}

\begin{proof}
  Similar to the previous.
\end{proof}

\begin{definition}
  A uniform space is \textbf{proximal} if $\pl D \win \proxgame{X}$.
\end{definition}

\begin{definition}
  For a space $X$ and a point $x\in X$, the \textbf{$W$-convergence-game} $\congame{X}{x}$ proceeds as follows. 

  In round $0$, $\pl O$ chooses a neighborhood $U_n$ of $x$, followed by $\pl P$ choosing a point $p_n\in \bigcap_{m\leq n}U_m$.

  Player $\pl O$ wins if $\<p_0,p_1,\dots\>$ converges.
\end{definition}

\begin{definition}
  A space is \textbf{$W$} if $\pl O \win \congame{X}{x}$ for all $x\in X$.
\end{definition}

\begin{definition}
  For each finite tuple $(m_0,\dots,m_{n-1})$, we define the \textbf{$k$-tactical fog-of-war}
    \[
      T_k(\<m_0,\dots,m_{n-1}\>)=\<m_{n-k},\dots,m_{n-1}\>
    \]
  and the \textbf{$k$-Mark\"ov fog-of-war}
    \[
      M_k(\<m_0,\dots,m_{n-1}\>)=\<\<m_{n-k},\dots,m_{n-1}\>,n\>
    \]

  So $P\ktactwin{k}G$ if and only if there exists a winning strategy for $P$ of the form $\sigma\circ T_k$, and $P\kmarkwin{k}G$ if and only if there exists a winning strategy of the form $\sigma\circ M_k$.
\end{definition}

\begin{theorem}
For all $x\in X$:
  \begin{itemize}
    \item
      $\pl D\win \proxgame{X} \Rightarrow \pl O \win \congame{X}{x}$
    \item
      $\pl D\ktactwin{2k} \proxgame{X} \Rightarrow \pl O \ktactwin{k} \congame{X}{x}$
    \item
      $\pl D\kmarkwin{2k} \proxgame{X} \Rightarrow \pl O \kmarkwin{k} \congame{X}{x}$
  \end{itemize}
\end{theorem}

\begin{proof}
Let $\sigma$ witness $\pl D \ktactwin{2k}\proxgame{X}$ (resp. $\pl D \kmarkwin{2k}\proxgame{X}$, $\pl D\win\proxgame{X}$). We define the $k$-tactical (resp. $k$-Mark\"ov, perfect info) strategy $\tau$ such that
  \[
    \tau\circ L_k(p)
      =
    \sigma\circ L_{2k}(\<x,p(0),\dots,x,p(|p|-1)\>)[x]
      \cap
    \sigma\circ L_{2k}(\<x,p(0),\dots,x,p(|p|-1),x\>)[x]
  \]
where $L_{2k}$ is the $2k$-tactical fog-of-war (resp. $2k$-Mark\"ov fog-of-war, identity) and $L_{k}$ is the $k$-tactical fog-of-war (resp. $k$-Mark\"ov fog-of-war, identity).

Let $p$ attack $\tau$. Consider the attack $q$ against the winning strategy $\sigma$ such that $q(2n)=x$ and $q(2n+1)=p(n)$, and let $D_n=\sigma\circ L_{2k}(q)$ and $E_n=\bigcap_{m\leq n}D_n$.

Certainly, $x\in E_{2n}[x]= E_{2n}[q(2n)]$ for any $n<\omega$. Note also for any $n<\omega$ that 
    \[
      p(n) \in 
      \bigcap_{m\leq n}\tau\circ L_k(p\rest n)
    \]
    \[
      =
      \bigcap_{m\leq n}\left(
        \sigma\circ L_{2k}(\<x,p(0),\dots,x,p(m-1)\>)[x]\cap
        \sigma\circ L_{2k}(\<x,p(0),\dots,x,p(m-1),x\>)[x]
      \right)
    \]
    \[
      =
      \bigcap_{m\leq n}\left(
        D_{2m}[x]\cap
        D_{2m+1}[x]
      \right) =
      \bigcap_{m\leq 2n+1} D_m[x]=E_{2n+1}[x]
    \]
so by the symmetry of $E_{2n+1}$, $x\in E_{2n+1}[p(n)]= E_{2n+1}[q(2n+1)]$. Thus $x\in \bigcap_{n<\omega} E_n[q(n)]\not=\emptyset$, and since $\sigma$ is a winning strategy, the attack $q$ converges. Since $q(2n)=x$, $q$ must converge to $x$. Thus its subsequence $p$ converges to $x$, and $\tau$ is a winning strategy in $\congame{X}{x}$.
\end{proof}

\begin{corollary}
For all $x\in X$:
  \begin{itemize}
    \item
      $\pl D\ktactwin{k} \proxgame{X} \Rightarrow \pl O \ktactwin{k} \congame{X}{x}$
    \item
      $\pl D\kmarkwin{k} \proxgame{X} \Rightarrow \pl O \kmarkwin{k} \congame{X}{x}$
  \end{itemize}
\end{corollary}

\begin{corollary}
  All proximal spaces are $W$-spaces.
\end{corollary}

\begin{theorem}
  Let $X\cup\{\infty\}$ be a uniformizable space such that $X$ is discrete. Then
  \begin{itemize}
    \item
      $\pl O\win \congame{X\cup\{\infty\}}{\infty} \Rightarrow \pl D \win \proxgame{X\cup\{\infty\}}$
    \item
      $\pl O\ktactwin{k} \congame{X\cup\{\infty\}}{\infty} \Rightarrow \pl D \ktactwin{k} \proxgame{X\cup\{\infty\}}$
    \item
      $\pl O\kmarkwin{k} \congame{X\cup\{\infty\}}{\infty} \Rightarrow \pl D \kmarkwin{k} \proxgame{X\cup\{\infty\}}$
  \end{itemize}
\end{theorem}

\begin{proof}
  Note that the topology on $X\cup\{\infty\}$ is induced by the uniformity with equivalence relation entourages $D(U)=\Delta\cup U^2$ for each open neighborhood $U$ of $\infty$.

  Let $\sigma$ witness $\pl D \ktactwin{k}\congame{X\cap\{\infty\}}{\infty}$ (resp. $\pl D \kmarkwin{k}\congame{X\cap\{\infty\}}{\infty}$, $\pl D\win\congame{X\cap\{\infty\}}{\infty}$). We define the $k$-tactical (resp. $k$-Mark\"ov, perfect info) strategy $\tau$ such that
    \[
      \tau\circ L(p)
        =
      D(\sigma\circ L(p))
    \]
  where $L$ is the $k$-tactical fog-of-war (resp. $k$-Mark\"ov fog-of-war, identity).

  Let $p\in (X\cup\{\infty\})^\omega$ attack $\tau$ such that $\bigcap_{n<\omega}\tau(p\rest n)[p(n)]\not=\emptyset$.

  If $\infty\in\bigcap_{n<\omega}\tau(p\rest n)[p(n)]$, it follows that $p$ is an attack on $\sigma$. Since $\sigma$ is a winning strategy, it follows that $q$ and its subsequence $p$ must coverge to $\infty$.

  Otherwise, $\infty\not\in\tau(p\rest N)[p(N)]$ for some $N<\omega$, and then $\tau(p\rest N)[p(N)]=\{p(N)\}$ implies $p\to p(N)$.

  Thus $\tau\circ L$ is a winning strategy.
\end{proof}

\begin{corollary}
  Let $X\cup\{\infty\}$ be a uniformizable space such that $X$ is discrete. Then
  \begin{itemize}
    \item
      $\pl O\win \congame{X\cup\{\infty\}}{\infty} \Leftrightarrow \pl D \win \proxgame{X\cup\{\infty\}}$
    \item
      $\pl O\ktactwin{k} \congame{X\cup\{\infty\}}{\infty} \Leftrightarrow \pl D \ktactwin{k} \proxgame{X\cup\{\infty\}}$
    \item
      $\pl O\kmarkwin{k} \congame{X\cup\{\infty\}}{\infty} \Leftrightarrow \pl D \kmarkwin{k} \proxgame{X\cup\{\infty\}}$
  \end{itemize}
\end{corollary}

\begin{proposition} For any $x\in X$ and $k\geq 1$,
  \begin{itemize}
    \item
      $\pl O\ktactwin{k}\congame{X}{x} \Leftrightarrow \pl O\tactwin\congame{X}{x}$
    \item
      $\pl O\kmarkwin{k}\congame{X}{x} \Leftrightarrow \pl O\markwin\congame{X}{x}$
  \end{itemize}
\end{proposition}

\begin{proof}
  If $\sigma$ witnesses $\pl O\ktactwin{k}\congame{X}{x}$, let $\tau(\emptyset)=\sigma(\emptyset)$ and
    \[
      \tau(\<q\>)
        =
      \bigcap_{i< k}
      \sigma(\<\underbrace{x,\dots,x}_{k-i-1},q,\underbrace{x,\dots,x}_{i}\>)
    \]

  This is easily verified to be a winning strategy. The proof for $\pl O\kmarkwin{k}\congame{X}{x}$ is analogous.
\end{proof}

\begin{corollary}
  Let $X\cup\{\infty\}$ be a uniformizable space such that $X$ is discrete, and $k\geq 1$. Then
  \begin{itemize}
    \item
      $\pl D\ktactwin{k}\proxgame{X\cup\{\infty\}} \Leftrightarrow O\tactwin\proxgame{X\cup\{\infty\}}$
    \item
      $\pl D\kmarkwin{k}\proxgame{X\cup\{\infty\}} \Leftrightarrow O\markwin\proxgame{X\cup\{\infty\}}$
  \end{itemize}
\end{corollary}

\begin{proposition} For any uniform space $X$,
  \begin{itemize}
    \item
      $\pl O\ktactwin{k}\proxgame{X} \Leftrightarrow \pl O\ktactwin{2}\proxgame{X}$
    \item
      $\pl O\kmarkwin{k}\proxgame{X} \Leftrightarrow \pl O\kmarkwin{2}\proxgame{X}$
  \end{itemize}
\end{proposition}

\begin{proof}
  If $\sigma$ witnesses $\pl O\ktactwin{k}\congame{X}{x}$, let $\tau(\emptyset)=\sigma(\emptyset)$ and
    \[
      \tau(\<q\>)
        =
      \bigcap_{i<k}
      \sigma(\<\underbrace{q,\dots,q}_{i}\>)
    \]
    \[
      \tau(\<q,q'\>)
        =
      \bigcap_{i< k}
      \sigma(\<\underbrace{q,\dots,q}_{k-i},\underbrace{q',\dots,q'}_{i}\>)
    \]

  This is easily verified to be a winning strategy. The proof for $\pl O\kmarkwin{k}\congame{X}{x}$ is analogous.
\end{proof}

\newpage

\begin{definition}
  The absolute proximal game $\aproxgame{X}$ is analogous to $\proxgame{X}$, except $\pl D$ may only win if $p$ converges.
\end{definition}

\begin{definition}
  A \textbf{uniformly locally compact} space is a uniformizable space with a \textbf{uniformly compact entourage} $M$ where $\cl{M[x]}$ is compact for all $x$.
\end{definition}

\begin{theorem}
  For any uniformly locally compact space $X$,
      $\pl D\win \proxgame{X} \Leftrightarrow \pl D\win \aproxgame{X}$
\end{theorem}

\begin{proof}
  Let $M$ be a uniformly locally compact entourage. Let $\sigma$ witness $\pl D\win \proxgame{X}$ such that $\sigma(a)\subseteq M$ always (so $\cl{\sigma(a)[x]}\subseteq\cl{M[x]}$ is compact), and $a\supseteq b$  implies $\sigma(a)\subseteq\frac{1}{4}\sigma(b)$.

  Let $\tau(p\rest n)=\frac{1}{2}\sigma(p\rest n)$. If $p$ attacks $\tau$ in $\aproxgame{X}$, then
    \[
      p(n+1)
        \in
      \tau(p\rest n)[p(n)]
        =
      \frac{1}{2}\sigma(p\rest n)[p(n)]
    \]

    and for

    \[
      x
        \in
      \cl{\sigma(p\rest (n+1))[p(n+1)]}
        \subseteq
      \cl{\frac{1}{4}\sigma(p\rest n)[p(n+1)]}
        \subseteq
      \frac{1}{2}\sigma(p\rest n)[p(n+1)]
    \]

  we can conclude $x\in\sigma(p\rest n)[p(n)]$. Thus

    \[
      \sigma(p\rest (n+1))[p(n+1)]
        \subseteq
      \cl{\sigma(p\rest (n+1))[p(n+1)]}
        \subseteq
      \sigma(p\rest n)[p(n)]
    \]

  Finally, note that $p$ attacks the winning strategy $\sigma$ in $\proxgame{X}$, but since the intersection of a chain of nonempty compact sets is nonempty:

    \[
      \bigcap_{n<\omega} \sigma(p\rest n)[p(n)]
        =
      \bigcap_{n<\omega} \cl{\sigma(p\rest n)[p(n)]}
        \not=
      \emptyset
    \]

  We conclude that $p$ converges.
\end{proof}

\begin{corollary}
  A uniformaly locally compact space $X$ is proximal if and only if $\pl D \win \aproxgame{X}$.
\end{corollary}

\begin{theorem}
  For any uniformly locally compact proximal space $X$, $\pl O\win \clusgame{X}{H}$ for all compact $H\subseteq X$.
\end{theorem}

\begin{proof}
  Let $\sigma$ witness $\pl D \win \aproxgame{X}$ such that $p\supseteq q$ implies $\sigma(p)\subseteq \frac{1}{4}\sigma(q)$.

  Let $o(t)$ be the subsequence of $t$ consisting of its odd-indexed terms.

  We define $T(\emptyset)$, etc. as follows:

  \begin{itemize}
    \item Let $\emptyset\in T(\emptyset)$.
    \item Choose $m_\emptyset<\omega$, $h_{\emptyset,i}\in H$ for $i<m_\emptyset$, and $h_{\emptyset,i,j}\in H\cap\cl{\frac{1}{4}\sigma(\emptyset)[h_{\emptyset,i}]}$ for $i,j<m_\emptyset$ such that
      \[
        \{\frac{1}{4}\sigma(\emptyset)[h_{\emptyset,i}]:i<m_\emptyset\}
      \]
    is a cover for $H$ and such that for each $i<m_\emptyset$
      \[
        \{\frac{1}{4}\sigma(\<h_{\emptyset,i}\>)[h_{\emptyset,i,j}]:j<m_\emptyset\}
      \]
    is a cover for $H\cap\cl{\frac{1}{4}\sigma(\emptyset)[h_{\emptyset,i}]}$.
    \item Let $\<i\>\in T(\emptyset)$, $\<i,h_{\emptyset,i}\>\in T(\emptyset)$, and $\<i,h_{\emptyset,i},j\>\in T(\emptyset)$ for $i,j<m_\emptyset$.
  \end{itemize}

  Suppose $T(a)$, etc. are defined. We then define $T(a\concat\<x\>)$, etc. for
    \[
      x\in \bigcup_{s\concat\<i,h_{s,i},j\>\in\max(T(a))} \frac{1}{4}\sigma(o(s)\concat\<h_{s,i}\>)[h_{s,i,j}]
    \]
  as follows:

  \begin{itemize}
    \item Let $T(a)\subseteq T(a\concat\<x\>)$.
    \item Choose $t=s\concat\<i,h_{s,i},j,x\>$ such that $s\concat\<i,h_{s,i},j\>\in\max(T(a))$ and $x\in \frac{1}{4}\sigma(o(s)\concat\<h_{s,i}\>)[h_{s,i,j}]$. 
    \item Note that, assuming $o(s)\concat\<h_{s,i}\>$ is a legal partial attack against $\sigma$, then
      \[
        x
          \in 
        \frac{1}{4}\sigma(o(s)\concat\<h_{s,i}\>)[h_{s,i,j}]
          \subseteq
        \frac{1}{4}\sigma(o(s))[h_{s,i,j}]
      \]
    and
      \[
        h_{s,i,j}
          \in 
        \cl{\frac{1}{4}\sigma(o(s))[h_{s,i}]}
          \subseteq 
        \frac{1}{2}\sigma(o(s))[h_{s,i}]
      \]
    implies
      \[
        x
          \in 
        \sigma(o(s))[h_{s,i}]
      \]
    and thus $o(s)\concat\<h_{s,i},x\>=o(t)$ is a legal partial attack against $\sigma$.
    \item Choose $m_t<\omega$, $h_{t,k}\in H\cap \cl{\frac{1}{4}\sigma(o(s)\concat\<h_{s,i}\>)[h_{s,i,j}]}$ for $k<m_t$, and $h_{t,k,l}\in H\cap\cl{\frac{1}{4}\sigma(t)[h_{t,k}]}$ for $k,l<m_t$ such that
      \[
        \{\frac{1}{4}\sigma(o(t))[h_{t,k}]:k<m_t\}
      \]
    is a cover for $H\cap \cl{\frac{1}{4}\sigma(o(s)\concat\<h_{s,i}\>)[h_{s,i,j}]}$ and such that for each $k<m_t$
      \[
        \{\frac{1}{4}\sigma(o(t)\concat\<h_{t,k}\>)[h_{t,i,j}]:l<m_t\}
      \]
    is a cover for $H\cap\cl{\frac{1}{4}\sigma(o(t))[h_{t,k}]}$.
    \item Note that, assuming $o(t)$ is a legal partial attack against $\sigma$, then
      \[
        h_{t,k}
          \in 
        \cl{\frac{1}{4}\sigma(o(s)\concat\<h_{s,i}\>)[h_{s,i,j}]}
          \subseteq
        \frac{1}{2}\sigma(o(s)\concat\<h_{s,i}\>)[h_{s,i,j}]
      \]
    and
      \[
        x
          \in 
        \frac{1}{4}\sigma(o(s)\concat\<h_{s,i}\>)[h_{s,i,j}]
      \]
    implies
      \[
        h_{t,k}
          \in
        \sigma(o(s)\concat\<h_{s,i}\>)[x]
      \]
    and thus $o(t)\concat\<h_{t,k}\>$ is a legal partial attack against $\sigma$.
    \item Let $t\in T(a\concat\<x\>)$, $t\concat\<k\>\in T(a\concat\<x\>)$, $t\concat\<k,h_{t,k}\>\in T(a\concat\<x\>)$, and $t\concat\<k,h_{t,k},l\>\in T(a\concat\<x\>)$ for $k,l<m_t$.
    \item Note that assuming
      \[
        \{\frac{1}{4}\sigma(o(s)\concat\<h_{s,i}\>)[h_{s,i,j}] : s\concat\<i,h_{s,i},j\>\in\max(T(a))\}
      \]
    covers $H$, then since
      \[
        \{\frac{1}{4}\sigma(o(t)\concat\<h_{t,k}\>)[h_{t,k,l}] : s\concat\<i,h_{s,i},j,x,k,h_{t,k},l\>\in\max(T(a\concat\<x\>))\setminus\max(T(a))\}
      \]
    covers $H\cap \frac{1}{4}\sigma(o(s)\concat\<h_{s,i}\>)[h_{s,i,j}]$, we have that
      \[
        \{\frac{1}{4}\sigma(o(t)\concat\<h_{t,k}\>)[h_{t,k,l}] : t\concat\<k,h_{t,k},l\>\in\max(T(a\concat\<x\>))\}
      \]
    covers $H$.
  \end{itemize}

  With this we may define the perfect information strategy $\tau$ for $\pl O$ in $\congame{X}{H}$ such that:
  \[
    \tau(p\rest n) = \bigcup_{s\concat\<i,h_{s,i},j\>\in\max(T(p\rest n))} \frac{1}{4}\sigma(o(s)\concat\<h_{s,i}\>)[h_{s,i,j}]
  \]

  If $p$ attacks $\tau$, then it follows that $T(p\rest n)$ is defined for all $n<\omega$, so let $T(p)=\bigcup_{n<\omega} T(p\rest n)$. We note $T(p)$ is an infinite tree with finite levels:
    \begin{itemize}
      \item $\emptyset$ has exactly $m_\emptyset$ successors $\<i\>$.
      \item $s\concat\<i\>$ has exactly one successor $t\concat\<i,h_{s,i}\>$
      \item $s\concat\<i,h_{s,i}\>$ has exactly $m_s$ successors $t\concat\<i,h_{s,i},j\>$
      \item $s\concat\<i,h_{s,i},j\>$ has either no successors or exactly one successor $t\concat\<i,h_{s,i},j,x\>$
      \item $t=s\concat\<i,h_{s,i},j,x\>$ has exactly $m_t$ successors $t\concat\<k\>$
    \end{itemize}

  Let $q'=\<i_0,h_0,j_0,x_0,i_1,h_1,j_1,x_1,\dots\>$ correspond to this infinite branch in $T(p)$, and let $q=o(q')=\<h_0,x_0,h_1,x_1,\dots\>$. Note that by the construction of $T(p)$, $q$ is an attack on the winning strategy $\sigma$ in $\aproxgame{X}$, so it must converge. Since every other term of $q$ is in $H$, it must converge to $H$. Then since $q$ is a subsequence of $p$, $p$ must cluster at $H$.
\end{proof}


\begin{corollary}
  For any uniformly locally compact proximal space, $\pl O\win \congame{X}{H}$ for all compact $H\subseteq X$.
\end{corollary}

\begin{proof}
  $\pl O \win \congame{X}{H}$ if and only if $\pl O \win \clusgame{X}{H}$.
\end{proof}

\begin{corollary}
  A compact uniform space $X$ is Corson compact if and only if it is proximal.
\end{corollary}

\begin{proof}
  A characterization of Corson compact is having a $W$-set diagonal. If $X$ is proximal compact, then $X^2$ is proximal compact, and its compact diagonal is a $W$-set.
\end{proof}









\newpage

\begin{theorem}
  $\pl O \prewin \congame{X}{H}$ if and only if there exists a countable base 
  around $H$.
\end{theorem}

\begin{proof}
  Let $\{U_n:n<\omega\}$ be a countable base around $H$. We define the 
  predetermined strategy $\sigma(n)=\bigcap_{m\leq n}U_m$. Let $p$ attack 
  $\sigma(n)$ - then if $U$ is any neighborhood of $H$, we may choose 
  $H\subseteq U_m\subseteq U$, and note that $\sigma(n)\subseteq U_m$ for 
  $n\geq m$, and thus $p(n)\in U_m\subseteq U$ for all $n\geq m$. Thus 
  $\sigma$ is a winning strategy.

  For the other direction, suppose there does not exist a countable base 
  around $H$, and let $\sigma(n)$ be an arbitrary predetermined strategy. 
  Since $\{\bigcap_{m\leq n}\sigma(m):n<\omega\}$ is not a countable base 
  around $H$, we may choose an open set $U$ around $H$ such that 
  $\bigcap_{m\leq n}\sigma(m)\not\subseteq U$ for all $n<\omega$. We may 
  easily verify that if $p(n)\in\bigcap_{m\leq n}\sigma(m)\setminus U$ for 
  all $n<\omega$, then $p$ is a successful counterattack to $\sigma$.
\end{proof}

\begin{corollary}
  $X$ is first countable if and only if $\pl O \prewin \congame{X}{x}$ for 
  all $x\in X$
\end{corollary}

\begin{corollary}
  $\pl D \prewin \proxgame{X}$ implies $X$ is first countable.
\end{corollary}

\begin{definition}
  Scattered Eberlein compact spaces are known as \term{strong Eberlein 
  compact} spaces.
\end{definition}

\begin{theorem}[folklore]
  Scattered compact first-countable spaces are metrizable.
\end{theorem}

\begin{corollary}
  If $X$ is scattered compact and $\pl O \prewin\congame{X}{x}$ for all
  $x\in X$ (or $\pl D \prewin\proxgame{X}$), then $X$ is metrizable.
\end{corollary}


\begin{example}
  $\pl D \not\prewin \proxgame{\oneptcomp\omega_1}$
\end{example}

\begin{proof}
  There does not exist a countable base around $\infty$, so 
    $\pl O \not\prewin \congame{X}{\omega_1}$.
\end{proof}

\begin{example}
  $\pl O \tactwin \congame{\oneptcomp\kappa}{\infty}$ and 
  $\pl D \tactwin \proxgame{\oneptcomp\kappa}$ for all cardinals $\kappa$
\end{example}

\begin{proof}
  For $\congame{\oneptcomp\kappa}{\infty}$, let 
    $\sigma()=\sigma(\infty)=\oneptcomp\kappa$ 
  and 
    $\sigma(x)=\oneptcomp\kappa\setminus\{x\}$ 
  otherwise.
\end{proof} 


\newpage

\begin{theorem}
  If $H$ is a closed subset of $X$, then 
    $
      \pl D \limitwin \proxgame{X}
        \Rightarrow
      \pl D \limitwin \proxgame{H}
    $
  where $\limitwin$ is any of $\win$, $\ktactwin{k}$, or $\kmarkwin{k}$.
\end{theorem}

\begin{proof}
  Let $\sigma\circ L$ witness $\pl D \limitwin \proxgame{X}$. We define
  $\tau\circ L$ for $\pl D$ in $\proxgame{H}$ as follows:
    \[
      \tau\circ L(p\rest n)
        =
      \sigma\circ L(p\rest n)
        \cap
      H^2
    \]

  Let $p$ attack $\tau\circ L$. $p$ also attacks the winning strategy 
  $\sigma\circ L$, so either 
    \[
      \bigcap_{n<\omega}
      \left(\bigcap_{m\leq n}\tau\circ L(p\rest n)\right)
      [p(n)]
        \subseteq
      \bigcap_{n<\omega}
      \left(\bigcap_{m\leq n}\sigma\circ L(p\rest n)\right)
      [p(n)]
        =
      \emptyset
    \]
  or $p$ converges in $X$, and thus converges in $H$.
\end{proof}

\begin{theorem}
  If $\pl D\limitwin \proxgame{X_i}$ for $i<\omega$, then
  $\pl D\limitwin \proxgame{\prod_{i<\omega}X_i}$, where $\limitwin$ is either
  $\win$ or $\kmarkwin{k}$.
\end{theorem}

\begin{proof}
  A subbase for $\prod_{i<\omega}X_i$ is
    \[
      \{\pi_i^{-1}(D):i<\omega, D\in\mc D_i\}
    \]
  where $\pi_i$ is the natural projection from 
  $\left(\prod_{i<\omega}X_i\right)^2$ onto $X_i^2$. (See Bell.)

  For $p\in \left(\prod_{i<\omega}X_i\right)^\omega$, let $p_i\in X_i^\omega$
  such that $p_i(n)=p(n)(i)$.

  Let $\sigma_i\circ L$ witness $\pl D\limitwin \proxgame{X_i}$ for
  $i<\omega$, and assume without loss of generality that 
  $\sigma_i\circ L$ always yields $X_i^2$ before round $i$.

  Then we define the strategy $\tau\circ L$ for $\pl D$ in
  $\proxgame{\prod_{i<\omega}X_i}$ as follows:
    \[
      \tau\circ L(p\rest n)
        =
      \bigcap_{i\leq n} \pi_i^{-1}(\sigma_i\circ L(p_i\rest n))
    \]

  Let $p$ attack $\tau\circ L$. If
    $
      \bigcap_{n<\omega}
      \left(\bigcap_{m\leq n}\sigma_i(p_i\rest n)\right)[p_i(n)]
      = \emptyset
    $
  for any $i<\omega$, it easily follows that 
    $
      \bigcap_{n<\omega}
      \left(\bigcap_{m\leq n}\tau(p\rest n)\right)[p(n)]
      = \emptyset
    $.

  Otherwise, we assume that for each $i<\omega$, $p_i$ converges to some
  $x_i\in X_i$. Thus $p$ converges to $x=\<x_0,x_1,\dots\>$.
\end{proof}

Note: I expect I should be able to do some clever things 
assuming $S(\kappa,\omega,\omega)$ to get a similar result for sigma
products of dimension $\kappa$.

\begin{example}
  $\pl D \markwin \proxgame{(\oneptcomp\kappa)^\omega}$
\end{example}

\begin{proof}
  $\pl D \tactwin \proxgame{\oneptcomp\kappa}$ + previous result
\end{proof}




\newpage

\begin{theorem}
  For any predetermined absolutely proximal space $X$, 
  $\pl O\prewin \congame{X}{H}$ for all compact $H\subseteq X$.
\end{theorem}

\begin{proof}
  Let $\sigma(n)$ be a winning predetermined strategy for $\pl D$ in the 
  absolutely proximal game such that 
  $\sigma(n+1)\subseteq \frac{1}{4}\sigma(n)$.
  For a given tree $T$, let $\max(T)$ denote its maximal nodes.

  \bigskip

  First we define $T(0)\subseteq \omega^{\leq 2}$.

  \begin{itemize}
    \item Let $\emptyset\in T(0)$.
    \item Choose 
      $m_{\emptyset}<\omega$, 
      $h_{\<i\>}\in H$ for $i<m_{\emptyset}$, and 
      $h_{\<i,j\>}\in 
        H\cap\cl{\frac{1}{4}\sigma(0)[h_{\<i\>}]}$ 
      for $i,j<m_{\emptyset}$ such that
        \[
          \left\{\frac{1}{4}\sigma(0)[h_{\<i\>}]:i<m_{\emptyset}\right\}
        \]
      is a cover for $H$ and such that for each $i<m_{\emptyset}$
        \[
          \left\{\frac{1}{4}\sigma(1)[h_{\<i,j\>}]:j<m_{\emptyset}\right\}
        \]
      is a cover for 
      $H\cap\cl{\frac{1}{4}\sigma(0)[h_{\<i\>}]}$.
    \item Let $\<i\>$ and $\<i,j\>$ be in $T(0)$ for $i,j<m_{\emptyset}$.
  \end{itemize}

  \bigskip

  Now suppose $T(n)\subseteq \omega^{\leq 2n+2}$ is defined. 
  We then define $T(n+1)\subseteq \omega^{\leq 2n+4}$ as follows:

  \begin{itemize}
    \item Let $T(n)\subseteq T(n+1)$.
    \item For each $t\in\max(T(n))$, choose 
        $m_t<\omega$, 
        $h_{t\concat\<i\>}\in H\cap \cl{\frac{1}{4}\sigma(2n+2)[h_t]}$ 
          for $i<m_t$, 
        and 
          $
            h_{t\concat\<i,j\>} 
              \in 
            H\cap\cl{\frac{1}{4}\sigma(2n+3)[h_{t\concat\<i\>}]}
          $ 
        for $i,j<m_t$ such that
      \[
        \left\{\frac{1}{4}\sigma(2n+2)[h_{t\concat\<i\>}]:i<m_t\right\}
      \]
    is a cover for $H\cap \cl{\frac{1}{4}\sigma(2n+1)[h_t]}$ 
    and such that for each $i<m_t$
      \[
        \left\{\frac{1}{4}\sigma(2n+3)[h_{t\concat\<i,j\>}]:i<m_t\right\}
      \]
    is a cover for $H\cap\cl{\frac{1}{4}\sigma(2n+2)[h_{t\concat\<i\>}]}$.
    \item For each $t\in\max(T(n))$ and each $i,j<m_t$, put $t\concat\<i\>$
      and $t\concat\<i,j\>$ in $T(n+1)$.
  \end{itemize}

  \bigskip

  We now define the predetermined strategy $\tau$ for $\pl O$ in 
  $\clusgame{X}{H}$ such that:
  \[
    \tau(n) 
      = 
    \bigcup_{t\in\max(T(n))} \frac{1}{4}\sigma(2n+1)[h_t]
  \]

  In order for $\tau$ to be a legal strategy, we must show that $\tau(n)$ 
  contains $H$. For $n=0$,
  \[
    \tau(0)
      =
    \bigcup_{i,j<m_\emptyset} \frac{1}{4}\sigma(1)[h_{\<i,j\>}]
  \]

  Since $\left\{\frac{1}{4}\sigma(1)[h_{\<i,j\>}]:j<m_{\emptyset}\right\}$ is 
  a cover for $H\cap\cl{\frac{1}{4}\sigma(0)[h_{\<i\>}]}$, and since 
  $\left\{\frac{1}{4}\sigma(0)[h_{\<i\>}]:i<m_{\emptyset}\right\}$ is a cover
  for $H$, $\tau(0)$ must contain $H$. A similar argument follows for 
  $\tau(n)$.

  Let $p$ be an attack against $\tau$ such that 
  $p(n)\in \bigcap_{m\leq n}\tau(m)$. If we can construct an attack $q$ 
  against $\sigma$ which shares a subsequence with $p$, then $p$ must cluster.
  To find such a $q$, we construct a new tree $T'$.

  We begin by setting 
    \[
      T'(0)
        =
      \{s : s\leq\<i,h_{\<i\>},j\> \text{ for } i,j<m_\emptyset\}
    \]

  Since 
    \[
      p(0)
        \in
      \frac{1}{4}\sigma(1)[h_{\<i,j\>}]
    \]
  for some $i,j<m_\emptyset$, we may find $t\in \max(T'(0))$ such that 
  $p(0)\in\frac{1}{4}\sigma(|o(t)|)[h_{e(t)}]$.

  Assume that $T'(n)$ is defined such that there is some $t\in \max(T'(n))$
  where $p(n)\in\frac{1}{4}\sigma(|o(t)|)[h_{e(t)}]$. Then let
    \[
      T'(n+1)
        =
      T'(n)
        \cup
      \{
        s: s\leq t\concat\<p(n),i,h_{e(t)\concat\<i\>},j\>
        \text{ for } i,j < m_{e(t)}
      \}
    \]

  Note that 
    \[
      p(n+1)
        \in
      \bigcap_{m\leq n+1} \tau(m)
        =
      \bigcap_{m\leq n+1} \left(
        \bigcup_{t\in\max(T(m))} \frac{1}{4}\sigma(2m+1)[h_t]
      \right)
    \]

  (Need details on why there is some $t$ in $\max(T'(n+1))$ such that 
  $p(n+1)\in\frac{1}{4}\sigma(|o(t)|)[h_{e(t)}]$.)

  (Follow this up with choosing an infinite branch $q'\in T'$, and showing
  that $q=o(q')$ is an attack on $\sigma$ similar to perfect info result.)

\end{proof}











\newpage

\begin{example}
  Let $X=I\times 2$ be the Alexandrov double interval. Then 
  $\pl D \not\prewin \proxgame{X}$, but $\pl D \markwin \proxgame{X}$.
\end{example}

\begin{proof}
  We assume that the uniformity on $X$ is given by entourages 
  \[
    D(\epsilon,F) = 
    \{\<x,0\>,\<y,0\>: |x-y|<\epsilon \}
      \cup
    \{\<x,1\>,\<y,0\>: |x-y|<\epsilon \vee x\not\in F\}
  \]
  \[
      \cup
    \{\<x,0\>,\<y,1\>: |x-y|<\epsilon \vee y\not\in F\}
      \cup
    \{\<x,1\>,\<y,1\>: x=y\}
  \]

  That is, points are $D(\epsilon,F)$-close if they are the same point, or 
  the first coordinates are within $\epsilon$ of each other while neither 
  second coordinate is in $F$.

  Suppose $\pl D$ had a predetermined winning strategy 
  $\sigma(n)=D(\epsilon_n,F_n)$. Then 
  $\pl P$ can choose $x\not\in \bigcup_{n<\omega} F_n$, and play $\<x,1\>$ 
  during even rounds, and $\<x_{2n+1},0\>$ where $|x-x_{2n+1}|<\epsilon_{2n}$
  during odd rounds, preventing convergence.

  However, assume $\pl D$ uses the Mark\"ov strategy 
  $\sigma(x,n)=D(2^{-n},\{x\})$.
  If $\pl P$ repeats a point of the form $\<x,1\>$, then since 
  $D(2^{-n},\{x\})[\<x,1\>]=\{\<x,1\>\}$, $\pl P$ must repeat $\<x,1\>$ for
  the rest of the game, and $\pl D$ wins. Otherwise, $\pl P$ cannot repeat 
  points played in $I\times\{1\}$, and as the first
  coordinates form a Cauchy sequence and converge to some $z$, any open set
  about $\<z,0\>$ contains all but finitely many points of $\pl P$'s sequence,
  and $\pl D$ wins.
\end{proof}

\begin{theorem}
  For any uniformly locally compact space $X$,
      $\pl D\prewin \proxgame{X} \Leftrightarrow \pl D\prewin \aproxgame{X}$
\end{theorem}

\begin{proof}
  Let $M$ be a uniformly locally compact entourage. 
  Let $\sigma$ witness $\pl D\prewin \proxgame{X}$ such that 
  $\sigma(n)\subseteq M$ always 
  (so $\cl{\sigma(a)[x]}\subseteq\cl{M[x]}$ is compact), 
  $\sigma(n+1)\subseteq\frac{1}{4}\sigma(n)$.

  Let $\tau(n)=\frac{1}{2}\sigma(n)$. If $p$ attacks $\tau$ in 
  $\aproxgame{X}$, then
    \[
      p(n+1)
        \in
      \tau(n)[p(n)]
        =
      \frac{1}{2}\sigma(n)[p(n)]
    \]

    and for

    \[
      x
        \in
      \cl{\sigma(n+1)[p(n+1)]}
        \subseteq
      \cl{\frac{1}{4}\sigma(n)[p(n+1)]}
        \subseteq
      \frac{1}{2}\sigma(n)[p(n+1)]
    \]

  we can conclude $x\in\sigma(n)[p(n)]$. Thus

    \[
      \sigma(n+1)[p(n+1)]
        \subseteq
      \cl{\sigma(n+1)[p(n+1)]}
        \subseteq
      \sigma(n)[p(n)]
    \]

  Finally, note that $p$ attacks the winning strategy $\sigma$ in 
  $\proxgame{X}$, but since the intersection of a chain of nonempty compact 
  sets is nonempty:

    \[
      \bigcap_{n<\omega} \sigma(n)[p(n)]
        =
      \bigcap_{n<\omega} \cl{\sigma(n)[p(n)]}
        \not=
      \emptyset
    \]

  We conclude that $p$ converges.
\end{proof}

\begin{proposition}
If $\pl D \prewin \proxgame{X}$, then $X$ has a $G_\delta$ diagonal.
\end{proposition}

\begin{proof}
  If $\pl D \prewin \proxgame{X}$ with strategy $\sigma$, then consider 
  $\<x,y\>\in \bigcap_{n<\omega} \sigma(n)$. It follows that 
  $\<x,y,x,y,\dots\>$ attacks $\sigma$, and 
    $
      \{x,y\}\subseteq \bigcap_{n<\omega} 
      \sigma(n)[x]\cap\bigcap_{n<\omega}\sigma(n)[y]
      \not=0
    $ 
  so it must converge, and $x=y$. Thus
  $\bigcap_{n<\omega}\sigma(n) = \Delta$ is $G_\delta$.
\end{proof}

\begin{example}
The Sorgenfrey line $S$ has a $G_\delta$ diagonal but $\pl P\win\proxgame{S}$.
\end{example}

\begin{corollary}
  For $X$ with uniformity $\mathbb{D}$ inducing the compact Hausdorff topology 
  $\tau$, the following are equivalent:
    \begin{enumerate}[(a)]
      \item $\pl D \prewin \proxgame{X}$
      \item $\pl D \prewin \aproxgame{X}$
      \item $X$ has a $G_\delta$ diagonal
      \item $\mathbb{D}$ is metrizable
      \item $\tau$ is metrizable
    \end{enumerate}
\end{corollary}

\begin{proof}
  For compact Hausdorff spaces, it is well known that there is exactly one
  uniformity inducing the topology. Thus $(d)\Leftrightarrow(e)$. Since $X$
  is uniformly locally compact, $(a)\Leftrightarrow(b)$. Also, compact spaces
  with a $G_\delta$ diagonal are metrizable, so $(c)\Rightarrow(e)$. 
  Bell noted $(d)\Rightarrow(a)$ 
  for arbitrary uniform spaces, and the previous proposition shows
  $(a)\Rightarrow(c)$.
\end{proof}

\begin{theorem}
  A uniformly locally compact space with a $G_\delta$ diagonal is metrizable.
\end{theorem}

\begin{proof}
  Based on several folklore results.

  Uniformly locally compact implies the topological sum of $\sigma$-compact
  spaces implies paracompact. Locally compact plus $G_\delta$ diagonal implies
  locally metrizable. Locally metrizable plus paracompact characterizes
  metrizable.
\end{proof}

\begin{corollary}
  If $X$ is uniformly locally compact, then $\pl D \prewin \proxgame{X}$ 
  implies $X$'s topology is metrizable.
\end{corollary}





\newpage

\begin{example}
  Let $R$ be the Michael Line. Then $\pl P\win \proxgame{X}$.
\end{example}

\begin{proof}
  Let $\pi_n$ be the $n$th digit of $\pi$. 
  
  During round $0$, $\pl P$ may choose $m(0)=0$ and $p(0)=\pi_0$, 
  and during round $n+1$,
  $\pl P$ may choose $m(n+1)>m(n)$ and
  $p(n+1)=p(n)+\frac{\pi_{n+1}}{10^{m(n+1)}}$ such that $p$ is a legal attack.
  
  It follows that $p$ ``converges'' to 
  $x=\sum_{n<\omega}\frac{\pi_n}{10^{m(n)}}$, except $x$ is an irrational
  number.
\end{proof}

\end{document}