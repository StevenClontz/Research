\documentclass[11pt]{article}

\pdfpagewidth 8.5in
\pdfpageheight 11in

\setlength\topmargin{0in}
\setlength\headheight{0in}
\setlength\headsep{0.4in}
\setlength\textheight{8in}
\setlength\textwidth{6in}
\setlength\oddsidemargin{0in}
\setlength\evensidemargin{0in}
\setlength\parindent{0.25in}
\setlength\parskip{0.1in} 
 
\usepackage{amssymb}
\usepackage{amsfonts}
\usepackage{amsmath}
\usepackage{mathtools}
\usepackage{amsthm}

\usepackage{fancyhdr}

\usepackage{enumerate}

      \theoremstyle{plain}
      \newtheorem{theorem}{Theorem}
      \newtheorem{lemma}[theorem]{Lemma}
      \newtheorem{corollary}[theorem]{Corollary}
      \newtheorem{proposition}[theorem]{Proposition}
      \newtheorem{conjecture}[theorem]{Conjecture}
      \newtheorem{question}[theorem]{Question}
      
      \theoremstyle{definition}
      \newtheorem{definition}[theorem]{Definition}
      \newtheorem{example}[theorem]{Example}
      \newtheorem{game}[theorem]{Game}
      
      \theoremstyle{remark}
      \newtheorem{remark}[theorem]{Remark}



\pagestyle{fancy}
\renewcommand{\headrulewidth}{0.5pt}
\renewcommand{\footrulewidth}{0pt}
\lfoot{\small \jobname.tex -- Updated on \today}
\chead{\small http://github.com/StevenClontz/Research}
\rfoot{\thepage}
\cfoot{}


% Strategy uparrow shortcuts
\newcommand{\win}{\uparrow}
\newcommand{\prewin}{\uparrow_{\text{pre}}}
\newcommand{\markwin}{\uparrow_{\text{mark}}}
\newcommand{\tactwin}{\uparrow_{\text{tact}}}
\newcommand{\ktactwin}[1]{\uparrow_{#1\text{-tact}}}
\newcommand{\kmarkwin}[1]{\uparrow_{#1\text{-mark}}}
\newcommand{\codewin}{\uparrow_{\text{code}}}
\newcommand{\limitwin}{\uparrow_{\text{limit}}}

\newcommand{\oneptcomp}[1]{#1^*}
\newcommand{\oneptlind}[1]{#1^\dagger}

\newcommand{\congame}[2]{Con_{O,P}(#1,#2)}
\newcommand{\clusgame}[2]{Clus_{O,P}(#1,#2)}

\newcommand{\lfkpgame}[1]{LF_{K,P}(#1)}
\newcommand{\lfklgame}[1]{LF_{K,L}(#1)}

\newcommand{\pfgame}[1]{PF_{F,C}(#1)}

\newcommand{\mengame}[1]{Cov_{C,F}(#1)}
\newcommand{\rothgame}[1]{Cov_{C,S}(#1)}
\newcommand{\altrothgame}[1]{Cov_{P,O}(#1)}

\newcommand{\fillgame}[1]{Fill^{\subseteq}_{M,N}(#1)}
\newcommand{\sfillgame}[1]{Fill^{\subsetneq}_{M,N}(#1)}

\newcommand{\kfillgame}[1]{Fill^{\subseteq}_{C,F}(#1)}
\newcommand{\ksfillgame}[1]{Fill^{\subsetneq}_{C,F}(#1)}

\newcommand{\sigmaprodr}[1]{\Sigma\mathbb{R}^{#1}}
\newcommand{\sigmaprodtwo}[1]{\Sigma2^{#1}}

\newcommand{\concat}{^\frown}
\newcommand{\rest}{\restriction}

\newcommand{\cl}[1]{\overline{#1}}

\newcommand{\pow}[1]{\mc{P}(#1)}

\newcommand{\<}{\langle}
\renewcommand{\>}{\rangle}

\newcommand{\mc}[1]{\mathcal{#1}}

\newcommand{\Lim}{\mathrm{Lim}}
\newcommand{\Suc}{\mathrm{Suc}}

\newcommand{\ds}{\displaystyle}

\newcommand{\rank}{\textrm{rank}}

\newcommand{\scish}{$\sigma$-compactish }

\begin{document}

  \begin{definition}
    $X$ is \textbf{Menger} if for all open covers $\mathcal{U}_0,\mathcal{U}_1,\dots$ there exist finite subcollections $\mathcal{F}_n \subseteq \mathcal{U}_n$ such that $\bigcup_{n<\omega} \mathcal{F}_n$ is a cover of $X$.
  \end{definition}

  \begin{proposition}
    $\sigma$-compact $\Rightarrow$ Menger $\Rightarrow$ Lindelof
  \end{proposition}

  \begin{definition}
    In the two-player game $\mengame{X}$ player $C$ chooses open covers $\mathcal{U}_n$ of $X$, followed by player $F$ choosing a finite subcollection $\mathcal{F}_n\subseteq\mathcal{U}_n$. $F$ wins if $\bigcup_{n<\omega} \mathcal{F}_n$ is a cover of $X$.
  \end{definition}

  \begin{theorem}
    $X$ is Menger if and only if $C \not\win \mengame{X}$.
  \end{theorem}

  \begin{proof}
    Result due to (???)

    First, suppose $X$ wasn't Menger. Then there would exist open covers $\mathcal{U}_0,\mathcal{U}_1,\dots$ of $X$ such that for any choice of finite subcollections $\mathcal{F}_n\subseteq\mathcal{U}_n$, $\bigcup_{n<\omega} \mathcal{F}_n$ isn't a cover of $X$. Thus $C\prewin\mengame{X} \Rightarrow S \not\win\mengame{X}$.

    The other direction is based upon Gruenhage's topological game presentation. Assume $X$ is Menger, and consider a strategy for $C$ in $\mengame{X}$.

    Since $X$ is Lindelof, we can assume $C$ plays only countable covers of $X$. Then, since $F$ is choosing finite subsets, we may assume $F$ chooses some initial segement of the countable cover. In turn, we can assume $C$ plays an increasing open cover $\{U_0,U_1,\dots\}$ where $U_n\subseteq U_{n+1}$. And in that case, it's suffient to assume $F$ simply chooses a singleton subset of each cover. And finally, since choices made by $F$ are already covered, we can assume that every open set in a cover played by $C$ covers the sets chosen by $F$ previously.

    As a result, we have the following figure of a tree of plays which I need to draw:

    (Insert figure here.)

    Note that for $a,b\in\omega^{<\omega}$ and $m\leq n$, we know:
      \begin{enumerate}[(a)]
        \item $U_{a\concat m}\subseteq U_{a\concat n}$ \\ (for example, $U_{1627} \subseteq U_{1629}$ - increasing the final digit yields supersets)
        \item $U_a \subseteq U_{a\concat b}$ \\ (for example, $U_{1627}\subseteq U_{162789}$ - appending any sequence to the end yields supersets)
        \item $U_{a\concat m} \subseteq U_{a\concat n} \subseteq U_{a\concat n\concat b} \subseteq U_{a\concat n\concat b\concat m}$ \\ (for example: $U_{1627} \subseteq U_{1629283287}$ - injecting a subseqence with initial number larger than the original's final number, prior to the final number, yields supersets) 
      \end{enumerate}

    We may observe that if $F$ can find an $f:\omega\to\omega$ such that $\bigcup_{n<\omega}U_{f\rest (n+1)} = X$, she can use $\{U_{f\rest 0}\}, \{U_{f\rest 1}\}, \dots$ to counter $C$'s strategy.

    Let $\ds V_k^n = \bigcap_{a\in\omega^{\leq n}} U_{a\concat k}$. We claim that (1) $V_k^n$ is open, (2) $\mathcal{V}^n=\{V_0^n,V_1^n,\dots\}$ is increasing, and (3) $\mathcal{V}^n$ is a cover. Proofs:

    \begin{enumerate}
      \item
      Since due to (c) for each $b\in \omega^{\leq n} \setminus k^{\leq n}$, there is an $a\in k^{\leq n}$ with $U_{a\concat k} \subseteq U_{b\concat k}$: \[\ds V_k^n = \bigcap_{a\in\omega^{\leq n}} U_{a\concat k} = \bigcap_{a\in k^{\leq n}} U_{a\concat k} \cap \bigcap_{b\in \omega^{\leq n} \setminus k^{\leq n}} U_{b\concat k} = \bigcap_{a\in k^{\leq n}} U_{a\concat k}\] making $V_k^n$ a finite intersection of open sets.

      \item
      We show $V_k^0\subseteq V_{k+1}^0$:
      \[
        V_k^0 = U_k \subseteq U_{k+1} = V_{k+1}^0
      \]
      and then assume $V_k^{n}\subseteq V_{k+1}^{n}$:
      \[
        V_k^{n+1} = 
        \bigcap_{a\in{\omega}^{\leq n+1}} U_{a\concat k} = 
        V_k^{n} \cap \bigcap_{a\in{\omega}^{n+1}} U_{a\concat k} \subseteq
        V_{k+1}^{n} \cap \bigcap_{a\in{\omega}^{n+1}} U_{a\concat (k+1)} =
        V_{k+1}^{n+1}
      \]

      \item
      We easily see that $\mathcal{V}^0=\{U_0,U_1,\dots\}$ is a cover, and then assume $\mathcal{V}^n$ is a cover.

      Let $x\in X$ and pick $l<\omega$ such that $x \in V^n_l$. For $a \in l^{n+1}$ choose $l_a$ such that $x\in U_{a\concat l_a}$, giving
      \[ x \in \bigcap_{a\in l^{n+1}} U_{a\concat l_a} \]
      
      We will assume $k>l,l_a$ for all $a\in l^{\leq n+1}$. 

      For any $a\in k^{n+1}\setminus l^{n+1}$ note that $a = b\concat c$ where $b\in l^{\leq n}$ and $c$ begins with a number $l$ or greater:
      \[
        V^n_l \subseteq 
        U_{b\concat l} \subseteq 
        U_{b\concat c} \subseteq
        U_{b\concat c\concat l_a} = 
        U_{a\concat l_a}
      \]

      Thus:
      \[ 
        x \in 
        V^n_l \cap \left(\bigcap_{a\in l^{n+1}} U_{a\concat l_a}\right) 
      \]
      \[
        =
        V^n_l \cap \left(\bigcap_{a\in k^{n+1}\setminus l^{n+1}} U_{a\concat l_a}\right) \cap \left(\bigcap_{a\in l^{n+1}} U_{a\concat l_a}\right)
      \]
      \[
        =
        V^n_l \cap \left(\bigcap_{a\in k^{n+1}} U_{a\concat l_a}\right)
      \]
      \[
        \subseteq
        V^n_k \cap \left(\bigcap_{a\in k^{n+1}} U_{a\concat k}\right)
      \]
      \[
        =
        V^{n+1}_k
      \]
    \end{enumerate}

    Finally, apply Menger to $\mathcal{V}^n$, resulting in the cover $\{V_{f(0)}^0, V_{f(1)}^1, \dots\}$, noting \[X = \bigcup_{n<\omega}V_{f(n)}^n \subseteq \bigcup_{n<\omega} U_{(f\rest n)\concat f(n)} = \bigcup_{n<\omega} U_{f\rest (n+1)}\]
  \end{proof}

  \begin{proposition}
    $X$ is compact if and only if $F \tactwin \mengame{X}$ if and only if $F \ktactwin{k} \mengame{X}$
  \end{proposition}

  \begin{proof}
    Assume $X$ is compact. For each open cover played by $C$, pick a finite subcover, and this yields a winning tactical strategy.

    Assume $F$ has a winning $k$-tactical strategy. For any open cover, have $C$ play only it during the entire game. $F$'s only choice must be a finite subcover.
  \end{proof}

  \begin{proposition}
    If $X$ is $\sigma$-compact then $F \markwin \mengame{X}$
  \end{proposition}

  \begin{proof}
    Let $X=\bigcup_{n<\omega} X_n$ for compact $X_n$. On round $n$, $F$ picks the finite subcover of $C$'s open cover of $X_n$.
  \end{proof}

  For Menger's game, there is no useful distinction between a $k$-Markov strategy for $F$, and a $2$-Markov strategy.

  \begin{theorem}
  For any topological space $X$ and all $k \geq 2$, $F \kmarkwin{k} \mengame{X}$ if and only if $F \kmarkwin{2} \mengame{X}$.
  \end{theorem}

  \begin{proof}
  Assume $\sigma(\mc{U}_0,\dots,\mc{U}_{k-1},n)$ is a winning $k$-Markov strategy. Define the $2$-Markov strategy $\tau(\mc{U},\mc{V},n)$ so that it contains $\sigma(\mc{W}_0,\dots,\mc{W}_{k-2},\mc{V},m)$ for the following conditions on $\mc{W}_0,\dots,\mc{W}_{k-2},m$:
    \begin{itemize}
    \item Each $\mc{W}_i \in \{\mc{U},\mc{V}\}$
    \item $m \leq (n+1)k$; in particular, for $i<k$, 
      \[
        \sigma(\mc{W}_0,\dots,\mc{W}_{k-2},\mc{V},(n+1)k+i)
        \subseteq 
        \tau(\mc{U},\mc{V},n+1) 
      \]
    \end{itemize}

  Considering an arbitrary play $\mc{U}_0, \mc{U}_1, \dots$ by $C$ versus $\tau$, we note that $\sigma$ defeats the play 
  \[
    \underbrace{\mc{U}_0,\mc{U}_0,\dots,\mc{U}_0}_{k},
    \underbrace{\mc{U}_1,\mc{U}_1,\dots,\mc{U}_1}_{k}
    \dots
  \]

  So we have that
    \[
      \bigcup_{i<k, n<\omega} \sigma(
      \underbrace{\mc{U}_n,\dots,\mc{U}_n}_{k-i-1},
      \underbrace{\mc{U}_{n+1},\dots,\mc{U}_{n+1}}_{i+1},
      (n+1)k+i)
    \]
  is a cover for $X$, and as 
    \[
      \sigma(
      \underbrace{\mc{U}_n,\dots,\mc{U}_n}_{k-i-1},
      \underbrace{\mc{U}_{n+1},\dots,\mc{U}_{n+1}}_{i+1},
      (n+1)k+i)
      \subseteq 
      \tau(\mc{U}_n,\mc{U}_{n+1},n+1)
    \]
  $\tau$ defeats the play $\mc{U}_0, \mc{U}_1, \dots$.
  \end{proof}

  But there are spaces for which there is no Markov strategy, but there is a $2$-Markov strategy.

  In a question I posed to G, he answered:

  \begin{lemma}
    For all functions $\tau:\omega_1\times\omega \rightarrow [\omega_1]^{<\omega}$, there exists a sequence $\alpha_0, \alpha_1, \dots < \omega_1$ such that $\{\tau(\alpha_n,n): n<\omega\}$ is not a cover for $\{\beta:\forall n<\omega (\beta < \alpha_n)\}$.
  \end{lemma} 

  \begin{proof}
    Let $P_n = \{\beta: \beta < \alpha \Rightarrow \beta \in \tau(\alpha, n)\}$. Observe that each $P_n$ is finite; else there is some $\alpha$ larger than every member of some countably infinite $P_n^*\subseteq P_n$ such that $P_n^* \subseteq \tau(\alpha, n)$.

    Choose $\beta \not\in \bigcup_{n<\omega} P_n$. Then for each $n<\omega$, pick $\alpha_n>\beta$ such that $\beta \not\in \tau(\alpha_n, n)$.
  \end{proof}

  Note that the one-point Lindel\"ofication of discrete $\omega_1$, $\oneptlind{\omega_1}$, is not $\sigma$-compact. With the above lemma, we may see that:

  \begin{example}
    $F \win \mengame{\oneptlind{\omega_1}}$ but $F \not\markwin \mengame{\oneptlind{\omega_1}}$.
  \end{example}

  \begin{proof}
    First, we see $F$ has a simple perfect information strategy: in response to the initial cover of $\oneptlind{\omega_1}$, $F$ chooses a co-countable neighborhood of $\infty$. On successive turns she may pick a single set from $C$'s covers to cover the countable remainder.

    Now, suppose that $\sigma(\mc{U}, n)$ was a winning Markov strategy and aim for a contradiction. Consider the covers \[\mc{U}(\alpha) = \{[\alpha,\omega_1)\cup\{\infty\}\} \cup \{\{\beta\}:\beta < \alpha\}\] and define $\tau(\alpha,n)$ to be the union of singletons chosen by $\sigma(\mathcal{U}(\alpha),n)$. 

    Using the sequence $\alpha_0, \alpha_1,\dots<\omega_1$ from the previous lemma, we consider the play $\mc{U}(\alpha_0),\mc{U}(\alpha_1),\dots$.

    As $\sigma$ was a winning strategy, $\{\sigma(\mathcal{U}(\alpha_n),n): n<\omega\}$ must cover $\oneptlind{\omega_1}$, and thus $\{\tau(\alpha_n,n): n<\omega\}$ must cover $\{\beta:\forall n<\omega (\beta < \alpha_n)\}$, contradiction.
  \end{proof}

  Telgarski showed in ``On Games of Topsoe'' that a metrizable space is $\sigma$-compact if and only if there exists a winning strategy for $F$ in the Menger game, and Scheepers gave a more direct proof later. We generalize Scheeper's proof to handle a number of cases.

  % \begin{definition}
  %   A space $X$ is $H$-closed if for every open cover of $X$, there exists a finite subset of the cover whose union is dense in $X$.
  % \end{definition}

  % \begin{proposition}
  %   For regular spaces, $X$ is $H$-closed if and only if $X$ is compact.
  % \end{proposition}

  % \begin{proof}
  %   The backward implication is trivial, so let $X$ be $H$-closed. For any open cover $\mc U$ of $X$ and point $x\in X$, choose $x\in V_x\subseteq \cl{V_x}\subseteq U_x\in \mc U$. Then if $\mc F = \{V_{x_i}:i<n\}$ is a finite subset of $\mc V=\{V_x:x\in X\}$ with dense union in $X$, then $\mc G = \{U_{x_i}:i<n\}\subseteq \mc U$ is a finite subcover of $X$.
  % \end{proof}

  \begin{definition}
    A set $R\subseteq X$ is relatively compact to the topological space $X$ if for every open cover of the entire space $X$, there is a finite subcover of the set $R$.
  \end{definition}

  \begin{proposition}
    If $X$ is regular, then $Y$ is relatively compact if and only if $\cl{Y}$ is compact.
  \end{proposition}

  \begin{proof}
    The reverse implication is trivial. 

    Assume $Y$ is relatively compact, let $\mc U$ be an open cover of $\cl Y$, and define $x\in V_x\subseteq \cl{V_x}\subseteq U_x \in \mc U$ for each $x\in X$. Then if we take a cover $\mc F = \{V_{x_i} : i<n\}$ of $Y$ by relative compactness, then $\{U_{x_i}:i<n\}$ is a finite cover of $\cl Y$, showing compactness.
  \end{proof}

  \begin{lemma}
    Let $\sigma(\mc{U}, n)$ be a winning Markov strategy for $F$ in $\mengame{X}$, and $\mathfrak{C}$ collect all open covers of $X$. Then for
      \[
        R_n = \bigcap_{\mc{U}\in\mathfrak{C}} \bigcup\sigma(\mc{U},n)
      \]
    it follows that $R_n$ is relatively compact to $X$, and $\bigcup_{n<\omega} R_n = X$.
  \end{lemma}

  \begin{proof}
    First, we see that $\sigma(\mc{U},n)$ witnesses the relative compactness of $R_n$. Suppose that $x \not\in R_n = \bigcap_{\mc{U}\in\mathfrak{C}} \bigcup\sigma(\mc{U},n)$ for any $n<\omega$. Then for each $n$, pick $\mc{U}_n\in\mathfrak{C}$ such that $x\not\in \bigcup\sigma(\mc{U}_n,n)$. Then $\sigma$ does not defeat the play $\mc{U}_0,\mc{U}_1,\dots$
  \end{proof}

  \begin{theorem}
    A space $X$ is $\sigma$-(relatively compact) if and only if $F \markwin \mengame{X}$.
  \end{theorem}

  \begin{proof}
    For the forward implication, let $X=\bigcup_{n<\omega}R_n$ for $R_n$ relatively compact, and define $\sigma(\mc U,n)$ to be a finite subcover of $R_n$. The previous lemma proves the other direction.
  \end{proof}

  \begin{corollary}
    For regular spaces $X$, the following are equivalent:
      \begin{enumerate}[(a)]
        \item $X$ is $\sigma$-compact
        \item $X$ is $\sigma$-(relatively compact)
        \item $F \markwin \mengame{X}$
      \end{enumerate}
  \end{corollary}

  \begin{theorem}
    For second-countable $X$, the following are equivalent:
      \begin{enumerate}[(a)]
        \item $X$ is $\sigma$-(relatively compact)
        \item $F \win \mengame{X}$
        \item $F\markwin \mengame{X}$
      \end{enumerate}
  \end{theorem}

  \begin{proof}.
    We need only show $(b)\Rightarrow(a)$, so let $\sigma(\mc{U}_0,\dots,\mc{U}_{n-1})$ be a winning strategy for $F$, and observe that since $X$ is second-countable, we may assume all covers are countable. Let $\mathfrak{C}$ be the collection of all countable covers of $X$. We define $R_s,\mc{U}_s$ for $s\in\omega^{<\omega}$ as follows:
      \begin{itemize}
        \item $\ds R_\emptyset = \bigcap_{\mc{U}\in\mathfrak{C}} \left(\bigcup \sigma(\mc{U})\right)$
        \item Note that there are only countably many distinct finite subsets $\sigma(\mc{U})$ of the countable collection $\mc U$. Thus define each $\mc U_{\<n\>}$ so that
          \[
            R_\emptyset = 
            \bigcap_{n<\omega}\left(\bigcup\sigma(\mc{U}_{\<n\>})\right)
          \]
        \item $\ds R_s = \bigcap_{\mc{U}\in\mathfrak{C}} \left(\bigcup \sigma(\mc{U}_{s\rest1},\mc{U}_{s\rest2},\dots,\mc{U}_s,\mc{U})\right)$
        \item Again, note that there are only countably many distinct finite subsets $\sigma(\mc{U}_{s\rest1},\mc{U}_{s\rest2},\dots,\mc{U}_s,\mc{U})$ of the countable collection $\mc U$. Thus define each $\mc U_{s\concat\<n\>}$ so that 
          \[
            R_s = 
            \bigcap_{n<\omega} \left(\bigcup \sigma(\mc{U}_{s\rest1},\mc{U}_{s\rest2},\dots,\mc{U}_s,\mc{U}_{s\concat\<n\>})\right)
          \]
      \end{itemize}

    We quickly confirm that each $R_s$ is relatively compact as for each open cover $\mc{U}$ of $X$ we have the finite subcover $\sigma(\mc{U}_{s\rest1},\mc{U}_{s\rest2},\dots,\mc{U}_s,\mc{U})$ of $R_s$.

    Finally, we claim that $X = \bigcup_{s\in\omega^{<\omega}} R_s$. If not, let $x$ be missed by every $R_s$, and define $f\in\omega^\omega$ such that $x\not\in \bigcup\sigma(\mc{U}_{f\rest1},\dots,\mc{U}_{f\rest n})$ for any $n$. Then $\mc{U}_{f\rest1},\mc{U}_{f\rest2},\dots$ is a counter to the winning strategy $\sigma$, a contradiction.
  \end{proof}

  \begin{corollary}
    For metric spaces $X$, the following are equivalent:
      \begin{enumerate}[(a)]
        \item $X$ is $\sigma$-compact
        \item $X$ is $\sigma$-(relatively compact)
        \item $F \win \mengame{X}$
        \item $F \markwin \mengame{X}$
      \end{enumerate}
  \end{corollary}

  \begin{example}
  Let $R$ be given the topology from example 63 from Counterexamples in Topology, the topology generated by open intervals with countable sets removed. This space is non-regular, non-$\sigma$-compact, and Lindel\"of. It is also Menger as $F \win \mengame{R}$, but $F \not\markwin \mengame{R}$.
  \end{example}

  \begin{proof}
  From Counterexamples: The irrationals are open, but contain no closed neighborhood, showing non-regular. Compact subsets are exactly finite subsets, showing non-$\sigma$-compact.

  Take open covers $\mc{U}_0,\mc{U}_1,\dots$. Define $\sigma(\mc{U}_0,\dots,\mc{U}_{2n})$ to be a finite subcover of $[-n,n]\setminus C_n$ for some countable $C_n=\{c_{n,0}, c_{n,1}, \dots\}$. For $\sigma(\mc{U}_0,\dots,\mc{U}_{2n+1})$, use any subcover of $\{c_{i,j} : i,j < n\}$. It is easily seen that $\sigma$ is a winning perfect information strategy.

  There cannot be a winning Markov strategy $\sigma(\mc{U},n)$, however. Define 
      \[
        R_n = \bigcap_{\mc{U}\in\mathfrak{C}} \bigcup\sigma(\mc{U},n)
      \]
  where $\mathfrak{C}$ is the collection of open covers of $R$. For any $x_0,x_1,\dots \in R$, we may define the open cover $\mc{U} = \{R \setminus \{x_i: i \not= n\} : n < \omega \}$, and observe that $\bigcup \sigma(\mc{U},n) \supseteq R_n$ contains only finitely many $x_i$. Thus $R_n$ is finite, but since the previous lemma requires $\bigcup_{n<\omega} R_n = R$ if $\sigma$ is a winning strategy, there exists a counter to $\sigma$.
  \end{proof}

  \begin{example}
  Let $R$ be given the topology from example 67 from Counterexamples in Topology, the topology generated by open intervals with or without the rationals removed. This space is non-regular, non-$\sigma$-compact, and Lindel\"of.

  This space is an example of non-$\sigma$-compact but $F \markwin \mengame{R}$ (and is thus also Menger).
  \end{example}

  \begin{proof}
  From Counterexamples: The rationals are closed, but the closure of any open neighborhood is the whole real line, so they cannot be separated from any irrational point. Compact sets in this topology are nowhere dense in the Euclidean topology, so there cannot be countably many which union to the whole space. $\{(a,b)\setminus D : a,b\in\mathbb{Q},D\in\{\emptyset,\mathbb{Q}\}\}$ is a countable base for the space, and second-countability implies Lindel\"of.

  % To see Menger, let $\mc{U}_0,\mc{U}_1,\dots$ be a sequence of open covers, and choose open sets $U_{2n}\in\mc{U}_{2n}$ which cover $q_n$ for each $q_n\in\mathbb{Q}$ (which in turn are open in the Euclidean line). Since the complement of the union $\bigcup_{n<\omega} U_{2n}$ must be closed discrete in the Euclidean line, it is countable, so open sets $U_{2n+1}\in\mc{U}_{2n+1}$ may be found to cover the remainder.

  To see $F \markwin \mengame{R}$, we define $\sigma(\mc{U}_{2n},2n)$ to be a finite cover of $[-n,n]\setminus\mathbb{Q}$, and $\sigma(\mc{U}_{2n+1},2n+1)$ to be a finite cover of $\{q_n\}$ for each $q_n\in\mathbb{Q}$.
  \end{proof}

  We define a new property ``\scish'' to describe a sufficient condition for $F \kmarkwin{2}\mengame{X}$.

  \begin{definition}
    Let $\mc U$ be a cover of $X$. We say $C\subseteq X$ is $\mc U$-compact if there exists a finite subcover of $\mc U$ which covers $C$.

    Let $\mathfrak{C}$ collect all the open covers of $X$. We say $X$ is \scish if there exists a function $f:\mathfrak{C}\times\omega\to\mc{P}(X)$ such that:
      \begin{itemize}
        \item $f(\mc{V},n)$ is $\mc{V}$-compact
        \item $f(\mc{V},n)\subseteq f(\mc{V},n+1)$
        \item $\bigcup_{n<\omega}f(\mc{V},n)=X$
        \item The set
          \[
            g(\mc{U},\mc{V},n) = \bigcup_{m\geq n} f(\mc{U},m)\setminus ( f(\mc{U},m-1)\cup f(\mc{V},m))
          \]
         is $\mc{V}$-compact
      \end{itemize}
  \end{definition}

  Obviously $\sigma$-compact implies \scish implies Lindel\"of. We shall see that the non-$\sigma$-compact space $\omega_1^\dagger$ is \scish.

   \begin{lemma}
    There exist injective functions $f_\alpha:\alpha\to\omega$ such that if $\alpha<\beta$, then \[f_\beta\rest\alpha =^* f_\alpha\] that is, $f_\beta\rest\alpha$ and $f_\alpha$ agree on all but finitely many ordinals. (In addition, the range of each $f_\alpha$ is co-infinite.)
  \end{lemma}

  \begin{proof}
    Taken from Kunen (used for the construction of an $\omega_1$-Aronszajn tree).

    We begin with the empty function $f_0:0\to\omega_1$ which satisfies the hypothesis, and assume $f_\alpha$ is defined by induction. Let $f_{\alpha+1} = f_\alpha \cup \{\<\alpha,n\>\}$ where $n$ is not defined for $f_\alpha$, and this satisfies the hypothesis.

    Finally, suppose $\gamma$ is the limit of $\alpha_0,\alpha_1,\dots$, and $f_\alpha$ is defined for $\alpha<\gamma$. Let $g_0=f_{\alpha_0}$, and define $g_n : \alpha_n \to \omega$ to be injective, $g_n=^* f_{\alpha_n}$, and $g_{n+1} \rest \alpha_n = g_n$. $g = \bigcup_{n<\omega} g_n$ is an injective function from $\gamma \to \omega$ and $g =^* f_\alpha$ for $\alpha <\gamma$, but the range need not be coinfinte. So let
    \[
      f_\gamma(\beta) = \left\{
      \begin{array}{ll}
        g(\alpha_{2n}) & \beta = \alpha_n \\
        g(\beta) & \text{otherwise}
      \end{array}
      \right.
    \]
    which frees up $\{g(\alpha_{2n+1}):n<\omega\}$ from the range of $f_\gamma$, but still agrees with all but finitely many points compared to previous $f$'s.
  \end{proof}

  \begin{theorem}
    The one-point Lindel\"ofication of the uncountable discrete space, $\oneptlind{\omega_1}$, is \scish.
  \end{theorem}

  \begin{proof}
    Take the injective funcions $f_\alpha$ from Kunen's lemma such that $f_\alpha\rest\beta =^* f_\beta$. Let $\gamma(\mc U)$ identify the least ordinal such that $[\gamma(\mc U),\omega_1)\cup\{\infty\}$ is in a refinement of $\mc U$. Then $f(\mc U,n)=f^{-1}_{\gamma(\mc U)}([0,n]) \cup [\gamma(\mc U),\omega_1)\cup\{\infty\}$ is easily seen to witness the property.
  \end{proof}

  \begin{theorem}
    If $X$ is \scish, then $F \kmarkwin{2}\mengame{X}$.
  \end{theorem}

  \begin{proof}
    Let $\sigma(\mc{U}_n,\mc{U}_{n+1},n+1)$ cover $f(\mc{U}_{n+1},n+1)$ and $g(\mc{U}_n,\mc{U}_{n+1},n+1)$. If $\mc{U}_0,\mc{U}_1,\dots$ is any play by $C$, then for each $x\in X$, we note that $x\in f(\mc{U}_0,N)$ for some $N$. So either $x\in \bigcap_{m\leq N} f(\mc{U}_m,N)$ and is covered by the strategy during round $N$, or for some $m< N$, $x \in f(\mc{U}_m,N)\setminus (f(\mc{U}_{m},N-1)\cup f(\mc{U}_{m+1},N))$ and is covered by the strategy during round $m+1$.
  \end{proof}

  \begin{corollary}
    $F\kmarkwin{2}\mengame{\oneptlind{\omega_1}}$
  \end{corollary}

  \newpage

  \begin{definition}
    $X$ is \textbf{Rothberger} if for all open covers $\mathcal{U}_0,\mathcal{U}_1,\dots$ there exist open sets $U_n\in \mathcal{U}_n$ such that $\{U_n:n<\omega\}$ is a cover of $X$.
  \end{definition}

  \begin{proposition}
    Rothberger $\Rightarrow$ Menger
  \end{proposition}

  \begin{definition}
    In the two-player game $\rothgame{X}$ player $C$ chooses open covers $\mathcal{U}_n$ of $X$, followed by player $S$ choosing an open set $U_n\in\mathcal{U}_n$. $S$ wins if $\{U_n:n<\omega\}$ is a cover of $X$.
  \end{definition}

  \begin{theorem}
    $X$ is Rothberger if and only if $C\not\win\rothgame{X}$.
  \end{theorem}

  \begin{proof}
    Due to Pawlikowski
  \end{proof}

  \begin{definition}
    A space $X$ is scattered if every subspace contains an isolated point. By convention, $\ds X = \bigcup_{\alpha<\rank(X)}X^\alpha$ where $X^\alpha$ is the set of isolated points of $\ds X\setminus\bigcup_{\beta<\alpha}X^\beta$.
  \end{definition}

  \begin{proposition}
    A space $X$ is scattered if and only if every closed subspace contains an isolated point.
  \end{proposition}

  \begin{proposition}
    The rank of a compact scattered $T_1$ space is a successor ordinal, and $X^{\rank(X)-1}$ is finite.
  \end{proposition}

  \begin{proof}
    Suppose that the rank of $X$ was a limit ordinal $\beta$. Then by choosing $\beta_n\to\beta$, we may pick a point $x_n\in X^{\beta_n}$, and $\{x_n: n<\omega\}$ may be shown to be a closed discrete subspace of $X$.

    It's easily seen that $X^{\rank(X)-1}$ must be finite - it is a closed discrete subspace of compact $X$.
  \end{proof}

  \begin{theorem}
    The following are equivalent for compact $T_2$ $X$:
      \begin{enumerate}[(a)]
        \item $X$ is Rothberger
        \item $X$ is scattered
        \item $O\win\rothgame{X}$
      \end{enumerate}
  \end{theorem}

  \begin{proof}
    To show $(a)\Rightarrow(b)$, we use Aurichi's proof in \textit{D-Spaces}: suppose $X$ has a closed subspace without isolated points. Then it is compact and contains a closed copy of the Cantor set, which is not Rothberger, contradiction.

    To show $(b)\Rightarrow(c)$, if $X$ is scattered, suppose during a particular round $n$, player $S$ observes that the uncovered subspace $Y \subseteq X$ is nonempty. Then as $Y$ is also compact scattered,  select one of the finite points in $Y^{\rank(Y)-1}$, label it $x_n$, and choose an open set containing $x_n$ from the given cover.

    We claim that if $S$ follows this strategy, player $S$ will observe that the uncovered subspace $Y\subseteq X$ is empty during some round. If not, consider the $x_n$ chosen by $Y$ by the end of the game - the rank of each point within $X$ is nonincreasing, and does not contain a constant final sequence, contradiction.

    Of course, $(c)\Rightarrow(a)$ is trivial.
  \end{proof}

  \begin{definition}
    In the two-player game $\altrothgame{X}$ player $P$ chooses points $x_n\in X$, followed by player $O$ choosing an open neighborhood $U_n$ of $x_n$. $P$ wins if $\{U_n:n<\omega\}$ is a cover of $X$.
  \end{definition}

  \begin{theorem} $\altrothgame{X}$ is equivalent to $\rothgame{X}$. That is:

    \begin{itemize}
      \item $P\win\altrothgame{X}$ if and only if $S\win\rothgame{X}$
      \item $O\win\altrothgame{X}$ if and only if $C\win\rothgame{X}$.
    \end{itemize}
  \end{theorem}

  \begin{proof}
    Due to Galvin.

    \begin{itemize}
      \item
      Let $\sigma$ be a strategy for $S$ in $\rothgame{X}$. 

      We define a strategy for $P$ in $\altrothgame{X}$ as follows: during round $0$, $P$ chooses a point $x_0$ for which every open neighborhood is of the form $U_0 = \sigma(\mc{U}_0)$ for some open cover $\mc{U}_0$. (If not, let $U_x$ witness for every $x\in X$, and note that $\sigma(\{U_x:x\in X\})$ is a contradiction.) %TODO make this clearer.

      During round $n+1$, $P$ chooses point $x_{n+1}$ for which every open neighborhood is of the form $U_{n+1}=\sigma(\mc{U}_0,\dots,\mc{U}_n,\mc{U}_{n+1})$ for some open cover $\mc{U}_{n+1}$. (If not, let $U_x$ witness for every $x\in X$, and note that $\sigma(\mc{U}_0,\dots,\mc{U}_n,\{U_x:x\in X\})$ is a contradiction.)

      If $\sigma$ was a winning strategy for $S$, then the open sets chosen by $P$ are of the form $\{\sigma(\mc U_0),\sigma(\mc U_0,\mc U_1),\dots\}$ and are an open cover of $X$.

      \item
      Let $\sigma$ be a strategy for $P$ in $\altrothgame{X}$.

      We define a strategy for $S$ in $\rothgame{X}$ as follows: during round $n$, if $S$ has chosen $U_0,\dots,U_{n-1}$ in previous rounds, $S$ chooses an open set covering the point $\sigma(U_0,\dots,U_{n-1})$. If $\sigma$ was a winning strategy for $P$, then for any open sets $U_0,U_1,\dots$ containing $\sigma(\cdot),\sigma(U_0),\dots$, the collection $\{U_0,U_1,\dots\}$ is a cover for $X$.

      \item
      Let $\sigma$ be a strategy for $C$ in $\rothgame{X}$.

      We define a strategy for $O$ in $\altrothgame{X}$ as follows: during round $n$, if $O$ has chosen $U_0,\dots,U_{n-1}$ in previous rounds, $O$ chooses an open set from the cover $\sigma(U_0,\dots,U_{n-1})$ containing the point chosen by $P$ that round. If $\sigma$ was a winning strategy for $C$, then for any open sets $U_0,U_1,\dots$ from the covers $\sigma(\cdot),\sigma(U_0),\dots$, the collection $\{U_0,U_1,\dots\}$ is not a cover for $X$.

      \item
      Let $\sigma$ be a strategy for $O$ in $\altrothgame{X}$.

      We define a strategy for $C$ in $\rothgame{X}$ as follows: during round $0$, $C$ chooses $\mc{U}_0 = \{\sigma(x) : x\in X\}$. In response, $S$ chooses some $\sigma(x_0)$. During round $n+1$, if $S$ has chosen $\sigma(x_0),\dots,\sigma(x_0,\dots,x_n)$ in previous rounds, $C$ chooses $\mc{U}_{n+1} = \{\sigma(x_0,\dots,x_n,x) : x\in X\}$. If $\sigma$ was a winning strategy for $O$, then $\{\sigma(x_0),\sigma(x_0,x_1),\dots\}$ is not a cover for $X$.

    \end{itemize}
  \end{proof}

  \begin{theorem}
    The following are equivalent for points-$G_\delta$ $X$:
      \begin{enumerate}[(a)]
        \item $S\win\rothgame{X}$
        \item $S\markwin\rothgame{X}$
        \item $|X|=\omega$
      \end{enumerate}
  \end{theorem}

  \begin{proof}
    $(c)\Rightarrow(b)\Rightarrow(a)$ are all trivial. Galvin has shown $(a)\Rightarrow(c)$ in \textit{Indeterminacy}, but a direct proof follows. Let $\sigma$ be a strategy for $S$ in $\rothgame{X}$.

    Let $G_{x,m}$ designate open sets such that $\{x\}=\bigcap_{m<\omega}G_{x,m}$ for all $x\in X$.

    Suppose for $s\in\omega^{<\omega}$, $\mc U_t$ is defined for each $t\leq s$. Then $C$ may find a point $x_s$ such that for each $m<\omega$, there exists an open cover $\mc U_{s\concat\<m\>}$ where $\sigma(\mc U_{s\rest1},\dots,\mc U_{s},\mc U_{s\concat\<m\>})=G_{x_s,m}$. (If not, let $m(x)$ witness the contrary for each $x$, and consider $\sigma(\mc U_{s\rest0},\dots,\mc U_s,\{G_{x,m(x)}:x\in X\})$ to see the contradiction.)

    If $x\not\in\{x_s:s\in\omega^{<\omega}\}$, then $C$ may counter the strategy $\sigma$ by choosing $f\in\omega^\omega$ where $x\not\in G_{x_{f\rest n},f(n)}$, and playing $\mc U_{f\rest1},\mc U_{f\rest2},\dots$, for which 
      \[
        x\not\in 
        \bigcup_{n<\omega} G_{x_{f\rest n},f(n)}=
        \bigcup_{n<\omega} \sigma(\mc U_{f\rest1},\dots,\mc U_{f\rest n+1})
      \]

    Thus if $\sigma$ is a winning strategy, then $X=\{x_s:s\in\omega^{<\omega}\}$ is countable.
  \end{proof}

  \begin{proposition}
    $S\win\rothgame{\oneptlind{\omega}}$
  \end{proposition}

  \begin{question}
    $S\kmarkwin{2}\rothgame{\oneptlind{\omega}}$?
  \end{question}

  \begin{question}
    $S\kmarkwin{2}\rothgame{X}\Leftrightarrow S\kmarkwin{k}\rothgame{X}$?
  \end{question}

  \begin{question}
  For any space TFAE:
    \begin{itemize}
      \item $S\markwin\rothgame{X}$
      \item $|X|=\omega$
    \end{itemize}
  \end{question}

\end{document}