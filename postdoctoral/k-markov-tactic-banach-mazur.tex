\documentclass[11pt]{article}

\usepackage{amssymb}
\usepackage{amsfonts}
\usepackage{amsmath}
\usepackage{mathtools}
\usepackage{amsthm}

\usepackage[letterpaper,margin=1in]{geometry}

\usepackage{enumerate}

      \theoremstyle{plain}
      \newtheorem{theorem}{Theorem}
      \newtheorem{lemma}[theorem]{Lemma}
      \newtheorem{corollary}[theorem]{Corollary}
      \newtheorem{proposition}[theorem]{Proposition}
      \newtheorem{conjecture}[theorem]{Conjecture}
      \newtheorem{question}[theorem]{Question}
      \newtheorem{claim}[theorem]{Claim}

      \theoremstyle{definition}
      \newtheorem{definition}[theorem]{Definition}
      \newtheorem{example}[theorem]{Example}
      \newtheorem{observation}[theorem]{Observation}
      \newtheorem{game}[theorem]{Game}

      \theoremstyle{remark}
      \newtheorem{remark}[theorem]{Remark}

      \theoremstyle{plain}
      \newtheorem*{theorem*}{Theorem}
      \newtheorem*{lemma*}{Lemma}
      \newtheorem*{corollary*}{Corollary}
      \newtheorem*{proposition*}{Proposition}
      \newtheorem*{conjecture*}{Conjecture}
      \newtheorem*{question*}{Question}
      \newtheorem*{claim*}{Claim}

      \theoremstyle{definition}
      \newtheorem*{definition*}{Definition}
      \newtheorem*{example*}{Example}
      \newtheorem*{observation*}{Observation}
      \newtheorem*{game*}{Game}

      \theoremstyle{remark}
      \newtheorem*{remark*}{Remark}

\title{Tactics and Marks in Banach Mazur Games}
\author{Steven Clontz}

\usepackage{../clontzDefinitions}

\newcommand{\bmPoGame}[2]{BM_{po}(#1,#2)}

\begin{document}

\maketitle

  My notes on Galvin/Telgarsky's Theorem 5 from \cite{MR831181}.

  \begin{definition}
    Let \(\mb P\) be partially ordered by \(\leq\).
    Let \(\mb P^{\downarrow}=\{f\in\mb P^\omega : f(n)\geq f(n+1)\}\).
    Then for \(f,g\in\mb P^\downarrow\), we say that \(f,g\) zip into each
    other if for all \(m<\omega\) there exists \(n<\omega\) such that
    \(f(m)\geq g(n)\) and \(g(m)\geq f(n)\).
  \end{definition}

  \begin{definition}
    \(\bmPoGame{\mb P}{W}\) is a game defined for all non-empty partial orders
    \(\mb P\) and all subsets \(W\subseteq\mb P^\downarrow\).
    During round \(0\), \(\plI\) chooses \(a_0\in\mb P\),
    and then \(\plII\) chooses \(b_0\leq a_0\); during around \(n+1\),
    \(\plI\) chooses \(a_{n+1}\leq b_n\), and then \(\plII\) chooses
    \(b_{n+1}\leq a_{n+1}\). \(\plII\) wins this game if
    \(\<a_0,a_1,\dots\>\in W\).
  \end{definition}

  \begin{theorem}
    Let \(W\subseteq\mb P^\downarrow\) be closed under zipping.
    \(\plII\markwin\bmPoGame{\mb P}{W}\) if and only if
    \(\plII\tactwin\bmPoGame{\mb P}{W}\).
  \end{theorem}

  \begin{proof}
    Let \(\tau(p,n+1)\) be a winning mark for \(\plII\), where \(p\)
    is the most recent move by \(\plI\) and \(n+1\)
    is the number of moves made by \(\plI\).
    Define \(\tau^0(p)=p\) and \(\tau^{n+1}(p)=\tau(\tau^n(p),n+1)\).
    Let \(\preceq\) well-order \(\mb P\).

    For \(p,q\in \mb P\), say \(p\geq_n q\) if there exist
    \(s_m(p)\in\mb P\) for \(m\leq n\)
    such that
    \[
      p
    \geq
      s_m(p)
    \geq
      \tau(s_m(p),n+1)
    \geq
      q
    .\]
    Note that \(p'\geq p\geq_n q\geq q'\) implies \(p'\geq_n q'\),
    and \(p\geq_n \tau^n(p)\).

    Say \(p\geq_\omega q\) whenever \(p\geq_n q\) for all \(n<\omega\).
    If \(p\geq_\omega l(p)\) for some \(l(p)\), then say \(p\) is long;
    otherwise call \(p\) short.

    For \(p\) short, let
    \[
      \mu(p)
        =
      \min_{\preceq}\{
        r\text{ short}
          :
        r\geq p
      \}
    \]
    and since \(\mu(p)\not\geq_n p\) for some \(n\), let
    \[
      N(p)
        =
      \min\{
        n<\omega
      :
        \mu(p)\not\geq_n p
      \}
    .\]
    Note that whenever \(\mu(p)=\mu(q)\) for \(p\geq_n q\),
    it follows that \(\mu(p)\geq_n q\) and therefore \(N(p)<N(q)\).

    We define
    \[
      \sigma( p)
        =
      \begin{cases}
        l(p) & p \text{ is long} \\
        \tau^{N(p)+1}( p) &  p \text{ is short}
     \end{cases}
    .\]

    Suppose \(\sigma\) is legally attacked by \(a\in\mb P^\omega\).
    For \(n\leq\omega\), if \(a(n)\) is long, then \(a(n)\geq_n l(a(n))\).
    Therefore,
    \[
      a(n)
        \geq
      s_n(a(n))
        \geq
      \tau(s_n(a(n)),n+1)
        \geq
      l(a(n))
        =
      \sigma(a(n))
        \geq
      a(n+1)
    .\]
    Thus if \(a(n)\) is long for \(n<\omega\), it follows that
    \(c\in\mb P^\downarrow\) defined by \(c(n)=s_n(a(n))\)
    is a legal attack against \(\tau\). Since \(\tau\) is winning,
    \(c\in W\), and since \(c\) zips into \(a\),
    \(a\in W\) as well.

    Otherwise, we may choose a final subsequence \(b\) of \(a\) such that
      \begin{itemize}
        \item \(b(n)\) is short for all \(n<\omega\),
              since \(a(m)\) short implies \(a(n+m)\) short for all
              \(n<\omega\).
        \item \(\mu(b(n))=\mu'\) is fixed for all \(n<\omega\), since
              there cannot be an infinite \(\preceq\)-decreasing sequence.
      \end{itemize}
    As a result,
    \[
      b(n)
        \geq_{N(b(n))}
      \tau^{N(b(n))+1}(b(n))
        =
      \sigma(b(n))
        \geq
      b(n+1)
    \]
    and therefore \(N(b(n))<N(b(n+1))\). In particular, \(N(b(n))\geq n\).

    Thus for \(n<\omega\),
    \[
      b(n)
        \geq
      \tau^{n}(b(n))
        \geq
      \tau(\tau^{n}(b(n)),n+1)
        \geq
      \tau^{N(b(n))+1}(b(n))
        =
      \sigma(b(n))
        \geq
      b(n+1)
    .\]
    As a result, \(c\in\mb P^\downarrow\) defined by \(c(n)=\tau^n(b(n))\)
    is a legal attack against the winning strategy \(\tau\). Therefore
    \(c\in W\), and since \(c\) zips into \(b\) and \(a\), we conclude
    \(a\in W\).
  \end{proof}

  \begin{observation}
    When \(\mb P=T(X)\setminus\{\emptyset\}\) is ordered by set-inclusion
    and \(W=\{U\in\mb P^\downarrow:\bigcap_{n<\omega}U(n)\not=\emptyset\}\),
    then \(\bmPoGame{\mb P}{W}\) is exactly the topological Banach Mazur game
    \(\bmGame{X}\). Note \(W\) is closed under zipping.
  \end{observation}

  \begin{corollary}
  \(\plII\markwin\bmGame{X}\) if and only if
  \(\plII\tactwin\bmGame{X}\).
  \end{corollary}

  \begin{observation}
    When \(\mb P=\{(U,x):U\in T(X)\setminus\{\emptyset\},x\in U\}\)
    is ordered by \((U,x)\geq(V,y)\) whenever \(x\in V\subseteq U\)
    and \(W=\{\<U(n),x(n)\>_{n<\omega}\in\mb
    P^\downarrow:\bigcap_{n<\omega}U(n)\not=\emptyset\}\),
    then \(\bmPoGame{\mb P}{W}\) is almost the Choquet game, except the first
    player also gets to choose a point. Note \(W\) is closed under zipping.
  \end{observation}

  \bibliographystyle{plain}
  \bibliography{../bibliography}


\end{document}
