\documentclass[11pt]{article}

\usepackage{amssymb}
\usepackage{amsfonts}
\usepackage{amsmath}
\usepackage{mathtools}
\usepackage{amsthm}

      \theoremstyle{plain}
      \newtheorem{theorem}{Theorem}
      \newtheorem{lemma}[theorem]{Lemma}
      \newtheorem{corollary}[theorem]{Corollary}
      \newtheorem{proposition}[theorem]{Proposition}
      \newtheorem{conjecture}[theorem]{Conjecture}
      \newtheorem{question}[theorem]{Question}
      \newtheorem{claim}[theorem]{Claim}

      \theoremstyle{definition}
      \newtheorem{definition}[theorem]{Definition}
      \newtheorem{example}[theorem]{Example}
      \newtheorem{game}[theorem]{Game}

      \theoremstyle{remark}
      \newtheorem{remark}[theorem]{Remark}

      \theoremstyle{plain}
      \newtheorem*{theorem*}{Theorem}
      \newtheorem*{lemma*}{Lemma}
      \newtheorem*{corollary*}{Corollary}
      \newtheorem*{proposition*}{Proposition}
      \newtheorem*{conjecture*}{Conjecture}
      \newtheorem*{question*}{Question}
      \newtheorem*{claim*}{Claim}

      \theoremstyle{definition}
      \newtheorem*{definition*}{Definition}
      \newtheorem*{example*}{Example}
      \newtheorem*{game*}{Game}

      \theoremstyle{remark}
      \newtheorem*{remark*}{Remark}

% Strategy uparrow shortcuts
\newcommand{\win}{\uparrow}
\newcommand{\prewin}{\underset{\text{pre}}{\uparrow}}
\newcommand{\markwin}{\underset{\text{mark}}{\uparrow}}
\newcommand{\tactwin}{\underset{\text{tact}}{\uparrow}}
\newcommand{\kmarkwin}[1]{\underset{#1\text{-mark}}{\uparrow}}
\newcommand{\ktactwin}[1]{\underset{#1\text{-tact}}{\uparrow}}
\newcommand{\codewin}{\underset{\text{code}}{\uparrow}}
\newcommand{\flexmarkwin}{\underset{\text{flexmark}}{\uparrow}}
\newcommand{\semiflexmarkwin}{\underset{\text{semiflexmark}}{\uparrow}}
\newcommand{\limitwin}{\underset{\text{limit}}{\uparrow}}
\newcommand{\notwin}{\not\uparrow}
\newcommand{\notprewin}{\underset{\text{pre}}{\not\uparrow}}
\newcommand{\notmarkwin}{\underset{\text{mark}}{\not\uparrow}}
\newcommand{\nottactwin}{\underset{\text{tact}}{\not\uparrow}}
\newcommand{\notkmarkwin}[1]{\underset{#1\text{-mark}}{\not\uparrow}}
\newcommand{\notktactwin}[1]{\underset{#1\text{-tact}}{\not\uparrow}}
\newcommand{\notcodewin}{\underset{\text{code}}{\not\uparrow}}
\newcommand{\notflexmarkwin}{\underset{\text{flexmark}}{\not\uparrow}}
\newcommand{\notsemiflexmarkwin}{\underset{\text{semiflexmark}}{\uparrow}}
\newcommand{\notlimitwin}{\underset{\text{limit}}{\not\uparrow}}

\newcommand{\onePtComp}[1]{#1^\star}
\newcommand{\onePtLind}[1]{#1^\dagger}


% Games
\newcommand{\gruConGame}[2]{Gru_{O,P}^{\to}\left({#1},{#2}\right)}
\newcommand{\gruConGameHard}[2]{Gru_{O,P}^{\to,\star}\left({#1},{#2}\right)}
\newcommand{\gruClusGame}[2]{Gru_{O,P}^{\leadsto}\left({#1},{#2}\right)}
\newcommand{\gruClusGameHard}[2]{Gru_{O,P}^{\leadsto,\star}\left({#1},{#2}\right)}

\newcommand{\gruKPGame}[1]{Gru_{K,P}\left({#1}\right)}
\newcommand{\gruKLGame}[1]{Gru_{K,L}\left({#1}\right)}

\newcommand{\cloPFGame}[1]{PtFin_{F,C}\left({#1}\right)}

\newcommand{\menGame}[1]{Men_{C,F}\left({#1}\right)}
\newcommand{\rothGame}[1]{Roth_{C,S}\left({#1}\right)}
\newcommand{\rothAltGame}[1]{Roth_{P,O}\left({#1}\right)}

\newcommand{\barmanDowFanGame}[2]{BD_{B,F}\left({#1},{#2}\right)}


\newcommand{\schFillStrictGame}[1]{Sch^{\cup,\subset}_{C,F}\left({#1}\right)}
\newcommand{\schFillGame}[1]{Sch^{\cup,\subseteq}_{C,F}\left({#1}\right)}
\newcommand{\schFillWeakGame}[1]{Sch^{\cup}_{C,F}\left({#1}\right)}
\newcommand{\schFillInitialGame}[1]{Sch^{1,\subseteq}_{C,F}\left({#1}\right)}
\newcommand{\schFillIntGame}[1]{Sch^{\cap}_{C,F}\left({#1}\right)}

\newcommand{\bellUniGame}[1]{Bell_{D,P}^{\textrm{uni}}\left({#1}\right)}
\newcommand{\bellConGame}[1]{Bell_{D,P}^{\to,\emptyset}\left({#1}\right)}
\newcommand{\bellConHardGame}[1]{Bell_{D,P}^{\to,\emptyset,\star}\left({#1}\right)}
\newcommand{\bellAbsConGame}[1]{Bell_{D,P}^{\to}\left({#1}\right)}
\newcommand{\bellAbsConHardGame}[1]{Bell_{D,P}^{\to,\star}\left({#1}\right)}

\newcommand{\sPrinciple}[2]{S(#1,#2)}
\newcommand{\sGame}[2]{G(#1,#2)}



\newcommand{\concat}{{^\frown}}
\newcommand{\rest}{\restriction}
\newcommand{\last}{\downharpoonright}

\newcommand{\cl}[1]{\overline{#1}}

\newcommand{\pow}[1]{\mc{P}(#1)}

\newcommand{\<}{\langle}
\renewcommand{\>}{\rangle}

\newcommand{\al}[1]{{#1}^*}

\newcommand{\mc}[1]{\mathcal{#1}}
\newcommand{\mb}[1]{\mathbb{#1}}
\newcommand{\mf}[1]{\mathfrak{#1}}

\newcommand{\po}{\mathbb{P}}
\newcommand{\pok}{\po_\kappa}

\newcommand{\Lim}{\mathrm{Lim}}
\newcommand{\Suc}{\mathrm{Suc}}

\newcommand{\ds}{\displaystyle}

\newcommand{\st}[2]{st\left(#1,#2\right)}

\newcommand{\alcomp}{\al\parallel}

\newcommand{\rank}{\textrm{rank}}
\newcommand{\dom}{\textrm{dom}}

\renewcommand{\mod}{\,\textrm{mod}}

\newcommand{\zip}{\bowtie}
\newcommand{\ran}[1]{\text{range}(#1)}

\newcommand{\cf}[1]{\textrm{cf}(#1)}

\newcommand{\alcompS}[1]{\mc A(#1)}
\newcommand{\alcompSP}[1]{\mc A'(#1)}
\newcommand{\splendidS}[1]{\mc S(#1)}
\newcommand{\kurepaS}[1]{\mc K(#1)}


\newcommand{\scish}{almost-$\sigma$-(relatively compact)}

\usepackage{mathrsfs}
\newcommand{\pl}[1]{\mathscr{#1}}



\newcommand{\term}{\textit}


\newcommand{\bakerGame}[1]{{Bak}_{A,B}(#1)}
\newcommand{\bmGame}[1]{{BM}_{E,N}(#1)}


% http://tex.stackexchange.com/a/251107/79754
\usepackage{amsmath}
\usepackage{graphicx}
\makeatletter
\newcommand{\sumlower}{\mathop{\mathpalette\@sumlower{\,}}\slimits@}
\newcommand{\@sumlower}[2]{%
  \vphantom{\sum}%
  \sbox\z@{$\m@th#1\sum$}%
  \dimen@=\ht\z@ \advance\dimen@\dp\z@
  \dimen\tw@=\wd\z@
  \displaystyle\dimen@=.6\dimen@
  \ifx#1\displaystyle\dimen@=.9\dimen@\fi
  \ooalign{%
    \hidewidth
    $\vcenter{%
     \vspace{.1\dimen@}%
     \hbox{$\m@th\@demotestyle#1#2$\kern.3\dimen\tw@}%
     \ifx#1\scriptstyle\kern-.25ex\fi}$\hidewidth\cr
    $\vcenter{\hbox{%
      \resizebox{!}{\dimen@}{$\m@th\sigma$}%
    }\ifx#1\scriptstyle\kern-.25ex\fi}$\cr
  }%
}
\newcommand\@demotestyle[1]{%
  \ifx#1\displaystyle
  \else
    \ifx#1\textstyle
      \scriptstyle
    \else
      \scriptscriptstyle
    \fi
  \fi
}
\newcommand{\sumstar}{\mathop{\sum\kern-1ex\star}\slimits@}
\makeatother

\newcommand{\SigmaProd}{\sum}
\newcommand{\SigmaPower}[2]{{\Sigma#1^#2}}
\newcommand{\SigmaPowerR}[1]{{\SigmaPower{\mathbb{R}}{#1}}}
\newcommand{\SigmaPowerTwo}[1]{{\SigmaPower{2}{#1}}}

\newcommand{\SigmaStarProd}{\sumstar}
\newcommand{\SigmaStarPower}[2]{{\Sigma^\star#1^#2}}
\newcommand{\SigmaStarPowerR}[1]{{\SigmaStarPower{\mathbb{R}}{#1}}}
\newcommand{\SigmaStarPowerTwo}[1]{{\SigmaStarPower{2}{#1}}}

\newcommand{\sigmaProd}{\sumlower}
\newcommand{\sigmaPower}[2]{{\sigma#1^#2}}
\newcommand{\sigmaPowerR}[1]{{\sigmaPower{\mathbb{R}}{#1}}}
\newcommand{\sigmaPowerTwo}[1]{{\sigmaPower{2}{#1}}}

\newcommand{\supp}{\textrm{supp}}




\begin{document}

\title{\(2\)-Markov Strategies in Selection Games}
\author{Steven Clontz}

\maketitle
\abstract{asdf}

\section{Introduction}

Let \(\sPrinciple{\mc A}{\mc B}\) be the statement that whenever
\(A_n\in\mc A\) for \(n<\omega\), there exist \(B_n\in[A_n]^{<\omega}\)
such that \(\bigcup\{B_n:n<\omega\}\in\mc B\). This
\term{selection principle} characterizes a property of a topological space
\(X\) when \(\mc A,\mc B\) are defined in terms of \(X\). For example,
if \(\mc O_X\) is the collection of open covers of \(X\), then
\(\sPrinciple{\mc O_X}{\mc O_X}\) is the well-known Menger covering property.

This property may be made stronger by considering the following two-player
game of length \(\omega\): \(\sGame{\mc A}{\mc B}\). During each round
\(n<\omega\), the first player \(\pl A\) chooses \(A_n\in\mc A\), followed
by \(\pl B\) choosing \(B_n\in[A_n]^{<\omega}\). \(\pl B\) wins the game
if \(\bigcup\{B_n:n<\omega\}\in\mc B\); otherwise \(\pl A\) wins.
If \(\pl B\) has a \term{winning strategy} for the game (a function
which defines a move for each finite sequence of previous moves by
\(\pl A\), and beats every possible response by \(\pl A\)), then we write
\(\pl B\win\sGame{\mc A}{\mc B}\).

These concepts were first introduced by Scheepers in [MR1378387].
Of course,
\(\pl B\win\sGame{\mc A}{\mc B}\Rightarrow \sPrinciple{\mc A}{\mc B}\),
but the converse need not hold, since each \(B_n\) may be defined in
\(\sPrinciple{\mc A}{\mc B}\) using knowledge of all \(A_n\), not just
those ``previously played''. Thus for each topological property \(P\)
characterized by \(\sPrinciple{\mc A}{\mc B}\), we denote the (possibly)
stronger property \(\pl B\win\sGame{\mc A}{\mc B}\) as \term{strategic \(P\)}.

Such notions may be made even stronger using
\term{limited information strategies}. A \term{\(k\)-Markov strategy} for
\(\pl B\) uses only the last \(k\) moves of \(\pl A\) and the round number.
When \(\pl B\) has a winning \(k\)-Markov strategy for \(\sGame{\mc A}{\mc B}\),
we write \(\pl B\kmarkwin{k}\sGame{\mc A}{\mc B}\).
Similarly, for each topological property \(P\)
characterized by \(\sPrinciple{\mc A}{\mc B}\), we denote
property \(\pl B\kmarkwin{k}\sGame{\mc A}{\mc B}\) as \term{\(k\)-Markov \(P\)}.
When \(k=1\), we may omit the \(k\).

\section{\(k\)-Markov implies \(2\)-Markov}

In the case of the selection game \(\sGame{\mc A}{\mc B}\), we may see that
a \((k+2)\)-Markov strategy may always be improved to a \(2\)-Markov strategy,
as shown by the author in [clontzMengerPreprint] with regards to
\(\sGame{\mc O_X}{\mc O_X}\).

\begin{theorem}
  For each \(k<\omega\), \(\pl B \kmarkwin{(k+2)} \sGame{\mc A}{\mc B}\)
  if and only if \(\pl B \kmarkwin{2} \sGame{\mc A}{\mc B}\).
\end{theorem}

\begin{proof}
  Let \(\sigma\) be a winning \((k+2)\)-mark. We define the \(2\)-mark
  \(\tau\) as follows:
    \[
      \tau(\<A\>,0)
        =
      \bigcup_{m<k+1}
        \sigma(\<\underbrace{A,\dots,A}_{m+1}\>,m)
    \]
    \[
      \tau(\<A,A'\>,n+1)
        =
      \bigcup_{m<k+1}
        \sigma(\<
          \underbrace{A,\dots,A}_{k+1-m},
          \underbrace{A',\dots,A'}_{m+1}
        \>,(n+1)(k+1)+m)
    \]

  Let \(\<A_0,A_1,\dots\>\) be an attack by \(\pl A\) against \(\tau\).
  Then consider the attack
    \[
      \<
        \underbrace{A_0,\dots,A_0}_{k+1},
        \underbrace{A_1,\dots,A_1}_{k+1},
        \dots
      \>
    \]
  by \(\pl A\) against \(\sigma\). Since \(\sigma\) is a winning \((k+2)\)-mark,
    \[
      \bigcup_{m<k+1}
        \sigma(\<\underbrace{A_0,\dots,A_0}_{m+1}\>,m)
      \cup
      \bigcup_{n<\omega,m<k+1}
        \sigma(\<
          \underbrace{A_n,\dots,A_n}_{k+1-m},
          \underbrace{A_{n+1},\dots,A_{n+1}}_{m+1}
        \>,(n+1)(k+1)+m)
    \]
    \[
      =
      \tau(\<A_0\>,0)
      \cup
      \bigcup_{n<\omega}
      \tau(\<A_n,A_{n+1}\>,n+1)
        \in
      \mc B
    \]
  Thus \(\tau\) is a winning \(2\)-mark.
\end{proof}

The following natrual question is open:

\begin{question}
  Do there exist (interesting/topological) \(\mc A,\mc B\) such that
  \(\pl B\win\sGame{\mc A}{\mc B}\) but
  \(\pl B\notkmarkwin{2}\sGame{\mc A}{\mc B}\)?
\end{question}

\section{Menger game results}

Consider the case that \(\mc A=\mc B=\mc O_X\),
i.e. the Menger game. The following summarize results from
[MR1129143]
[clontzMengerPreprint] and [clontzDowAlcompPreprint].

\begin{definition}
  For any cardinal \(\kappa\), let \(\onePtLind\kappa=\kappa\cup\{\infty\}\)
  denote the \term{one-point Lindel\"of-ication} of discrete \(\kappa\),
  where points in \(\kappa\) are isolated, and the neighborhoods of
  \(\infty\) are co-countable.
\end{definition}

\begin{theorem}
  \(\pl B\win\sGame{\mc O_{\onePtLind\kappa}}{\mc O_{\onePtLind\kappa}}\).
\end{theorem}

\begin{definition}
  For two functions \(f,g\) we say \(f\) is \term{almost compatible} with
  \(g\) if \(|\{x\in\dom(f)\cap\dom(g):f(x)\not=g(x)\}|<\omega\).
\end{definition}

\begin{definition}
  \(\alcompSP\kappa\) states that there exists a collection of pairwise
  almost compactible finite-to-one functions
  \(\{f_A\in\omega^A:A\in[\kappa]^{\leq\omega}\}\).
\end{definition}

\begin{theorem}
  \(\alcompSP{\omega_n}\) holds for all \(n<\omega\).
\end{theorem}

\begin{theorem}
  \(\alcompSP\kappa\) implies
  \(\pl B\kmarkwin{2}\sGame{\mc O_{\onePtLind\kappa}}{\mc O_{\onePtLind\kappa}}\).
\end{theorem}

\begin{theorem}
  For any cardinal \(\kappa\),
  \(\kappa\) Cohen reals may be added to a model of \(ZFC+CH\)
  while preserving \(\alcompSP{\mf c}\).
\end{theorem}

\begin{theorem}
  There exists a model of \(ZFC\) where \(\alcompSP{\omega_\omega}\) fails.
\end{theorem}

\begin{theorem}
  \(
    \pl B
      \kmarkwin{2}
    \sGame{
      \mc O_{\onePtLind{\omega_\omega}}
    }{
      \mc O_{\onePtLind{\omega_\omega}}
    }
  \).
\end{theorem}

It remains open whether
  \(
    \pl B
      \kmarkwin{2}
    \sGame{
      \mc O_{\onePtLind{\omega_{\omega+1}}}
    }{
      \mc O_{\onePtLind{\omega_{\omega+1}}}
    }
  \)
might fail when \(\alcompSP{\omega_\omega}\) fails. Due to the above,
any attempt to show
  \(
    \pl B
      \notkmarkwin{2}
    \sGame{
      \mc O_{\onePtLind{\kappa}}
    }{
      \mc O_{\onePtLind{\kappa}}
    }
  \)
cannot happen solely within \(ZFC\).

\end{document}