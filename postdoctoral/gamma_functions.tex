\documentclass[11pt]{article}

\usepackage{../clontzStyle}
\usepackage{../clontzDefinitions}

\renewcommand{\int}{\textrm{int}}
\renewcommand{\cl}{\textrm{cl}}
\newcommand{\lexTimes}{\times_{\textrm{lex}}}
\newcommand{\vect}{\vec}

\begin{document}

  \begin{definition}
    Let a V-map be a u.s.c. idempotent surjection.
  \end{definition}

  \begin{definition}
    For any LOS \(\<L,\leq\>\), let \(\check L\) be the collection of
    leftward subsets of \(L\)
    (subsets for which \(b\in L,a\leq b\Rightarrow a\in L\))
    linearly ordered by \(\subseteq\), and let \(\hat L\) be the collection
    of left-closed subsets of \(L\) (leftward subsets which are closed)
    linearly ordered by \(\subseteq\).
  \end{definition}

  \begin{proposition}
    \(\check L\), \(\hat L\) are compact.
  \end{proposition}

  \begin{proof}
    Each subset \(S\) has an infimum \(\cap S\) and a supremum \(\cup S\)
    (or \(\cl(\cap S)\)).
  \end{proof}

  Note that \(\check L\)
  is not a ``compactification'' as \(L\) does not necessarily
  embed as a dense subspace of \(\check L\): if \(L=I\), we might attempt to embed
  \(t\mapsto [0,t)\), but then note that the subspace topology induces the
  reverse Sorgenfrey interval as \(([0,s),[0,t])=([0,s),[0,t)]\) is open.
  However \(\hat L\) is
  the typical way of compactifying a linearly ordered space \(L\),
  provided \(L\) lacks a least element (otherwise the empty set is an [easily
  removable] isolated point in \(\hat L\)).

  \begin{definition}
    For any compact LOTS \(K\) with minimum \(0\) and maximum \(1\),
    let \(\gamma\) be the V-map on \(K\) where \(\gamma(0)=K\) and \(\gamma(t)=\{1\}\)
    for \(t>0\).
  \end{definition}

  \begin{definition}
    For any LOTS \(M\) with minimum element \(0\),
    let \(\nu\) be the V-map on \(M\) where \(\nu(0)=K\) and \(\nu(t)=\{t\}\)
    for \(t>0\).
  \end{definition}

  Note for \(K=M=2\) that \(\gamma=\nu\).

  \begin{theorem}
    \(X=\varprojlim \{2, \nu, L\}\cong \check L\)
  \end{theorem}

  \begin{proof}
    We start by placing an order on \(X\). Let \(\vect x<\vect y\) if
    there exists \(a\in L\) with \(\vect x(a)=0,\vect y(a)=1\). We claim this is
    a total order inducing the topology on \(X\).

    We first observe that if \(\vect x(b)=1\), then for all \(a\leq b\),
    \(\vect x(a)\in\nu(1)=\{1\}\). If \(\vect x\not=\vect y\), then assume
    without loss of generality that \(\vect x(a)=0,\vect y(a)=1\), so
    \(\vect x<\vect y\). Also, whenever \(\vect x(b)=1\), we have that \(b<a\),
    so \(\vect y(b)=1\), preventing \(\vect y<\vect x\). Finally if
    \(\vect x<\vect y\) and \(\vect y<\vect z\), take \(a,b\) with
    \(\vect x(a)=0\), \(\vect y(a)=1\),\(\vect y(b)=0\),\(\vect z(b)=1\). It
    follows that \(a<b\) so \(\vect z(a)=1\) and \(\vect x<\vect z\).

    Consider the basic open set \(B(\vect x,F)\) for a finite set
    \(F\in [L]^{<\omega}\)
    about the sequence \(\vect x\in X\) which contains all sequences
    \(\vect y\) agreeing with \(\vect x\) on \(F\). If \(\vect x(a)=1\) for all
    \(a\in F\), then let \(\vect w\in X\) be \(0\) on the maximum of \(F\),
    and \(1\) for anything less. It follows that
    \(B(\vect x,F)=(\vect w,\rightarrow)\). If \(\vect x(a)=0\) for all
    \(a\in F\), then let \(\vect y\in X\) be \(1\) on the minimum of \(F\),
    and \(0\) for anything greater. It follows that
    \(B(\vect x,F)=(\leftarrow,\vect y)\). Finally if \(\vect x(a)=1\) and
    \(\vect x(b)=0\) for \(a<b\) in \(F\) and nothing between \(a,b\) is in
    \(F\), then let \(\vect w\in X\) be \(0\) on \(a\)
    and \(1\) for anything less, and let \(\vect y\in X\) be \(1\) on \(b\)
    and \(0\) for anything greater. It follows that
    \(B(\vect x,F)=(\vect w,\vect y)\).

    Let \(\phi\) evaluate each \(\vect x\in X\subseteq 2^L\) as the
    characteristic function for a subset of \(L\). It's easy to see that
    \(\phi\) is an order isomorphism between \(\<X,\leq\>\) and
    \(\<\check L,\subseteq\>\).
  \end{proof}

  \begin{corollary}
    \(
      \varprojlim \{2, \nu, \alpha\}
      \cong
      \alpha+1
    \)
    for every ordinal \(\alpha\).
  \end{corollary}

  \begin{proof}
    Since \(\check\alpha=\alpha+1\) (actually equals, not just homeomorphic!),
    we get \(\varprojlim^\star \{2, \nu, \alpha\}
      \cong \check\alpha =
    \alpha+1\) for free.
  \end{proof}

  We introduce an alternate definition of an arbitrarily indexed
  inverse limit.

  \begin{definition}
    Let \(\varprojlim^\star\{X,f,L\}\subseteq\varprojlim\{X,f,L\}\) satisfy
    that \(\vect x(a)=\lim_{t\to a}\vect x(t)\) for all \(a\in L\)
    (for any open neighborhood
    \(U\)of \(\vect x(a)\) there is \(b<a\) where \(\vect x(t)\in U\)
    for all \(t\in(b,a]\)).
  \end{definition}

  \begin{theorem}
    \(Y=\varprojlim^\star \{2, \nu, L\}\cong \hat L\).
  \end{theorem}

  \begin{proof}
    Consider \(Y\) as a subspace of \(X=\varprojlim \{2, \nu, L\}\) with
    the linear order described above. We claim that if \(\phi\) is the
    characteristic function for a subset of \(L\), then \(\phi\)
    is an order isomorphsim between \(\<Y,\leq\>\) and
    \(\<\hat L,\subseteq\>\).

    Let \(A\) be a left-closed subset of \(L\). Let \(\vect x(a)=1\) when
    \(a\in A\) and \(\vect x(a)=0\) otherwise. Then \(\vect x\in Y\) and
    \(\phi(\vect x)=A\).

    Let \(\vect x,\vect y\in Y\). If
    \(\phi(\vect x)=\phi(\vect y)=A\), then \(A\) is a
    left-closed set where \(\vect x(a)=\vect y(a)=1\) for \(a\in A\)
    and \(\vect x(a)=\vect y(a)=0\) otherwise, so \(\vect x=\vect y\).

    Finally let \(\vect x<\vect y\), so there exists \(a\in L\) with
    \(\vect x(a)=0\), \(\vect y(a)=1\). Then
    \(
      \phi(\vect x)
        \subseteq
      (\leftarrow,a)
        \subseteq
      \phi(\vect y)
    \). Thus \(\phi\) preserves order.
  \end{proof}

  \begin{corollary}
    \(
      \varprojlim^\star \{2, \nu, \alpha\}
      \cong
      \alpha+1
    \)
    for every infinite limit or finite ordinal \(\alpha\).
  \end{corollary}

  \begin{proof}
    If \(\alpha\) is finite, then of course all (leftward) sets are
    closed and we get \(\hat\alpha=\check\alpha=\alpha+1\) for free.
    Otherwise, since \(\alpha\) lacks a greatest point, \(\hat\alpha\)
    is homeomorphic to its usual compactification \(\alpha+1\).
  \end{proof}

  In fact, \(\hat\alpha=\alpha+1\setminus L(\alpha)\) where \(L(\alpha)\)
  is the colleciton of all limit ordinals less than \(\alpha\), which also
  shows \(\hat\alpha\cong\alpha\) for infinite successor ordinals \(\alpha\).

  \begin{theorem}
    Let \(M\) be a LOTS with minimum \(0\).
    \(
      Z
        =
      \varprojlim^\star \{M, \nu, L\}
        \cong
      \{-\infty\}\cup (\hat L\setminus\emptyset)\lexTimes M'
    \)
    where \(M'=M\setminus\{0\}\),
    \(\lexTimes\) induces the lexicographic ordering on the product,
    and \(-\infty\) is a least element.
  \end{theorem}

  \begin{proof}
    Let \(\rho(\vect x)=\{a\in L:\vect x(a)>0\}\);
    this is obviously leftwards, and it's closed for all \(\vect x\in Z\).
    Then we give \(Z\) the linear order where \(\vect x<\vect y\) if
    \(\rho(\vect x)\subsetneq\rho(\vect y)\) or both
    \(\rho(\vect x)=\rho(\vect y)\)
    and \(\vect x(l)<\vect y(l)\) for some \(l\in L\).

    Define \(r:Z\to \hat L\lexTimes M\)
    by \(r(\vect 0)=-\infty\) and \(r(\vect x)=\<\rho(\vect x),m\>\) where
    \(\vect x(l)=m>0\) for some \(l\in L\) otherwise.

    We first show that \(r\) is a bijection. Let \(A\) be left-closed
    and \(m\in M'\). Then let \(\vect x_{A,m}\) satisfy \(\vect x_{A,m}(a)=m\)
    for \(a\in A\) and \(\vect x_{A,m}(a)=0\) otherwise and note
    \(\vect x_{A,m}\in Z\), \(r(\vect x_{A,m})=\<A,m\>\). Showing one-to-one
    is similarly trivial.

    Finally let \(\vect 0<\vect x<\vect y\).
    If \(\rho(\vect x)\subsetneq\rho(\vect y)\), then \(r(\vect x)<r(\vect y)\).
    Otherwise \(\rho(\vect x)=\rho(\vect y)\), but \(0<\vect x(l)<\vect y(l)\)
    for some \(l\in L\) and therefore \(r(\vect x)<r(\vect y)\).
  \end{proof}

  This gives us the previous result as a corollary:
  \(
    \varprojlim^\star \{2, \nu, L\}
      \cong
    \{-\infty\}\cup(\hat L\setminus\{\emptyset\})\lexTimes\{1\}
      \cong
    \hat L
  \). Unfortunately, the result is not always compact.

  \begin{example}
    \(
      \varprojlim^\star \{I, \nu, I\}
        \cong
      \{-\infty\}\cup (0,1]\lexTimes(0,1]
    \)
  is not compact.
  \end{example}

  \begin{proof}
    The infinite set \(\{\<1-\frac{1}{2^n},1-\frac{1}{2^n}\>:0<n<\omega\}\)
    is closed discrete.
  \end{proof}

\newpage
\bibliographystyle{plain}
\bibliography{../bibliography}

\end{document}