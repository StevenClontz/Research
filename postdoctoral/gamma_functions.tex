\documentclass[11pt]{article}

\usepackage{../clontzStyle}
\usepackage{../clontzDefinitions}

\renewcommand{\int}{\textrm{int}}
\renewcommand{\cl}{\textrm{cl}}
\newcommand{\lexTimes}{\times_{\textrm{lex}}}

\begin{document}

  \begin{definition}
    Let a V-map be a u.s.c. idempotent surjection.
  \end{definition}

  \begin{definition}
    For any LOS \(\<L,\leq\>\), let \(\check L\) be the collection of
    leftward subsets of \(L\)
    (subsets for which \(b\in L,a\leq b\Rightarrow a\in L\))
    linearly ordered by \(\subseteq\), and let \(\hat L\) be the collection
    of left-open subsets of \(L\) (leftward subsets which are open)
    linearly ordered by \(\subseteq\).
  \end{definition}

  \begin{proposition}
    \(\check L\), \(\hat L\) are compact.
  \end{proposition}

  \begin{proof}
    Each subset \(S\) has a supremum \(\cup S\) and infimum \(\cap S\)
    (or \(\int(\cap S)\)).
  \end{proof}

  Note that \(\check L\)
  this is not a ``compactification'' as \(L\) does not necessarily
  embed as a dense subspace of \(\check L\): if \(L=I\), we might attempt to embed
  \(t\mapsto [0,t)\), but then note that the subspace topology induces the
  reverse Sorgenfrey interval as \(([0,s),[0,t])=([0,s),[0,t)]\) is open.
  However \(\hat L\) is basically
  the typical way of compactifying a linearly ordered space \(L\) (actually,
  left-closed is more typical, but this works similarly and fits our applications
  later),
  provided \(L\) lacks a maximum element (otherwise the whole space is an [easily
  removable] isolated point in \(\hat L\)).

  \begin{definition}
    For any compact LOTS \(K\) with minimum \(0\) and maximum \(1\),
    let \(\gamma\) be the V-map on \(K\) where \(\gamma(0)=K\) and \(\gamma(t)=\{1\}\)
    for \(t>0\).
  \end{definition}

  \begin{definition}
    For any LOTS \(M\) with minimum element \(0\),
    let \(\nu\) be the V-map on \(M\) where \(\nu(0)=K\) and \(\nu(t)=\{t\}\)
    for \(t>0\).
  \end{definition}

  Note for \(K=M=2\) that \(\gamma=\nu\).

  \begin{theorem}
    \(X=\varprojlim \{2, \nu, L\}\cong \check L\)
  \end{theorem}

  \begin{proof}
    We start by placing an order on \(X\). Let \(\vec x<\vec y\) if
    there exists \(a\in L\) with \(\vec x(a)=0,\vec y(a)=1\). We claim this is
    a total order inducing the topology on \(X\).

    We first observe that if \(\vec x(b)=1\), then for all \(a\leq b\),
    \(\vec x(a)\in\nu(1)=\{1\}\). If \(\vec x\not=\vec y\), then assume
    without loss of generality that \(\vec x(a)=0,\vec y(a)=1\), so
    \(\vec x<\vec y\). Also, whenever \(\vec x(b)=1\), we have that \(b<a\),
    so \(\vec y(b)=1\), preventing \(\vec y<\vec x\). Finally if
    \(\vec x<\vec y\) and \(\vec y<\vec z\), take \(a,b\) with
    \(\vec x(a)=0\), \(\vec y(a)=1\),\(\vec y(b)=0\),\(\vec z(b)=1\). It
    follows that \(a<b\) so \(\vec z(a)=1\) and \(\vec x<\vec z\).

    Consider the basic open set \(B(\vec x,F)\) for a finite set
    \(F\in [L]^{<\omega}\)
    about the sequence \(\vec x\in X\) which contains all sequences
    \(\vec y\) agreeing with \(\vec x\) on \(F\). If \(\vec x(a)=1\) for all
    \(a\in F\), then let \(\vec w\in X\) be \(0\) on the maximum of \(F\),
    and \(1\) for anything less. It follows that
    \(B(\vec x,F)=(\vec w,\rightarrow)\). If \(\vec x(a)=0\) for all
    \(a\in F\), then let \(\vec y\in X\) be \(1\) on the minimum of \(F\),
    and \(0\) for anything greater. It follows that
    \(B(\vec x,F)=(\leftarrow,\vec y)\). Finally if \(\vec x(a)=1\) and
    \(\vec x(b)=0\) for \(a<b\) in \(F\) and nothing between \(a,b\) is in
    \(F\), then let \(\vec w\in X\) be \(0\) on \(a\)
    and \(1\) for anything less, and let \(\vec y\in X\) be \(1\) on \(b\)
    and \(0\) for anything greater. It follows that
    \(B(\vec x,F)=(\vec w,\vec y)\).

    Let \(\phi\) evaluate each \(\vec x\in X\subseteq 2^L\) as the
    characteristic function for a subset of \(L\). It's easy to see that
    \(\phi\) is an order isomorphism between \(\<X,\leq\>\) and
    \(\<\check L,\subseteq\>\).
  \end{proof}

  We introduce an alternate definition of an arbitrarily indexed
  inverse limit.

  \begin{definition}
    Let \(\varprojlim^\star\{X,f,L\}\subseteq\varprojlim\{X,f,L\}\) satisfy
    that \(\vec x(a)=\lim_{t\to a}\vec x(t)\) for all \(a\in L\)
    (for any open neighborhood
    \(U\)of \(\vec x(a)\) there is \(b<a\) where \(\vec x(t)\in U\)
    for all \(t\in(b,a]\)).
  \end{definition}

  \begin{theorem}
    \(Y=\varprojlim^\star \{2, \nu, L\}\cong \hat L\).
  \end{theorem}

  \begin{proof}
    Consider \(Y\) as a subspace of \(X=\varprojlim \{2, \nu, L\}\) with
    the linear order described above. We claim that if \(\phi\) is the
    characteristic function for a subset of \(L\), then \(\int\circ\phi\)
    is an order isomorphsim between \(\<Y,\leq\>\) and
    \(\<\hat L,\subseteq\>\).

    Let \(A\) be a left-open subset of \(L\). Then let \(\vec x(a)=1\) when
    \(a\in\cl A\) and \(0\) otherwise. Then \(\vec x\in Y\),
    \(\phi(\vec x)=\cl A\), and \(\int\circ\phi(\vec x)=\int\cl A=A\).

    Let \(\vec x,\vec y\in Y\). If
    \(\int\circ\phi(\vec x)=\int\circ\phi(\vec y)=A\), then \(A\) is a
    left-open set where \(\vec x(a)=\vec y(a)=1\) for \(a\in A\). It follows
    that if \(b\in\cl A\), then as \(\vec x,\vec y\in Y\),
    \(\vec x(b)=\vec y(b)=1\) as well. And if \(c\not\in\cl A\), then
    as \(c\not\in\phi(x)\),
    \(\vec x(c)=\vec y(c)=0\), so \(\vec x=\vec y\).

    Finally let \(\vec x<\vec y\), so there exists \(a\in L\) with
    \(\vec x(a)=0,\vec y(a)=1\). Then
    \(
      \int\circ\phi(\vec x)
        \subseteq
      \phi(\vec x)
        \subseteq
      (\leftarrow,a)
        \subseteq
      \int\circ\phi(\vec y)
    \). Thus \(\int\circ\phi\) preserves order.
  \end{proof}

  \begin{corollary}
    \(
      \varprojlim \{2, \nu, \kappa\}
      \cong
      \varprojlim^\star \{2, \nu, \kappa\}
      \cong
      \kappa+1
    \)
  \end{corollary}

  \begin{proof}
    Since \(\hat\kappa=\kappa+1\) (actually equals, not just homeomorphic!),
    we get \(\varprojlim^\star \{2, \nu, \kappa\}
      \cong \hat\kappa =
    \kappa+1\) for free. Then observe that all leftward sets in \(\kappa\) are
    open: \(A=[0,\min(\kappa+1\setminus A))\).
    Thus we have
    \(\varprojlim \{2, \nu, \kappa\}\cong\check\kappa=\hat\kappa=\kappa+1\).
  \end{proof}

  % \begin{theorem}
  %   Let \(K\) be a compact LOTS with minimum \(0\) and maximum \(1\). Then
  %   \(Z=\varprojlim^\star \{K, \nu, L\}\cong \hat L\lexTimes K / \sim\)
  %   where \(\lexTimes\) induces the lexicographic ordering on the product,
  %   \(\<k,1\>\sim\<m,0\>\) when \(k<m\) and \((k,m)=\emptyset\).
  % \end{theorem}

  % \begin{proof}
  %   Similar to before, we introduce a linear order to \(Z\) where
  %   \(\vec x<\vec y\) if there exists \(a\in L\) where \(\vec x(a)=0\) and
  %   \(\vec y(a)>0\).
  % \end{proof}

\newpage
\bibliographystyle{plain}
\bibliography{../bibliography}

\end{document}