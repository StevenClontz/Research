\documentclass[11pt]{article}

\usepackage{../clontzStyle}
\usepackage{../clontzDefinitions}

\begin{document}

  \begin{definition}
    A space $X$ is \term{strong Eberlein compact} if it embeds in
    $\sigmaProdTwo\kappa=\{x\in2^\kappa:|\{\alpha:x(\alpha)=1\}|<\omega\}$.
  \end{definition}

  \begin{theorem}[Gruenhage]
    For compact spaces $X$,
    $X$ is strong Eberlein compact if and only if
    $X$ is scattered and $X$ is a $W$-space
    ($\pl O\win\gruConGame{X}{x}$ for all $x\in X$).
  \end{theorem}

  \begin{theorem}
    $\pl D\tactwin\bellAbsConGame{\sigmaProdTwo\kappa}$.
  \end{theorem}

  \begin{proof}
    Let $\text{supp}(x)=\{\alpha:x(\alpha)=1\}\in[\kappa]^{<\omega}$.

    Define the tactic $\sigma$ for $\pl D$ such that
      \[
        \sigma(\<x\>)
          =
        \bigcap\{
          P_\alpha(\Delta) : \alpha\in\text{supp}(x)
        \}
      \]

    Fix a legal attack $p:\omega\to\sigmaProdTwo\kappa$, and let
    $\alpha<\kappa$. If $p_\alpha:\omega\to\sigmaProdTwo\kappa$ defined by
    $p_\alpha(n)=p(n)(\alpha)$ converges, then $\sigma$ is a winning
    tactic. So assume
    $p_\alpha(n)=1$ for some $n$, and as $\alpha\in\text{supp}(p(n))$,
    $\sigma(p(n))\subseteq P_\alpha(\Delta)$. As $p$ is a legal attack,
    it follows that $p_\alpha(m)=p_\alpha(m+1)$ for all $m>n$, so
    $p_\alpha$ converges. Otherwise $p_\alpha(n)=0$ for all $n$ so
    $p_\alpha$ converges.
  \end{proof}

  \begin{corollary}
    If $X$ is strong Eberlein compact, then $\pl D\tactwin\bellAbsConGame{X}$.
  \end{corollary}

  \begin{theorem}\label{cantorCopy}
    If $X$ contains a copy of the Cantor set, then
    $\pl D\nottactwin\bellConGame{X}$.
  \end{theorem}

  \begin{proof}
    The result follows from showing that
    $\pl D\nottactwin\bellAbsConGame{2^\omega}$ (any copy of the Cantor
    set within a Hausdorff space is a compact and thus closed subspace).
    Let $\sigma$ be a tactic for $\pl D$ in $\bellAbsConGame{2^\omega}$
    and let $D_k=\{\<f,g\>:f\rest k = g\rest k\}$. Since $\{D_k:k<\omega\}$
    is a base for the uniformity on $2^\omega$, we may fix $k(f)<\omega$
    for each $f\in2^\omega$ such that $D_{k(f)}\subseteq\sigma(\<f\>)$.

    Then there exists $k<\omega$ such that $\{f:k=k(f)\}$ is uncountable,
    and therefore there exist distinct $f,g$ such that $k=k(f)=k(g)$ and
    $f\rest k=g\rest k$. Then $p:\omega\to2^\omega$ defined by
    $p(2n)=f$ and $p(2n+1)=g$ is an attack against $\sigma$ which
    obviously doesn't converge. This attack is legal since
    $f\in D_k[g]\subseteq\sigma(\<g\>)[g]$ and
    $g\in D_k[f]\subseteq\sigma(\<f\>)[f]$.
  \end{proof}

  \begin{lemma}
    Every non-scattered Corson compact space contains a homeomorphic
    copy of the Cantor set.
  \end{lemma}

  \begin{proof}
    Every non-scattered space contains a closed subspace without
    isolated points. Let $X$ be such a subspace, and assume that this
    Corson compact is embedded in $\SigmaProdR\kappa$. Let
    $B_{\alpha,\epsilon}(x)=\{y: d(x(\alpha),y(\alpha))<\epsilon\}$.
    For each $x\in X$ and $n<\omega$, let $\beta(x,n)<\kappa$ be defined
    such that
    $\{\alpha:x(\alpha)\not=0\}=\{\beta(x,n):n<\omega\}$.

    Choose an arbitrary $x_\emptyset\in X$ and $\epsilon_0>0$, and
    and let $A_0=\emptyset$.

    Suppose then that for some $n<\omega$,
    $x_s\in X$ is defined for all $s\in 2^n$,
    and $\epsilon_n>0$ and $A_n\in[\kappa]^{<\omega}$ are defined.
    Since each $x_s$ is not isolated in $X$, let $U_s$ be the open
    set
      \[
        U_s
          =
        X
          \cap
        \bigcap_{\alpha\in A_{|s|}} B_{\alpha,\epsilon_{|s|}}(x_s)
      \]
    and choose $x_{s\concat\<0\>},x_{s\concat\<1\>}\in U_s$ distinct.
    Then let $\alpha_s<\kappa$ such that
    $x_{s\concat\<0\>}(\alpha_s)\not=x_{s\concat\<1\>}(\alpha_s)$.
    Let
      \[
        A_{n+1}
          =
        \{\alpha_s:s\in2^{\leq n}\}
          \cup
        \{\beta(x_s,i):s\in2^{\leq n},i\leq n\}
      \]

    Then choose $0<\epsilon_{n+1}<\frac{1}{2}\epsilon_n$ such that
    \[
      B_{\alpha_s,\epsilon_{n+1}}(x_s\concat\<0\>)
        \cap
      B_{\alpha_s,\epsilon_{n+1}}(x_s\concat\<1\>)
        =
      \emptyset
    \]
    and
    \[
      \overline{
        \bigcap_{\alpha\in A_{n+1}}
        B_{\alpha,\epsilon_{n+1}}(x_s\concat\<0\>)
      }
        \cup
      \overline{
        \bigcap_{\alpha\in A_{n+1}}
        B_{\alpha,\epsilon_{n+1}}(x_s\concat\<1\>)
      }
        \subseteq
      \bigcap_{\alpha\in A_n} B_{\alpha,\epsilon_n}(x_s)
    \]
    for all $s\in 2^n$.

    Let $x_f=\lim_{n<\omega} x_{f\rest n}\in X$
    for each $f\in 2^\omega$. We claim $C=\{x_f:f\in 2^\omega\}$
    is a copy of the Cantor set. This will follow if we can show that
    $\{U_s:s\in 2^{<\omega}\}$ is a base for $C$, since it has
    the structure of the Cantor tree.

    Consider $x_f$ for some $f\in 2^\omega$, and a subbasic open ball
    $B_{\alpha,\epsilon}(x_f)$. Observe that
    $x_f\in\bigcap_{n<\omega} U_{f\rest n}$ since
    $x_{f\rest n}\in U_{f\rest m}$ for all $m<n<\omega$.

    If $\alpha\in\{\beta(x_s,n):s\in2^{<\omega},n<\omega\}$, choose
    $k<\omega$ with $\alpha\in A_k$. Then choose $l<\omega$ such that
    $\epsilon_l<\epsilon$. Then
    $U_{f\rest(l+k)}\subseteq B_{\alpha,\epsilon}(x_f)$.

    Otherwise, $x_s(\alpha)=0$ for all $s\in2^{<\omega}$, so
    $x_g(\alpha)=0$ for all $g\in2^\omega$ and therefore
    $C\subseteq B_{\alpha,\epsilon}(x_f)$.
  \end{proof}

  \begin{corollary}
    For compact spaces $X$,
    $X$ is strong Eberlein compact if and only if
    $\pl D\tactwin\bellConGame{X}$.
  \end{corollary}

  \begin{proof}
    Suppose $X$ is not strong Eberlien compact; then $X$ is either
    not a $W$-space or not scattered.
    If $\pl D\notwin\bellConGame{X}$, then the result follows immediately,
    which only leaves non-scattered proximal compact spaces to be considered.
    But non-scattered proximal compacts are non-scattered Corson compacts,
    and thus contain
    a copy of the Cantor set, so the result follows from Theorem
    \ref{cantorCopy}.
  \end{proof}

  \newpage

  \begin{definition}
    A space $X$ is \term{Eberlein compact} if it embeds in
    $
      \SigmaStarProdR\kappa
        =
      \{
        x\in2^\kappa
      :
        |\{\alpha:|x(\alpha)|\geq \epsilon\}|<\omega
        \text{ for all } \epsilon > 0
      \}
    $.
  \end{definition}

  \begin{theorem}
    $\pl D\markwin\bellAbsConGame{\SigmaStarProdR\kappa}$.
  \end{theorem}

  \begin{proof}
    Let
    $
      \text{supp}_\epsilon(x)
        =
      \{\alpha:|x(\alpha)|\geq \epsilon\}
        \in
      [\kappa]^{<\omega}
    $. Let $D_\epsilon$ be the entourage of the diagonal formed by
    balls of radius $\epsilon$.

    Define the tactic $\sigma$ for $\pl D$ such that
      \[
        \sigma(\<x\>,n)
          =
        \bigcap\{
          P_\alpha(D_{2^{-n}}) : \alpha\in\text{supp}_{2^{-n}}(x)
        \}
      \]

    % Fix a legal attack $p:\omega\to\sigmaProdTwo\kappa$, and let
    % $\alpha<\kappa$. If $p_\alpha:\omega\to\sigmaProdTwo\kappa$ defined by
    % $p_\alpha(n)=p(n)(\alpha)$ converges, then $\sigma$ is a winning
    % tactic. So assume
    % $p_\alpha(n)=1$ for some $n$, and as $\alpha\in\text{supp}(p(n))$,
    % $\sigma(p(n))\subseteq P_\alpha(\Delta)$. As $p$ is a legal attack,
    % it follows that $p_\alpha(m)=p_\alpha(m+1)$ for all $m>n$, so
    % $p_\alpha$ converges. Otherwise $p_\alpha(n)=0$ for all $n$ so
    % $p_\alpha$ converges.
  \end{proof}












  \newpage

  Miscellaneous:

  \begin{example}
    $\pl D\tactwin \bellConGame{\oneptcomp\kappa}$,
    so $\pl D\markwin \bellConGame{(\oneptcomp\kappa)^\omega}$.
  \end{example}

  % probably false
  % \begin{theorem}
  %   If $\alcompS\kappa$ holds, then
  %   $\pl D\kmarkwin2 \bellConGame{\SigmaProdTwo\kappa}$.
  % \end{theorem}

  % \begin{proof}
  %   For $x\in\SigmaProdTwo\kappa$, let
  %     \[
  %     \text{diff}(x,y)
  %       =
  %     \{\alpha:x(\alpha)\not=y(\alpha)\}
  %       \in
  %     [\kappa]^{\leq\omega}
  %     \]

  %   For an entourage $D$ of a space $X_\alpha$, let
  %   $P_\alpha(D)=\{\<x,y\>:\<x(\alpha),y(\alpha)\>\in D\}$ be an
  %   entourage of $\prod_{\alpha<\kappa}X_\alpha$.
  %   Let $f_A:A\to\omega$ witness $\alcompS\kappa$ for
  %   $A\in[\kappa]^{<\omega}$.

  %   Then define a $2$-mark $\sigma$ for $\pl D$ in
  %   $\bellConGame{\SigmaProdTwo\kappa}$ such that:
  %     \[
  %       \sigma(\<x,y\>,n+2)
  %         =
  %       \bigcap\{
  %       P_{\alpha}(\Delta)
  %         :
  %       f_{\text{diff}(x,y)}(\alpha)\not=f_{\text{supp}(y)}(\alpha)
  %       \}
  %     \]

  %   Let $\alpha<\kappa$ and consider an arbitrary legal attack
  %   $p:\omega\to\SigmaProdTwo\kappa$. The result follows if we show
  %   that $p_\alpha:\omega\to2$ where $p_\alpha(n)=p(n)(\alpha)$
  %   converges.

  %   If $p_\alpha(n)=0$ or $1$ for cofinitely many $n$, then we are done.
  %   Otherwise,
  % \end{proof}

\end{document}