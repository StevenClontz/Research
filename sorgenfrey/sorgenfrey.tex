\documentclass[11pt]{article}

\pdfpagewidth 8.5in
\pdfpageheight 11in

\setlength\topmargin{0in}
\setlength\headheight{0in}
\setlength\headsep{0.4in}
\setlength\textheight{8in}
\setlength\textwidth{6in}
\setlength\oddsidemargin{0in}
\setlength\evensidemargin{0in}
\setlength\parindent{0.25in}
\setlength\parskip{0.1in} 
 
\usepackage{amssymb}
\usepackage{amsfonts}
\usepackage{amsmath}
\usepackage{mathtools}
\usepackage{amsthm}

\usepackage{fancyhdr}

\usepackage{enumerate}

      \theoremstyle{plain}
      \newtheorem{theorem}{Theorem}
      \newtheorem{lemma}[theorem]{Lemma}
      \newtheorem{corollary}[theorem]{Corollary}
      \newtheorem{proposition}[theorem]{Proposition}
      \newtheorem{conjecture}[theorem]{Conjecture}
      \newtheorem{question}[theorem]{Question}
      
      \theoremstyle{definition}
      \newtheorem{definition}[theorem]{Definition}
      \newtheorem{example}[theorem]{Example}
      \newtheorem{game}[theorem]{Game}
      
      \theoremstyle{remark}
      \newtheorem{remark}[theorem]{Remark}



\pagestyle{fancy}
\renewcommand{\headrulewidth}{0.5pt}
\renewcommand{\footrulewidth}{0pt}
\lfoot{\small \jobname.tex -- Updated on \today}
\chead{\small http://github.com/StevenClontz/Research}
\rfoot{\thepage}
\cfoot{}


% Strategy uparrow shortcuts
\newcommand{\win}{\uparrow}
\newcommand{\prewin}{\uparrow_{\text{pre}}}
\newcommand{\markwin}{\uparrow_{\text{mark}}}
\newcommand{\tactwin}{\uparrow_{\text{tact}}}
\newcommand{\ktactwin}[1]{\uparrow_{#1\text{-tact}}}
\newcommand{\kmarkwin}[1]{\uparrow_{#1\text{-mark}}}
\newcommand{\codewin}{\uparrow_{\text{code}}}
\newcommand{\limitwin}{\uparrow_{\text{limit}}}

\newcommand{\oneptcomp}[1]{#1^*}
\newcommand{\oneptlind}[1]{#1^\dagger}

\newcommand{\congame}[2]{Con_{O,P}(#1,#2)}
\newcommand{\clusgame}[2]{Clus_{O,P}(#1,#2)}

\newcommand{\lfkpgame}[1]{LF_{K,P}(#1)}
\newcommand{\lfklgame}[1]{LF_{K,L}(#1)}

\newcommand{\pfgame}[1]{PF_{F,C}(#1)}

\newcommand{\mengame}[1]{Cov_{C,F}(#1)}
\newcommand{\rothgame}[1]{Cov_{C,S}(#1)}
\newcommand{\altrothgame}[1]{Cov_{P,O}(#1)}

\newcommand{\fillgame}[1]{Fill^{\subseteq}_{M,N}(#1)}
\newcommand{\sfillgame}[1]{Fill^{\subsetneq}_{M,N}(#1)}

\newcommand{\kfillgame}[1]{Fill^{\subseteq}_{C,F}(#1)}
\newcommand{\ksfillgame}[1]{Fill^{\subsetneq}_{C,F}(#1)}

\newcommand{\sigmaprodr}[1]{\Sigma\mathbb{R}^{#1}}
\newcommand{\sigmaprodtwo}[1]{\Sigma2^{#1}}

\newcommand{\concat}{^\frown}
\newcommand{\rest}{\restriction}

\newcommand{\cl}[1]{\overline{#1}}

\newcommand{\pow}[1]{\mc{P}(#1)}

\newcommand{\<}{\langle}
\renewcommand{\>}{\rangle}

\newcommand{\mc}[1]{\mathcal{#1}}

\newcommand{\Lim}{\mathrm{Lim}}
\newcommand{\Suc}{\mathrm{Suc}}

\newcommand{\ds}{\displaystyle}

\newcommand{\rank}{\textrm{rank}}

\newcommand{\scish}{$\sigma$-compactish }

\begin{document}

In this paper we investigate an open question posed to us by Gruenhage:

\begin{question}
Let $P$ be the subspace of the Sorgenfrey line containing only irrational numbers. Does there exist a base $\mc{B}$ for $P$ such that for every $\mc{C}\subseteq\mc{B}$ which is also a base for $X$, there exists a locally finite subcover $\mc{C}'\subseteq\mc{C}$?
\end{question}

We begin by tackling a simpler (solved) problem:

\begin{proposition}
Let $R$ be the Sorgenfrey line, the set of real numbers with the topology generated by the base $\mc{B} = \{[a,b): a<b\}$ (where $[a,b) = \{x : a\leq x < b\}$).

For every $\mc{C}\subseteq\mc{B}$ which is also a base for $R$, there exists a pairwise disjoint subcover $\mc{C}'\subseteq\mc{C}$ (and thus a locally finite subcover).
\end{proposition}

\begin{proof}
We begin by letting $b_{n,0}=n$ for each $n<\omega$, and if $b_{n,\alpha}$ is defined for some ordinal $\alpha<\omega_1$ and $b_{n,\alpha}<n+1$, we define its successor $b_{n,\alpha+1}$ as follows:
  \begin{itemize}
    \item $b_{n,\alpha} < b_{n,\alpha+1} \leq n+1$
    \item $[b_{n,\alpha},b_{n,\alpha+1})\in \mathcal{C}$ 

    (This is possible as $\mc C$ is a base, and there must be some element of $\mc C$ which contains $b_{n,\alpha}$ and is a subset of $[b_{n,\alpha},n+1)$.)
  \end{itemize}
(If $b_{n,\alpha}=n+1$, then let $b_{n,\alpha+1}=n+1$ as well.) Finally, if $\alpha<\omega_1$ is a limit ordinal, let $\ds b_{n,\alpha} = \lim_{\beta\to\alpha}b_{n,\beta}$.

Let $C_{n,\alpha} = [b_{n,\alpha},b_{n,\alpha+1})$. We claim $\mc{C}' = \{C_{n,\alpha} : n<\omega,\alpha<\omega_1\}$ is a pairwise disjoint cover of $R$. Pairwise disjoint is evident by definition. To see that it is a cover, suppose it wasn't and missed some $x\in[n,n+1)$. Then we have an uncountable increasing sequence of numbers $\{b_{n,\alpha}:\alpha<\omega_1\}$, which contradicts the countable chain condition on the real line.
\end{proof}

This idea can actually be generalized for any dense subspace, but it takes some machinery:

\begin{theorem}
Let $P$ be a dense subspace of the Sorgenfrey line with the induced topology, which has the base $\mc{B} = \{[a,b) : a<b, a\in P\}$.

For every $\mc{C}\subseteq\mc{B}$ which is also a base for $P$, there exists a pairwise disjoint subcover $\mc{C}'\subseteq\mc{C}$ (and thus a locally finite subcover).
\end{theorem}
\newpage
\begin{proof}
It suffices to show for $[0,1)\cap P$. We begin by constructing a collection of functions $S\subseteq \omega^{\omega_1}$ and numbers $c_s$, $d_s$ defined by those functions as follows:
  \begin{itemize}
    \item Let $S_0 = \{\emptyset\}$. Let $d_{\emptyset}=0$ and $c_{\<-1\>}=1$.
    \item Suppose $S_\alpha$ has been defined, as well as $d_s\leq c_{s\concat\<-1\>}$. For $s\in S$, consider the following:
      \begin{itemize}
        \item If $d_s = c_{s\concat\<-1\>}$, do nothing.
        \item If $d_s < c_{s\concat\<-1\>}$ and $d_s\in P$, let $S_{\alpha+1}$ contain $s\concat\<0\>$ and define $c_{s\concat\<0\>}$, $d_{s\concat\<0\>}$, $c_{s\concat\<0,-1\>}$ such that 
          \[
            d_s = c_{s\concat\<0\>} < d_{s\concat\<0\>} \leq c_{s\concat\<0,-1\>} = c_{s\concat\<-1\>}
          \]
        where $[c_{s\concat\<0\>}, d_{s\concat\<0\>})\in \mc{C}$.
        \item If $d_s < c_{s\concat\<-1\>}$ and $d_s\in P$, let $S_{\alpha+1}$ contain $s\concat\<n\>$ for all $n<\omega$ and define $c_{s\concat\<n\>}$, $d_{s\concat\<n\>}$, $c_{s\concat\<n,-1\>}$ for all $n<\omega$ such that
          \[
            d_s < \dots \leq c_{s\concat\<2,-1\>} = c_{s\concat\<1\>} < d_{s\concat\<1\>} \leq c_{s\concat\<1,-1\>} = c_{s\concat\<0\>} < d_{s\concat\<0\>} \leq c_{s\concat\<0,-1\>} = c_{s\concat\<-1\>}
          \]
        where $[c_{s\concat\<n\>}, d_{s\concat\<n\>})\in \mc{C}$ for all $n<\omega$ and $c_{s\concat\<n\>} \rightarrow d_s$.
      \end{itemize}
    \item Suppose $\alpha$ is a limit ordinal and $S_\beta$ has been defined for all $\beta<\alpha$. If $s\in\omega^\alpha$ and $t\in \bigcup_{\beta<\alpha}S_\beta$ for all $t<s$, let $S_\alpha$ contain $s$ and define $d_s = \lim_{t<s}d_t$ and $c_{s\concat\<-1\>} = \lim_{t<s} c_{t\concat\<-1\>}$.
  \end{itemize}

  By construction, $\mc{C}' = \{[c_s,d_s) : s\in S\}$ is a disjoint subcollection of $\mc{C}$. We claim it also must cover $[0,1)\cap P$.

  Suppose not: $x\in[0,1)\cap P$ is not contained in $[c_s,d_s)$ for any $s\in S$. Note that $d_\emptyset<x$ (or else $x\in[c_{\<0\>},d_{\<1\>}$)). Assume $n_\beta<\omega$ is defined for all $\beta<\alpha$, and consider $s\in\omega^\alpha$ where $s(\beta)=n_\beta$. If $d_s<x$, we claim there is a minimal $n_\alpha<\omega$ where $d_{s\concat\<n_\alpha\>}<x$.
    \begin{itemize}
      \item This possible when $d_s\not\in P$ since $d_{s\concat\<n\>}\to d_s$.
      \item This is also possible when $d_s\in P$ since $[c_{s\concat\<0\>},d_{s\concat\<0\>})$ does not contain $x$, and thus $d_s=c_{s\concat\<0\>}<d_{s\concat\<0\>}\leq x$. If $d_{s\concat\<0\>} = x$ then $d_{s\concat\<0\>}\in P$ and $[d_{s\concat\<0\>},d_{s\concat\<0,0\>})=[c_{s\concat\<0,0\>},d_{s\concat\<0,0\>})$ contains $x$, which is a contradiction.
    \end{itemize}

  Finally, we notice that by defining $f\in\omega^\omega_1$ such that $f(\alpha)=n_\alpha$, then $d_{f(\alpha)}$ is an increasing sequence defined for all $\alpha<\omega_1$, which is a contradiction.
\end{proof}

\end{document}